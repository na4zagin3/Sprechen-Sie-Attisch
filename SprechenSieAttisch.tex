%% LyX 2.1.3 created this file.  For more info, see http://www.lyx.org/.
%% Do not edit unless you really know what you are doing.
\documentclass[english,french,latin,german,twoside]{scrartcl}
\usepackage{fontspec}
%\setmainfont[Mapping=tex-text,Numbers=OldStyle]{Old Standard}
\setmainfont[Ligatures=TeX]{Old Standard}
\usepackage[b5paper]{geometry}
\geometry{verbose,lmargin=3cm}
\pagestyle{plain}
\usepackage{verbatim}
\usepackage{url}
\usepackage{setspace}
\setstretch{1.1}
\usepackage[unicode=true,
 bookmarks=true,bookmarksnumbered=false,bookmarksopen=false,
 breaklinks=false,pdfborder={0 0 1},backref=false,colorlinks=false]
 {hyperref}
\hypersetup{pdftitle={Sprachen Sie Attich?},
 pdfauthor={J. Sengbusch}}

\makeatletter

%%%%%%%%%%%%%%%%%%%%%%%%%%%%%% LyX specific LaTeX commands.
%% Because html converters don't know tabularnewline
\providecommand{\tabularnewline}{\\}

%%%%%%%%%%%%%%%%%%%%%%%%%%%%%% Textclass specific LaTeX commands.
\usepackage{enumitem}		% customizable list environments
\newlength{\lyxlabelwidth}      % auxiliary length 

%%%%%%%%%%%%%%%%%%%%%%%%%%%%%% User specified LaTeX commands.
\usepackage{ifpdf}
\usepackage{ifluatex}
\usepackage{ifxetex}
\usepackage{xspace}
\usepackage{etoolbox}
\usepackage{trace}
\usepackage{paracol}
\usepackage{bigdelim}
\usepackage{lineno}
\usepackage{indentfirst}
\usepackage{quoting}
\usepackage{version}
\usepackage{calc}
\usepackage{everypage}
\usepackage{multicol}
% \usepackage{titlesec}
\usepackage{tocloft}

% \renewcommand\cftchapafterpnum{\vskip0pt}
\renewcommand\cftsecafterpnum{\vskip-0.5em}
\usepackage{microtype}

\includeversion{frakturfonttypesetting}
\excludeversion{romanfonttypesetting}

\newif\ifFraktur
\Frakturtrue

\usepackage{multirow}
\ifpdf
  \ifluatex
    %\usepackage{luatexja-fontspec}
   % \usepackage[hiragino-pro,deluxe]{luatexja-preset}
  \else
    %\usepackage[pdftex]{hyperref}
  \fi
\else
  \ifxetex
    \usepackage{zxjatype}
    %\setjamainfont{YuMincho}
    %\setjasansfont{YuGothic}
    \setjamonofont{IPAGothic}
  \else
    %\usepackage[dvipdfmx]{hyperref}
    \usepackage{pxjahyper}
  \fi
\fi

\newcommand*{\germanlanguagename}{german}
\newcommand*{\greeklanguagename}{greek}
\newcommand*{\ifsperrsatzlanguage}[2]{%
\ifboolexpr{%
  test {\ifdefstrequal{\languagename}{\germanlanguagename}}
   or
  test {\ifcsstrequal{\languagename}{\germanlanguagename}}
  or
  test {\ifdefstrequal{\languagename}{\greeklanguagename}}
  or
  test {\ifcsstrequal{\languagename}{\greeklanguagename}}
}{#1}{#2}%
}
\ifdefined\newfontfamily
  %\newfontfamily\greekfontsf[Script=Greek]{Linux Biolinum O}
  \newfontfamily\greekfontsf[Script=Greek,Ligatures={TeX,Common,Contextual,Rare,Historic}]{Linux Biolinum O}
  %\newfontfamily\greekfontsf[Script=Greek,Ligatures={TeX,Common,Contextual,Rare,Historic}]{Linux Biolinum}
  \ifFraktur
    %\newfontfamily\greekfont[Script=Greek,Ligatures={TeX,Common,Contextual,Rare,Historic}]{Anaktoria}
    \newfontfamily\greekfont[Script=Greek,Ligatures={TeX,Common,Contextual,Rare,Historic}]{Theano Old Style}
    \newfontfamily\greekfont[Script=Greek,Ligatures={TeX}]{Old Standard Italic}
    \newfontfamily\greekfont[Script=Greek,Ligatures={TeX,Contextual}]{Junicode}
    %\newfontfamily\greekfont[Script=Greek,Ligatures={TeX,Common,Contextual,Rare,Historic}]{Asea}
    %\newfontfamily\greekfont[Script=Greek,Ligatures={TeX,NoCommon}]{Times Italic}
    \newfontfamily\germanfont[Script=Latin,Language=German,Mapping=tex-text,Numbers=OldStyle,CharacterVariant={11,13,14},Ligatures=Historic]{UnifrakturMaguntia}
    \newfontfamily\germanfontsf[Script=Latin,Language=German,Mapping=tex-text,CharacterVariant={11,13,14}]{UnifrakturCook}
    \setsansfont[Script=Latin,Language=German,Mapping=tex-text,CharacterVariant={11,13,14}]{UnifrakturCook}

    \let\origemshape=\emshape
    \let\origemph=\emph
    % orig:
    %\renewcommand\emshape{\xspace\addfontfeature{LetterSpace=20.0,WordSpace=1.5,Ligatures={NoCommon}}}
    % new:
   % \renewcommand\emshape{\nobreak\discretionary{}{}{\hbox{\xspace}}\addfontfeature{LetterSpace=20.0,WordSpace=1.5,Ligatures={NoCommon}}}
    \renewcommand\emshape{%
%\ifx\languagename\germanlanguagename%
\ifboolexpr{test {\ifsperrsatzlanguage}}{%
 % \addfontfeature{LetterSpace=15.0,WordSpace=1.5,Ligatures={NoCommon}}%
  \spaceskip=0.8em plus 0.2em minus 0.4em\relax
  \addfontfeature{LetterSpace=17.0,Ligatures={NoCommon}}%
  \renewcommand{\textcompwordmark}{\-\hspace*{0.15em}}%
}{\origemshape}}
    \preto\emph{%
      %\ifx\languagename\germanlanguagename\ifvmode\else\xspace\fi\fi
      \ifsperrsatzlanguage{\ifvmode\else\xspace\fi}{}%
      %\ifboolexpr{test {\ifsperrsatzlanguage} and ( not bool{vmode} ) }{%
      %  \xspace
      %}{}%
    }
  \else
    %\newfontfamily\greekfont[Script=Greek,Ligatures={TeX,Common,Contextual,Rare,Historic}]{Asea}
    \newfontfamily\greekfont[Script=Greek,Ligatures={TeX,Common,Contextual,Rare,Historic}]{Theano Old Style}

  \fi
\fi

%\newfontfamily\greekfont[Script=Greek,Ligatures={TeX,Common,Contextual,Rare,Historic}]{Asea}
%\renewcommand{\emph}[1]{{\addfontfeature{LetterSpace=20.0}#1}}

%\renewcommand\emshape{\ifx\languagename\germanlanguagename\xspace\addfontfeature{LetterSpace=20.0,WordSpace=1.5,Ligatures={NoCommon}}\else\origemshape\fi}

\begin{frakturfonttypesetting}

\end{frakturfonttypesetting}

\newcommand\frontmatter{%
    \cleardoublepage
  %\@mainmatterfalse
  \pagenumbering{roman}}

\newcommand\mainmatter{%
    \cleardoublepage
 % \@mainmattertrue
  \pagenumbering{arabic}}

\newcommand\backmatter{%
  \if@openright
    \cleardoublepage
  \else
    \clearpage
  \fi
 % \@mainmatterfalse
   }

%\gappto\enumerate{%
%\setlength{\listparindent}{0.2in}%
%\setlength{\labelwidth}{0pt}%
%\setlength{\itemindent}{0.2in}%
%}

\quotingsetup{vskip=0pt,rightmargin=0pt}

% from http://tex.stackexchange.com/questions/36974/adding-an-open-quote-mark-to-the-start-of-each-line-in-a-multiline-quotation
%\newcommand{\leftquotes}{\def\makeLineNumber{%
%  \ifnum\value{linenumber}=1 \else\hskip\leftmargin\llap{\quotedblbase}\hss\fi}}
\newcommand{\leftquotes}{\def\makeLineNumber{%
  \hskip\leftmargin\llap{\quotedblbase}\hss}}
\newenvironment{quotedquotation}
  {\quoting\linenumbers\leftquotes}
  {\endquoting}

%\setlength{\parsep}{0pt}

\newsavebox{\mygermanalignbox}
\newsavebox{\mygreekalignbox}

\newsavebox{\mygermanaligntext}{}
\newcommand{\mygreekaligntext}{}

\newlength{\mygermanalign}
\newlength{\mygreekalign}

\newif\ifmyafterpage
\myafterpagetrue

\newcommand{\mysetaligntextsub}[3]{%
\global\savebox{#1}{#3}%
\global\setlength{#2}{\widthof{#1}}%
}
\newcommand{\mysetaligntextt}[2]{%
\expandafter\mysetaligntextsub{\csname my#1alignbox\endcsname}{\csname my#1align\endcsname}{#2}%
}
\newcommand{\mysetaligntext}[2]{%
\global\sbox{\csname my#1alignbox\endcsname}{#2}%
%\expandafter\global\savebox{\csname my#1alignbox\endcsname}{#2}%
%\global\setlength{\csname my#1align\endcsname}{\wd\csname my#1alignbox\endcsname}%
\global\csname my#1align\endcsname=\wd\csname my#1alignbox\endcsname%
}

\NewDocumentCommand \mysetalign { s m }
{%
\ifmyafterpage%
\usebox{\csname my#2alignbox\endcsname}%
\else%
\advance\leftskip\csname my#2align\endcsname%
\fi%
\IfBooleanTF {#1}%
{\global\myafterpagefalse }%
{}%
}
%\newcommand{\mysetalign}[1]{}
\newcommand{\myaligntext}[1]{\ifmyafterpage\else\fi}
%
\AddEverypageHook{\global\myafterpagetrue}
%\rfoot{\global\myafterpagetrue}
\newcommand{\StarOrnament}{\begin{center}\begin{minipage}{10em}\centering$\ast$\hspace{5em}$\ast$\\$\ast$\end{minipage}\end{center}\par}

\newcommand{\source}[1]{%
  \nobreak\parbox[t]{\linewidth}{\raggedleft #1}% Placing a quote source
}%
\setlength{\tabcolsep}{0em}

\newsavebox{\mycontinuousitemlinebox}
\newenvironment{continuousitemline}{%
  \bgroup%
  \global\sbox{\mycontinuousitemlinebox}{\theenumi.}%
  \addtolength{\leftskip}{\wd\mycontinuousitemlinebox}%
  \addtolength{\leftskip}{1ex}%
  %\addtolength{\leftskip}{2ex}%
  \addtolength{\leftskip}{1em}%
  \addtolength{\parindent}{-1em}%
  %\llap{"\usebox{\mycontinuousitemlinebox}"}%
  %\rule{\wd\mycontinuousitemlinebox}{1pt}%
  %\rule[1ex]{1ex}{1pt}%
%
}{%
  \egroup%
}


\newenvironment{continuousexamples}{%
  \bgroup%
  \setlength{\leftskip}{\parindent}%
%  \addtolength{\leftskip}{2ex}%
  \addtolength{\leftskip}{2em}%
  \addtolength{\parindent}{-4em}%
}{%
  \egroup%
}
\renewcommand*{\ldots}{. . .}

\makeatother

\usepackage{xunicode}
\usepackage{polyglossia}
\setdefaultlanguage[spelling=old,babelshorthands=true]{german}
\setotherlanguage{english}
\setotherlanguage{french}
\setotherlanguage[variant=ancient]{greek}
\setotherlanguage{latin}
\setotherlanguage{japanese}

\newcommand{\trDE}[1]{#1}
\newcommand{\trLA}[1]{#1}
\newcommand{\trJA}[1]{}

\begin{document}
\frontmatter
\renewcommand\thepart{\Alph{part}}


\title{{\fontfamily\rmdefault\selectfont Sprechen Sie Attisch?}}


\subtitle{\rule[1.2ex]{0.2\columnwidth}{1pt}\\
Moderne Conversation\\
\emph{in altgriechischer Umgangs\textcompwordmark{}sprache}\\
nach den besten attischen Autoren}


\author{{\footnotesize{}von}\bigskip{}
\\
\textlatin{\textbf{E. Joannides,}}\\
\textlatin{{\small{}Dr. phil.}}\smallskip{}
\textlatin{{\small{}}}\\
{\tiny\begin{minipage}{\columnwidth}
\begin{flushright}
\textlatin{--- --- Ridentem discere Graeca\\
Quid vetat? --- ---}
\end{flushright}
\end{minipage}}\smallskip{}
\\
\rule[1.2ex]{0.2\columnwidth}{1pt}}


\publishers{\textbf{Leipzig, 1889.}\\
\emph{C.\ A.\ Koch's Verlag.}\\
(J. Sengbusch)}


\date{\relax}

\maketitle
\pagebreak{}

\vfill{}


\begin{center}
{\Huge{}(Das originale Buch hat Ankündigungen hier.)}
\par\end{center}{\Huge \par}

\vfill{}


\pagebreak{}

\section*{Vorbemerkungen\addcontentsline{toc}{section}{\emph{Vorbemerkungen} über die Bedeutung der attischen Umgangs\textcompwordmark{}sprache für das Erlernen des Griechischen}}

Griechisch gilt den Allermeisten für eine im Grunde unlernbare Sprache,
deren man nimmermehr so mächtig werden könne, wie einer neueren, die
man leidlich beherrscht. Vorliegendes Büchlein, das fröhlicher Ferienlaune
seinen Ursprung verdankt, möchte den Gegenbeweis führen, indem es
einem ersten Versuch macht, attische Umgangs\textcompwordmark{}sprache
in ihren gebräuchlichsten Wendungen zu lehren.

Wer die Umgangs\textcompwordmark{}sprache eines Volkes kennt, hat
den Schlüssel zum Verständniß seiner Schriftwerke gleich den Volks\textcompwordmark{}genossen
selbst.

Der attische Knabe brachte zur Lectüre griechischer Dichter, der attische
Bauer in sein Theater oder in die Volks\-ver\-samm\-lung nur die
Kenntniß der attischen Umgangs\textcompwordmark{}sprache in ihrer
einfachsten Form mit; \emph{sie} befähigte zum Verständniß sophokleïscher
Dramen und perikleïsche Reden. Die Sprache des Alltags\textcompwordmark{}lebens
lieferte diejenigen Analogien, welche zum Erfassen der höheren Erzeugnisse
in Rede und Schrift nothwendig waren.

Man hat oft behauptet, daß es erstaunlich wenig Worte und Wendungen
sind, mit denen der gemeine Mann in seiner Muttersprache aus\textcompwordmark{}kommt
und die ihn befähigen, auch das zu verstehen, was für ihn Neubildung
ist. Sollte es nicht möglich sein, dem Athener seinen verhältnißmäßig
kleinen Urvorrath abzulauschen, somit die Sprache in ihrem \emph{Kerne}
zu erfassen und diese Worte und Wendungen demjenigen, der Griechisch
wirklich lernen will, geläufig zu machen?

Aristophanes bietet für diesen Zweck in denjenigen Partien, wo er
den gemeinen Mann im volks\textcompwordmark{}thümlichen Verkehrs\textcompwordmark{}tone
reden läßt, sprachlichen Stoff genug, und auch in der übrigen Literatur
finden sich verstreut Stellen, welche für treue Nachahmungen der Sprache
des gemeinen Lebens gelten müssen. Die Aufgabe kann also nicht unlös\textcompwordmark{}bar
sein, wenn auch das vorliegende Schrift\textcompwordmark{}chen nur
erst einen kleinen Beitrag zu ihrer Lösung bringt.

Die Worte und Wendungen in den nachstehenden Gesprächen sind in der
Hauptsache der aristphanischen Sprache entnommen. Einiges mußte aus
der späteren Gräcität beigefügt werden. Die dem Neugriechischen entlehnten
Ergänzungen, welche zur Bezeichnung moderner Begriffe verwandt wurden,
sind durch {*} besonders kenntlich gemacht.

Auch wer nicht die Absicht hat, attisch conversiren zu lernen, wird
mit vielem Nutzen für sein Verständniß des Griechischen sich mit der
attischen Umgangs\textcompwordmark{}sprache beschäftigen. Denn während
man auf unseren Gymnasien im Lateinischen fast nur solche Schriften
liest, welche der höheren Kunst\textcompwordmark{}sprache angehören
--- man denke nur and Cicero und Tacitus --- und in welchen die Volks\textcompwordmark{}sprache
kaum hier und da er\textcompwordmark{}kennbar ist, werden wir im
Griechischen weit mehr auf die Sprache des gewöhnlichen Lebens hin\textcompwordmark{}gewiesen.
Im Griechischen lesen wir Gespräche bei den Dramatikern, Gespräche
bei Plato; die Stimme des gemeinsten Mannes, --- schon \emph{dies}
nöthigt sie, seiner Sprache nahe zu bleiben, und schon dies muß die
Kenntniß der Aus\textcompwordmark{}drucks\textcompwordmark{}weise
des täglichen Lebens im Griechischen nützlich machen zum feinfühligeren
Verständniß der Texte.

Zweitens aber ist die \emph{Färbung} der Sprache und die Stil\textcompwordmark{}gattung
eines Literatur\textcompwordmark{}werkes nur demjenigen recht erkennbar,
der ermessen kann, wie weit dessen Sprache sich \emph{abhebt} von
der Alltags\textcompwordmark{}sprache. Wer das Deutsche nur aus Schiller
gelernt hätte, dem würde das Verständniß ab\textcompwordmark{}gehen
für die Eigenart und die Höhe der Schiller'schen Diction. Erst wer
von der Sprache der \emph{Alltäglichkeit} aus an sie herantritt,
bringt den Maßstab für sie mit. Es wird im Griechischen nicht anders
sein.

Drittens zwingt ganz besonders die Beschäftigung mit der griechischen
Um\textcompwordmark{}gangs\textcompwordmark{}sprache zur \emph{Vergleichung}
des deutschen und griechischen Aus\textcompwordmark{}druckes und
fördert dadurch die Sicherheit und Natürlichkeit der Über\textcompwordmark{}setzungen
aus dem Griechischen, die auf der Leichtigkeit und Bereit\textcompwordmark{}schaft
der Wort\textcompwordmark{}ver\textcompwordmark{}gleichungen der
beruht. Was man den \emph{Geist} der Sprache nennt, das zeigt sich
am Auf\textcompwordmark{}fallendsten da, wo die Vergleichung der
Sprachen unter einander \emph{leicht} und \emph{nahe}liegend ist:
das ist auf dem Gebiete des Alltäglichen. Den jocosen Ton, der sich
von selbst ergiebt, sobald man die alltägliche Aus\textcompwordmark{}drucks\textcompwordmark{}weise
des modernen Lebens mit der Sprechweise der Alten in Vergleich stellt,
wird man als bei diesem Studium unvermeidlich um der Sache willen
mit in den Kauf nehmen.

Endlich aber sei darauf hingewiesen, daß nichts dem Erlernen des
Griechischen an unseren Gymnasien so viele \emph{Gegner} geschaffen,
als eben die Thatsache, daß Griechisch im Grunde für eine unlernbare
Sprache gilt. Was der belgische Professor Emil de Laveleye über die
von ihm beobachteten Ergebnisse des Gymnasial\textcompwordmark{}unter\textcompwordmark{}richtes
sagt: \quotedblbase \textfrench{résultat net et incontestable: on
sait peu le latin et point du tout le grec,}`` das, behaupten Viele,
trifft annähernd auch bei den deutschen Gymnasien zu. Erstaunlich
Wenige, die \quotedblbase Griechisch gelernt`` haben, wissen mit
einiger Bestimmt\textcompwordmark{}heit anzugeben, wie der Attiker
die einfachsten Begriffe, z.\,B. \quotedblbase Ich werde zu dir
kommen``, aus\textcompwordmark{}zudrücken pflegt. Wenn im Lateinischen
Jemand nicht sofort auf \quotedblbase \textlatin{veniam}`` käme,
würde man meinen, daß ihm die allerersten Anfangs\textcompwordmark{}gründe
mangeln, und wenn er nicht verstünde, \quotedblbase \textlatin{veniam}``
und \quotedblbase \textlatin{ibo}`` aus\textcompwordmark{}ein\textcompwordmark{}an\textcompwordmark{}der\textcompwordmark{}zu\textcompwordmark{}halten,
so würde man über Unzulänglichkeit des Unterrichtes mit vollem Rechte
Klage führen und glauben, daß solche Unsicherheit auch dem sicheren
Erfassen des \emph{Sinnes} lateinischer \emph{Schriftwerke Eintrag
}thun müsse. Aber  im Griechischen? Man mache den Versuch, und man
wird überraschend Wenige finden, die das im Gebrauche des Attikers
alltägliche \quotedblbase \textgreek[variant=ancient]{ἥξω παρὰ σέ}``
in Be\textcompwordmark{}reit\textcompwordmark{}schaft haben. Man
studirt im Griechischen eifrig die Sprach\emph{gesetze}, aber gar
wenig die \emph{Sprache}, und doch lernt man es nicht um der grammatischen
Schulung willen, --- für diese sorgt aus\textcompwordmark{}reichend
das Latein, --- sondern der Sprache wegen. Man setze einem jungen
Manne, der die Schule  mit dem Zeugniß der Reife im Griechischen
verlassen hat, ein Glas griechischen Weines vor: er wird schwerlich
im Stande sein, auf Griechisch mit nur einigermaßen passendem Worte
dafür zu danken, oder zu sagen, daß ihm der Wein gut schmeckt. Allerdings
ist solche Sprachfertigkeit nicht das Ziel und die Aufgabe des griechischen
Unterrichts im Gymnasium aber daß sie bei den langen und angestrengten
Studien nicht nebenbei mit abfällt und so völlig fern zu bleiben
scheint, läßt das Gefühl des Griechischkönnens nicht aufkommen. Der
\quotedblbase Reife`` ist sich gar wohl bewußt, daß es ihm unsägliche
Mühe macht, ganz einfache Gedanken in wirklich griechischen Wendungen
wiederzugeben. Das macht unzufrieden und trägt viel dazu bei, dem
Griechischen Gegner zu schaffen. Auch aus diesem Grunde soll mein
Büchlein zeigen, daß es leicht angeht, sich mit den Kenntnissen,
die das Gymnasium bietet, des Griechischen so zu bemächtigen, daß
man sich darin verständlich machen könnte. 

Die Hauptsache aber bleibt: die allergewöhnlichsten Wörter und Wendungen
in der Ver\textcompwordmark{}kehrs\textcompwordmark{}sprache des
täglichen Lebens sind der Urvorrath, der Krystallisations\textcompwordmark{}kern,
an den und um den sich die weiteren sprachlichen Bildungen angesetzt
und angeschlossen haben. Schon darum verdienen sie unsere Achtung.
\emph{Hier} gilt es, die Sprache zu fassen, für den, der sie wirklich
lernen will.

Eras\textcompwordmark{}mus und die Leute seiner Zeit, deren Kenntniß
des Griechischen wir bewundern, lernten es durch Ver\textcompwordmark{}kehr
mit Griechisch sprechenden Lehrern aus den Gesprächen über Gegenstände
des gewöhnlichen Lebens. Aus der Grammatik und Lectüre allein hat
noch Niemand Griechisch wirklich gelernt. Aber die Sprache verdient
es, daß wer sie lernen will, sie wirklich und nicht bloß zum Scheine
zu lernen sucht; denn Griechisch ist, wie der treffliche Wilhelm
Roscher, der berühmte Leipziger Nationalökonom, in seinem Buche über
Thukydides einst gesagt hat,

\begin{quotedquotation}\noindent die Sprache aller Sprachen, worin
die köstlichsten Menschenworte geredet sind. Die feierliche Grandezza
des Spaniers, die feine Süßigkeit des Italieners, des Franzosen geläufige
Anmuth, des Engländers pathetische Kraft, des Deutschen unergründlicher
Reichthum, ja selbst die Würde der römischen Senatorensprache, hier
sind sie vereinigt, sind geläutert im Feuer des Geistes und zum edelsten
Erze zusammengeschmolzen.\unskip``\end{quotedquotation} 

\begin{quote}

\end{quote}
\pagebreak{}\gappto\captionsgerman{%
\renewcommand{\contentsname}{Inhalts\textcompwordmark{}verzeichniß}%
\renewcommand{\partname}{Gespräche}%
} 
\captionsgerman
\renewcommand{\contentsname}{Inhalts\textcompwordmark{}verzeichniß}
\renewcommand{\partname}{Gespräche}

\begin{quote}
\setlength{\columnseprule}{0.5pt}\begin{multicols}{2}\tableofcontents{}

\end{multicols}

\pagebreak{}\mainmatter
\end{quote}

\section*{Kleine Regeln und Beobachtungen\addcontentsline{toc}{section}{\emph{Kleine Regeln und Beobachtungen}}}

\setlist[enumerate]{parsep=0pt,itemsep=0pt,partopsep=0pt,topsep=0pt}
\begin{enumerate}[leftmargin=0pt,rightmargin=0pt,listparindent =1cm,labelindent=1cm,labelsep=1ex,labelwidth={*},itemindent={*},align=left]
\item Nichts erleichtert es so sehr, eine Sprache zu beherrschen, als wenn
man ihre \emph{Schwächen} erspäht. Erst wenn wir ermittelt haben,
was einer Sprache fehlt, verstehen wir recht, warum sie gerade diese
oder jene Wendung vorzieht, diese oder jene Verbindung von Begriffen
liebt, warum sie in dieser oder jener Weise von der Aus\textcompwordmark{}drucks\textcompwordmark{}weise
unserer eigenen Sprache abweicht. Wir erfassen als\textcompwordmark{}dann
ein gutes Theil von ihrem \quotedblbase Geiste``, wie man den Inbegriff
ihrer Besonderheiten so gern nennt.


Eine bemerkens\textcompwordmark{}werthe Schwäche der griechischen
Sprache nun ist es, daß ihr bei allem Formenreichthum doch ein bequem
zu verwendendes \emph{\,Passivum fehlt}. Die Übereinstimmung eines
großen Theiles der passiven Formen mit den medialen erschwert ihre
Anwendung, weil Deutlichkeit das erste Gesets der Sprache ist, und
vielen Zeitwörtern fehlen überdies die allein dem Passivum eigenen
Formen.


Um die eigenthümliche Färbung der griechischen Sprache nachzuahmen,
hat man daher zu allererst Folgendes zu beachten:


\emph{Man meide thunlichst die den medialen gleichlautenden passiven
Formen und achte darauf, wie der Grieche diese zu ersetzen pflegt.}


Nur die durch den Zusammenhang sofort als solche erkennbaren und
gewisse in häufigen Gebrauch gekommene Passiva der bezeichneten Art
sind unbedenklich anzuwenden.


Umschreibungen des Passivums geschehen.
\begin{enumerate}
\item durch active Verba, z.\,B.

\begin{quote}
belehrt werden \textgreek[variant=ancient]{μανθάνειν,}\\
gerühmt werden \textgreek[variant=ancient]{εὐδοκιμεῖν,}\\
geplagt werden \textgreek[variant=ancient]{κάμνειν,}\\
vor Gericht gestellt werden \textgreek[variant=ancient]{εἰσιέναι εἰς
δικαστήριον,}\\
verklagt werden \textgreek[variant=ancient]{φεύγειν,}\\
gehalten werden für \ldots{} \textgreek[variant=ancient]{δοκεῖν,}\\
es wird mir etwas zugefügt \textgreek[variant=ancient]{πάσχω τι,}\\
vertrieben werden \textgreek[variant=ancient]{ἐκπίπτειν,}\\
einer Sache beraubt werden \textgreek[variant=ancient]{ἀπολλύναι τι},\\
getödtet werden \textgreek[variant=ancient]{ἀποθνήσκειν},\\
sie wurden vertrieben \textgreek[variant=ancient]{ἀνέστησαν},\\
es wurde mir geantwortet \textgreek[variant=ancient]{ἤκουσα},\\
es wird mir Gutes erwiesen \textgreek[variant=ancient]{εὖ πάσχω},\\
ich ward durch's Loos gewählt \textgreek[variant=ancient]{ἔλαχον},\\
ich ward freigesprochen \textgreek[variant=ancient]{ἀπέφυγον},\\
ich ward geschmäht \textgreek[variant=ancient]{κακῶς ἤκουσα},\\
ich ward (von Mitleid) ergriffen \textgreek[variant=ancient]{(ἔλεός)
με εἰσῄει}.
\end{quote}
\item vielfach durch \textgreek[variant=ancient]{γέγνεσθαι;} es steht für
gemacht, veranstaltet, bewerkstelligt werden, übertragen, verliehen,
erkauft, erworben werden, verübt w., gefeiert w. (von Festen), geboren
w. und andere Passiva.
\item durch Substantiva mit Verben, z.\,B.

\begin{quote}
gelobt werden \textgreek[variant=ancient]{ἔπαινον ἔχειν,}\\
es wird (viel) gesprochen \textgreek[variant=ancient]{λόγος ἐστὶ (πολύς),}\\
bestraft werden \textgreek[variant=ancient]{δίκην διδόναι,}\\
es wird gezürnt u. \textgreek[variant=ancient]{ὀργὴ γίγνεται} dgl.\ mehr;
\end{quote}
\item durch Adjektiva mit \textgreek[variant=ancient]{εἶναι,} z.\,B.

\begin{quote}
gesehen werden \textgreek[variant=ancient]{καταφανῆ εἰναι},\\
es wird dir nicht geglaubt \textgreek[variant=ancient]{ἄπιστος εἶ}
u. dgl. mehr.
\end{quote}
\end{enumerate}
\item Im Griechischen fehlt die Genauigkeit in der Bezeichnung des Objectes,
wie sie den modernen Sprachen eigen ist. Die letzteren setzen, wenn
zwei verbundene Verba das gleiche Object in verschiedenem Casus erfordern,
zum zweiten Verbum anstatt der Wiederholung des Nomens das persönliche
Pronomen (seiner, ihm, ihn, ihrer, ihr, sie, es, ihnen) als Object,
\emph{der Grieche läßt die Stelle des gemeinsamen Objectes beim zweiten
Verbum unbezeichnet, gleichviel in welchem Casus es stehen müßte.}


Das dem französischen \textfrench{en} entsprechende Object (welchen,
welche, welches) wird im Griechischen nicht aus\textcompwordmark{}gedrückt,
z.\,B.: Sie werden das Gold aus Lydien holen lassen müssen, wenn
sie welches haben wollen \textgreek[variant=ancient]{ἐκ Λυδίας μεταστέλλεσθαι
τὸ χρυσίον δεήσει αὐτοὺς, ἢν ἐπιθυμήσωσιν.}

\item Dem Griechen fehlt, wie dem Lateiner, das Mittel zur Hervorhebung
einzelner Satztheile, welches unsere Sprache, ähnlich anderen modernen
Sprachen, darin besitzt, daß sie den hervorzuhebenden Begriff zum
Prädivcte eines neuen Satzes meist mit dem unpersönlichen Subject
es macht, während die übrigen Satztheile in einem abhängigen Satze
vermittelst eines Relativs oder einer Conjunction angefügt werden.
\emph{Im Griechischen muß die der Hervorhebung eines Begriffes dienende
Zerlegung eines Satzes in zwei unterbleiben,} z.\,B.:  Es ist derselbe,
der dies sagt \textgreek[variant=ancient]{ὁ αὐτὸς ταῦτα λέγει}. Wer
ist der Mann, den du rufst? \textgreek[variant=ancient]{τίνα τὸν ἄνδρα
καλεῖς;} Ist es wahr, daß du das gethan hast? \textgreek[variant=ancient]{ἆρ᾽
ἀληθῶς τοῦτ᾽ ἐποίησας;} Wie ist es möglich, daß\ldots{} \textgreek[variant=ancient]{πῶς\ldots{};}
wie kommt es, daß\ldots{} \textgreek[variant=ancient]{πῶς\ldots{}; }
\item Coordinirte Sätze und coordinirte Satztheile kann der Grieche nicht
unverbunden lassen. Asyndetisches Nebeneinanderstellen von Satztheilen
kommt nur selten und zwar als Aus\textcompwordmark{}druck lebhafter
Erregung zur Anwendung.


In ununterbrochener Rede ist \emph{jeder neue Satz} durch eine passende
Conjunction (\textgreek[variant=ancient]{δέ, καί οὖν, γάρ} etc.) an
das Voraus\textcompwordmark{}gehende \emph{anzuschließen.}


Der Lernende ist davor zu warnen, \textgreek[variant=ancient]{μέν}
für eine diese Verbin- dung mit dem \emph{Voraus\textcompwordmark{}gehenden}
ersetzende Conjunction zu halten, da es nur zum Hinweis auf das \emph{Folgende}
dient.


Anfügung ohne Bindewort ist in ununterbrochener Rede nur gestattet:
\begin{enumerate}
\item an den Stellen, wo wir im Deutschen den Doppelpunkt als Interpunctions\textcompwordmark{}zeichen
setzen;
\item wenn der neue Satz mit stark betontem Demonstrativum oder
\item wenn der neue Satz mit \textgreek[variant=ancient]{εἶτα} (= und dann)
oder \textgreek[variant=ancient]{ἔπειτα} beginnt;
\item wo wir im Deutschen mit \emph{\quotedblbase nicht aber``} fortsahren;
es steht dann häufig bloßes \textgreek[variant=ancient]{οὐ} (beziehentlich
\textgreek[variant=ancient]{μή}\,), (weil \textgreek[variant=ancient]{οὐ}
mit \textgreek[variant=ancient]{δέ} \quotedblbase und nicht`` oder
\quotedblbase nicht einmal`` bedeutet), oft jedoch auch \textgreek[variant=ancient]{οὐ
μέντοι.} 
\end{enumerate}
\item Man merke: Nun so r denn = \textgreek[variant=ancient]{ἀλλά,}


\bgroup\addtolength{\leftskip}{\parindent}\addtolength{\leftskip}{3em}\setlength{\parindent}{-1em}\setlength{\arraycolsep}{0pt}o
dann ... = \textgreek[variant=ancient]{ἄρα,}


da kam, da sagte = \textgreek[variant=ancient]{καὶ ἦλθε, καὶ εἰπεν,}


jedoch = \textgreek[variant=ancient]{μέντοι,}


denn sonst \ldots{} =\textgreek[variant=ancient]{γάρ,}


denn (folgernd), z.B. höre \emph{denn, so} ward er \emph{denn} ..
= \textgreek[variant=ancient]{δή,}


doch wohl (ohne Zweifel) = \textgreek[variant=ancient]{δήπου,}


und schon = \textgreek[variant=ancient]{καὶ δή} (\textgreek[variant=ancient]{δή}
= \textgreek[variant=ancient]{ἤδη}), vgl. \textgreek[variant=ancient]{πάλαι
δή} schon längst, \textgreek[variant=ancient]{νῦν δή} jetzt eben,


wohl aber = \textgreek[variant=ancient]{δὲ,}


\begin{tabular}{lc}
dann erst & \ldelim\}{2}{1em}[]\tabularnewline
erst dann & \tabularnewline
\end{tabular} =\footnote{Ich setzte das Gleichheitszeichen.} \textgreek[variant=ancient]{οὕτω
δή,}


\ldots{} allerdings = \textgreek[variant=ancient]{\ldots{} μήν,}


indessen \ldots{} = \textgreek[variant=ancient]{οὐ μὴν ἀλλά,}


wahrscheinlich (adv.) =\footnote{Ich setzte das Gleichheitszeichen.}
\textgreek[variant=ancient]{ἦ που \ldots{}}


oder (nach Negationen) = \textgreek[variant=ancient]{οὐδέ, μνδέ,}


doch (lat. \textlatin{quaeso}) = \textgreek[variant=ancient]{δῆτα,}


nicht sowohl \textgreek[variant=ancient]{\ldots{}} als vielmehr =
\footnote{Ich habe das geschwungene Klammer gespiegelt.}%
\begin{tabular}{ll}
\ldelim\{{2}{1em}[] & \begin{greek}[variant=ancient]%
οὐ τοσοῦτον ὅσον \ldots{}\end{greek}%
\tabularnewline
 & \begin{greek}[variant=ancient]%
οὐ τὸ πλέον \ldots{} ἀλλὰ \ldots{}\end{greek}%
\tabularnewline
\end{tabular}


\egroup Aus der Thatsache, daß \quotedblbase o dann ...`` sich
überall passend durch \textgreek[variant=ancient]{ἄρα} geben läßt,
folgt noch keines\textcompwordmark{}wegs, daß umgekehrt \textgreek[variant=ancient]{ἄρα}
sich überall passend durch \quotedblbase o dann \ldots{}`` übersetzen
lasse.

\item \emph{Großes} Glück \textgreek[variant=ancient]{πολλὴ εὐδαιμονία.}


\begin{continuousitemline}


Großes Mißgeschick \textgreek[variant=ancient]{πολλὴ δυστυχία.}


Großer Überfluß \textgreek[variant=ancient]{πολλὴ ἀφθονία.}


Große Thorheit \textgreek[variant=ancient]{πολλὴ μωρία.}


Große Unwissenheit \textgreek[variant=ancient]{πολλὴ ἀμαθία.}


Große Unvernunft \textgreek[variant=ancient]{πολλὴ ἀλογία.}


Große Geschäftigkeit \textgreek[variant=ancient]{πολλὴ πραγματεία.}


Sehr große Muthlosigkeit \textgreek[variant=ancient]{πλείστη ἀθυμία.}\par\end{continuousitemline}

\item %
\begin{tabular}[t]{lc}
So ein trefflicher & \rdelim\}{5}{1em}[{\textgreek[variant=ancient]{τοιοῦτος.}}]\tabularnewline
So ein abscheulicher & \tabularnewline
So ein erfahrener & \tabularnewline
So ein beschränkter & \tabularnewline
So ein gefährlicher & \tabularnewline
\end{tabular}


\begin{continuousitemline}\qquad{}u. s. w.


\begin{tabular}[t]{lc}
So ein trefflicher & \rdelim\}{5}{1em}[{ \textgreek[variant=ancient]{τοιοῦτος.}}]\tabularnewline
So ein abscheulicher & \tabularnewline
So ein erfahrener & \tabularnewline
So ein beschränkter & \tabularnewline
So ein gefährlicher & \tabularnewline
\end{tabular}


\qquad{}u. s. w.


\begin{tabular}[t]{lc}
So Verwerfliches & \rdelim\}{2}{1em}[{ \textgreek[variant=ancient]{τοιαῦτα.}}]\tabularnewline
So Löbliches & \tabularnewline
\end{tabular}


\qquad{}u. s. w.


\begin{tabular}[t]{lc}
es klingt schön & \rdelim\}{3}{1em}[{ \textgreek[variant=ancient]{ἡδύ ἐστιν.}}]\tabularnewline
es schmeckt gut & \tabularnewline
es riecht gut & \tabularnewline
\end{tabular}


\begin{tabular}[t]{lc}
(jetzt) so spät & \rdelim\}{2}{1em}[{ \textgreek[variant=ancient]{τηνικάδε.}}]\tabularnewline
(jetzt) so früh & \tabularnewline
\end{tabular}


\quad{}Der gewöhnliche Aus\textcompwordmark{}druck für


\begin{tabular}[t]{lc}
hoffen & \rdelim\}{2}{1em}[\textgerman{ ist \textgreek[variant=ancient]{οἴεσθαι},}]\tabularnewline
fürchten & \tabularnewline
\end{tabular}


\begin{tabular}[t]{lc}
versprechen & \rdelim\}{4}{1em}[\textgerman{ ist \textgreek[variant=ancient]{φάναι.}}]\tabularnewline
brohen & \tabularnewline
antworten & \tabularnewline
erwidern & \tabularnewline
\end{tabular}


\ldots{} fuhr er fort, = \textgreek[variant=ancient]{ἔφη.}\par\end{continuousitemline}

\item Ein Freund \textgreek[variant=ancient]{φίλος \emph{τις.}}


\begin{continuousitemline}Ein redlicher Freund \textgreek[variant=ancient]{χρηστός
τις \emph{ἄνθρωπος} φίλος.}\par\end{continuousitemline}

\item Unsere 500 Schüler \textgreek[variant=ancient]{\emph{οἱ} ἡμέτεροι
πεντακόσιοι μαθηταί.}


\begin{continuousitemline}Meine drei besten Schüler \textgreek[variant=ancient]{\emph{οἱ}
τρεῖς ἄριστοι τῶν μαθητῶν μου.}\par\end{continuousitemline}

\item Ich verlange kein Geld, sondern Zuneigung (Liebe) \textgreek[variant=ancient]{αἰτῶ
\emph{οὐκ} ἀργύριον, ἀλλ᾽ εὔνοιαν.}
\item Ich \emph{habe} gehabt \textgreek[variant=ancient]{εἶχον,} z. B.
ich habe ebenfalls diese Klasse einmal gehabt \textgreek[variant=ancient]{κἀγὼ
εἶχον τὴν τάξιν ταύτην ποτέ.} Er \emph{ist} gestern bei mir gewesen
\textgreek[variant=ancient]{παρ᾽ ἐμοὶ χθὲς ἦν.}


Das \emph{Perfectum} von \emph{sein} und \emph{haben} und allen ein
Dauer ausdrückenden Verben wird im Griechischen durch das Imperfectum,
bei den übrigen Verben meist durch den Aorist, seltener durch das
Perfectum wiedergegeben. Läßt sich zu dem Verbum ein Adverb der Vergangenheit
(z. B. damals) hinzudenken, so steht Aorist; läßt sich ein Adverb
der Gegenwart (z. B. nunmehr, bereits) hinzudenken, nur dann steht
Perfectum.


\begin{continuousexamples}\emph{Hast} du das Geld gefunden? (\textlatin{sc.}
nunmehr) \textgreek[variant=ancient]{ἆρ᾽ εὕρηκας τἀργύριον;}


Ja, ich \emph{habe} es gefunden (\textlatin{sc.} nunmehr) \textgreek[variant=ancient]{εὕρηκανὴ
Δία.}


Wo \emph{hast} du es gefunden? (\textlatin{sc.} damals als du es fandest)
\textgreek[variant=ancient]{ποῦ εὗρες;}


Ich \emph{habe} es (\textlatin{sc.} damals) in dem Garden gefunden
\textgreek[variant=ancient]{ἐν τῷ κήπῳ εὗρον. }\par\end{continuousexamples}

\item Der Infinitiv Aoristi bezeichnet nach den Verben des Sagens und Meinens
die Vergangenheit, z. B.


\begin{continuousexamples}\textgreek[variant=ancient]{φησὶν εὑρεῖν}
er behauptet er \emph{habe} gefunden.\par\end{continuousexamples}

\item Bedeutet \emph{daß} soviel wie \emph{mache(t)} daß, so wird es durch
\textgreek[variant=ancient]{ὅπως}\footnote{\begin{latin}%
orig. \textgreek[variant=ancient]{οπως}\end{latin}%
} mit dem \textlatin{Indic. Fut.} ausgedrückt.


\begin{continuousexamples}Daß es nur kein Mensch erfährt! \textgreek[variant=ancient]{ὅπως
ταῦτα μηδεὶς ἀνθρώπων πεύσεται!}\par\end{continuousexamples}

\item Mit \textgreek[variant=ancient]{ἐξ οὗ} oder \textgreek[variant=ancient]{ἐπεί}
= \emph{seit} verträgt sich kein \textgreek[variant=ancient]{οὐ} oder
\textgreek[variant=ancient]{μή}\footnote{\begin{latin}%
orig. \textgreek[variant=ancient]{μη}\end{latin}%
}\textgreek[variant=ancient]{.}


\begin{continuousexamples}Seit wir uns \emph{nicht} gesehen, hat
es viel geregnet: \textgreek[variant=ancient]{ἐξ οὗ }oder\textgreek[variant=ancient]{
ἐπεὶ εἴδομεν ἀλλήλους ὕδωρ ἀγένετο πολύ.}\par\end{continuousexamples}

\item Wo sich statt \emph{sein} denken läßt \emph{gehen,} wird \textgreek[variant=ancient]{παρεῖναι
εἰς} angewandt.


\begin{continuousexamples}Sind Sie oft im Theater gewesen? \textgreek[variant=ancient]{ἦ
πολλάκις παρῆσθα εἰς τὸ θέατρον;}\par\end{continuousexamples}

\item Indefinita werden nach Negationen gern negativ, \textgreek[variant=ancient]{πω}
jedoch bleibt unverändert.
\item Ja = doch (franz. \textfrench{si!}) dem Unglauben oder mangelhaften
Glauben versichernd: \textgreek[variant=ancient]{ναί! }
\item \emph{Zu, allzu} bleibt meist unübersetzt; z. B. Wir sind zu wenige
\textgreek[variant=ancient]{ὀλίγοι ἐσμέν,} du hast zu menig geschrieben
\textgreek[variant=ancient]{ὀλίγον ἔγραψας.} \textgreek[variant=ancient]{Τὸ
ὕδωρ ψυχρὸν ὥστε λούσασθαί ἐστιν} (zu kalt). \textgreek[variant=ancient]{Νέοι
ἔτι ἐσμὲν ὥστε τοῦτ᾽ εἰδέναι} (zu jung, als daß wir wissen könnten).


\emph{Nicht genug} \textgreek[variant=ancient]{ὀλίγος.} Er hat nicht
genug zu leben \textgreek[variant=ancient]{βίον ἔχει ὀλίγον.} Ich
habe nicht genug Geld \textgreek[variant=ancient]{ἀργύριον ἔχω ὀλίγον.}


\emph{Genug} = ausreichend wird adjectivisch meist durch \textgreek[variant=ancient]{ἱκανός}
ausgedrückt. Geld genug \textgreek[variant=ancient]{ἱκανὸν ἀργύριον.}
Ich denke, zwanzig Schüler sind genug \textgreek[variant=ancient]{ἱκανοὺς
νομίζω μαθηδὰς εἴκοσιν.}


\emph{Genug} = in Menge \textgreek[variant=ancient]{οὐκ ὀλίγος.}

\item Ein anderer = noch ein weiterer \textgreek[variant=ancient]{ἕτερος;
}ein anderer = irgend welcher andere \textgreek[variant=ancient]{ἄλλος.}


Ich war dort und viele andere \textgreek[variant=ancient]{ἐγὼ παρεγενόμην
καὶ ἕτεροι πολλοί.} Nun, es giebt ja andere gute Bücher genug \textgreek[variant=ancient]{ἀλλ᾽
ἔστιν ἔτερα νὴ Δία χρηστά βιβλία οὐκ ὀλίγα.}


\begin{continuousitemline}Keine andere Sache \textgreek[variant=ancient]{οὐκ
ἄλλο πρᾶγμα.}


Wer sonst? \textgreek[variant=ancient]{τίς ἄλλος;}\par\end{continuousitemline}

\item Immer noch = \textgreek[variant=ancient]{ἔτι καὶ νῦν,}


\begin{continuousitemline}noch welches \textgreek[variant=ancient]{ἄλλο,}


noch einige \textgreek[variant=ancient]{ἄλλοι,}


noch irgend einer \textgreek[variant=ancient]{ἄλλος τις.}


Hat er noch (sonstiges) Geld? \textgreek[variant=ancient]{ἆρ᾽ ἔχει
ἀργύριον ἄλλο;}


Er hat \emph{welches} \textgreek[variant=ancient]{ἔχει.}\par\end{continuousitemline}

\item Ihr \emph{beiden} alten Herren \textgreek[variant=ancient]{ὦ \emph{δύο}
πρεσβύτα.}


\begin{continuousitemline}\emph{Diese beiden} alten Herren hier \textgreek[variant=ancient]{τὼ
πρεσβύτα τώδε.}


\emph{Diese beiden} \textgreek[variant=ancient]{τώδε (ἄμφω).}\par\end{continuousitemline}


\textgreek[variant=ancient]{ἄμφω} verlangt stets den \emph{Dual}
des beigefügten Substantivs, \textgreek[variant=ancient]{ἀμφότερος}
steht meist mit seinem Substantiv im \emph{Plural.}

\item allein (= allein für sich) \textgreek[variant=ancient]{αὐτός,}


\begin{continuousitemline}allein (= der einzige) \textgreek[variant=ancient]{μόνος.}


Wir sind allein (unter uns) \textgreek[variant=ancient]{αὐτοί ἐσμεν.}


Wir sind die einzigen \textgreek[variant=ancient]{μόνοι ἐσμέν.}\par\end{continuousitemline}


Ich habe die (schriftliche) Arbeit allein gemacht \textgreek[variant=ancient]{αὐτὸς
ἐγὼ ταῦτα ἔγραψα.} Dagegen \textgreek[variant=ancient]{μόνος ἐγὼ ταῦτα
ἔγραψα} ich bin der Einzige, der diese Arbeit gemacht hat.

\item Ich habe mehr \emph{von diesen} (z. B. Söhne) wie von jenen (Töchter)
\textgreek[variant=ancient]{πλείους ἔχω τούτους ἢ ἐκείνας} (doch auch
\textgreek[variant=ancient]{ἐκείνους ἢ ταύτας}).
\item Wollen = Lust haben, sich entschließen \textgreek[variant=ancient]{ἐθέλειν.}


\begin{continuousitemline}Wollen = wünschen \textgreek[variant=ancient]{βούλεσθαι.}


Er hat keine Lust \textgreek[variant=ancient]{οὐκ ἐθέλει.}


(Sehnlich) wünschen \textgreek[variant=ancient]{ἐπιθυμεῖν. }


Wollen = darüber sein \textgreek[variant=ancient]{μέλλειν.}\par\end{continuousitemline}


Wohin eilen sie? Ich will einen Brief zum Briefkasten tragen \textgreek[variant=ancient]{ποῖ
θεῖς; ἐπιστολὴν μέλλω φέρειν εἰς τὸ κιβώτιον (γραμματοκιβώτιον). Ich
will gehen εἶμι oder βαδιοῦμαι.} 


\begin{continuousitemline}Ich will gehen \textgreek[variant=ancient]{εἶμι}
oder \textgreek[variant=ancient]{βαδιοῦμαι.}\par\end{continuousitemline}

\item Wo ist dein Bruder? \textgreek[variant=ancient]{ποῦ ᾽σθ᾽ ὁ σὸς ἀδελφός;}
\item Bei = franz. \textfrench{chez} \textgreek[variant=ancient]{παρά} mit
\textlatin{Dat.}


\begin{continuousitemline}Zu = franz. \textfrench{chez} \textgreek[variant=ancient]{παρά}
mit \textlatin{Acc.}\par\end{continuousitemline}

\item %
\begin{tabular}[t]{ccccc}
Mitnehmen, & ~ & mitbringen & ~ & (von Sachen) \textgreek[variant=ancient]{φέρειν,}\tabularnewline
,, &  & ,, &  & (von Personen) \textgreek[variant=ancient]{ἄγειν.}\tabularnewline
\end{tabular}


\begin{continuousitemline}Ich will das Buch mitbringen \textgreek[variant=ancient]{οἴσω
τὸ βιβλίον.}


Ich will dich mit (zu ihm) nehmen \textgreek[variant=ancient]{ἄξω
σε παρ᾽ αυτόν.}\par\end{continuousitemline}

\item Ich gehe (hin) \textgreek[variant=ancient]{βαδίζω,}


\begin{continuousitemline}ich komme (her) \textgreek[variant=ancient]{ἔρχομαι,}


ich bin hergegangen \textgreek[variant=ancient]{ἐλήλυθα,}


ich bin gekommen \textgreek[variant=ancient]{ἥκω,}


ich bin wieder da \textgreek[variant=ancient]{ἥκω,}


bis ich wieder da bin \textgreek[variant=ancient]{μέχρι ἃν ἥκω,}


ich gehe \emph{(weiter)} \textgreek[variant=ancient]{χωρῶ,}


ich will ihn \emph{besuchen} \textgreek[variant=ancient]{εἶμι (εἴσειμι)
ὡς αὐτόν,}


ich werde kommen \textgreek[variant=ancient]{ἥξω.}


Ich will gehen, \emph{um} ihn zu befragen \textgreek[variant=ancient]{εἶμι
ἐρωτήσων αὐτόν.}


Ich komme her, \emph{um} mit\textcompwordmark{}zuspeisen \textgreek[variant=ancient]{ἔρχομαι
δειπνήσων.}


\emph{aus}gehen \textgreek[variant=ancient]{θύραζε ἐξιέναι} oder
\textgreek[variant=ancient]{θ. βαδίζειν.}\par\end{continuousitemline}

\item Die \emph{guten} Schüler \textgreek[variant=ancient]{οἱ ἀγαθοὶ τῶν
μαθητῶν.}


\begin{continuousitemline}Die guten \emph{Schüler} \textgreek[variant=ancient]{οἱ
ἀγαθοὶ μαθηταί.}\par\end{continuousitemline}

\item \emph{Da} kommt der junge Mann herbai! \textgreek[variant=ancient]{τὸ
μειράκιον \emph{τοδὶ} (τόδε) προσέρχεται!} 
\item Ich habe \emph{nichts zu} essen \textgreek[variant=ancient]{οὐκ ἔχω
καταφαγεῖν.}
\item \emph{hier,} den Ort des \emph{Sprechenden} bezeichnend, haißt \textgreek[variant=ancient]{ἐνθάδε,}


\begin{continuousitemline}hier (dem Ort des Sprechenden \emph{nahe})
\textgreek[variant=ancient]{ἐνταῦθα,}


hier (= \emph{an Ort und Stelle,} am Orte selbst) \textgreek[variant=ancient]{αὐτοῦ.}\par\end{continuousitemline}

\item Jemanden kennen \textgreek[variant=ancient]{γιγνώσκειν τινά.}
\item Zwar nicht groß, aber schön \textgreek[variant=ancient]{μέγας μὲν
οὔ, καλὸς δέ.}
\item Er hat eine breite Stirn \textgreek[variant=ancient]{πλατὺ ἔχει \emph{τὸ}
μέτωπον.}


Sie hat allerliebste Hände \textgreek[variant=ancient]{\emph{τὰς}
χεῖρας ἔχει παγκάλας.}

\item Beabsichtigen, gedenken \textgreek[variant=ancient]{ἐπινο\emph{εῖν}}
oder \textgreek[variant=ancient]{διανο\emph{εῖσθαι.}}
\item Ich lerne die Gedichte Homers \emph{auswendig} \textgreek[variant=ancient]{μανθάνω
τὰ Ὁμήρου ἔπη.}


\begin{continuousitemline}Ich \emph{kann} die Ilias \emph{auswendig}
\textgreek[variant=ancient]{ἐπίσταμαι Ἰλιάδα.}


Ich könnte die Odyssee \emph{auswendig hersagen} \textgreek[variant=ancient]{δυναίμην
ἂν Ὀδύσσειαν ἀπὸ στόματος εἰπεῖν. }\par\end{continuousitemline}

\item Mein Vater hat mich gezwungen, die Odyssee auswendig zu lernen \textgreek[variant=ancient]{ὁ
πατὴρ ἠνάγκασέ με Ὀδύσσειαν \emph{μαθεῖν}} = that\textcompwordmark{}sächlich
mit dem Lernen zu Stande zu kommen; \textgreek[variant=ancient]{ἠνάγκασέ
με \emph{μανθάνειν}} bedeutet nur: er zwang mich, mit dem Lernen mich
zu beschäftigen, zu befassen, zu bemühen. 
\item \textgreek[variant=ancient]{Εὖ λέγει }er hat Recht.


\begin{continuousitemline}\textgreek[variant=ancient]{καλῶς λέγει}
er spricht gut.\par\end{continuousitemline}

\item Ich habe mehr Geld als du, aber Karl hat \emph{das} meiste \textgreek[variant=ancient]{ἐγὼ
μὲν ἀργύριον ἔχω πλέον ἢ σύ, πλεῖστον δὲ Κάρολος.}
\item Der Mann, \emph{dessen} Brief du liest \textgreek[variant=ancient]{ὁ
ἀνήρ, οὖ ἀναγιγνώσκεις \emph{τὴν} ἐπιστολήν.}


\begin{continuousitemline}Wessen Brief liest du? \textgreek[variant=ancient]{\emph{τὴν}
τίνος ἐπιστολὴν ἀναγιγνώσκεις;}\par\end{continuousitemline}

\item Setzest du deinen Hut auf? \textgreek[variant=ancient]{ἦ περιτίθεσαι
\emph{τὸν} πῖλον;}


\begin{continuousitemline}Zieh deine Stiefel aus! \textgreek[variant=ancient]{ἀποδύου
\emph{τὰς} ἐμβάδας!}\par\end{continuousitemline}


Das Possessiv ist durch das Medium bereits ausgedrückt. 

\item Er wird dich von \emph{deinem} Augenleiden befreien \textgreek[variant=ancient]{ἀπαλλάξει
σε τῆς ὀφθαλμίας.}


\begin{continuousitemline}Ein einziger Tag hat mir \emph{meinen}
ganzen Wohlstand geraubt \textgreek[variant=ancient]{μία ἡμέρα με
\emph{τὸν} πάντα ὄλβον ἀφείλετο.}


Er hat mir \emph{mein} Geld gestahlen \textgreek[variant=ancient]{ὑπείλετό
μου τἀργύρια.}\par\end{continuousitemline}


Bei den Verben \emph{nehmen} und dergl. darf kein Possessiv übersetzt
werden, sobald die durch dasselbe bezeichnete Person bereits genannt
ist.

\item Brauchst du \emph{etwas?} \textgreek[variant=ancient]{δέει \emph{τίνος;}}


\begin{continuousitemline}Giebt es \emph{was} Neues? \textgreek[variant=ancient]{λέγεται
\emph{τί} καινόν;}\par\end{continuousitemline}

\item Woher kommst du? \textgreek[variant=ancient]{πόθεν ἥκεις;} Aus dem
Garten \textgreek[variant=ancient]{ἐκ τοῦ κήπου.} Aus welchem? \textgreek[variant=ancient]{ἐκ
\emph{τοῦ} ποίου;}


Wenn \textgreek[variant=ancient]{ποῖος} auf einen mit Artikel versehenen
Gattungsnamen \textlatin{(Substantivum appellativum)} oder einen ihn
vertretenden Satz zurückweist, so nimmt es den Artikel an. Weg bleibt
der Artikel in der Regel nur dann, wenn \textgreek[variant=ancient]{ποῖος}
Prädicat ist. 

\item \emph{Geld} in kleineren Summen \textgreek[variant=ancient]{ἀργύριον.}


\begin{continuousitemline}Geld = Kapitalien \textgreek[variant=ancient]{χρήματα.
}\par\end{continuousitemline}

\item \textgreek[variant=ancient]{τάχα} ent\textcompwordmark{}spricht
genau dem in unserer Volks\textcompwordmark{}sprache üblichen \emph{am
Ende} (= schließlich, möglicher Weise)


\textgreek[variant=ancient]{ταχύ, ταχέως} schnell, bald,


\textgreek[variant=ancient]{διὰ ταχέων} bald. 

\item \emph{Unter} = zwischen drin \textgreek[variant=ancient]{ἐν,} z. B.
\textgreek[variant=ancient]{ἐν τοῖς Χριστιανοῖς πολλοί εἰσιν Ἰουδαῖοι.
ἐν νέοις ἀνὴρ γέρων.}
\item \emph{Nicht sonderlich} \textgreek[variant=ancient]{οὐ πάνυ.} Er strengt
sich nicht sonderlich an \textgreek[variant=ancient]{οὐ πάνυ σπουδάζει.}
\item Die natürliche Stellung des Adverbs ist im Griechischen \emph{vor}
dem durch dasselbe zu bestimmenden Begriffe. Abweichung von dieser
Stellung dient zur Hervorhebung des Adverbs. Steht das Adverb mit
Nachdruck zuletzt, so ersetzt diese Stellung das deutsche \emph{und
zwar:} \textgreek[variant=ancient]{χάριν σωθέντες ὑπὸ σοῦ σοὶ ἂν ἔχοιμεν
δικαίως} (und zwar pflichtschuldigst). 
\item Indirecte Ausrufesätze werden in der lateinischen Grammatik den indirecten
Fragesätzten gleichgestellt; im Griechischen unterscheiden sie sich
aber von den indirecten Fragesätzen dadurch, daß diese letzteren mit
dem indirecten oder directen Frageworte beginnen, die Ausrufesätze
hingegen mit dem Relativum, und zwar mit dem \emph{einfachen} Relativum.
\item Der Deutsche fragt: \emph{Wohin} setzt er sich? der Grieche: \emph{Wo?}
Wohin wollen wir uns setzen? \textgreek[variant=ancient]{ποῦ καθιζησόμεθα;}
\item \emph{Alle Welt} \textfrench{(tout le monde)} heißt \textgreek[variant=ancient]{πάντες
ἄνθρωποι} (ohne Artikel).
\item \emph{Um zu} wird gern durch \textgreek[variant=ancient]{βουλόμενος}
aus|gedrückt.
\item Ich habe bekommen = \textgreek[variant=ancient]{ἔχω,} z. B. ich habe
von meinem Vater 10 Mk. bekommen, \textgreek[variant=ancient]{δέκα
μάρκας ἔχω παρὰ τοῦ πατρός.}
\item \emph{Lieber als} \ldots{} = eher als \ldots{} heißt \textgreek[variant=ancient]{μᾶλλον
ἢ }\ldots{}
\item \emph{Vorhin} heißt \textgreek[variant=ancient]{τότε.} 
\item \textgreek[variant=ancient]{μέν} steht anderen Bindewörtern voran,
also nicht \textgreek[variant=ancient]{πολλοὶ γὰρ μὲν }\ldots{}\textgreek[variant=ancient]{,}
sondern \textgreek[variant=ancient]{πολλοὶ μὲν γὰρ }\ldots{}\textgreek[variant=ancient]{,}
ebenso \textgreek[variant=ancient]{μέν γε, μὲν δή }\ldots{}\textgreek[variant=ancient]{,
μὲν οὖν }\ldots{}\textgreek[variant=ancient]{, μέντοι.}
\item Den bringlichen Imperativ, welchen wir durch \emph{so} (mach') \emph{doch}
ausdrücken, giebt der Grieche durch (das sehr oft und gern angewendete)
\textgreek[variant=ancient]{οὐ} mit Futurum, z. B. so schweig' doch!
\textgreek[variant=ancient]{οὐ σιγήσει;} Negation ist dabei \textgreek[variant=ancient]{μή,}
z. B. so mach' doch kein Gerede! \textgreek[variant=ancient]{οὐ μὴ
λαλήσεις; }so halte dich doch nicht auf! \textgreek[variant=ancient]{οὐ
μὴ διατρίψεις;}
\item Satzverbindungen wie folgende: \quotedblbase Wenn ich nach Dresden
komme und über die Brücke gehe, so sehe ich das Denkmal August des
Starken`` werden im Griechischen zerlegt in: \quotedblbase Wenn
ich nach Dresden komme, so sehe ich, wenn ich über die Brücke komme,
das Denkmal.`` Trotzdem gehen die \emph{beiden} Nebensätze dem Hauptsatze
voran. 
\item Der gewöhnliche Ausdruck für \emph{\quotedblbase ich bitte``} ist
\textgreek[variant=ancient]{πρὸς (τῶν) θεῶν,} wofür auch \textgreek[variant=ancient]{πρὸς
τοῦ Διός} u. Ähnliches eintritt. \textgreek[variant=ancient]{πρὸς
θεῶν} ist keines\textcompwordmark{}wegs, wie gewöhnlich angegeben
wird, \emph{\quotedblbase Versicherung} bei den Göttern``, sondern
\emph{Bitt}formel.
\item Es giebt nicht bloß, wie es nach den Grammatiken scheint, einen Irrealis
der Gegenwart und Irrealis der Vergangenheit (z. B. ich wäre (jetzt)
zufrieden, ich wäre (damals) zufrieden gewesen, wenn . . .), sondern
es muß auch einen Irrealis der \emph{Zukunft} geben. Ich sage z. B.:
\quotedblbase Wenn ich morgen in New-York wäre, würde ich mich an
dem Feste betheiligen,`` obgleich ich weiß, daß ich morgen unmöglich
dort sein kann. Diesen Irrealis der Zukunft drückt der Grieche im
Nebensatze durch \textgreek[variant=ancient]{εἰ} mit dem Optativ,
im regierenden Satze durch Optativ mit \textgreek[variant=ancient]{ἄν}
aus. 
\end{enumerate}
\emph{Anmerkung} In Beispielen, wie \textgreek[variant=ancient]{φαίη
δ᾽ ἂν ἡ θανοῦσα, εἰ φωνὴν λάβοι} steht also nicht der Optativ ungewöhnlich
für das Präteritum, sondern er bezeichnet regelrecht, wie in zahllosen
ähnlichen Fällen, den Irrealis der Zukunft: \quotedblbase wenn die
Verstorbene \emph{künftig einmal} wiederkäme, so würde sie es bestätigen.`` 


\begin{center}
{\Huge{}\rule[1.2ex]{0.2\columnwidth}{1pt}}
\par\end{center}{\Huge \par}

\clearpage{}



\leftskip=0.1in\parindent=-0.1in\tabcolsep=0pt
%\hangindent=2em\hangafter=2
%\titlespacing{\section}{0pt}{\parskip}{-\parskip}
\begin{paracol}{2}
	\switchcolumn*[

\part{%
	\trDE{Allgemeinen Inhalts.}%
	\trJA{一般的事項}%
}

]
\switchcolumn*[
\section{%
	\trDE{Guten Tag!}%
	\trJA{こんにちは!}%
}

]%
\indent
\trDE{Ah! Guten Tag!}%
\trJA{おお、こんにちは!}

\switchcolumn

\begin{greek}[variant=ancient]%
ὦ χαῖρε!

\end{greek}%
\switchcolumn*

\trDE{Guten Morgen, Karl!}%
\trJA{おはやう、カール\footnote{ΚάρολοςはKarlのラテン語形Carolusをギリシャ語化した形のやうである。}!}

\switchcolumn

\begin{greek}[variant=ancient]%
χαῖρ' ὦ Κάρολε!

\end{greek}%
\switchcolumn*

\trDE{Guten Morgen, Gustav! (Erwiderung)}%
\trJA{おはやう、グスタフ!(返答)}

\switchcolumn

\begin{greek}[variant=ancient]%
καὶ σύγε ὦ Γούσταβε!

\end{greek}%
\switchcolumn*

\trDE{Seien Sie mir schön willkommen!}%
\trJA{Seien Sie mir schön willkommen!}% ToDo

\switchcolumn

\begin{greek}[variant=ancient]%
ὦ χαῖρε, φίλτατε!

\end{greek}%
\switchcolumn*

\trDE{Ah! freue mich außerordentlich!}%
\trJA{Ah! freue mich außerordentlich!}% ToDo

\switchcolumn

\begin{greek}[variant=ancient]%
ἀσπάζομαι!

\end{greek}%
\switchcolumn*

\trDE{Freue mich außerordentlich, Herr Müller!}%
\trJA{ようこそ、ミュラーさん。歓迎します。}% ToDo

\switchcolumn

\begin{greek}[variant=ancient]%
Μύλλερον ἀσπάζομαι!

\end{greek}%
\switchcolumn*

\trDE{Ganz auf meiner Seite!}%
\trJA{私もです。}% ToDo

\switchcolumn

\begin{greek}[variant=ancient]%
κἄγογέ σε!

\end{greek}%
\switchcolumn*

\trDE{Guten Tag! Guten Tag! Wie freue ich mich, daß Sie gekommen sind, Verehrtester!}%
\trJA{今日は、今日は、友よ。お越し下さって、嬉しく思います。}% ToDo


\switchcolumn

\begin{greek}[variant=ancient]%
χαῖρε, χαῖρε, ὡς ἀσμένῳ μοι ἦλθες, ὦ φίλτατε!

\end{greek}%
\switchcolumn*

\trDE{Ah! Guten Tag! Was bringen Sie?}%
\trJA{おお、こんにちは。何を持って来たのですか。}% ToDo

\switchcolumn

\begin{greek}[variant=ancient]%
ὦ χαῖρε, τί φέρεις!

\end{greek}%
\switchcolumn*

\trDE{Ah! Guten Tag, Perikles; was steht zu Diensten?}%
\trJA{おお、今日は、ペリクレースよ。何を持って来たのですか。}% ToDo

\switchcolumn

\begin{greek}[variant=ancient]%
ὦ χαῖρε, Περίκλεις, τί ἔστιν;

\end{greek}%
\switchcolumn*

\trDE{Giebt's was Neues?}%
\trJA{何か新しいことはありませんか?}% ToDo

\switchcolumn

\begin{greek}[variant=ancient]%
λέγεται τί καινόν; (νεώτερον, \textgerman[spelling=old,babelshorthands=true]{Schlimmes})

\end{greek}%
\switchcolumn*

\trDE{Guten Abend, meine Herren (meine Damen)! Meine (jungen) Damen!}%
\trJA{今晩は。諸君、(御夫人方)、御令嬢!}% ToDo

\switchcolumn

\begin{greek}[variant=ancient]%
χαίρετε, ὦ φίλοι (ὦ δέσποιναι)! ὦ κόραι!

\end{greek}%
\switchcolumn*

\trDE{Paul läßt Sie grüßen.}%
\trJA{パウロに挨拶するやう言われました。}% ToDo

\switchcolumn

\begin{greek}[variant=ancient]%
Παῦλος ἐπέστειλε φράσαι χαίρειν σοι.

\end{greek}%
\switchcolumn*

\trDE{Mein lieber Herr!}%
\trJA{おお、友(男)よ。}% ToDo

\switchcolumn

\begin{greek}[variant=ancient]%
ὦ φίλ' ἄνερ!

\end{greek}%
\switchcolumn*[


\section{%
\trDE{Wie geht's?}%
\trJA{元気?}% ToDo
}

]\indent %
\begin{tabular}{lc}
\trDE{Wie geht es Ihnen?}%
\trJA{御機嫌如何ですか?}% ToDo
	& \ldelim\}{2}{1em}[]\tabularnewline
\trDE{Was machen Sie?}%
\trJA{何をされているのですか?}% ToDo
	& \tabularnewline
\end{tabular}

\switchcolumn

\begin{greek}[variant=ancient]%
\vspace{0.5em}
τί πράττεις;

\end{greek}%
\switchcolumn*

\trDE{Danke, es geht mir ganz wohl.}%
\trJA{頗る良好です。}% ToDo

\switchcolumn

\begin{greek}[variant=ancient]%
πάντ' ἀγαθὰ πράττω, ὦ φίλε.

\end{greek}%
\switchcolumn*

\trDE{Ich bin besser daran, als gestern.}%
\trJA{昨日より良くなりました。}% ToDo

\switchcolumn

\begin{greek}[variant=ancient]%
ἄμεινον πράττω ἢ χθές.

\end{greek}%
\switchcolumn*

\trDE{Wie geht es Ihrem Vater?}%
\trJA{父君はお元気ですか?}% ToDo

\switchcolumn

\begin{greek}[variant=ancient]%
τί πράττει ὁ πατήρ σου;

\end{greek}%
\switchcolumn*

\trDE{Es geht ihm recht gut.}%
\trJA{よくやっております。}% ToDo

\switchcolumn

\begin{greek}[variant=ancient]%
εὐδαιμόνως πράττει.

\end{greek}%
\switchcolumn*

\trDE{Wie steht es sonst bei euch?}%
\trJA{他に何かございましたか?}% ToDo

\switchcolumn

\begin{greek}[variant=ancient]%
τί δ' ἄλλο παρ' ὑμῖν;

\end{greek}%
\switchcolumn*

\trDE{Wie befinden Sie sich?}%
\trJA{如何なされましたか?}% ToDo

\switchcolumn

\begin{greek}[variant=ancient]%
πῶς ἔχεις;

\end{greek}%
\switchcolumn*

\trDE{Schlecht.}%
\trJA{悪うございます。}% ToDo

\switchcolumn

\begin{greek}[variant=ancient]%
ἔχω κακῶς.

\end{greek}%
\switchcolumn*

\trDE{Ich habe keine Freude mehr am Leben.}%
\trJA{生まれてこの方、喜びなど一つもありはしませんでした。}% ToDo

\switchcolumn

\begin{greek}[variant=ancient]%
οὐδεμίαν ἔχω τῷ βίῳ χάριν.

\end{greek}%
\switchcolumn*

\trDE{Es geht mir (wirthschaftlich) nicht gut.}%
\trJA{(経済的に)よくありません。}% ToDo

\switchcolumn

\begin{greek}[variant=ancient]%
κακῶς πράττω.

\end{greek}%
\switchcolumn*

\trDE{Es steht schlecht mit mir.}%
\trJA{あまりよくありません。}% ToDo

\switchcolumn

\begin{greek}[variant=ancient]%
φαῦλόν ἐστι τὸ ἐμὸν πρᾶγμα.

\end{greek}%
\switchcolumn*

\trDE{Wie lebt sich's in Leipzig?}%
\trJA{ライプツィヒでの生活はいかがですか。}% ToDo

\switchcolumn

\begin{greek}[variant=ancient]%
τίς ἐσθ' ὁ ἐν Λειψίᾳ{*} βίος;

\end{greek}%
\switchcolumn*

\trDE{Ganz hübsch.}%
\trJA{誠に宜しゅうございます。}% ToDo

\switchcolumn

\begin{greek}[variant=ancient]%
οὐκ ἄχαρις.

\end{greek}%
\switchcolumn*[


\section{%
	\trDE{Was fehlt Ihnen?}%
	\trJA{何か悪い所はありますか?}%
}

]\indent %
\begin{tabular}{lc}
\trDE{Was fehlt Ihnen?}%
\trJA{何か悪い所はありますか?}%
& \ldelim\}{2}{1em}[]\tabularnewline
\trDE{Was ist mit Ihnen?}%
\trJA{何かありましたか?}%
& \tabularnewline
\end{tabular}

\switchcolumn

\begin{greek}[variant=ancient]%
\vspace{0.5em}
τί πράττεις;

\end{greek}%
\switchcolumn*

\trDE{Es geht mir merkwürdig.}%
\trJA{著しく(良い)。}%

\switchcolumn

\begin{greek}[variant=ancient]%
πάσχω θαυμαστόν.

\end{greek}%
\switchcolumn*

\trDE{Was haben Sie für Schmerzen.}%
\trJA{何処か痛みますか。}

\switchcolumn

\begin{greek}[variant=ancient]%
τί κάμνεις.

\end{greek}%
\switchcolumn*

\trDE{Was ist Ihnen zugestoßen?}%
\trJA{何があったのですか?}%

\switchcolumn

\begin{greek}[variant=ancient]%
τί πέπονθας.

\end{greek}%
\switchcolumn*

\trDE{Wie ist es Ihnen ergangen?}%
\trJA{何があったのですか?}%

\switchcolumn

\begin{greek}[variant=ancient]%
τί ἔπαθες.

\end{greek}%
\switchcolumn*

\trDE{Warum seufzen Sie?}%
\trJA{何故嘆いているのか?}%

\switchcolumn

\begin{greek}[variant=ancient]%
τί στένεις.

\end{greek}%
\switchcolumn*

\trDE{Warum sind Sie so verstimmt?}%
\trJA{何故苛立っているのか?}%

\switchcolumn

\begin{greek}[variant=ancient]%
τί δυσφορεῖς.

\end{greek}%
\switchcolumn*

\trDE{Sieh nicht so finster aus, mein Lieber!}%
\trJA{そう陰気になるな、我が子よ!}%

\switchcolumn

\begin{greek}[variant=ancient]%
μὴ σκυθρώπαζε, ὦ τέκνον!

\end{greek}%
\switchcolumn*

\trDE{Ich langweile mich hier.}%
\trJA{ここに居るのは飽きた。}%

\switchcolumn

\begin{greek}[variant=ancient]%
ἄχθομαι ἐνθάδε παρών.

\end{greek}%
\switchcolumn*

\trDE{Sie scheinen mir zu frieren.}%
\trJA{君は凍えているやうだ。}%

\switchcolumn

\begin{greek}[variant=ancient]%
ῥιγῶν μοι δοκεῖς.

\end{greek}%
\switchcolumn*

\trDE{Mir ist schwindlig.}%
\trJA{眩暈がする。}%

\switchcolumn

\begin{greek}[variant=ancient]%
ἰλιγγιῶ.

\end{greek}%
\switchcolumn*

\trDE{Ich habe Kopf\textcompwordmark{}schmerz.}%
\trJA{頭痛がする。}%

\switchcolumn

\begin{greek}[variant=ancient]%
ἀλγῶ τὴν κεφαλήν\footnote{\begin{latin}%
orig. \textgreek[variant=ancient]{κεφαλὴν}\end{latin}%
}!

\end{greek}%
\switchcolumn*

\trDE{Sie haben jedenfalls Katzenjammer.}%
\trJA{確実に、君は二日酔いではないな。}%

\switchcolumn

\begin{greek}[variant=ancient]%
οὐκ ἔσθ' ὅπως οὐ κραιπαλᾷς.

\end{greek}%
\switchcolumn*

\trDE{An welcher Krankheit leben Sie?}%
\trJA{何の病気を患っているのか?}%

\switchcolumn

\begin{greek}[variant=ancient]%
τίνα νόσον νοσεῖς;

\end{greek}%
\switchcolumn*

\trDE{Sie haben doch wohl die Seekrankheit.}%
\trJA{たぶん、君は船酔いだ。}%

\switchcolumn

\begin{greek}[variant=ancient]%
ναυτιᾷς δήπου.

\end{greek}%
\switchcolumn*

\trDE{Du bekommst den Schnupfen.}%
\trJA{君は(鼻)風邪を引いている。}%

\switchcolumn

\begin{greek}[variant=ancient]%
κόρυζά σε λαμβάνει.

\end{greek}%
\switchcolumn*

\trDE{Ich leide an den Augen.}%
\trJA{眼炎を患っている。}%

\switchcolumn

\begin{greek}[variant=ancient]%
ὀφθαλμιῶ.

\end{greek}%
\switchcolumn*

\trDE{Bist du müde?}%
\trJA{もう疲れた?}%

\switchcolumn

\begin{greek}[variant=ancient]%
ἆρα κέκμηκας;

\end{greek}%
\switchcolumn*

\trDE{Mir thun die Beine weh von dem weiten Wege.}%
\trJA{長い旅路を来たので脚が痛い。}%

\switchcolumn

\begin{greek}[variant=ancient]%
ἀλγῶ τὰ σκέλη μακρὰν ὁδὸν διεληλυθώς.

\end{greek}%
\switchcolumn*

\trDE{Du bist besser zu Fuße als ich.}%
\trJA{歩くことにかけては、君は僕より強い。}%

\switchcolumn

\begin{greek}[variant=ancient]%
κρείττων εἶ μου σὺ βαδίζειν.

\end{greek}%
\switchcolumn*

\trDE{Sie wird ohnmächtig.}%
\trJA{彼女\footnote{唐突に現れた彼女は一体誰なんだろうか。}は気を失いつつある。}% TODO

\switchcolumn

\begin{greek}[variant=ancient]%
ὡρακιᾷ.

\end{greek}%
\switchcolumn*[


\section{%
	\trDE{Leben Sie wohl!}%
	\trJA{お元気で!}%
}

]

\indent
\trDE{Leben Sie wohl!}%
\trJA{お元気で!}%

\switchcolumn

\begin{greek}[variant=ancient]%
ὑγίαινε!

\end{greek}%
\switchcolumn*

\trDE{Ich will gehen, leben Sie wohl!}%
\trJA{さて、私は行く。お元気で!}%

\switchcolumn

\begin{greek}[variant=ancient]%
ἀλλ' εἶμι, σὺ δ' ὑγίαινε!

\end{greek}%
\switchcolumn*

\trDE{Leben Sie wohl (Erwiderung)!}%
\trJA{君も!}%

\switchcolumn

\begin{greek}[variant=ancient]%
καὶ σύγε!

\end{greek}%
\switchcolumn*

\trDE{Leben Sie recht wohl!}%
\trJA{お幸せに!}%

\switchcolumn

\begin{greek}[variant=ancient]%
χαῖρε πολλά!

\end{greek}%
\switchcolumn*

\trDE{Geben Sie mir eine Hand!}%
\trJA{右手\footnote{δεξιά, ἡは右手の意であるが、歓迎や挨拶の時にはδεξιὰν διδόναι [προτείνειν, ἐμβάλλειν]するものらしい。ホメーロスでは、差し出すものはδεξιάであってχείρは使われていなかったらしい。LSJ δεξιάより。}を差し出して下さい!}%

\switchcolumn

\begin{greek}[variant=ancient]%
ἔμβαλέ μοι τὴν δεξιάν!

\end{greek}%
\switchcolumn*

\trDE{Nun so leben Sie denn wohl und behalten Sie mich in gutem Andenken!}%
\trJA{では、さらば。私のことを憶えておいてくれ。}%

\switchcolumn

\begin{greek}[variant=ancient]%
ἀλλὰ χαῖρε πολλὰ καὶ μέμνησό μου!

\end{greek}%
\switchcolumn*

\trDE{Auf Wiedersehen!}%
\trJA{ではまた!}%

\switchcolumn

\begin{greek}[variant=ancient]%
εἰς αὖθις!

\end{greek}%
\switchcolumn*

\trDE{Viel Vergnügen!}%
\trJA{楽しんで来てね!}%

\switchcolumn

\begin{greek}[variant=ancient]%
ἴθι χαίρων!

\end{greek}%
\switchcolumn*

\trDE{Gute Nacht!}%
\trJA{お休み!}%

\switchcolumn

\begin{greek}[variant=ancient]%
ὑγίαινε!\textgerman[spelling=old,babelshorthands=true]{ (Auch am Morgen
beim Abschied).}

\end{greek}%
\switchcolumn*[


\section{%
	\trDE{Ich bitte}%
	\trJA{すみません!}%
}

]
\indent
\trDE{Verzeihen Sie!}%
\trJA{すみません!}%

\switchcolumn

\begin{greek}[variant=ancient]%
συγγνώμην ἔχε!

\end{greek}%
\switchcolumn*

\trDE{Ent\textcompwordmark{}schuldigen Sie!}%
\trJA{ごめんなさい!}%

\switchcolumn

\begin{greek}[variant=ancient]%
σύγγνωθί μοι.

\end{greek}%
\switchcolumn*

\trDE{Es ist meins. Geben Sie mir es, bitte!}%
\trJA{それは私のものだ。こちらに渡してくださいませんか。}%

\switchcolumn

\begin{greek}[variant=ancient]%
ἐστι τὸ ἐμόν. ἀλλὰ δός μοι, ἀντιβολῶ!

\end{greek}%
\switchcolumn*

\trDE{Ich bitte Sie, geben Sie es mir!}%
\trJA{どうか、それを私にくださいませんか?}%

\switchcolumn

\begin{greek}[variant=ancient]%
δός μοι πρὸς τῶν θεῶν!

\end{greek}%
\switchcolumn*

\trDE{Ich bitte Sie inständigst.}%
\trJA{どうか、よろしくお願いします。}%

\switchcolumn

\begin{greek}[variant=ancient]%
πρὸς τοῦ Διός, ἀντιβολῶ σε.

\end{greek}%
\switchcolumn*

\trDE{Ich bitte um Himmels\textcompwordmark{}willen!.}%
\trJA{天地神明に誓って。}%

\switchcolumn

\begin{greek}[variant=ancient]%
πρὸς πάντων θεῶν!

\end{greek}%
\switchcolumn*

\trDE{Thun Sie mir den Gefallen!}%
\trJA{頼み事があるのだが。}%

\switchcolumn

\begin{greek}[variant=ancient]%
χάρισαί μοι!

\end{greek}%
\switchcolumn*

\trDE{Nun, so thun Sie uns denn den Gefallen.}%
\trJA{我々の為に頼まれてくれないか。}%

\switchcolumn

\begin{greek}[variant=ancient]%
ἀλλὰ χάρισαι ἡμῖν!

\end{greek}%
\switchcolumn*

\trDE{Thun Sie mir einen kleinen Gefallen!}%
\trJA{ちょっとした頼み事があるのだが。}%

\switchcolumn

\begin{greek}[variant=ancient]%
χάρισαι βραχύ τί μοι!

\end{greek}%
\switchcolumn*

\trDE{Was soll ich Ihnen zu Gefallen thun?}%
\trJA{何をしたら良い?}%

\switchcolumn

\begin{greek}[variant=ancient]%
τί σοι χαρίσωμαι.

\end{greek}%
\switchcolumn*

\trDE{Sei so gut und gieb mir's.}%
\trJA{私にそれをくれないか?(どれくらい叮嚀なんだろ……?)}%

\switchcolumn

\begin{greek}[variant=ancient]%
βούλει μοι δοῦναι;

\end{greek}%
\switchcolumn*

\emph{Den} Gefallen will ich Ihnen thun.

\switchcolumn

\begin{greek}[variant=ancient]%
χαριοῦμαί σοι τοῦτο.

\end{greek}%
\switchcolumn*

Gleich!

\switchcolumn

\begin{greek}[variant=ancient]%
ταῦτα!

\end{greek}%
\switchcolumn*

Recht gern!

\switchcolumn

\begin{greek}[variant=ancient]%
φθόνος οὐδείς!

\end{greek}%
\switchcolumn*

Sagen Sie es doch gefälligst den Anderen!

\switchcolumn

\begin{greek}[variant=ancient]%
οὐ δῆτα γενναίως τοῖς ἄλλοις ἐρεῖς;

\end{greek}%
\switchcolumn*

Bitte, sag' es ihm doch!

\switchcolumn

\begin{greek}[variant=ancient]%
εἰπὲ δῆτα αὐτῷ πρὸς τῶν θεῶν!

\end{greek}%
\switchcolumn*

\emph{Darf ich mir erlauben} Ihnen einzuschenken?

\switchcolumn

\begin{greek}[variant=ancient]%
\emph{βούλει} ἐγχέω σοι πιεῖν;

\end{greek}%
\switchcolumn*[


\section{Ich danke}

]\indent Ich danke!

\switchcolumn

\begin{greek}[variant=ancient]%
ἐπαινῶ.

\end{greek}%
\switchcolumn*

Ich danke Ihnen!

\switchcolumn

\begin{greek}[variant=ancient]%
ἐπαινῶ τὸ σόν!

\end{greek}%
\switchcolumn*

Ich danke Ihnen für Ihre freundliche Gesinnung.

\switchcolumn

\begin{greek}[variant=ancient]%
ἐπαινῶ τὴν σὴν πρόνοιαν.

\end{greek}%
\switchcolumn*

Haben Sie vielen Dank dafür!

\switchcolumn

\begin{greek}[variant=ancient]%
εὖ γ' ἐμοίησας!

\end{greek}%
\switchcolumn*

Sie sind sehr gütig.

\switchcolumn

\begin{greek}[variant=ancient]%
γενναῖος εἶ.

\end{greek}%
\switchcolumn*

Ich werde Ihnen nur dankbar sein, wenn Sie das thun.

\switchcolumn

\begin{greek}[variant=ancient]%
χάριν γε εἴσομαι, ἐὰν τοῦτο ποιῇς.

\end{greek}%
\switchcolumn*

Ich bin Ihnen zu Danke verpflichtet.

\switchcolumn

\begin{greek}[variant=ancient]%
κεκάρισαί μοι.

\end{greek}%
\switchcolumn*

Der Himmel segne Sie tausendmal!

\switchcolumn

\begin{greek}[variant=ancient]%
πόλλ' ἀγαθὰ γένοιό σοι!

\end{greek}%
\switchcolumn*

Danke schön! (auch ablehnend.)

\switchcolumn

\begin{greek}[variant=ancient]%
καλῶς!

\end{greek}%
\switchcolumn*

Ich danke bestens! (des\textcompwordmark{}gl.)

\switchcolumn

\begin{greek}[variant=ancient]%
κάλλιστα· ἐπαινῶ.

\end{greek}%
\switchcolumn*

Bravo! Bravo!

\switchcolumn

\begin{greek}[variant=ancient]%
εὖγε! εὖγι.

\end{greek}%
\switchcolumn*

Wie herrlich!

\switchcolumn

\begin{greek}[variant=ancient]%
ὡς ἡδύ!

\end{greek}%
\switchcolumn*

Hurrah! (Freudenruf.)

\switchcolumn

\begin{greek}[variant=ancient]%
ἀλαλαί!

\end{greek}%
\switchcolumn*

Das macht nichts. Das ist einerlei.

\switchcolumn

\begin{greek}[variant=ancient]%
οὐδὲν διαφέρει.

\end{greek}%
\switchcolumn*

\begin{tabular}{lc}
Das kümmert mich wenig. & \ldelim\}{2}{1em}[]\tabularnewline
Daran liegt mir wenig. & \tabularnewline
\end{tabular}

\switchcolumn

\begin{greek}[variant=ancient]%
\vspace{0.5em}
ὀλίγον μέλει μοι.

\end{greek}%
\switchcolumn*

Was geht das \emph{mich} an?

\switchcolumn

\begin{greek}[variant=ancient]%
τί δ' ἐμοὶ ταῦτα;

\end{greek}%
\switchcolumn*

Was geht \emph{Sie} das an?.

\switchcolumn

\begin{greek}[variant=ancient]%
τί δ' σοὶ τοῦτο;

\end{greek}%
\switchcolumn*

Sie interessirt es wahrscheinlich nicht.

\switchcolumn

\begin{greek}[variant=ancient]%
σοὶ δ' ἴσως οὐδὲν μέλει.

\end{greek}%
\switchcolumn*

Da sieh \emph{du} zu!

\switchcolumn

\begin{greek}[variant=ancient]%
αὐτὸς σκόπει σύ!

\end{greek}%
\switchcolumn*

Es ist einmal so Sitte.

\switchcolumn

\begin{greek}[variant=ancient]%
νόμος γάρ ἐστιν.

\end{greek}%
\switchcolumn*[


\section{Können Sie Griechisch?}

]\indent Können Sie Griechisch?

\switchcolumn

\begin{greek}[variant=ancient]%
ἐπίστασαι ἑλληνίζειν;

\end{greek}%
\switchcolumn*

Ein wenig.

\switchcolumn

\begin{greek}[variant=ancient]%
ὀλίγον τι.

\end{greek}%
\switchcolumn*

Natürlich!

\switchcolumn

\begin{greek}[variant=ancient]%
εἰκότως γε!

\end{greek}%
\switchcolumn*

Ja freilich!

\switchcolumn

\begin{greek}[variant=ancient]%
μάλιστα!

\end{greek}%
\switchcolumn*

Ja gewiß!

\switchcolumn

\begin{greek}[variant=ancient]%
ἔγωγε νὴ Δία!

\end{greek}%
\switchcolumn*

Darin bin ich stark.

\switchcolumn

\begin{greek}[variant=ancient]%
ταύτῃ κράτιστός εἰμι.

\end{greek}%
\switchcolumn*

Schön!

\switchcolumn

\begin{greek}[variant=ancient]%
καλῶς!

\end{greek}%
\switchcolumn*

Da wollen wir einmal Griechisch mit einander sprechen!

\switchcolumn

\begin{greek}[variant=ancient]%
διαλεχθῶμεν οὖν ἑλληνικῶς!

\end{greek}%
\switchcolumn*

Meinetwegen.

\switchcolumn

\begin{greek}[variant=ancient]%
οὐδὲν κωλύει.

\end{greek}%
\switchcolumn*

Was meinen sie?

\switchcolumn

\begin{greek}[variant=ancient]%
τί λέγεις;

\end{greek}%
\switchcolumn*

Verstehen Sie, was ich meine?

\switchcolumn

\begin{greek}[variant=ancient]%
ξυνίης τὰ λεγόμενα;

\end{greek}%
\switchcolumn*

Haben Sie verstanden, was ich meine?

\switchcolumn

\begin{greek}[variant=ancient]%
ξυνῆκας, ὃ λέγω;!

\end{greek}%
\switchcolumn*

Nein, ich verstehe es nicht.

\switchcolumn

\begin{greek}[variant=ancient]%
οὐ ξυνίημι μὰ Δία.

\end{greek}%
\switchcolumn*

Wiederholen Sie es gefälligst noch einmal!

\switchcolumn

\begin{greek}[variant=ancient]%
αὖθις ἐξ ἀρχῆς λέγε, ἀντιβολῶ!

\end{greek}%
\switchcolumn*

Seien Sie so gut und sprechen sie langsamer!

\switchcolumn

\begin{greek}[variant=ancient]%
βούλει σχολαίτερον λέγειν;

\end{greek}%
\switchcolumn*[


\section{Fragen}

]\indent Was giebt's?

\switchcolumn

\begin{greek}[variant=ancient]%
τί δ' ἔστιν;

\end{greek}%
\switchcolumn*

Wie?

\switchcolumn

\begin{greek}[variant=ancient]%
τί λέγεις;

\end{greek}%
\switchcolumn*

\emph{Was} denn?

\switchcolumn

\begin{greek}[variant=ancient]%
τί δή;

\end{greek}%
\switchcolumn*

Was \emph{denn}?

\switchcolumn

\begin{greek}[variant=ancient]%
τί δαί;

\end{greek}%
\switchcolumn*

\emph{Wie} denn?

\switchcolumn

\begin{greek}[variant=ancient]%
πῶς δή;

\end{greek}%
\switchcolumn*

Wie \emph{denn}?

\switchcolumn

\begin{greek}[variant=ancient]%
πῶς δαί;

\end{greek}%
\switchcolumn*

Warum denn?

\switchcolumn

\begin{greek}[variant=ancient]%
ὁτιὴ τί δή; τιὴ τί δή;

\end{greek}%
\switchcolumn*

Wes\textcompwordmark{}halb?

\switchcolumn

\begin{greek}[variant=ancient]%
τίνος ἕνεκα;

\end{greek}%
\switchcolumn*

In wiefern?

\switchcolumn

\begin{greek}[variant=ancient]%
τίνι τρόπῳ;

\end{greek}%
\switchcolumn*

Wieso denn?

\switchcolumn

\begin{greek}[variant=ancient]%
πῶς δή;

\end{greek}%
\switchcolumn*

Bitte, wo?

\switchcolumn

\begin{greek}[variant=ancient]%
ποῦ δῆτα;

\end{greek}%
\switchcolumn*

Wohin? Woher?

\switchcolumn

\begin{greek}[variant=ancient]%
ποῖ; πόθεν;

\end{greek}%
\switchcolumn*

Wann?

\switchcolumn

\begin{greek}[variant=ancient]%
πηνίκα;

\end{greek}%
\switchcolumn*

Er straft ihn.

\switchcolumn

\begin{greek}[variant=ancient]%
κολάζει αὐτόν.

\end{greek}%
\switchcolumn*

Wofür?

\switchcolumn

\begin{greek}[variant=ancient]%
τί δράσαντα;

\end{greek}%
\switchcolumn*

Wodurch?

\switchcolumn

\begin{greek}[variant=ancient]%
τί δρῶν;

\end{greek}%
\switchcolumn*

Zu welchem Zwecke denn?

\switchcolumn

\begin{greek}[variant=ancient]%
ἵνα δὴ τί;

\end{greek}%
\switchcolumn*

Um was handelt es sich?

\switchcolumn

\begin{greek}[variant=ancient]%
τί τὸ πρᾶγμα;

\end{greek}%
\switchcolumn*

Meinen Sie nicht auch?

\switchcolumn

\begin{greek}[variant=ancient]%
οὐ καὶ σοὶ δοκεῖ;

\end{greek}%
\switchcolumn*

Wär's möglich?

\switchcolumn

\begin{greek}[variant=ancient]%
πῶς φής;

\end{greek}%
\switchcolumn*

Wo blieb' \emph{ich}?

\switchcolumn

\begin{greek}[variant=ancient]%
τί ἐγὼ δέ;

\end{greek}%
\switchcolumn*

Laß doch einmal sehen!

\switchcolumn

\begin{greek}[variant=ancient]%
φέρ' ἴδω!

\end{greek}%
\switchcolumn*

Nun, machen sie Fort\textcompwordmark{}schritte?

\switchcolumn

\begin{greek}[variant=ancient]%
τί δέ, ἐπιδώσεις λαμβάνεις;

\end{greek}%
\switchcolumn*[


\section{Wie heißen Sie?}

]\indent Wie heißen Sie?

\switchcolumn

\begin{greek}[variant=ancient]%
ὄνομά σοι τί ἐστιν;

\end{greek}%
\switchcolumn*

Wie heißen Sie mit Vor- und Zunamen?

\switchcolumn

\begin{greek}[variant=ancient]%
τίνα σοι ὀνόματα.

\end{greek}%
\switchcolumn*

Wie heißen Sie eigentlich?

\switchcolumn

\begin{greek}[variant=ancient]%
τί σοί ποτ' ἔστ' ὄνομα;

\end{greek}%
\switchcolumn*

Wer \emph{sind} Sie?

\switchcolumn

\begin{greek}[variant=ancient]%
σὺ δὲ τίς εἰ;

\end{greek}%
\switchcolumn*

Wer sind \emph{Sie}?

\switchcolumn

\begin{greek}[variant=ancient]%
τίς δ' εἶ σύ;

\end{greek}%
\switchcolumn*

Wer sind Sie eigentlich?

\switchcolumn

\begin{greek}[variant=ancient]%
σὺ δ' εἶ τίς ἐτεόν;

\end{greek}%
\switchcolumn*

Ich heiße Müller.

\switchcolumn

\begin{greek}[variant=ancient]%
ὄνομά μοι Μύλλερος.

\end{greek}%
\switchcolumn*

Wer ist eigentlich der hier?

\switchcolumn

\begin{greek}[variant=ancient]%
τίς ποθ' ὅδε;

\end{greek}%
\switchcolumn*

Wer muß das nur sein?

\switchcolumn

\begin{greek}[variant=ancient]%
τίς ἄρα ποτ' ἐστίν;

\end{greek}%
\switchcolumn*

Und wo sind Sie her?

\switchcolumn

\begin{greek}[variant=ancient]%
καὶ ποδαπός;

\end{greek}%
\switchcolumn*

Wo wohnen Sie?

\switchcolumn

\begin{greek}[variant=ancient]%
ποῦ κατοικεῖς;

\end{greek}%
\switchcolumn*

Ich wohne ganz in der Nähe.

\switchcolumn

\begin{greek}[variant=ancient]%
ἐγγύτατα οἰκῶ.

\end{greek}%
\switchcolumn*

Ich wohne weit.

\switchcolumn

\begin{greek}[variant=ancient]%
τηλοῦ οἰκῶ.

\end{greek}%
\switchcolumn*

Nennen Sie mich nicht bei Namen!

\switchcolumn

\begin{greek}[variant=ancient]%
μὴ κάλει μου τοὔνομα!

\end{greek}%
\switchcolumn*

So rufen Sie mich doch nicht, ich bitte Sie!

\switchcolumn

\begin{greek}[variant=ancient]%
οὐ μὴ καλεῖς με; ἱκετεύω!

\end{greek}%
\switchcolumn*[


\section{Wieviel Uhr ist es?}

]\indent Wie viel Uhr ist es?

\switchcolumn

\begin{greek}[variant=ancient]%
τίς ὥρα ἐστίν;

\end{greek}%
\switchcolumn*

Wie spät ist es am Tage?

\switchcolumn

\begin{greek}[variant=ancient]%
πηνίκ' ἐστὶ τῆς ἡμέρας;

\end{greek}%
\switchcolumn*

Es ist um Eins.

\switchcolumn

\begin{greek}[variant=ancient]%
ἐσὶ μία ὥρα.

\end{greek}%
\switchcolumn*

Es ist um Zwei (Dri, Vier).

\switchcolumn

\begin{greek}[variant=ancient]%
εἰσὶ δύο (τρεῖς, τέσσαρες) ὦραι.!

\end{greek}%
\switchcolumn*

Es ist ½2 Uhr.

\switchcolumn

\begin{greek}[variant=ancient]%
ἐστὶ μία ὥρα καὶ ἡμίσεια.

\end{greek}%
\switchcolumn*

Um welche Zeit?

\switchcolumn

\begin{greek}[variant=ancient]%
πηνίκα;

\end{greek}%
\switchcolumn*

Um ein Uur.

\switchcolumn

\begin{greek}[variant=ancient]%
τῇ πρώτῃ ὥρᾳ.

\end{greek}%
\switchcolumn*

Um zwei.

\switchcolumn

\begin{greek}[variant=ancient]%
τῇ δευτέρᾳ (ὥρα).

\end{greek}%
\switchcolumn*

Es ist noch weiter (später).

\switchcolumn

\begin{greek}[variant=ancient]%
περαιτέρω ἐστίν.

\end{greek}%
\switchcolumn*

Es ist ein Viertel nach Sieben.

\switchcolumn

\begin{greek}[variant=ancient]%
εἰσὶν ἑπτὰ ὦραι καὶ τέταρτον.

\end{greek}%
\switchcolumn*

Es ist drei Viertel auf Eins.

\switchcolumn

\begin{greek}[variant=ancient]%
εἰσὶ δώδεκα (ὦραι) καὶ τρία τέταρτα.

\end{greek}%
\switchcolumn*

Um die dritte Stunde.

\switchcolumn

\begin{greek}[variant=ancient]%
περὶ τὴν τρίτην ὥραν.!

\end{greek}%
\switchcolumn*

Gegen halb fünf.

\switchcolumn

\begin{greek}[variant=ancient]%
περὶ τὴν τετάρτην καὶ ἡμίσειαν!

\end{greek}%
\switchcolumn*

Ich werde um ¾11 Uhr kommen.

\switchcolumn

\begin{greek}[variant=ancient]%
ἥξω εἰς τὴν δεκάτην καὶ τρία τέταρτα.

\end{greek}%
\switchcolumn*[


\section{Tages\textcompwordmark{}zeiten}

]\indent Zu Mittag.

\switchcolumn

\begin{greek}[variant=ancient]%
ἐν μεσημβρίᾳ.

\end{greek}%
\switchcolumn*

Vormittags.

\switchcolumn

\begin{greek}[variant=ancient]%
πρὸ μεσημβρίας.

\end{greek}%
\switchcolumn*

Nachmittags.

\switchcolumn

\begin{greek}[variant=ancient]%
μετὰ μεσημβρίαν.

\end{greek}%
\switchcolumn*

Es ist hell.

\switchcolumn

\begin{greek}[variant=ancient]%
φῶς ἐστιν.

\end{greek}%
\switchcolumn*

Es ist (wird) dunkel.

\switchcolumn

\begin{greek}[variant=ancient]%
σκότος γίγνεται.

\end{greek}%
\switchcolumn*

Im Finstern.

\switchcolumn

\begin{greek}[variant=ancient]%
ἐν (τῷ) σκότῳ.

\end{greek}%
\switchcolumn*

Abends.

\switchcolumn

\begin{greek}[variant=ancient]%
\emph{τῆς ἑσπέρας.}

\end{greek}%
\switchcolumn*

Gestern Abend.

\switchcolumn

\begin{greek}[variant=ancient]%
\emph{ἑσπέρας.}

\end{greek}%
\switchcolumn*

Heute Abend. (künstig.)

\switchcolumn

\begin{greek}[variant=ancient]%
\emph{εἰς ἑσπέραν.}

\end{greek}%
\switchcolumn*

Abends spät.

\switchcolumn

\begin{greek}[variant=ancient]%
νύκτωρ ὀψέ.

\end{greek}%
\switchcolumn*

Den Tag über.

\switchcolumn

\begin{greek}[variant=ancient]%
δι' ἡμέρας.

\end{greek}%
\switchcolumn*

Die ganze Nacht hindurch.

\switchcolumn

\begin{greek}[variant=ancient]%
ὅλην τὴν νύκτα.

\end{greek}%
\switchcolumn*

Vom frühen Morgen an.

\switchcolumn

\begin{greek}[variant=ancient]%
ἐξ ἑωθινοῦ.

\end{greek}%
\switchcolumn*

Von früh an.

\switchcolumn

\begin{greek}[variant=ancient]%
ἐξ ἕω.

\end{greek}%
\switchcolumn*

Gleich von früh an.

\switchcolumn

\begin{greek}[variant=ancient]%
ἕωθεν εὐθύς.

\end{greek}%
\switchcolumn*

Heute Morgens.

\switchcolumn

\begin{greek}[variant=ancient]%
ἕωθεν.

\end{greek}%
\switchcolumn*

Morgen früh.

\switchcolumn

\begin{greek}[variant=ancient]%
αὔριον ἕωθεν.

\end{greek}%
\switchcolumn*

Heute.

\switchcolumn

\begin{greek}[variant=ancient]%
τῇδε τῇ ἡμέρᾳ. --- τήμερον\footnote{\begin{latin}%
\textgreek[variant=ancient]{ὁ τυπογράφος ἔγραψα τὸν οὐ γεγραμμένον
τόνον.}\end{latin}%
}.

\end{greek}%
\switchcolumn*

Gestern.

\switchcolumn

\begin{greek}[variant=ancient]%
χθές. ἐχθές.

\end{greek}%
\switchcolumn*

Morgen.

\switchcolumn

\begin{greek}[variant=ancient]%
αὔριον.

\end{greek}%
\switchcolumn*

Übermorgen.

\switchcolumn

\begin{greek}[variant=ancient]%
ἕνης. εἰς ἕνηςν

\end{greek}%
\switchcolumn*

Vorgestern.

\switchcolumn

\begin{greek}[variant=ancient]%
τρίτην ἡμέραν. (\textgerman[spelling=old,babelshorthands=true]{auch}
νεωστί).

\end{greek}%
\switchcolumn*[


\section{Jetzt\textcompwordmark{}zeit. Feste}

]\indent In der jetzigen Zeit.

\switchcolumn

\begin{greek}[variant=ancient]%
ἐν τῷ νῦν χρόνῳ.

\end{greek}%
\switchcolumn*

Gerade wie früher.

\switchcolumn

\begin{greek}[variant=ancient]%
ὥσπερ καὶ πρὸ τοῦ.

\end{greek}%
\switchcolumn*

Auf welchen Tag?

\switchcolumn

\begin{greek}[variant=ancient]%
ἐς\footnote{\begin{latin}%
\textgreek[variant=ancient]{ὁ τυπογράφος θαυμάζω τοῦ ἑνεκα γέγραφε
οὐ «εἰς» ὃν Ἀττικοὶ ἔχραον, ἀλλὰ «ἐς» ὃν Ἰωνικοί.}\end{latin}%
} τίνα ἡμέραν.

\end{greek}%
\switchcolumn*

Für sogleich.

\switchcolumn

\begin{greek}[variant=ancient]%
ἐς αὐτίκα μάλα.

\end{greek}%
\switchcolumn*

Vor Kurzem.

\switchcolumn

\begin{greek}[variant=ancient]%
τὸ ἔναγχος.

\end{greek}%
\switchcolumn*

Lange genug.

\switchcolumn

\begin{greek}[variant=ancient]%
ἰκανὸν χρόνον.

\end{greek}%
\switchcolumn*

Heute über 14 Tage.

\switchcolumn

\begin{greek}[variant=ancient]%
μεθ' ἡμέρας \emph{μεντεκαὶδεκα} ἀπὸ τῆς τήμερον.

\end{greek}%
\switchcolumn*

Heuer.

\switchcolumn

\begin{greek}[variant=ancient]%
τῆτες.

\end{greek}%
\switchcolumn*

Vor'm Jahr.

\switchcolumn

\begin{greek}[variant=ancient]%
πέρυσιν.

\end{greek}%
\switchcolumn*

Über's Jahr.

\switchcolumn

\begin{greek}[variant=ancient]%
εἰς νέωτα.

\end{greek}%
\switchcolumn*

Alle vier Jahre.

\switchcolumn

\begin{greek}[variant=ancient]%
δι' ἔτους \emph{πέμπτου.}

\end{greek}%
\switchcolumn*

Monatlich.

\switchcolumn

\begin{greek}[variant=ancient]%
κατὰ μῆνα.

\end{greek}%
\switchcolumn*

Der Frühling. Der Sommer.

\switchcolumn

\begin{greek}[variant=ancient]%
τὸ ἔαρ. τὸ θέρος.

\end{greek}%
\switchcolumn*

Der Herbst. Der Winter.

\switchcolumn

\begin{greek}[variant=ancient]%
τὸ φθινόπωρον. ὁ χειμών.

\end{greek}%
\switchcolumn*

Zur Winters\textcompwordmark{}zeit.

\switchcolumn

\begin{greek}[variant=ancient]%
χειμῶνος ὄντος.

\end{greek}%
\switchcolumn*

Das Fest.

\switchcolumn

\begin{greek}[variant=ancient]%
ἡ ἑορτή.

\end{greek}%
\switchcolumn*

Weihnachten.

\switchcolumn

\begin{greek}[variant=ancient]%
τὰ Χριστούγεννα.{*}

\end{greek}%
\switchcolumn*

Neujahr.

\switchcolumn

\begin{greek}[variant=ancient]%
ἡ πρώτη τοῦ ἔτους.

\end{greek}%
\switchcolumn*

Fastnacht.

\switchcolumn

\begin{greek}[variant=ancient]%
αἱ ἀπόκρεω.{*}

\end{greek}%
\switchcolumn*

Charfreitag.

\switchcolumn

\begin{greek}[variant=ancient]%
ἡ μεγάλη παρασκευή.{*}

\end{greek}%
\switchcolumn*

Ostern.

\switchcolumn

\begin{greek}[variant=ancient]%
τὸ πάσχα.{*}

\end{greek}%
\switchcolumn*

Pfingsten.

\switchcolumn

\begin{greek}[variant=ancient]%
ἡ πεντηκοστή.

\end{greek}%
\switchcolumn*

Geburts\textcompwordmark{}tag.

\switchcolumn

\begin{greek}[variant=ancient]%
τὸ γενέθλια.

\end{greek}%
\switchcolumn*

Jahres\textcompwordmark{}tag (Stiftungs\textcompwordmark{}fest)..

\switchcolumn

\begin{greek}[variant=ancient]%
ἡ ἐπέτειος ἑορτή.

\end{greek}%
\switchcolumn*[\centering\rule{1.5in}{1pt}]

Die Monate:

\switchcolumn

\begin{greek}[variant=ancient]%
οἱ μῆνες: Ἰανουάριος. Φεβρουάριος. Μάρτιος. Ἀπρίλιος. Μάϊος. Ἰούνιος.
Ἰούλιος. Αὔγουστος. Σεπτέμβριος. Ὀκτώβριος. Νοεμβριος. Δεκέμβριος.

\end{greek}%
\switchcolumn*[


\section{Das Wetter}

]\indent Was haben wir für Wetter?

\switchcolumn

\begin{greek}[variant=ancient]%
ποῖος ὁ \emph{ἀὴρ} τό νῦν;

\end{greek}%
\switchcolumn*

Das Wetter ist schön.

\switchcolumn

\begin{greek}[variant=ancient]%
εὐδία ἐστίν.

\end{greek}%
\switchcolumn*

Es ist herrliches Wetter.

\switchcolumn

\begin{greek}[variant=ancient]%
εὐδία ἐστὶν ἡδίστη.

\end{greek}%
\switchcolumn*

Die Sonne scheint.

\switchcolumn

\begin{greek}[variant=ancient]%
ἐξέχει εἵλη ἔχομεν ἥλιον. φαίνεται ὁ ἥλιος. ἥλιος λάμπει.

\end{greek}%
\switchcolumn*

Es ist warm.

\switchcolumn

\begin{greek}[variant=ancient]%
θάλμος ἐστίν.

\end{greek}%
\switchcolumn*

Es ist windig. (Der Wind geht.)

\switchcolumn

\begin{greek}[variant=ancient]%
ἄνεμος γίγνεται.

\end{greek}%
\switchcolumn*

Es weht ein starker Wind.

\switchcolumn

\begin{greek}[variant=ancient]%
ἄνεμος πνεῖ \emph{μέγας.}

\end{greek}%
\switchcolumn*

Wir haben Nord-, Süd-, Ost-, Westwind.

\switchcolumn

\begin{greek}[variant=ancient]%
ἄνεμος γίγνεται βόρειος, νότιος\footnote{\begin{latin}%
\textgreek[variant=ancient]{ὁ τυπογράφος ἔγραψα τὸν οὐ γεγραμμένον
τόνον.}\end{latin}%
}, ἀνατολικός, δυτικός.!

\end{greek}%
\switchcolumn*

Es umwölkt sich.

\switchcolumn

\begin{greek}[variant=ancient]%
ξυννεφεῖ.

\end{greek}%
\switchcolumn*

Es sprüht.

\switchcolumn

\begin{greek}[variant=ancient]%
ψακάζει.

\end{greek}%
\switchcolumn*

Es regnet.

\switchcolumn

\begin{greek}[variant=ancient]%
ὕει.

\end{greek}%
\switchcolumn*

Es gießt sehr.

\switchcolumn

\begin{greek}[variant=ancient]%
ὄμβρος πολὺς γίγνεται.

\end{greek}%
\switchcolumn*

Es donnert.

\switchcolumn

\begin{greek}[variant=ancient]%
βροντᾷ.

\end{greek}%
\switchcolumn*

Wir haben ein Gewitter.

\switchcolumn

\begin{greek}[variant=ancient]%
βρονταὶ γίγνονται καὶ κεραυνοί.

\end{greek}%
\switchcolumn*

Es blitzt stark.

\switchcolumn

\begin{greek}[variant=ancient]%
ἀστράπτει πολὺ νὴ Δία.

\end{greek}%
\switchcolumn*

Es hat eingeschlagen.

\switchcolumn

\begin{greek}[variant=ancient]%
ἔπεσε σκηπτός. ἔπεσε κεραυνός.

\end{greek}%
\switchcolumn*

Es ist kalt. (sehr kalt.)

\switchcolumn

\begin{greek}[variant=ancient]%
ψῦχός ἐστιν. (ψ. ἐστι μέγεστον.)

\end{greek}%
\switchcolumn*

Es schneit! hu!

\switchcolumn

\begin{greek}[variant=ancient]%
νίφει· βαβαιάξ!

\end{greek}%
\switchcolumn*

Es schneit sehr.

\switchcolumn

\begin{greek}[variant=ancient]%
χιών γίγνεται πολλή.

\end{greek}%
\switchcolumn*

Es friet.

\switchcolumn

\begin{greek}[variant=ancient]%
χρύος γίγνεται.

\end{greek}%
\switchcolumn*

Warum machst du den (Sonnen-)Schirm zu?.

\switchcolumn

\begin{greek}[variant=ancient]%
τί πάλιν ξυνάγεις τὸ σκιάδειον;

\end{greek}%
\switchcolumn*

Mach' ihn wieder auf!

\switchcolumn

\begin{greek}[variant=ancient]%
ἐκπέτασον αὐτό!

\end{greek}%
\switchcolumn*

Her mit dem Schirm!

\switchcolumn

\begin{greek}[variant=ancient]%
φέρε τὸ σκιάδειον!

\end{greek}%
\switchcolumn*

Halte den Schirm über mich!

\switchcolumn

\begin{greek}[variant=ancient]%
ὑπέρεχέ μου τὸ σκιάδειον.

\end{greek}%
\switchcolumn*

Nimm dich hier vor dem Schmutze in Acht!

\switchcolumn

\begin{greek}[variant=ancient]%
τὸν πηλὸν τουτονὶ φύλαξαι!

\end{greek}%
\switchcolumn*[


\section{Abreise}

]\indent Wann reisen Sie nach Berlin?

\switchcolumn

\begin{greek}[variant=ancient]%
πότε \emph{ἄπει} εἰς Βερόλινον{*} (Λόνδινον, Βιέννην{*} \textgerman[spelling=old,babelshorthands=true]{Wien,}
Γαστάϊν{*} , Παρισίους, Πετρούπολιν{*}, εἰς Ἑλβητίαν, Κίσσιγγεν{*},
Δρέσδην{*}, Βρυξέλας{*}, Μόναχον \textgerman[spelling=old,babelshorthands=true]{München});

\end{greek}%
\switchcolumn*

Um 12. November.

\switchcolumn

\begin{greek}[variant=ancient]%
τῇ δωδεκάτῃ Νοεμβρίου.

\end{greek}%
\switchcolumn*

Nach Leipzig sind Sie bis\textcompwordmark{}her noch nicht gekommen.

\switchcolumn

\begin{greek}[variant=ancient]%
εἰς Λειψίαν{*} οὔπω ἐλήλυθας.

\end{greek}%
\switchcolumn*

In den Ferien hätte ich Lust auf's Land zu gehen.

\switchcolumn

\begin{greek}[variant=ancient]%
ἐν τῷ ἀναπαύλης χρόνῳ ἐπιθυμῶ ἐλθεῖν εἰς ἀγρόν.

\end{greek}%
\switchcolumn*

Mit welcher Gelegenheit wollen Sie reisen?

\switchcolumn

\begin{greek}[variant=ancient]%
τίς σοι γενήσεται πόρος τῆς ὁδοῦ;

\end{greek}%
\switchcolumn*

Um vier Uhr mit dem Bahnzuge.

\switchcolumn

\begin{greek}[variant=ancient]%
τῇ τετάρτῃ ὥρᾳ χρώμενος τῇ ἁμαξοστοιχίᾳ\footnote{\begin{latin}%
\textgreek[variant=ancient]{τῷ τυπογράφῳ ἄσκοπος τὸ γράμμα «ἁ» ἦν.}\end{latin}%
}.{*}

\end{greek}%
\switchcolumn*

O, dann ist es Zeit zu gehen.

\switchcolumn

\begin{greek}[variant=ancient]%
ὥρα βαδίζειν ἄρ' ἐστίν.

\end{greek}%
\switchcolumn*

Es ist Zeit auf den Bahnhof zu gehen.

\switchcolumn

\begin{greek}[variant=ancient]%
ὥρα ἐστὶν εἰς τὸν (σιδηροδρομικὸν{*}) σταθμὸν βαδίζειν.

\end{greek}%
\switchcolumn*

Es wäre längst Zeit gewesen!

\switchcolumn

\begin{greek}[variant=ancient]%
ὥρα\footnote{\begin{latin}%
\textgreek[variant=ancient]{ὁ τυπογράφος ἔγραψα τὸν οὐ γεγραμμένον
τόνον.}\end{latin}%
} ἦν πάλαι.

\end{greek}%
\switchcolumn*

Nun, so reisen Sie glücklich!

\switchcolumn

\begin{greek}[variant=ancient]%
ἀλλ' ἴθι χαίρων!

\end{greek}%
\switchcolumn*

Adieu!

\switchcolumn

\begin{greek}[variant=ancient]%
χαῖρε καὶ σύ!

\end{greek}%
\switchcolumn*

Er ist abgereist.

\switchcolumn

\begin{greek}[variant=ancient]%
οἴχεται.

\end{greek}%
\switchcolumn*

Mein Bruder ist seit 5 Monaten fort.

\switchcolumn

\begin{greek}[variant=ancient]%
ὁ ἐμὸς ἀδελφὸς πέντε μῆνας ἄπεστιν.

\end{greek}%
\switchcolumn*

Er ist auf der Reise.

\switchcolumn

\begin{greek}[variant=ancient]%
ἀποδημῶν ἐστιν.

\end{greek}%
\switchcolumn*[


\section{Gehen. Weg.}

]\indent Kommen Sie mit!

\switchcolumn

\begin{greek}[variant=ancient]%
ἕπου!

\end{greek}%
\switchcolumn*

Kommen Sie mit mir!

\switchcolumn

\begin{greek}[variant=ancient]%
ἕπου μετ' ἐμοῦ!

\end{greek}%
\switchcolumn*

Der Bahnhof ist nicht weit.

\switchcolumn

\begin{greek}[variant=ancient]%
ἔστ' οὐ μεκρὰν ἄποθεν ὁ σταθμός.

\end{greek}%
\switchcolumn*

Nun, so wollen wir gehen.

\switchcolumn

\begin{greek}[variant=ancient]%
ἄγε νυν ἴωμεν.

\end{greek}%
\switchcolumn*

Wir wollen fortgehen

\switchcolumn

\begin{greek}[variant=ancient]%
ἀπίωμεν.

\end{greek}%
\switchcolumn*

Wir wollen weitergehen.

\switchcolumn

\begin{greek}[variant=ancient]%
χωρῶμεν.

\end{greek}%
\switchcolumn*

Vorwärts!

\switchcolumn

\begin{greek}[variant=ancient]%
χώρει!

\end{greek}%
\switchcolumn*

Wir wollen Euch voraus\textcompwordmark{}gehen.

\switchcolumn

\begin{greek}[variant=ancient]%
προίωμεν ὑμῶν.

\end{greek}%
\switchcolumn*

Ich werde eine Droschke nehmen.

\switchcolumn

\begin{greek}[variant=ancient]%
ἁμάξῃ χρήσομαι.

\end{greek}%
\switchcolumn*

Ich werde vielmehr den Omnibus benutzen.

\switchcolumn

\begin{greek}[variant=ancient]%
ἐγὼ μὲν οὖν χρήσομαι τῷ λεωφορείῳ{*}.

\end{greek}%
\switchcolumn*

Ich meinerseits gehe zu Fuße.

\switchcolumn

\begin{greek}[variant=ancient]%
βαδίζω ἔγωγε.

\end{greek}%
\switchcolumn*

Du reitest.

\switchcolumn

\begin{greek}[variant=ancient]%
ὀχεῖ!

\end{greek}%
\switchcolumn*

Sagen Sie, auf welchem Wege kommen wir am schnellsten nach dem Bahnhofe?

\switchcolumn

\begin{greek}[variant=ancient]%
φράζε, ὅπῃ τάχιστα ἀφιξόμεθα εἰς τὸν σταθμόν;

\end{greek}%
\switchcolumn*

Wir können den Weg nicht finden.

\switchcolumn

\begin{greek}[variant=ancient]%
οὐ δυνάμεθα ἐξευρεῖν τὴν ὁδόν.

\end{greek}%
\switchcolumn*

Ich weiß nicht mehr, wo wir sind.

\switchcolumn

\begin{greek}[variant=ancient]%
οὐκέτι οἶδα, ποῖ γῆς ἐσμεν..

\end{greek}%
\switchcolumn*

Sie haben den Weg verfehlt.

\switchcolumn

\begin{greek}[variant=ancient]%
τῆς ὁδοῦ ἡμάρτηκας.

\end{greek}%
\switchcolumn*

Ach, du mein Gott!

\switchcolumn

\begin{greek}[variant=ancient]%
ὦ φίλιο θεοί!

\end{greek}%
\switchcolumn*

Gehen Sie die Straße hier, so werden Sie sogleich auf den Marktplatz
kommen.

\switchcolumn

\begin{greek}[variant=ancient]%
ἴθι τὴν ὁδὸν ταυτηνί καὶ τὐθὺς ἐπὶ τὴν ἀγορὰν ἥξεις.

\end{greek}%
\switchcolumn*

Und was dann?

\switchcolumn

\begin{greek}[variant=ancient]%
εἶτα τί;

\end{greek}%
\switchcolumn*

Dann müssen Sie rechts (links) gehen.

\switchcolumn

\begin{greek}[variant=ancient]%
εἶτα βαδιστέα σοι ἐπὶ δεξιά (ἐπ' ἀριστερά).

\end{greek}%
\switchcolumn*

Gerade aus!

\switchcolumn

\begin{greek}[variant=ancient]%
ὀρθήν!

\end{greek}%
\switchcolumn*

Wie weit ist es etwa?

\switchcolumn

\begin{greek}[variant=ancient]%
πόση τις ἡ ὁδός;

\end{greek}%
\switchcolumn*

Danke.

\switchcolumn

\begin{greek}[variant=ancient]%
καλῶς.

\end{greek}%
\switchcolumn*

Nun, da wollen wir uns beeilen.

\switchcolumn

\begin{greek}[variant=ancient]%
ἀλλὰ σπεύδωμεν.

\end{greek}%
\switchcolumn*

Gehen Sie zu!

\switchcolumn

\begin{greek}[variant=ancient]%
χώρει!

\end{greek}%
\switchcolumn*

Wir sind erst nach dem zweiten Läuten gekommen.

\switchcolumn

\begin{greek}[variant=ancient]%
ὕστερον ἤλθομεν τοῦ δευτέρου σημείου.

\end{greek}%
\switchcolumn*[


\section{Warte!}

]\indent Du, halt einmal!

\switchcolumn

\begin{greek}[variant=ancient]%
ἐπίσχες, οὗτος!

\end{greek}%
\switchcolumn*

Warte einmal!

\switchcolumn

\begin{greek}[variant=ancient]%
ἔχε νυν ἥσυχος!

\end{greek}%
\switchcolumn*

Halt! Bleib' stehen!

\switchcolumn

\begin{greek}[variant=ancient]%
μέν' ἥσυχος! στῆθι!

\end{greek}%
\switchcolumn*

Nicht von der Stelle!

\switchcolumn

\begin{greek}[variant=ancient]%
ἔχ' ἀστέμας αὐτοῦ!

\end{greek}%
\switchcolumn*

So warte doch!

\switchcolumn

\begin{greek}[variant=ancient]%
\emph{οὐ μενεῖς;}

\end{greek}%
\switchcolumn*

Warte eine Weile auf mich!

\switchcolumn

\begin{greek}[variant=ancient]%
ἐπανάμεινον μ' ὀλίγον χρόνον.

\end{greek}%
\switchcolumn*

Ich werde gleich wiederkommen.

\switchcolumn

\begin{greek}[variant=ancient]%
ἀλλ' ἥξω ταχέως.

\end{greek}%
\switchcolumn*

Wo soll ich dich erwarten?

\switchcolumn

\begin{greek}[variant=ancient]%
ποῦ ἀναμεῶ;

\end{greek}%
\switchcolumn*

Komm' nur schnell wieder!

\switchcolumn

\begin{greek}[variant=ancient]%
ἧκέ νυν ταχύ!

\end{greek}%
\switchcolumn*

Da bin ich wieder.

\switchcolumn

\begin{greek}[variant=ancient]%
ἰδού, πάρειμι.

\end{greek}%
\switchcolumn*

Bist du wieder da?

\switchcolumn

\begin{greek}[variant=ancient]%
ἥκεις;

\end{greek}%
\switchcolumn*

Ich bin dir doch nicht zu lange gewesen?

\switchcolumn

\begin{greek}[variant=ancient]%
μῶν ἐπισχεῖν σοι δοκῶ;

\end{greek}%
\switchcolumn*

Wo bist du nur so lange geblieben?

\switchcolumn

\begin{greek}[variant=ancient]%
ποῦ ποτ' ἦσθα ἀπ' ἐμοῦ (ἀφ' ἡμῶν) τὸν πολὺν τοῦτον χρόνον;

\end{greek}%
\switchcolumn*[


\section{Komm her!}

]\indent Komm her!

\switchcolumn

\begin{greek}[variant=ancient]%
δεῦρ' ἐλθέ!

\end{greek}%
\switchcolumn*

Komm hierher!

\switchcolumn

\begin{greek}[variant=ancient]%
ἐλθὲ δεῦρο!

\end{greek}%
\switchcolumn*

Geh' her!

\switchcolumn

\begin{greek}[variant=ancient]%
χώρει δεῦρο!

\end{greek}%
\switchcolumn*

Geh' hierher, zu mir!

\switchcolumn

\begin{greek}[variant=ancient]%
βάδιζε δεῦρο, ὡς ἐμέ!

\end{greek}%
\switchcolumn*

Du kommst wie gerusen.

\switchcolumn

\begin{greek}[variant=ancient]%
ἥκεις ὥσπερ κατὰ θεῖον.

\end{greek}%
\switchcolumn*

Woher kommst du?

\switchcolumn

\begin{greek}[variant=ancient]%
πόθεν βαδίζεις;

\end{greek}%
\switchcolumn*

Aber wo kommst du eigentlich her?

\switchcolumn

\begin{greek}[variant=ancient]%
ἀτὰρ πόθεν ἥκεις ἐτεόν;

\end{greek}%
\switchcolumn*

Ich komme von Müllers.

\switchcolumn

\begin{greek}[variant=ancient]%
\emph{ἐκ Μυλλέρου} ἔρχομαι.

\end{greek}%
\switchcolumn*

Geh' mit mir hinein!

\switchcolumn

\begin{greek}[variant=ancient]%
εἴσιθι αμ'\footnote{ὁ τυπογράφος ἔγραψα τὸν οὐ γεγραμμένον τόνον.}
ἐμοί.

\end{greek}%
\switchcolumn*

Ich bitte dich, noch bei uns zu bleiben.

\switchcolumn

\begin{greek}[variant=ancient]%
δέομαί σου περαμεῖναι ἡμῖν.

\end{greek}%
\switchcolumn*

Das geht nicht!

\switchcolumn

\begin{greek}[variant=ancient]%
ἀλλ' οὐχ οἷόν τε!

\end{greek}%
\switchcolumn*

Wohin gehst du?

\switchcolumn

\begin{greek}[variant=ancient]%
ποῖ βαδίζεις;

\end{greek}%
\switchcolumn*

So bleib' doch da!

\switchcolumn

\begin{greek}[variant=ancient]%
οὐ παραμενεῖς;

\end{greek}%
\switchcolumn*

Wir lassen dich nicht fort.

\switchcolumn

\begin{greek}[variant=ancient]%
οὔ σ' ἀφήσομεν.

\end{greek}%
\switchcolumn*

Ich will zum Friseur.

\switchcolumn

\begin{greek}[variant=ancient]%
\emph{βούλομαι εἰς} τὸ κουρεῖον.

\end{greek}%
\switchcolumn*

Wir lassen dich durchaus nicht fort.

\switchcolumn

\begin{greek}[variant=ancient]%
οὐκ ἀφήσομέν σε μά δία οὐδέποτε!

\end{greek}%
\switchcolumn*

Laßt mich los!

\switchcolumn

\begin{greek}[variant=ancient]%
μέθεσθέ μου!

\end{greek}%
\switchcolumn*

Kommt schnell zu mir her!

\switchcolumn

\begin{greek}[variant=ancient]%
ἴτε δεῦρ' ὡς ἐμὲ ταχέως.

\end{greek}%
\switchcolumn*

Heute Abend will ich kommen.

\switchcolumn

\begin{greek}[variant=ancient]%
εἰς ἑσπέραν ἥξω.

\end{greek}%
\switchcolumn*

Weg ist er!

\switchcolumn

\begin{greek}[variant=ancient]%
φροῦδός ἐστιν!

\end{greek}%
\switchcolumn*

Wo ist er denn hin?

\switchcolumn

\begin{greek}[variant=ancient]%
ποῖ γὰρ οἴχεται;

\end{greek}%
\switchcolumn*

Er ist fort zum Friseur.

\switchcolumn

\begin{greek}[variant=ancient]%
εἰς τὸ κουρεῖον οἴχεται.

\end{greek}%
\switchcolumn*

Er geht heim.

\switchcolumn

\begin{greek}[variant=ancient]%
οἴκαδ' ἔρχεται.

\end{greek}%
\switchcolumn*

Wir wollen wieder heimgehen.

\switchcolumn

\begin{greek}[variant=ancient]%
ἀπίωμεν οἴκαδ' αὖθις.

\end{greek}%
\switchcolumn*

Er will ihnen entgegen gehen.

\switchcolumn

\begin{greek}[variant=ancient]%
ἀπαντῆσαι αὐτοῖς βούλεται.

\end{greek}%
\switchcolumn*

Er ist ihr begegnet.

\switchcolumn

\begin{greek}[variant=ancient]%
ξυνήντησεν αὐτῇ.

\end{greek}%
\switchcolumn*

Wo wollen wir uns treffen?

\switchcolumn

\begin{greek}[variant=ancient]%
ποῖ ἀπαντησόμεθα;

\end{greek}%
\switchcolumn*

Hier.

\switchcolumn

\begin{greek}[variant=ancient]%
ἐνθάδε.

\end{greek}%
\switchcolumn*[


\section{Bier her!}

]\indent Kellner! Kellner!

\switchcolumn

\begin{greek}[variant=ancient]%
παῖ! παῖ!

\end{greek}%
\switchcolumn*

Wo steckt denn die Bedienung?

\switchcolumn

\begin{greek}[variant=ancient]%
οὐ περιδραμεῖταί τις δεῦρο τῶν παίδων;

\end{greek}%
\switchcolumn*

Sie da, Kellner, wohin laufen Sie? --- Nach Gläsern.

\switchcolumn

\begin{greek}[variant=ancient]%
οὗτος σὺ, παῖ, ποῖ θεῖς; --- Ἐπ' ἐκπώματα.

\end{greek}%
\switchcolumn*

Kommen Sie hierher!

\switchcolumn

\begin{greek}[variant=ancient]%
ἐλθὲ δεῦρο!

\end{greek}%
\switchcolumn*

Bringen Sie mir einmal schnell Bier und Hasenbraten!

\switchcolumn

\begin{greek}[variant=ancient]%
ἔνεγκέ μοι ταχέως ζῦθον καὶ λαγῷα.

\end{greek}%
\switchcolumn*

Ganz wohl, mein Herr!

\switchcolumn

\begin{greek}[variant=ancient]%
\emph{ταῦτα,} ὦ δέσποτα.

\end{greek}%
\switchcolumn*

So, da bringe ich Alles.

\switchcolumn

\begin{greek}[variant=ancient]%
ἰδού, ἅπαντ' ἐγὼ φέρω.

\end{greek}%
\switchcolumn*

Das Bier schmeckt gut!

\switchcolumn

\begin{greek}[variant=ancient]%
ὡς ἡδὺς ὁ ζῦθος!

\end{greek}%
\switchcolumn*

Es schmeckt mir nicht.

\switchcolumn

\begin{greek}[variant=ancient]%
οὐκ ἀρέσκει με.

\end{greek}%
\switchcolumn*

Das Bier \emph{schmeckt} sehr stark nach Pech.

\switchcolumn

\begin{greek}[variant=ancient]%
\emph{ὄζει} πίττης ὁ ζῦθος ὀξύτατον.!

\end{greek}%
\switchcolumn*

Bier her, Kellner! --- Schleunigst!

\switchcolumn

\begin{greek}[variant=ancient]%
πέρε σὺ ζῦθον ὁ παῖς! --- πάσῃ τέχνῃ!

\end{greek}%
\switchcolumn*

So beeilen Sie sich doch!

\switchcolumn

\begin{greek}[variant=ancient]%
οὐ θᾶττον ἐγκονήσεις;

\end{greek}%
\switchcolumn*

Sie sorgen schlecht für uns.

\switchcolumn

\begin{greek}[variant=ancient]%
κακῶς ἐπιμελεῖ ἡμῶν!

\end{greek}%
\switchcolumn*

Kellner, schenken Sie mir noch einmal ein!

\switchcolumn

\begin{greek}[variant=ancient]%
παῖ, ἕτερον ἔγχεον!

\end{greek}%
\switchcolumn*

Schenken Sie mir auch ein!

\switchcolumn

\begin{greek}[variant=ancient]%
ἔγχει κἀμοί!

\end{greek}%
\switchcolumn*

Heute Abend wollen wir nach langer Zeit wieder einmal gehörig zechen.

\switchcolumn

\begin{greek}[variant=ancient]%
εἰς ἑσπέραν μεθυσθῶμεν διὰ χρόνου.

\end{greek}%
\switchcolumn*

Das Kneipen taugt nichts.

\switchcolumn

\begin{greek}[variant=ancient]%
κακὸν τὸ πίνειν!

\end{greek}%
\switchcolumn*

Man bekommt Katzenjammer von dem Bier.

\switchcolumn

\begin{greek}[variant=ancient]%
κραιπάλη γίγνεται ἀπὸ τοῖ ζύθου.

\end{greek}%
\switchcolumn*

Ich will Bier \emph{holen}.

\switchcolumn

\begin{greek}[variant=ancient]%
ἐπὶ ζῦθον εἶμι.

\end{greek}%
\switchcolumn*

Ich werde Sie nöthigenfalls rufen.

\switchcolumn

\begin{greek}[variant=ancient]%
καλέσω σε, εἴ τι δέοι.

\end{greek}%
\switchcolumn*

Ich gehe und hole mir noch eins.

\switchcolumn

\begin{greek}[variant=ancient]%
ἕτερον ἰὼν κομιοῦμαι.

\end{greek}%
\switchcolumn*

Hier haben Sie es!

\switchcolumn

\begin{greek}[variant=ancient]%
ἰδού, τουτὶ λαβέ.

\end{greek}%
\switchcolumn*

Schön. Sie sollen ein Trinkgeld von mir bekommen.

\switchcolumn

\begin{greek}[variant=ancient]%
καλῶς. εὐεργετήσω σε.

\end{greek}%
\switchcolumn*

Ich bin nicht im Stande hier zu bleiben.

\switchcolumn

\begin{greek}[variant=ancient]%
οὐχ οἷός τ' εἰμὶ ἐυθάδε μένειν.

\end{greek}%
\switchcolumn*

Der Rauch beißt mich in die Augen.

\switchcolumn

\begin{greek}[variant=ancient]%
ὁ καπνὸς δάκνει τὰ βλέφαρά μου.

\end{greek}%
\switchcolumn*

Komm', geh' mit!

\switchcolumn

\begin{greek}[variant=ancient]%
ἕπου μετ' ἐμοῦ.

\end{greek}%
\switchcolumn*

Der Rauch vertreibt mich.

\switchcolumn

\begin{greek}[variant=ancient]%
ὁ καπνός μ' ἐκπέμπει.

\end{greek}%
\switchcolumn*

Kellner, rechnen Sie einmal die Zeche zusammen!

\switchcolumn

\begin{greek}[variant=ancient]%
παῖ, λόγισαι ταῦτα.

\end{greek}%
\switchcolumn*

Sie hatten 6 Bier, Hasenbraten, Brot, macht 2½ Mark.

\switchcolumn

\begin{greek}[variant=ancient]%
εἴχετε ζύθου ἕξ (ποτήρια) καὶ λαγῷα καὶ ἄρτον· γίγνονται οὖν ἡμῖν
δύο μάρκαι{*} καὶ ἡμίσεια.

\end{greek}%
\switchcolumn*

Hier haben Sie!

\switchcolumn

\begin{greek}[variant=ancient]%
ἰδού, λαβέ.

\end{greek}%
\switchcolumn*

Ich taumele beim Gehen.

\switchcolumn

\begin{greek}[variant=ancient]%
σφαλλόμενος ἔρχομαι.

\end{greek}%
\switchcolumn*[


\section{Mich hungert}

]\indent Ich bekomme Hunger.

\switchcolumn

\begin{greek}[variant=ancient]%
λιμός με λαμβάνει.

\end{greek}%
\switchcolumn*

Ich habe nichts zu essen.

\switchcolumn

\begin{greek}[variant=ancient]%
οὐκ ἔχω καταφαγεῖν.

\end{greek}%
\switchcolumn*

Er hat einen Bärenhunger.

\switchcolumn

\begin{greek}[variant=ancient]%
βουλιμιᾷ.

\end{greek}%
\switchcolumn*

Ich komme vor Hunger um..

\switchcolumn

\begin{greek}[variant=ancient]%
ἀπόλωλα ὑπὸ λιμοῦ.

\end{greek}%
\switchcolumn*

Soll ich Ihnen etwas zu essen (zu trinken) geben?

\switchcolumn

\begin{greek}[variant=ancient]%
φέρε τί σοι δῶ \emph{φαγεῖν;} (πιεῖν;)

\end{greek}%
\switchcolumn*

Geben Sie mir etwas zu essen!

\switchcolumn

\begin{greek}[variant=ancient]%
δός μοι φαγεῖν!

\end{greek}%
\switchcolumn*

Ich will zu Tische gehen.

\switchcolumn

\begin{greek}[variant=ancient]%
βαδιοῦμαι ἐπὶ δεῖπνον.

\end{greek}%
\switchcolumn*

Sie haben noch nicht zu Mittag gegessen?

\switchcolumn

\begin{greek}[variant=ancient]%
οὔπω δεδείπνηκας;

\end{greek}%
\switchcolumn*

Nein!

\switchcolumn

\begin{greek}[variant=ancient]%
μὰ Δί' ἐγὼ μὲν οὔ.

\end{greek}%
\switchcolumn*

Ich muß fort zu Tische.

\switchcolumn

\begin{greek}[variant=ancient]%
δεῖ με χωρεῖν ἐπὶ δεῖπνον.

\end{greek}%
\switchcolumn*

Nun, so gehen Sie schnell zum Essen!

\switchcolumn

\begin{greek}[variant=ancient]%
ἀλλ' ἐπὶ δεῖπνον ταχὺ βάδιζε!

\end{greek}%
\switchcolumn*

Er kommt zu Tische.

\switchcolumn

\begin{greek}[variant=ancient]%
ἐπὶ δεῖπνον ἔρχεται.

\end{greek}%
\switchcolumn*

Der Tisch ist gedeckt.

\switchcolumn

\begin{greek}[variant=ancient]%
τὸ δεῖπνόν ἐστ' ἐπεσκευασμένον.!

\end{greek}%
\switchcolumn*

Die Tasse.

\switchcolumn

\begin{greek}[variant=ancient]%
τὸ κύπελλον.

\end{greek}%
\switchcolumn*

Der Teller.

\switchcolumn

\begin{greek}[variant=ancient]%
τὸ λεκάνιον.

\end{greek}%
\switchcolumn*

Die Schüssel.

\switchcolumn

\begin{greek}[variant=ancient]%
τὸ τρυβλίον.

\end{greek}%
\switchcolumn*

Das Messer.

\switchcolumn

\begin{greek}[variant=ancient]%
τὸ μαχαίριον.

\end{greek}%
\switchcolumn*

Die Gabel.

\switchcolumn

\begin{greek}[variant=ancient]%
τὸ πειρούνιον.{*}

\end{greek}%
\switchcolumn*

Die Serviette.

\switchcolumn

\begin{greek}[variant=ancient]%
τὸ χειρόμακτρον.

\end{greek}%
\switchcolumn*[


\section{Mahlzeit}

]\indent Ich lade dich zum Frühstück ein.

\switchcolumn

\begin{greek}[variant=ancient]%
ἐπ' ἄριστον καλῶ σε.

\end{greek}%
\switchcolumn*

Er hat mich zum Frühstück geladen.

\switchcolumn

\begin{greek}[variant=ancient]%
ἐπ' ἄριστον μ' ἐκάλεσεν.!

\end{greek}%
\switchcolumn*

Wir werden gut essen und trinken.

\switchcolumn

\begin{greek}[variant=ancient]%
εὐωχησόμεθα ἡμεῖς γε.

\end{greek}%
\switchcolumn*

Ich rechnete darauf, daß Sie kommen mürden.

\switchcolumn

\begin{greek}[variant=ancient]%
ἐλογιζόμην\footnote{\begin{latin}%
\textgreek[variant=ancient]{τῷ τυπογράφῳ ἄσκοπος τὸ γράμμα «ό» ἦν.}\end{latin}%
} ἐγώ σε παρέσεσθαι.

\end{greek}%
\switchcolumn*

Er frühstückt.

\switchcolumn

\begin{greek}[variant=ancient]%
ἀριστᾷ.

\end{greek}%
\switchcolumn*

\myafterpagetrue\mysetaligntext{german}{Es giebt{ }}\mysetalign{german}Braten. 

\switchcolumn

\begin{greek}[variant=ancient]%
\mysetaligntext{greek}{πάρεστι{ }}\mysetalign*{greek}κρέα ὠπτημένα.

\end{greek}%
\switchcolumn*\bgroup\mysetalign{german} Kalbs\textcompwordmark{}braten.

\egroup\switchcolumn\bgroup

\begin{greek}[variant=ancient]%
\mysetalign*{greek}(κρέα) μόσχεια.\mysetaligntext{greek}{πάρεστι
(κρέα){ }}

\end{greek}%
\egroup\switchcolumn*\bgroup

\mysetalign{german}Kinderbraten. 

\egroup\switchcolumn\bgroup

\begin{greek}[variant=ancient]%
\mysetalign*{greek}βόεια.

\end{greek}%
\egroup\switchcolumn*\bgroup

\mysetalign{german}Schweinebraten. 

\egroup\switchcolumn\bgroup

\begin{greek}[variant=ancient]%
\mysetalign*{greek}χοίρεια.

\end{greek}%
\egroup\switchcolumn*\bgroup

\mysetalign{german}Hammelbraten. 

\egroup\switchcolumn\bgroup

\begin{greek}[variant=ancient]%
\mysetalign*{greek}ἄρνεια.

\end{greek}%
\egroup\myafterpagefalse\switchcolumn*\bgroup

\mysetalign{german}Ziegenbraten. 

\egroup\switchcolumn\bgroup

\begin{greek}[variant=ancient]%
\mysetalign*{greek}ἐρίφεια.

\end{greek}%
\egroup\myafterpagefalse\switchcolumn*\bgroup

\mysetalign{german}Keule, Schinken. 

\egroup\switchcolumn\bgroup

\begin{greek}[variant=ancient]%
\mysetalign*{greek}κωλῆ.

\end{greek}%
\egroup\myafterpagefalse\switchcolumn*\bgroup

\mysetalign{german}Hasenbraten. 

\egroup\switchcolumn\bgroup

\begin{greek}[variant=ancient]%
\mysetalign*{greek}λαγῷα.

\end{greek}%
\egroup\myafterpagefalse\switchcolumn*\bgroup

\mysetalign{german}Geflügel. 

\egroup\switchcolumn\bgroup

\begin{greek}[variant=ancient]%
\mysetalign*{greek}ὀρνίθεια.

\end{greek}%
\egroup\myafterpagefalse\switchcolumn*\bgroup

\mysetalign{german}Aal. 

\egroup\switchcolumn\bgroup

\begin{greek}[variant=ancient]%
\mysetalign*{greek}ἐγχέλεια.

\end{greek}%
\egroup\switchcolumn*

Aal habe ich nicht gern; lieber äße ich Geflügel.

\switchcolumn

\begin{greek}[variant=ancient]%
οὐ χαίρω ἐγχέλεσιν, ἀλλ' \emph{ἥδιον}\footnote{\begin{latin}%
\textgreek[variant=ancient]{ὁ τυπογράφος ἔγραψα τὸν οὐ γεγραμμένον
ἦχον καὶ τόνον.}\end{latin}%
} ἂν φάγοιμι ὀρνίθεια.

\end{greek}%
\switchcolumn*

Das esse ich am liebsten.

\switchcolumn

\begin{greek}[variant=ancient]%
ταῦτα γὰρ \emph{ἥδιστ'} ἐσθίω.

\end{greek}%
\switchcolumn*

Das habe ich gestern gegessen.

\switchcolumn

\begin{greek}[variant=ancient]%
τοῦτο χθὲς ἔφαγον.

\end{greek}%
\switchcolumn*

Bringen Sie Krammets\textcompwordmark{}vögel für mich her!

\switchcolumn

\begin{greek}[variant=ancient]%
φέρε δεῦρο κίχλας ἐμοί!

\end{greek}%
\switchcolumn*

Kosten Sie einmal davon!

\switchcolumn

\begin{greek}[variant=ancient]%
γεῦσαι λαβών!

\end{greek}%
\switchcolumn*

Essen Sie einmal dies!

\switchcolumn

\begin{greek}[variant=ancient]%
φάγε τουτί!

\end{greek}%
\switchcolumn*

Nein, das bekommt mir gar nicht gut.

\switchcolumn

\begin{greek}[variant=ancient]%
μὰ τὸν Δία, οὐ γὰρ οὐδαμῶς μοι ξύμφορον.

\end{greek}%
\switchcolumn*

Knus\textcompwordmark{}pern Sie einmal dies!

\switchcolumn

\begin{greek}[variant=ancient]%
ἔντραγε τουτί!

\end{greek}%
\switchcolumn*

Genöthigt wird principiell nicht.

\switchcolumn

\begin{greek}[variant=ancient]%
οὐ προσαναγκάζομεν οὐδαμῶς.

\end{greek}%
\switchcolumn*

Das Fleisch schmeckt sehr gut.

\switchcolumn

\begin{greek}[variant=ancient]%
τὰ κρέα ἥδιστά ἐστιν.

\end{greek}%
\switchcolumn*

Das schmeckt gut.

\switchcolumn

\begin{greek}[variant=ancient]%
ὡς ἡδύ!

\end{greek}%
\switchcolumn*

Die Sause schmeckt sehr gut.

\switchcolumn

\begin{greek}[variant=ancient]%
ὡς ἡδὺ τὸ κατάχυσμα!

\end{greek}%
\switchcolumn*

Eins vermisse ich noch.

\switchcolumn

\begin{greek}[variant=ancient]%
ἕν ἔτι ποθῶ.

\end{greek}%
\switchcolumn*

Geben Sie mir doch ein Stück Brot!

\switchcolumn

\begin{greek}[variant=ancient]%
δός μοι δῆτα ὀλίγον τι ἄρτου!

\end{greek}%
\switchcolumn*

Und ein Stück Wurst\\
und Erbsenbrei.

\switchcolumn

\begin{greek}[variant=ancient]%
καὶ χορδῆς τι\\
καὶ ἔτνος\footnote{\begin{latin}%
\textgreek[variant=ancient]{ὁ τυπογράφος ἔγραψα τὸν οὐ γεγραμμένον
ἦχον καὶ τόνον.}\end{latin}%
} πίσινον.

\end{greek}%
\switchcolumn*

Der Nachtisch.

\switchcolumn

\begin{greek}[variant=ancient]%
τὸ ἐπίδειπνον.

\end{greek}%
\switchcolumn*

Was wollen wir zum Dessert essen?

\switchcolumn

\begin{greek}[variant=ancient]%
τί ἐπιδειπνήσομεν;

\end{greek}%
\switchcolumn*

Bringen Sie noch etwas Weißbrot mit Schweizerkäse!

\switchcolumn

\begin{greek}[variant=ancient]%
παράθες ἔτι ὀλίγον τι ἄρτου πυρίνου μετὰ τυροῦ ἑλβητικοῦ!

\end{greek}%
\switchcolumn*

Es wird Kuchen gebacken.

\switchcolumn

\begin{greek}[variant=ancient]%
πόπανα πέττεται.

\end{greek}%
\switchcolumn*

Da haben Sie auch ein Stück Speckkuchen.

\switchcolumn

\begin{greek}[variant=ancient]%
λαβὲ καὶ πλακοῦντος πίονος τόμον.

\end{greek}%
\switchcolumn*

Ich danke bestens! (Nein!)

\switchcolumn

\begin{greek}[variant=ancient]%
κάλλιστα· ἐπαινῶ.

\end{greek}%
\switchcolumn*

Auch ich habe genug.

\switchcolumn

\begin{greek}[variant=ancient]%
κἀμοί γ' ἅλις.

\end{greek}%
\switchcolumn*

Bringen Sie Wein! (Weiß-, Roth-.)

\switchcolumn

\begin{greek}[variant=ancient]%
φέρ' οἶνον (λευκόν, ἐρυθρόν).

\end{greek}%
\switchcolumn*

Der Wein hat Bouquet.

\switchcolumn

\begin{greek}[variant=ancient]%
ὀσμὴν ἔχει ὁ οἶνος ὁδί.

\end{greek}%
\switchcolumn*

Ich trinke diesen Wein hier gern.

\switchcolumn

\begin{greek}[variant=ancient]%
ἡδέως\footnote{\begin{latin}%
\textgreek[variant=ancient]{τῷ τυπογράφῳ ἄσκοπος τὸ γράμμα «ἡ» ἦν.}\end{latin}%
} πίνω τὸν οἶνον τονδί.

\end{greek}%
\switchcolumn*

Es ist noch Wein übrig geblieben.

\switchcolumn

\begin{greek}[variant=ancient]%
οἶνός ἐστι περιλελειμμένος.

\end{greek}%
\switchcolumn*

Wie viel etwa?

\switchcolumn

\begin{greek}[variant=ancient]%
πόσον τι;

\end{greek}%
\switchcolumn*

Über die Hälfte.

\switchcolumn

\begin{greek}[variant=ancient]%
ὑπὲρ ἥμισυ.

\end{greek}%
\switchcolumn*

Was soll ich damit machen?

\switchcolumn

\begin{greek}[variant=ancient]%
τί χρήσομαι τούτῳ;

\end{greek}%

	\switchcolumn*[


\part{In der Schule.}


\section{In die Schule!}

]\indent Es ist Zeit zu gehen!

\switchcolumn

\begin{greek}[variant=ancient]%
ὥρα προβαίνειν σοί ἐστιν.

\end{greek}%
\switchcolumn*

Es ist Zeit in's Gymnasium zu gehen!

\switchcolumn

\begin{greek}[variant=ancient]%
ὥρα ἐστὶν εἰς τὸ γυμνάσιον βαδίζειν.

\end{greek}%
\switchcolumn*

So mach' doch, daß du in's Gymnasium kommst!

\switchcolumn

\begin{greek}[variant=ancient]%
οὐκ ἂν φθάνοις εἰς τὸ γυνμάσιον ἰών;

\end{greek}%
\switchcolumn*

Halt dich nicht auf! --- Beeile dich!

\switchcolumn

\begin{greek}[variant=ancient]%
μή νυν διάτριβε! --- σπεῦδέ νυν!

\end{greek}%
\switchcolumn*

Du hast keine Zeit mehr zu verlieren.

\switchcolumn

\begin{greek}[variant=ancient]%
ὁ καιρός ἐστι μηκέτι μέλλειν.

\end{greek}%
\switchcolumn*

Mach' dir keine Sorge!

\switchcolumn

\begin{greek}[variant=ancient]%
μὴ φροντίσῃς.

\end{greek}%
\switchcolumn*

Nur nicht ängstlich!

\switchcolumn

\begin{greek}[variant=ancient]%
μηδὲν δείσῃς.

\end{greek}%
\switchcolumn*

Sei unbesorgt!

\switchcolumn

\begin{greek}[variant=ancient]%
μηδὲν φοβηθῆς.

\end{greek}%
\switchcolumn*[


\section{Zu spät gekommen!}

]Wir wollen beten!

\switchcolumn

\begin{greek}[variant=ancient]%
ἀλλ' εὐχώμεθα!

\end{greek}%
\switchcolumn*

Ich bin \emph{doch nicht etwa} zu spät gekommen?

\switchcolumn

\begin{greek}[variant=ancient]%
\emph{μῶν} ὕστερος πάρειμι;

\end{greek}%
\switchcolumn*

Ich bin zu spät gekommen.

\switchcolumn

\begin{greek}[variant=ancient]%
ὕστερος ἦλθον!

\end{greek}%
\switchcolumn*

Hilf Himmel! --- Ach, ich Ärmster!

\switchcolumn

\begin{greek}[variant=ancient]%
Ἄπολλον ἀποτρόπαιε! --- οἴμοι κακοδαίμων!

\end{greek}%
\switchcolumn*

Ich Unglücks\textcompwordmark{}wurm!

\switchcolumn

\begin{greek}[variant=ancient]%
κακοδαίμων ἐγώ!

\end{greek}%
\switchcolumn*

Verwünscht!

\switchcolumn

\begin{greek}[variant=ancient]%
οἴμοι τάλας!

\end{greek}%
\switchcolumn*

Wo kommen Sie denn nur her?

\switchcolumn

\begin{greek}[variant=ancient]%
πόθεν ἥκεις ἐτεόν;

\end{greek}%
\switchcolumn*

Sie sind wieder zu spät gekommen!

\switchcolumn

\begin{greek}[variant=ancient]%
ὕστερον αὖθις ἦλθες!

\end{greek}%
\switchcolumn*

Wes\textcompwordmark{}halb sind Sie jetzt erst gekommen?

\switchcolumn

\begin{greek}[variant=ancient]%
τοῦ ἕνεκα τηνικάδε ἀφίκου;

\end{greek}%
\switchcolumn*

Es hat noch nicht acht geschlagen.

\switchcolumn

\begin{greek}[variant=ancient]%
οὐ γάρ πω ἐσήμηνε τὴν ὀγδόην.

\end{greek}%
\switchcolumn*

Sie sind erst nach dem Läuten gekommen!

\switchcolumn

\begin{greek}[variant=ancient]%
ὕστερος σὺ ἦλθες τοῦ σημείου.

\end{greek}%
\switchcolumn*

Seien Sie nicht böse; meine Uhr geht falsch.

\switchcolumn

\begin{greek}[variant=ancient]%
μὴ ἀγανάκτει· τὸ γὰρ ὡρολόγιόν μου \emph{οὐκ ὀρθῶς χωρεῖ.}

\end{greek}%
\switchcolumn*

Wirklich? Zeigen Sie einmal!

\switchcolumn

\begin{greek}[variant=ancient]%
ἄληθες; ἀλλὰ δεῖξον! (\textgerman[spelling=old,babelshorthands=true]{nicht:}
ἀληθές;)

\end{greek}%
\switchcolumn*

Setzen Sie sich!

\switchcolumn

\begin{greek}[variant=ancient]%
κάθιζε!

\end{greek}%
\switchcolumn*[


\section{Schriftliche Arbeiten}

]Wollen einmal sehen, was Sie geschrieben haben!

\switchcolumn

\begin{greek}[variant=ancient]%
φέρ' ἴδω, τί ουν ἔγραψας.

\end{greek}%
\switchcolumn*

Hier ist es.

\switchcolumn

\begin{greek}[variant=ancient]%
\emph{ἰδού.}

\end{greek}%
\switchcolumn*

Wovon handelt der Aufsatz?

\switchcolumn

\begin{greek}[variant=ancient]%
ἐστὶ δὲ περὶ τοῦ τὰ γεγραμμένα;

\end{greek}%
\switchcolumn*

Geben Sie das Heft her, damit ich es lesen kann.

\switchcolumn

\begin{greek}[variant=ancient]%
\emph{φέρε} τὸ βιβλίον, ἵν' ἀναγνῶ.

\end{greek}%
\switchcolumn*

Wollen einmal sehen, was darin steht!

\switchcolumn

\begin{greek}[variant=ancient]%
φέρ' ἴδω, τί ἔνεστιν.

\end{greek}%
\switchcolumn*

Haben Sie einen Bleistift?

\switchcolumn

\begin{greek}[variant=ancient]%
ἔχεις κυκλομόλυβδον;

\end{greek}%
\switchcolumn*

Das R hier ist miserabel.

\switchcolumn

\begin{greek}[variant=ancient]%
τὸ ῥῶ τουτὶ μοχθηρόν.

\end{greek}%
\switchcolumn*

Was ist denn das eigentlich für ein Buchstabe?

\switchcolumn

\begin{greek}[variant=ancient]%
τουτὶ τί \emph{ποτ'} ἐστὶ γράμμα;

\end{greek}%
\switchcolumn*

Sie geben sich keine Mühe!

\switchcolumn

\begin{greek}[variant=ancient]%
οὐκ ἐπιμελὴς εἶ.

\end{greek}%
\switchcolumn*

Haben Sie das allein gemacht (verfaßt)?

\switchcolumn

\begin{greek}[variant=ancient]%
αὐτὸς δὺ ταῦτα ἔγραφες;

\end{greek}%
\switchcolumn*

Verfaßt ist es von mir, aber von meinem Vater corrigirt.

\switchcolumn

\begin{greek}[variant=ancient]%
συντέταχθαι μὲν ταῦτα ὑπ' ἐμοῦ, διώρθωται δὲ ὑπὸ τοῦ πατρός.

\end{greek}%
\switchcolumn*

Haben Sie alles berührt und nichts übergangen?

\switchcolumn

\begin{greek}[variant=ancient]%
ἦ πάντα ἐπελήλυθας κοὐδὲν παρῆλθες;

\end{greek}%
\switchcolumn*

Ich glaube wenigstens.

\switchcolumn

\begin{greek}[variant=ancient]%
δοκεῖ γοῦν μοι.

\end{greek}%
\switchcolumn*

Das steht nicht darin.

\switchcolumn

\begin{greek}[variant=ancient]%
οὐκ ἔνεστι τοῦτο.

\end{greek}%
\switchcolumn*

Ich habe die Nacht nicht geschlafen, sondern bis zum Morgen an meiner
Rede gearbeitet.

\switchcolumn

\begin{greek}[variant=ancient]%
οὐκ ἐκάθευδον τὴν νύκτα ἀλλὰ\footnote{\begin{latin}%
\textgreek[variant=ancient]{ὁ τυπογράφος ἔγραψα τὸν οὐ γεγραμμένον
τόνον.}\end{latin}%
} διεπονούμην πρὸς φῶς περὶ τὸν λόγον.

\end{greek}%
\switchcolumn*

Ich weiß schon, wie Sie es machen. 

\switchcolumn

\begin{greek}[variant=ancient]%
τούς τρόπους σου ἐπίσταμαι.

\end{greek}%
\switchcolumn*

Hier haben Sie zweimal dasselbe gesagt! 

\switchcolumn

\begin{greek}[variant=ancient]%
ἐνταῦθα δὶς ταὐτὸν εἶπες!

\end{greek}%
\switchcolumn*

Gleich von vornherein haben Sie einen kolossalen Bock gemacht. 

\switchcolumn

\begin{greek}[variant=ancient]%
\emph{εὐθὺς} ἡμάρτηκας θαυμασίως ὡς.

\end{greek}%
\switchcolumn*

Ihre Arbeit enthält 20 Fehler. 

\switchcolumn

\begin{greek}[variant=ancient]%
ἔχει τὸ σὸν εἴκοσιν ἁμαρτίας.

\end{greek}%
\switchcolumn*

Sie wissen von vielen Dingen nichts. 

\switchcolumn

\begin{greek}[variant=ancient]%
πολλά σε λανθάνει.

\end{greek}%
\switchcolumn*[


\section{Grammatisches}

]Weiter nun!

\switchcolumn

\begin{greek}[variant=ancient]%
ἴθι νυν.

\end{greek}%
\switchcolumn*

Ich will Sie einmal examiniren, wie es mit Ihnen im Griechischen steht. 

\switchcolumn

\begin{greek}[variant=ancient]%
βούλομαι λαβεῖν σου πεῖραν, ὅπως ἔχεις περὶ τῶν Ἑλληνικῶν.

\end{greek}%
\switchcolumn*

\emph{Wie} heißt der Genitiv von diesem Wort? 

\switchcolumn

\begin{greek}[variant=ancient]%
ποία \emph{ἐστὶν} ἡ γενικὴ ταύτης τῆς λέξεως;

\end{greek}%
\switchcolumn*

Der Nominativ, Dativ, Accusativ, Vocativ? 

\switchcolumn

\begin{greek}[variant=ancient]%
ἡ ὀνομαστική, δοτική, αἰτιατική, κλητική;

\end{greek}%
\switchcolumn*

Falsch! 

\switchcolumn

\begin{greek}[variant=ancient]%
μὴ δῆτα!

\end{greek}%
\switchcolumn*

Der Genitiv von diesem Worte ist ungebräuchlich. 

\switchcolumn

\begin{greek}[variant=ancient]%
ἡ γενικὴ τῆς λέξεως ταύτης ἄχρηστός ἐστιν.

\end{greek}%
\switchcolumn*

Ganz richtig! 

\switchcolumn

\begin{greek}[variant=ancient]%
ὀρθῶς γε!

\end{greek}%
\switchcolumn*

Wie heißt der Indicativ des Präsens von diesem Verb? 

\switchcolumn

\begin{greek}[variant=ancient]%
ποῖός ἐστιν ὁ ἐνεστὼς (χρόνος) τῆς ὁριστικῆς τοῦ ῥήματος τούτου;!

\end{greek}%
\switchcolumn*

Das will ich mir notiren. 

\switchcolumn

\begin{greek}[variant=ancient]%
μνημόσυνα ταῦτα γράψομαι.

\end{greek}%
\switchcolumn*

Ich schreibe mir das auf. 

\switchcolumn

\begin{greek}[variant=ancient]%
γράφομαι τοῦτο.

\end{greek}%
\switchcolumn*

Der Conjunctiv, Optativ, Imperativ. 

\switchcolumn

\begin{greek}[variant=ancient]%
ἡ ὑποτακτική, εὐκτική, προστακτική.

\end{greek}%
\switchcolumn*

Der Infinitiv, das Particip. 

\switchcolumn

\begin{greek}[variant=ancient]%
ἡ ἀπαρέμφατος, ἡ μετοχή.

\end{greek}%
\switchcolumn*

Das Imperfect, Perfect. 

\switchcolumn

\begin{greek}[variant=ancient]%
ὁ παρατατικός, ὁ παρακείμενος.

\end{greek}%
\switchcolumn*

Plus\textcompwordmark{}quamperfect, Aorist. 

\switchcolumn

\begin{greek}[variant=ancient]%
ὁ ὑπερσυντελικός, ἀόριστος.

\end{greek}%
\switchcolumn*

Futurum. (Erstes, zweites.) 

\switchcolumn

\begin{greek}[variant=ancient]%
ὁ μέλλων. (πρῶτος, δεύτερος.)

\end{greek}%
\switchcolumn*

Das Activ, Passiv. 

\switchcolumn

\begin{greek}[variant=ancient]%
τὸ ἐνεργητικόν, παθητικόν.

\end{greek}%
\switchcolumn*

Sie betonen falsch. 

\switchcolumn

\begin{greek}[variant=ancient]%
οὐκ ὀρθῶς τονοῖς.

\end{greek}%
\switchcolumn*

Der Accent (Acut, Gravis, Circumflex). 

\switchcolumn

\begin{greek}[variant=ancient]%
ἡ κεραία (ἡ ὀξεῖα, βαρεῖα, περισπωμένη).

\end{greek}%
\switchcolumn*

Der Artikel muß stehen. 

\switchcolumn

\begin{greek}[variant=ancient]%
δεῖ τοῦ ἄρθρου.

\end{greek}%
\switchcolumn*[


\section{Verkehrte Antworten}

]Geben Sie Acht! 

\switchcolumn

\begin{greek}[variant=ancient]%
πρόσεχε τὸν νοῦν!

\end{greek}%
\switchcolumn*

Beantworten sie mir, was ich fragen werde. 

\switchcolumn

\begin{greek}[variant=ancient]%
ἀπόκριναι, ἅττ' ἄν ἔρωμαι.

\end{greek}%
\switchcolumn*

Antworten Sie bestimmt! 

\switchcolumn

\begin{greek}[variant=ancient]%
ἀπόκριναι σαφῶς!

\end{greek}%
\switchcolumn*

Reden Sie laut. 

\switchcolumn

\begin{greek}[variant=ancient]%
λέξον \emph{μέγα.}

\end{greek}%
\switchcolumn*

Versuchen Sie etwas recht Scharfsinniges u. Gescheites zu sagen! 

\switchcolumn

\begin{greek}[variant=ancient]%
ἀποκινδύνευε λεπτόν τι καὶ σοφὸν λέγειν.

\end{greek}%
\switchcolumn*

Bitte, sprechen Sie weiter! 

\switchcolumn

\begin{greek}[variant=ancient]%
λέγοις ἂν ἄλλο.

\end{greek}%
\switchcolumn*

Fahren Sie fort! 

\switchcolumn

\begin{greek}[variant=ancient]%
λέγε, ὦ 'γαθέ!

\end{greek}%
\switchcolumn*

Nun, Sie scheinen nicht zu wissen, was Sie sagen sollen. 

\switchcolumn

\begin{greek}[variant=ancient]%
ἀλλ' οὐκ ἔχειν ἔοικας, ὅτι λέγῃς.

\end{greek}%
\switchcolumn*

Warum reden Sie nicht weiter? 

\switchcolumn

\begin{greek}[variant=ancient]%
τί σιωπᾷς;

\end{greek}%
\switchcolumn*

Sagen Sie mir, was Sie meinen! 

\switchcolumn

\begin{greek}[variant=ancient]%
εἰπέ μοι, ὅτι\footnote{\begin{latin}%
\textgreek[variant=ancient]{ὁ τυπογράφος ἔγραψα τὸν οὐ γεγραμμένον
ἦχον.}\end{latin}%
} λέγεις.

\end{greek}%
\switchcolumn*

Was reden Sie da für verkehrtes Zeug? 

\switchcolumn

\begin{greek}[variant=ancient]%
τί ταῦτα ληρεῖς;

\end{greek}%
\switchcolumn*

Sie schwatzen in's Blaue hinein! 

\switchcolumn

\begin{greek}[variant=ancient]%
ἄλλως φλυαρεῖς;

\end{greek}%
\switchcolumn*

Das ist was ganz Anderes! 

\switchcolumn

\begin{greek}[variant=ancient]%
οὐ ταὐτόν, ὦ 'τάν!

\end{greek}%
\switchcolumn*

Nicht darnach frage ich Sie! 

\switchcolumn

\begin{greek}[variant=ancient]%
οὐ τοῦτ' ἐρωτῶ σε.

\end{greek}%
\switchcolumn*

Doch (\textlatin{sc.} abbrechend) antworten Sie einmal auf meine Frage. 

\switchcolumn

\begin{greek}[variant=ancient]%
\emph{καὶ μὴν} ἐπερωτηθεὶς ἀπόκριναί μοι.

\end{greek}%
\switchcolumn*

Sie sprechen in Räthseln! 

\switchcolumn

\begin{greek}[variant=ancient]%
δι' αἰνιγμῶν λέγεις.

\end{greek}%
\switchcolumn*

Ist das Ihr Ernst oder scherzen Sie? 

\switchcolumn

\begin{greek}[variant=ancient]%
σπουδάζεις ταῦτα ἢ παίζεις;

\end{greek}%
\switchcolumn*

Unsinn! 

\switchcolumn

\begin{greek}[variant=ancient]%
οὐδὲν λέγεις!

\end{greek}%
\switchcolumn*

Machen Sie weiter kein Gerede! 

\switchcolumn

\begin{greek}[variant=ancient]%
μὴ λάλει!

\end{greek}%
\switchcolumn*

\vspace{0.5em}
Schweigen Sie! 

\switchcolumn

\begin{tabular}{ll}
\ldelim\{{2}{1em}[] & \begin{greek}[variant=ancient]%
σίγα!\end{greek}%
\tabularnewline
 & \begin{greek}[variant=ancient]%
σιώπα!\end{greek}%
\tabularnewline
\end{tabular}

\switchcolumn*

So schweigen Sie doch! 

\switchcolumn

\begin{greek}[variant=ancient]%
οὐ σιγήσει;

\end{greek}%
\switchcolumn*

O Sie Schwachkopf! 

\switchcolumn

\begin{greek}[variant=ancient]%
ὦ μῶρε σύ!

\end{greek}%
\switchcolumn*[


\section{Abbildungen}

]Ich will Ihnen eine Abbildung zeigen.

\switchcolumn

\begin{greek}[variant=ancient]%
εἰκόνα ὑμῖν ἐπιδείξω.

\end{greek}%
\switchcolumn*

Sehen Sie einmal hinunter! 

\switchcolumn

\begin{greek}[variant=ancient]%
βλέψατε κάτω!

\end{greek}%
\switchcolumn*

Sehen Sie hinauf! 

\switchcolumn

\begin{greek}[variant=ancient]%
βλέψατε ἄνω!

\end{greek}%
\switchcolumn*

Wo sehen Sie hin? 

\switchcolumn

\begin{greek}[variant=ancient]%
ποῖ βλέπεις;

\end{greek}%
\switchcolumn*

Sie sehen wo anders hin. 

\switchcolumn

\begin{greek}[variant=ancient]%
ἑτέρωσε βλέπεις.

\end{greek}%
\switchcolumn*

Sieh einmal hierher! 

\switchcolumn

\begin{greek}[variant=ancient]%
δεῦρο σκεψαι!

\end{greek}%
\switchcolumn*

\myafterpagetrue\mysetaligntext{german}{Ich höre ein Geräusch{ }}\mysetalign{german}dahinten. 

\switchcolumn

\begin{greek}[variant=ancient]%
καὶ μὴν αἰσθάνομαι ψόφου τινός ἐξόπισθεν.

\end{greek}%
\myafterpagefalse\switchcolumn*\bgroup

\mysetalign{german}da vorn. 

\egroup\switchcolumn

\begin{greek}[variant=ancient]%
ἐν τῷ πρόσθεν.

\end{greek}%
\switchcolumn*

Hören Sie auf zu schwatzen! 

\switchcolumn

\begin{greek}[variant=ancient]%
παῦσαι λαλῶν!

\end{greek}%
\switchcolumn*

So schwatzen Sie doch nicht! 

\switchcolumn

\begin{greek}[variant=ancient]%
οὐ μὴ λαλήσετε;

\end{greek}%
\switchcolumn*[


\section{Griechische Dichter}

]Sagen Sie mir nun die schönste Stelle aus der Antigone her! 

\switchcolumn

\begin{greek}[variant=ancient]%
ἐκ τῆς Ἀντιγόνης τὸ νῦν εἰπὲ τὴν καλλίστην ῥῆσιν ἀπολέγων.

\end{greek}%
\switchcolumn*

Den Anfang der Odyssee. 

\switchcolumn

\begin{greek}[variant=ancient]%
τὸ πρῶτον τῆς Ὀδυσσείας.

\end{greek}%
\switchcolumn*

Was bedeutet diese Stelle? 

\switchcolumn

\begin{greek}[variant=ancient]%
τί \emph{νοεῖ} τοῦτο;

\end{greek}%
\switchcolumn*

Sie sind nicht recht bei Troste! 

\switchcolumn

\begin{greek}[variant=ancient]%
κακοδαιμονᾷς.

\end{greek}%
\switchcolumn*

Wie naiv! 

\switchcolumn

\begin{greek}[variant=ancient]%
ὡς εὐηθικῶς!

\end{greek}%
\switchcolumn*

Wo haben Sie Ihren Verstand? 

\switchcolumn

\begin{greek}[variant=ancient]%
ποῦ τὸν νοῦν ἔχεις;

\end{greek}%
\switchcolumn*

Sie sind von Sinnen. 

\switchcolumn

\begin{greek}[variant=ancient]%
παραφρονεῖς!

\end{greek}%
\switchcolumn*

Diese Stelle hat Sophokles nicht so aufgefaßt, wie Sie sie auffassen.
Überlegen Sie es sich besser! 

\switchcolumn

\begin{greek}[variant=ancient]%
τὴν ῥῆσιν ταύτην οὐκ οὕτω Σοφοκλῆς ὑπελάμβανεν, ὡς σὺ ὑπολαμβάνεις.
ὅρα δὴ βέλτιον.

\end{greek}%
\switchcolumn*

Beachten Sie diesen Aus\textcompwordmark{}druck!\_

\switchcolumn

\begin{greek}[variant=ancient]%
σκόπει τὸ ῥῆμα τοῦτο!

\end{greek}%
\switchcolumn*

\textgreek[variant=ancient]{ἥκω} ist gleichbedeutend mit \textgreek[variant=ancient]{κατέρχομαι.} 

\switchcolumn

\begin{greek}[variant=ancient]%
ἥκω ταὐτόν ἐστι τῷ κατέρχομαι.

\end{greek}%
\switchcolumn*

Was soll das bedeuten? 

\switchcolumn

\begin{greek}[variant=ancient]%
τίς ὁ νοῦς.

\end{greek}%
\switchcolumn*

Jetzt sprechen sie vernünftig. 

\switchcolumn

\begin{greek}[variant=ancient]%
τουτὶ φρονίμως ἤδη λέγεις.

\end{greek}%
\switchcolumn*

Sie haben nunmehr den Sinn vollkommen inne. 

\switchcolumn

\begin{greek}[variant=ancient]%
πάντ' ἔχεις ἤδη.

\end{greek}%
\switchcolumn*

Sie haben gut combinirt. 

\switchcolumn

\begin{greek}[variant=ancient]%
εὖ γε ξυνέβαλες!

\end{greek}%
\switchcolumn*

Das ist ohne Zweifel \emph{das Schönste, was} Sophokles gedichtet
hat. 

\switchcolumn

\begin{greek}[variant=ancient]%
τοῦτο δήπου κάλλιστον πεποίηκε Σοφοκλῆς.

\end{greek}%
\switchcolumn*

Sophokles steht über Euripides. 

\switchcolumn

\begin{greek}[variant=ancient]%
Σοφοκλῆς πρότερός ἐστ' Εὐριπίδου.

\end{greek}%
\switchcolumn*

Doch ist dieser ebenfalls ein guter Dichter. 

\switchcolumn

\begin{greek}[variant=ancient]%
ὁ δ' ἀγαθὸς ποιητής ἐστι καὶ αὐτός.

\end{greek}%
\switchcolumn*

Ich bin kein Verehrer des Euripides. 

\switchcolumn

\begin{greek}[variant=ancient]%
οὐκ ἐπαινῶ Εὐριπίδην μὰ Δία.

\end{greek}%
\switchcolumn*

Fällt Ihnen nicht ein Vers des Euripides ein? 

\switchcolumn

\begin{greek}[variant=ancient]%
οὐκ ἀναμιμνήσκει ἴαμβον Εὐριπίδου;

\end{greek}%
\switchcolumn*

Das können sie ziemlich gut. 

\switchcolumn

\begin{greek}[variant=ancient]%
τουτὶ μὲν ἐπιεικῶς σύγ' ἐπίστασαι.

\end{greek}%
\switchcolumn*

Im Euripides sind Sie gut bewandert. 

\switchcolumn

\begin{greek}[variant=ancient]%
Εὐριπίδην πεπάτηκας ἀκριβῶς.

\end{greek}%
\switchcolumn*

Wo haben Sie das so gut gelernt? 

\switchcolumn

\begin{greek}[variant=ancient]%
πόθεν ταῦτ' ἔμαθες οὕτω καλῶς;

\end{greek}%
\switchcolumn*

Ich habe mir viele Stellen von Euripides abgeschrieben. 

\switchcolumn

\begin{greek}[variant=ancient]%
Εὐριπίδου ῥήσεις ἐξεγραψάμην πολλάς.

\end{greek}%
\switchcolumn*

Declamire mir ein Stück von einem neueren Dichter! 

\switchcolumn

\begin{greek}[variant=ancient]%
λέξον τι τῶν νεωτέρων.

\end{greek}%
\switchcolumn*

Sie verdienen es nicht, denn einen originellen Dichter wird man wohl
nicht mehr unter ihnen finden. 

\switchcolumn

\begin{greek}[variant=ancient]%
οὐκ ἔξιοί εἰσι τούτου, γόνιμον γὰρ ποιητὴν οὐκ ἂν ἔτι εὕροις ἐν αὐτοῖς.

\end{greek}%
\switchcolumn*

Welche Ansicht haben Sie über Äschylus? 

\switchcolumn

\begin{greek}[variant=ancient]%
περὶ Αἰσχύλου δὲ τίνα ἔχεις γνώμην;

\end{greek}%
\switchcolumn*

Den Äschylus stelle ich am höchsten unter den Dichtern. 

\switchcolumn

\begin{greek}[variant=ancient]%
Αἰσχύλον νομίζω πρῶτον ἐν ποιηταῖς.

\end{greek}%
\switchcolumn*

Kennen Sie dieses Lied von Simonides? 

\switchcolumn

\begin{greek}[variant=ancient]%
ἐπίστασαι τοῦτο τὸ ἆσμα Σιμωνίδου.

\end{greek}%
\switchcolumn*

Ja! 

\switchcolumn

\begin{greek}[variant=ancient]%
μάλιστα.

\end{greek}%
\switchcolumn*

Ja gewiß! 

\switchcolumn

\begin{greek}[variant=ancient]%
ἔγωγε νὴ Δία\textgerman[spelling=old,babelshorthands=true]{.}

\end{greek}%
\switchcolumn*

Soll ich es ganz hersagen? 

\switchcolumn

\begin{greek}[variant=ancient]%
βούλει πᾶν διεξέλθω;

\end{greek}%
\switchcolumn*

Ist nicht nöthig. 

\switchcolumn

\begin{greek}[variant=ancient]%
οὐδὲν δεῖ.

\end{greek}%
\switchcolumn*

Wie heißen diese Verse? (\textlatin{sc.} mit Namen) 

\switchcolumn

\begin{greek}[variant=ancient]%
ὄνομα δὲ τούτῳ τῷ μέτρῳ τί ἐστιν;

\end{greek}%
\switchcolumn*

Ich kann das Gedicht nicht. 

\switchcolumn

\begin{greek}[variant=ancient]%
τὸ ἆσμα οὐκ ἐπίσταμαι.

\end{greek}%
\switchcolumn*

Doch ich wende mich nun zu dem zweiten Act der Tragödie. 

\switchcolumn

\begin{greek}[variant=ancient]%
καὶ μὴν ἐπὶ τὸ δεύτερον τῆς τραγῳδίας\footnote{\begin{latin}%
\textgreek[variant=ancient]{ὁ τυπογράφος ἔγραψα τὸν οὐ γεγραμμένον
ἴοτα ὑπογραμμένον.}\end{latin}%
} μέρος τρέψομαι.

\end{greek}%
\switchcolumn*[


\section{Übersetzen}

]Suchen Sie in Ihrem Buche den Abschnitt über Sokrates auf! Es ist
Nr.\ 107.

\switchcolumn

\begin{greek}[variant=ancient]%
ζητεῖτε τὸ περὶ Σωκράτους λαβόντες τὸ βιβλίον. ἐστὶ δὲ τὸ ἐκατοστὸν
καὶ ἕβδομον. 

\end{greek}%
\switchcolumn*

Nun, so geben Sie Acht!.

\switchcolumn

\begin{greek}[variant=ancient]%
ἀλλὰ προσέχετε τὸν νοῦν.

\end{greek}%
\switchcolumn*

Wir wollen das (mündlich) in's Griechische übersetzen.

\switchcolumn

\begin{greek}[variant=ancient]%
λέγωμεν ἑλληνικῶς ταῦτα μεταβάλλοντες.

\end{greek}%
\switchcolumn*

Fangen Sie an, N.!

\switchcolumn

\begin{greek}[variant=ancient]%
ἴθι δή\footnote{\begin{latin}%
orig. \textgreek[variant=ancient]{δὴ}\end{latin}%
}, λέγε, ὦ Ν.

\end{greek}%
\switchcolumn*

Ich bin mit Ihrer Übersetzung zufrieden.

\switchcolumn

\begin{greek}[variant=ancient]%
ταῦτα μ' ἤρεσας λέγων.

\end{greek}%
\switchcolumn*

Von wem haben sie Griechisch gelernt?

\switchcolumn

\begin{greek}[variant=ancient]%
τίς σ' ἐδίδαξε τὴν ἑλληνικὴν φωνήν;

\end{greek}%
\switchcolumn*

Fahren Sie \emph{fort!}

\switchcolumn

\begin{greek}[variant=ancient]%
\emph{λέγε.}

\end{greek}%
\switchcolumn*

Das ist wieder ganz geschickt.

\switchcolumn

\begin{greek}[variant=ancient]%
τοῦτ' αὖ δεξιόν.

\end{greek}%
\switchcolumn*

Fahren \emph{Sie} fort!

\switchcolumn

\begin{greek}[variant=ancient]%
λέγε δὴ σύ, ὦ 'γαθέ.

\end{greek}%
\switchcolumn*

Sie übersetzen ungeschickt.

\switchcolumn

\begin{greek}[variant=ancient]%
σκαιῶς ταῦτα λέγεις.

\end{greek}%
\switchcolumn*

Das ist ein Jonisches Wort.

\switchcolumn

\begin{greek}[variant=ancient]%
τοῦτ' ἐστ' Ἰωνικὸν τὸ ῥῆμα.

\end{greek}%
\switchcolumn*

Sie übersetzen in Jonischem Dialekt.

\switchcolumn

\begin{greek}[variant=ancient]%
Ἰωνικῶς λέγεις.

\end{greek}%
\switchcolumn*

Nun, wie wollen Sie übersetzen?

\switchcolumn

\begin{greek}[variant=ancient]%
φέρε δή\footnote{\begin{latin}%
orig. \textgreek[variant=ancient]{δὴ}\end{latin}%
}, τί λέγεις;

\end{greek}%
\switchcolumn*

Machen Sie schnell u.\textgreek[variant=ancient]{}\footnote{\quotedblbase und``}\ übersetzen
Sie!

\switchcolumn

\begin{greek}[variant=ancient]%
ἀλλ' \emph{ἀνύσας} λέγε!

\end{greek}%
\switchcolumn*

Mit Ihnen ist nichts.

\switchcolumn

\begin{greek}[variant=ancient]%
σύγ' οὐδὲν εἶ.

\end{greek}%
\switchcolumn*

Es ist \emph{meine Pflicht,} daß ich Ihnen dies sage.

\switchcolumn

\begin{greek}[variant=ancient]%
δικαίως δὲ τοῦτό σοι λέγω.

\end{greek}%
\switchcolumn*

Sie können ja nicht drei Worte übersetzen, ohne Fehler zu machen.

\switchcolumn

\begin{greek}[variant=ancient]%
σὺ γὰρ οὐδὲ τρία ῥήματα ἑλληνικῶς εἰπεῖν οἷός τ' εἶ πρὶν ἐξαμαρτεῖν.

\end{greek}%
\switchcolumn*

Hören Sie auf!

\switchcolumn

\begin{greek}[variant=ancient]%
\emph{παῦε!}

\end{greek}%
\switchcolumn*

Übersetzen Sie dieses Stück auch schriftlich!.

\switchcolumn

\begin{greek}[variant=ancient]%
καὶ μεταγράφετε αὐτὸ τοῦτο ἑλληνιστί!

\end{greek}%
\switchcolumn*

Verstanden?

\switchcolumn

\begin{greek}[variant=ancient]%
\emph{μανθάνετε;}

\end{greek}%
\switchcolumn*

Ja wohl!

\switchcolumn

\begin{greek}[variant=ancient]%
πάνυ μανθάνομεν.

\end{greek}%
\switchcolumn*

Die Aufgabe.

\switchcolumn

\begin{greek}[variant=ancient]%
τὸ ἔργον.

\end{greek}%
\switchcolumn*

Wie fatal, daß ich das Heft vergessen habe.

\switchcolumn

\begin{greek}[variant=ancient]%
ἐς κόρακας! ὡς ἄχθομαι, ὅτι\footnote{\begin{latin}%
\textgreek[variant=ancient]{τῷ τυπογράφῳ ἄσκοπος τὸ γράμμα «ι» ἦν,
καὶ ὁ τυπογράφος ἔγραψα τὸν οὐ γεγραμμένον τόνον.}\end{latin}%
} ἐπελαθόμην τοὺς χάρτας (τὸ βιβλίον) προσφέρειν.

\end{greek}%
\switchcolumn*

Leih' mir eine Feder und Papier!

\switchcolumn

\begin{greek}[variant=ancient]%
χρῆσόν τί μοι γραφεῖον καὶ χάρτην.

\end{greek}%
\switchcolumn*[


\section{Beschäftigt}

]Jeder geht an seine Arbeit.

\switchcolumn

\begin{greek}[variant=ancient]%
πᾶς χωρεῖ πρὸς ἔργον.

\end{greek}%
\switchcolumn*

Was haben wir (beiden) denn nun weiter zu thun?

\switchcolumn

\begin{greek}[variant=ancient]%
ἄγε δή, τί νῷν ἐντευθενὶ ποιητέον;

\end{greek}%
\switchcolumn*

So, das wäre besorgt.

\switchcolumn

\begin{greek}[variant=ancient]%
ταυτὶ δέδραται.

\end{greek}%
\switchcolumn*

Ich will's besorgen.

\switchcolumn

\begin{greek}[variant=ancient]%
ταῦτα δράσω.

\end{greek}%
\switchcolumn*

Das will ich schon besorgen.

\switchcolumn

\begin{greek}[variant=ancient]%
μελήσει μοι ταῦτα.

\end{greek}%
\switchcolumn*

Da ist Alles, was du brauchst.

\switchcolumn

\begin{greek}[variant=ancient]%
ἰδοὺ πάντα, ὧν δέει.

\end{greek}%
\switchcolumn*

Hast du Alles, was du brauchst?

\switchcolumn

\begin{greek}[variant=ancient]%
ἆρ' ἔχεις ἅπαντα, ἅ δεῖ;

\end{greek}%
\switchcolumn*

Ja, ich habe Alles da, was ich brauche.

\switchcolumn

\begin{greek}[variant=ancient]%
πάντα νὴ Δία πάρεστι μοί, ὅσων δέομαι.

\end{greek}%
\switchcolumn*

Die Sache ist ganz einfach.

\switchcolumn

\begin{greek}[variant=ancient]%
\emph{φαυλότατον} ἔργον.

\end{greek}%
\switchcolumn*

Zu welchem Zwecke thut ihr dies?

\switchcolumn

\begin{greek}[variant=ancient]%
ἵνα δὴ τί τοῦτο δρᾶτε;

\end{greek}%
\switchcolumn*

So geht die Sache viel besser.

\switchcolumn

\begin{greek}[variant=ancient]%
χωρεῖ τὸ πρᾶγμα οὕτω\footnote{\begin{latin}%
\textgreek[variant=ancient]{ὁ τυπογράφος ἔγραψα τὸν οὐ γεγραμμένον
ἦχον και τόνον.}\end{latin}%
} πολλῷ\footnote{\begin{latin}%
\textgreek[variant=ancient]{τῷ τυπογράφῳ ἄσκοπος τὸ γράμμα «ῷ» ἦν.}\end{latin}%
} πᾶλλον.

\end{greek}%
\switchcolumn*

Sei fleißig bei der Arbeit!

\switchcolumn

\begin{greek}[variant=ancient]%
τῷ ἔργῳ πρόσεχε!

\end{greek}%
\switchcolumn*

Mach' es nicht wie die Andern!

\switchcolumn

\begin{greek}[variant=ancient]%
μὴ ποίει, ἅπερ οἱ ἄλλοι δρῶσιν!

\end{greek}%
\switchcolumn*

Die Arbeit geht nicht vorwärts.

\switchcolumn

\begin{greek}[variant=ancient]%
οὐ χωρεῖ τοὖργον.

\end{greek}%
\switchcolumn*

Was wollen Sie \emph{denn }thun?

\switchcolumn

\begin{greek}[variant=ancient]%
τί δαὶ ποιήσεις;

\end{greek}%
\switchcolumn*

Das Weitere ist \emph{Eure} Aufgabe.

\switchcolumn

\begin{greek}[variant=ancient]%
ὑμέτερον ἐντεῦθεν ἔργον.

\end{greek}%
\switchcolumn*

Hilf mir, wenn du (jetzt) keine Abhaltung hast!

\switchcolumn

\begin{greek}[variant=ancient]%
συλλαμβάνου, εἰ μή σέ τι κωλύει!

\end{greek}%
\switchcolumn*

Ich habe keine Zeit.

\switchcolumn

\begin{greek}[variant=ancient]%
\emph{οὐ σχολή} (μοι).

\end{greek}%
\switchcolumn*[


\section{Lob und Tadel}

]Wie denken Sie über diesen Schüler, Herr Rector? 

\switchcolumn

\begin{greek}[variant=ancient]%
τί οὖν ἐρεῖς περὶ τούτου τοῦ μαθητοῦ, ὦ γυμνασίαρχε;

\end{greek}%
\switchcolumn*

Der Mensch ist nicht unbegabt. 

\switchcolumn

\begin{greek}[variant=ancient]%
οὐ σκαιὸς ἄνθρωπος\footnote{orig. ἅνθρωπος}!

\end{greek}%
\switchcolumn*

Er scheint mir nicht unbegabt zu sein. 

\switchcolumn

\begin{greek}[variant=ancient]%
οὐ σκαιός μοι δοκεῖ εἶναι.

\end{greek}%
\switchcolumn*

Nein, er ist (vielmehr) recht befähigt. 

\switchcolumn

\begin{greek}[variant=ancient]%
δεξιὸς \emph{μὲν οὖν} ἐστιν.

\end{greek}%
\switchcolumn*

Und lerneifrig und geweckt. 

\switchcolumn

\begin{greek}[variant=ancient]%
καὶ φιλομαθὴς καὶ ἀγχίνους.

\end{greek}%
\switchcolumn*

Und wie ist der Andere? 

\switchcolumn

\begin{greek}[variant=ancient]%
ὁ δὲ ἕτερος ποῖός τις;

\end{greek}%
\switchcolumn*

Er gehört zur schlechten Sorte. 

\switchcolumn

\begin{greek}[variant=ancient]%
ἐστὶ τοῦ πονηροῦ κόμματος.

\end{greek}%
\switchcolumn*

Nun, mit diesem werde ich später ein Wort reden. 

\switchcolumn

\begin{greek}[variant=ancient]%
ἀλλὰ πρὸς τοῦτον μὲν ὕστερός ἐστί μοι λόγος.

\end{greek}%
\switchcolumn*

Er ist vergeßlich und schwer von Begriffen. 

\switchcolumn

\begin{greek}[variant=ancient]%
ἐπιλήσμων γάρ ἐστι καὶ βραδύς.

\end{greek}%
\switchcolumn*

Und er giebt sich keine Mühe. 

\switchcolumn

\begin{greek}[variant=ancient]%
καὶ οὐκ ἐπιμελής ἐστιν.

\end{greek}%
\switchcolumn*

Er ist der dümmste von allen. 

\switchcolumn

\begin{greek}[variant=ancient]%
ἠλιθιότατός ἐστι πάντων.

\end{greek}%
\switchcolumn*

Er hat sich ganz und gar geändert. 

\switchcolumn

\begin{greek}[variant=ancient]%
πολὺ πάνυ μεθέστηκεν.

\end{greek}%
\switchcolumn*

Ich weiß es wohl. 

\switchcolumn

\begin{greek}[variant=ancient]%
οἶδά τοι.

\end{greek}%
\switchcolumn*

Wir werden entsprechende Maßregeln ergreifen. 

\switchcolumn

\begin{greek}[variant=ancient]%
ποιήσομέν τι τῶν προὔργου.

\end{greek}%
\switchcolumn*

Er ist \quotedblbase dumm, faul und gefräßig.``

\switchcolumn

\begin{greek}[variant=ancient]%
ἠλίθιός τε καὶ ἀργὸς καὶ γάστρις ἐστιν.

\end{greek}%
\switchcolumn*

Er ist ganz verdreht. 

\switchcolumn

\begin{greek}[variant=ancient]%
μεγαγχολᾷ.

\end{greek}%
\switchcolumn*

Wie macht A. seine Sache? 

\switchcolumn

\begin{greek}[variant=ancient]%
ὁ δὲ Ἁ.\ πῶς παρέχει τὰ ἑαυτοῦ;

\end{greek}%
\switchcolumn*

Nach (seinen) Kräften. 

\switchcolumn

\begin{greek}[variant=ancient]%
καθ' ὅσον ἂν σθένῃ!

\end{greek}%
\switchcolumn*

Ziemlich gut. 

\switchcolumn

\begin{greek}[variant=ancient]%
ἐπιεικῶς.

\end{greek}%
\switchcolumn*\bgroup

\myafterpagetrue\mysetaligntext{german}{(Censuren:) { }}\mysetalign{german}1. 

\egroup\switchcolumn\bgroup

\mysetaligntext{greek}{}\mysetalign*{greek}\textgreek[variant=ancient]{εὖγε.}

\egroup\switchcolumn*\bgroup

\mysetalign{german}1b.

\egroup\switchcolumn\bgroup

\mysetalign*{greek}\textgreek[variant=ancient]{καλῶς.}

\egroup\switchcolumn*\bgroup

\mysetalign{german}2a.

\egroup\switchcolumn\bgroup

\mysetalign*{greek}\textgreek[variant=ancient]{ἀκριβῶς.}

\egroup\switchcolumn*\bgroup

\mysetalign{german}2.

\egroup\switchcolumn\bgroup

\mysetalign*{greek}\textgreek[variant=ancient]{ὀρθῶς.}

\egroup\switchcolumn*\bgroup

\mysetalign{german}2b.

\egroup\switchcolumn\bgroup

\mysetalign*{greek}\textgreek[variant=ancient]{ἐπιεικῶς.}

\egroup\switchcolumn*\bgroup

\mysetalign{german}3a.

\egroup\switchcolumn\bgroup

\mysetalign*{greek}\textgreek[variant=ancient]{μετρίως.}

\egroup\switchcolumn*\bgroup

\mysetalign{german}3.

\egroup\switchcolumn\bgroup

\mysetalign*{greek}\textgreek[variant=ancient]{μέσως.}

\egroup\switchcolumn*\bgroup

\mysetalign{german}3b.

\egroup\switchcolumn\bgroup

\mysetalign*{greek}\textgreek[variant=ancient]{φαύλως.}

\egroup\switchcolumn*\bgroup

\mysetalign{german}4.

\egroup\switchcolumn\bgroup

\mysetalign*{greek}\textgreek[variant=ancient]{οὐκ ὀρθῶς.}

\egroup\switchcolumn*[


\section{Singen}

]Singe etwas!

\switchcolumn

\begin{greek}[variant=ancient]%
ᾆδόν τι!

\end{greek}%
\switchcolumn*

Ich kann nicht singen. 

\switchcolumn

\begin{greek}[variant=ancient]%
μελῳδεῖν οὐκ ἐπίσταμαι\footnote{\begin{latin}%
orig. \textgreek[variant=ancient]{επίσταμαι}\end{latin}%
}!

\end{greek}%
\switchcolumn*

Singt einmal ein Lied! 

\switchcolumn

\begin{greek}[variant=ancient]%
μέλος τι ᾄσατε.

\end{greek}%
\switchcolumn*

Was gedenkt Ihr zu singen? 

\switchcolumn

\begin{greek}[variant=ancient]%
τί ἐπινοεῖτε ᾄδειν;

\end{greek}%
\switchcolumn*

Nun, was sollen wir denn singen? 

\switchcolumn

\begin{greek}[variant=ancient]%
ἀλλὰ τί δῆτ' ᾄδωμεν;

\end{greek}%
\switchcolumn*

Sagen Sie nur, was Sie gern hören. 

\switchcolumn

\begin{greek}[variant=ancient]%
εἰπὲ οἶστισι \emph{χαίρεις.}

\end{greek}%
\switchcolumn*

Ein herrliches Lied! 

\switchcolumn

\begin{greek}[variant=ancient]%
ὡς ἡδὺ τὸ μέλος!

\end{greek}%
\switchcolumn*

Wir wollen noch eins singen. 

\switchcolumn

\begin{greek}[variant=ancient]%
ἕτερον ᾀσόμεθα.

\end{greek}%
\switchcolumn*

Erlauben sie, daß ich ein Solo singe! 

\switchcolumn

\begin{greek}[variant=ancient]%
ἔασόν με μονῳδῆσαι.

\end{greek}%
\switchcolumn*

Singe, soviel du willst! 

\switchcolumn

\begin{greek}[variant=ancient]%
ἀλλ' ᾆδ' ὁπόσα βούλει.

\end{greek}%
\switchcolumn*

Hör' auf zu singen! 

\switchcolumn

\begin{greek}[variant=ancient]%
παῦσαι μελῳδῶν!

\end{greek}%
\switchcolumn*

Du singst immer nu \emph{vom} Wein. 

\switchcolumn

\begin{greek}[variant=ancient]%
οὐδὲν γὰρ ᾄδεις πλὴν οἶνον.

\end{greek}%
\switchcolumn*

Das gefällt mir. 

\switchcolumn

\begin{greek}[variant=ancient]%
τουτί μ' ἀρέσκει.

\end{greek}%
\switchcolumn*

Ihnen gefällt das? 

\switchcolumn

\begin{greek}[variant=ancient]%
σὲ δὲ τοῦτ' ἀρέσκει;

\end{greek}%
\switchcolumn*

Was Sie deben gesungen haben, werde ich sicherlich nie vergessen. 

\switchcolumn

\begin{greek}[variant=ancient]%
ὅσα ἄρτι ᾖσας, οὐ μὴ ἐπιλάθωμαί ποτε.!

\end{greek}%
\switchcolumn*

Ich will ein Lied dazu singen. 

\switchcolumn

\begin{greek}[variant=ancient]%
ἐπᾴσομαι μέλος τι.

\end{greek}%
\switchcolumn*[


\section{Sie haben Recht!}

]Sie haben Recht. 

\switchcolumn

\begin{greek}[variant=ancient]%
\emph{εὖ λέγεις.}

\end{greek}%
\switchcolumn*

Sie haben wirklich Recht. 

\switchcolumn

\begin{greek}[variant=ancient]%
εὖ τοι λέγεις.

\end{greek}%
\switchcolumn*

Sie könnten vielleicht Recht haben. 

\switchcolumn

\begin{greek}[variant=ancient]%
ἴσως ἄν τι λέγοις.

\end{greek}%
\switchcolumn*

Sie haben ganz Recht. 

\switchcolumn

\begin{greek}[variant=ancient]%
εὖ πάνυ λέγεις.

\end{greek}%
\switchcolumn*

Sie haben offenbar Recht. 

\switchcolumn

\begin{greek}[variant=ancient]%
εὖ λέγειν σὺ φαίνει.

\end{greek}%
\switchcolumn*

Ich denke, Sie haben Recht. 

\switchcolumn

\begin{greek}[variant=ancient]%
εὖ γέ μοι δοκεῖς λέγειν.

\end{greek}%
\switchcolumn*

Das ist auch meine Ansicht. 

\switchcolumn

\begin{greek}[variant=ancient]%
συνδοκεῖ ταῦτα κἀμοί.

\end{greek}%
\switchcolumn*

Es kommt mir allerdings auch so vor. 

\switchcolumn

\begin{greek}[variant=ancient]%
τοῦτο μὲν κἀμοί δοκεῖ.

\end{greek}%
\switchcolumn*

Das ist ganz klar. 

\switchcolumn

\begin{greek}[variant=ancient]%
τοῦτο περιφανέστατον.

\end{greek}%
\switchcolumn*

Das ist ein billiger Vorschlag. 

\switchcolumn

\begin{greek}[variant=ancient]%
δίκαιος ὁ λόγος.

\end{greek}%
\switchcolumn*

Glaub's gern. 

\switchcolumn

\begin{greek}[variant=ancient]%
\emph{πείθομαι.}

\end{greek}%
\switchcolumn*

Wie es scheint. 

\switchcolumn

\begin{greek}[variant=ancient]%
ὡς ἔοικεν.

\end{greek}%
\switchcolumn*

Dafür giebt es viele Beweise. 

\switchcolumn

\begin{greek}[variant=ancient]%
τούτων τεκμήριά ἐστι πολλά.

\end{greek}%
\switchcolumn*

Ich schließe es aus Thatsachen. 

\switchcolumn

\begin{greek}[variant=ancient]%
ἔργῳ τεκμαίρομαι.

\end{greek}%
\switchcolumn*[


\section{Ja!}

]Ja! (Ohne Zweifel!) 

\switchcolumn

\begin{greek}[variant=ancient]%
νὴ\footnote{\begin{latin}%
\textgreek[variant=ancient]{τυπογράφος ἔγραψα τὸν οὐ γεγραμμένον
τόνον.}\end{latin}%
} Δία!

\end{greek}%
\switchcolumn*

Ja wahrhaftig! 

\switchcolumn

\begin{greek}[variant=ancient]%
νὴ τοὺς θεούς! --- νὴ τὸν Ποσειδῶ!

\end{greek}%
\switchcolumn*

Ganz recht! 

\switchcolumn

\begin{greek}[variant=ancient]%
μάλιστά γε. --- νάνυ!

\end{greek}%
\switchcolumn*

Sehr richtig! 

\switchcolumn

\begin{greek}[variant=ancient]%
κομιδῆ μὲν οὖν!

\end{greek}%
\switchcolumn*

Natürlich! 

\switchcolumn

\begin{greek}[variant=ancient]%
εἰκότως! --- εἰκὸς γάρ!

\end{greek}%
\switchcolumn*

Ja natürlich! 

\switchcolumn

\begin{greek}[variant=ancient]%
εἰκότως γε (νὴ Δία)!

\end{greek}%
\switchcolumn*

Ganz gewiß! 

\switchcolumn

\begin{greek}[variant=ancient]%
εὖ ἴσθ' ὅτι!

\end{greek}%
\switchcolumn*

Ich? Freilich, Sie! 

\switchcolumn

\begin{greek}[variant=ancient]%
ἐγώ; σὺ \emph{μέντοι!}

\end{greek}%
\switchcolumn*

Kann sein! 

\switchcolumn

\begin{greek}[variant=ancient]%
\emph{οὐκ οἶδα.}

\end{greek}%
\switchcolumn*

Kann wohl sein! 

\switchcolumn

\begin{greek}[variant=ancient]%
ἕοικεν!

\end{greek}%
\switchcolumn*

Kein Wunder! 

\switchcolumn

\begin{greek}[variant=ancient]%
κοὐ θαῦμά γε!

\end{greek}%
\switchcolumn*

Und das ist gar kein Wunder! 

\switchcolumn

\begin{greek}[variant=ancient]%
καὶ θαῦμά γ' οὐδέν!

\end{greek}%
\switchcolumn*

Schön! 

\switchcolumn

\begin{greek}[variant=ancient]%
εὖ λέγεις!

\end{greek}%
\switchcolumn*

Du fragst noch? 

\switchcolumn

\begin{greek}[variant=ancient]%
οὐκ\footnote{\begin{latin}%
\textgreek[variant=ancient]{ὁ τυπογράφος ἔγραψα τὸν οὐ γεγραμμένον
ἦχον.}\end{latin}%
} οἶσθα;!

\end{greek}%
\switchcolumn*[


\section{Nein!}

]Nein! 

\switchcolumn

\begin{greek}[variant=ancient]%
\emph{οὐ μὰ Δία!}

\end{greek}%
\switchcolumn*

Nein, ich nicht. 

\switchcolumn

\begin{greek}[variant=ancient]%
μὰ Δί' ἐγὼ μὲν οὔ.

\end{greek}%
\switchcolumn*

Nein, sondern . . .

\switchcolumn

\begin{greek}[variant=ancient]%
οὔκ· ἀλλά . . .

\end{greek}%
\switchcolumn*

Nicht doch! 

\switchcolumn

\begin{greek}[variant=ancient]%
μὴ δῆτα!

\end{greek}%
\switchcolumn*

Thu's nicht! 

\switchcolumn

\begin{greek}[variant=ancient]%
μή νυν ποιήσῃς!

\end{greek}%
\switchcolumn*

Noch nicht! 

\switchcolumn

\begin{greek}[variant=ancient]%
μὴ δῆτά πώ γε.

\end{greek}%
\switchcolumn*

Nicht eher, als bis (dies geschieht) 

\switchcolumn

\begin{greek}[variant=ancient]%
οὔκ, ἢν μὴ (τοῦτο γένηται\footnote{\begin{latin}%
\textgreek[variant=ancient]{ὁ τυπογράφος ἔγραψα τὸν οὐ γεγραμμένον
τόνον.}\end{latin}%
}).

\end{greek}%
\switchcolumn*

Ja nicht! 

\switchcolumn

\begin{greek}[variant=ancient]%
μηδαμῶς!

\end{greek}%
\switchcolumn*

Ist nicht nöthig! 

\switchcolumn

\begin{greek}[variant=ancient]%
οὐδὲν δεῖ!

\end{greek}%
\switchcolumn*

Freilich nicht. 

\switchcolumn

\begin{greek}[variant=ancient]%
μὰ Δί' οὐ μέντοι.

\end{greek}%
\switchcolumn*

(Ich) leider nicht! 

\switchcolumn

\begin{greek}[variant=ancient]%
εἰ γὰρ ὤφελ(ον)!

\end{greek}%
\switchcolumn*

Du bist gescheit! (ironisch ablehnend.) 

\switchcolumn

\begin{greek}[variant=ancient]%
σωφρονεῖς! --- δεξιὸς εἶ!

\end{greek}%
\switchcolumn*

Kein Gedanke! 

\switchcolumn

\begin{greek}[variant=ancient]%
ἥκιστα!

\end{greek}%
\switchcolumn*

Am allerwenigsten! 

\switchcolumn

\begin{greek}[variant=ancient]%
ἥκιστά γε!

\end{greek}%
\switchcolumn*

Um keinen Preis! 

\switchcolumn

\begin{greek}[variant=ancient]%
ἥκιστα πάντων!

\end{greek}%
\switchcolumn*

Nein, und wenn Ihr Euch auf den Kopf stellt! 

\switchcolumn

\begin{greek}[variant=ancient]%
οὐκ ἂν μὰ Δία, εἰ κρέμαισθέ γε ὑμεῖς!

\end{greek}%
\switchcolumn*

Denken Sie, ich sei verrückt? 

\switchcolumn

\begin{greek}[variant=ancient]%
μελαγχολᾶν μ' οὕτως οἴκει;

\end{greek}%
\switchcolumn*

So steht die Sache nicht! 

\switchcolumn

\begin{greek}[variant=ancient]%
οὐχ οὗτος ὁ τρόπος!

\end{greek}%
\switchcolumn*

Wenn zehnmal! 

\switchcolumn

\begin{greek}[variant=ancient]%
ἀλλ' ὅμως!

\end{greek}%
\switchcolumn*

Sie haben \emph{nicht Recht!} 

\switchcolumn

\begin{greek}[variant=ancient]%
\emph{οὐκ ὀρθῶς} λέγεις.

\end{greek}%
\switchcolumn*

Ach was! (Blech!) 

\switchcolumn

\begin{greek}[variant=ancient]%
λῆρος!

\end{greek}%
\switchcolumn*

Das ist Unsinn! 

\switchcolumn

\begin{greek}[variant=ancient]%
οὐδὲν λέγεις!

\end{greek}%
\switchcolumn*

Aber das ist was ganz Anderes! 

\switchcolumn

\begin{greek}[variant=ancient]%
ἀλλ' οὐ ταὐτόν!

\end{greek}%
\switchcolumn*

Aber das gehört ja gar nicht hierher, was Sie sagen! 

\switchcolumn

\begin{greek}[variant=ancient]%
ἀλλ' οὐκ εἶπας ὅμοιον!

\end{greek}%

	\switchcolumn*[


\part{Handel und Wandel.}


\section{Er will Geld}

]Er will etwas haben. 

\switchcolumn

\begin{greek}[variant=ancient]%
αἰτεῖ λαβεῖν τι.

\end{greek}%
\switchcolumn*

Er hat Alles, was er braucht. 

\switchcolumn

\begin{greek}[variant=ancient]%
ἔχει ἅπαντα, ἃ δεῖ.

\end{greek}%
\switchcolumn*

Was wünschen Sie? 

\switchcolumn

\begin{greek}[variant=ancient]%
\emph{τοῦ δέει;}

\end{greek}%
\switchcolumn*

\vspace{0.5em}
Wes\textcompwordmark{}halb sind Sie hergekommen?

\switchcolumn

\begin{tabular}{ll}
\ldelim\{{2}{1em}[] & \begin{greek}[variant=ancient]%
τοῦ δεόμενος ἦλθες ἐνθαδί;\end{greek}%
\tabularnewline
 & \begin{greek}[variant=ancient]%
ἥκεις κατὰ τί;\end{greek}%
\tabularnewline
\end{tabular}

\switchcolumn*

Was hat Sie hergeführt? 

\switchcolumn

\begin{greek}[variant=ancient]%
ἐπὶ τί πάρει δεῦρο;

\end{greek}%
\switchcolumn*

Ich bitte Sie, leihen Sie mir 20 Mark! 

\switchcolumn

\begin{greek}[variant=ancient]%
δάνεισόν μοι πρὸς τῶν θεῶν εἴκοσι μάρκας{*}!

\end{greek}%
\switchcolumn*

Die Noth zwingt mich dazu. 

\switchcolumn

\begin{greek}[variant=ancient]%
ἡ ἀνάγκη με πιέζει.

\end{greek}%
\switchcolumn*

Nein! 

\switchcolumn

\begin{greek}[variant=ancient]%
μὰ Δί' ἐγὼ μὲν οὔ!

\end{greek}%
\switchcolumn*

Sie haben, was Sie brauchen. 

\switchcolumn

\begin{greek}[variant=ancient]%
ἔχεις ὧν δέει.

\end{greek}%
\switchcolumn*

So helfen Sie mir doch! 

\switchcolumn

\begin{greek}[variant=ancient]%
οὐκ ἀρἠξεις;

\end{greek}%
\switchcolumn*

Haben Sie Mitleid mit mir! 

\switchcolumn

\begin{greek}[variant=ancient]%
οἴκτειρόν με!

\end{greek}%
\switchcolumn*

Was wollen Sie mit dem Gelde machen? 

\switchcolumn

\begin{greek}[variant=ancient]%
\emph{τί χρήσει τῷ} ἀργυρίῳ;

\end{greek}%
\switchcolumn*

Ich will meinen Schuhmacher bezahlen. 

\switchcolumn

\begin{greek}[variant=ancient]%
ἀποδώσω τῷ σκυτοτόμῳ.

\end{greek}%
\switchcolumn*

Woher soll ich das Geld bekommen? 

\switchcolumn

\begin{greek}[variant=ancient]%
πόθεν τὸ ἀργύριον λήψομαι;

\end{greek}%
\switchcolumn*

Hier haben Sie es! 

\switchcolumn

\begin{greek}[variant=ancient]%
ἰδοὺ τουτὶ λαβέ!

\end{greek}%
\switchcolumn*

Haben Sie vielen Dank! 

\switchcolumn

\begin{greek}[variant=ancient]%
εὖ γ' ἐποίησας!

\end{greek}%
\switchcolumn*

Der Himmel segne Sie tausendmal! 

\switchcolumn

\begin{greek}[variant=ancient]%
πόλλ' ἀγαθὰ γένοιτό σοι!

\end{greek}%
\switchcolumn*

Seien sie nicht böse, mein Lieber! 

\switchcolumn

\begin{greek}[variant=ancient]%
μὴ ἀγανάκτει, ὦ 'γαθέ!

\end{greek}%
\switchcolumn*

\emph{Seien sie so gut} und sprechen Sie nicht davon! 

\switchcolumn

\begin{greek}[variant=ancient]%
\emph{οἶσθ' ὁ δρᾶσον;} μὴ διαλέγου περι τούτοθ μηδέν!

\end{greek}%
\switchcolumn*

Aber ich bitte Sie —! 

\switchcolumn

\begin{greek}[variant=ancient]%
ἀλλ' ὦ 'γαθέ —!

\end{greek}%
\switchcolumn*[


\section{Der Hausirer}

]Da kommt der Jude wieder!

\switchcolumn

\begin{greek}[variant=ancient]%
καὶ μὴν ὁδὶ ἐκεῖνος ὁ Ἰουδαῖος!

\end{greek}%
\switchcolumn*

Schöne Portemonnaies! Schlipfe\footnote{Redacteur: \quotedblbase Shlipfe`` wird im Original verwendet.}!
Messer!

\switchcolumn

\begin{greek}[variant=ancient]%
βαλάντια καλά! λαιμοδέτια!{*} μαχαίρια!

\end{greek}%
\switchcolumn*

Was soll ich für dies hier zahlen? 

\switchcolumn

\begin{greek}[variant=ancient]%
τί δῆτα καταθῶ τουτοί;

\end{greek}%
\switchcolumn*

Zwei Mark fünfzig. 

\switchcolumn

\begin{greek}[variant=ancient]%
δύο μάρκας{*} καὶ πεντήκοντα.

\end{greek}%
\switchcolumn*

Nein, das ist zuviel. 

\switchcolumn

\begin{greek}[variant=ancient]%
μὰ Δί', ἀλλ' ἔλαττον.

\end{greek}%
\switchcolumn*

Geben Sie zwei Mark darür! 

\switchcolumn

\begin{greek}[variant=ancient]%
δύο μάρκας τελεῖς;

\end{greek}%
\switchcolumn*

Hier haben Sie 1 Mark 50 Pf. 

\switchcolumn

\begin{greek}[variant=ancient]%
λαβὲ μάρκην καὶ ἡμίσειαν.

\end{greek}%
\switchcolumn*

Was kosten die Portemonnaies? 

\switchcolumn

\begin{greek}[variant=ancient]%
πῶς τὰ βαλάντια ὤνια;

\end{greek}%
\switchcolumn*

Für 4 Mark können Sie ein ganz schönes bekommen. 

\switchcolumn

\begin{greek}[variant=ancient]%
λήψει τεσσάρων μαρκῶν πάνυ καλόν.

\end{greek}%
\switchcolumn*

Nehmen Sie es wieder mit, ich kaufe es nicht. — Sie wollen \emph{zu
vie}l profitiren. 

\switchcolumn

\begin{greek}[variant=ancient]%
ἀπόφερε· οὐκ ὠνήσομαι. κερδαίνειν γὰρ βούλει \emph{πολύ.}

\end{greek}%
\switchcolumn*

Was bieten Sie gutwillig? 

\switchcolumn

\begin{greek}[variant=ancient]%
\emph{αὐτὸς} σὺ τί δίδως;

\end{greek}%
\switchcolumn*

Was ich biete? Zwei Mark würde ich geben. 

\switchcolumn

\begin{greek}[variant=ancient]%
ὅτι δίδωμι; δοίην ἂν δύο μάρκας.

\end{greek}%
\switchcolumn*

Da nehmen Sie es; denn es ist immer besser als nichts zu lösen. 

\switchcolumn

\begin{greek}[variant=ancient]%
ἔνεγκε τοίνυν· κρεῖττον γάρ ἐστιν ἢ μηδὲν λαβεῖν.

\end{greek}%
\switchcolumn*

Wir werden den Kerl nicht wieder los! 

\switchcolumn

\begin{greek}[variant=ancient]%
ἅνθρωπος οὐκ ἀπαλλαχθήσεται ἡμῶν.

\end{greek}%
\switchcolumn*

Das Messer taugt nichts; ich würde nicht 1 Mark dafür geben. 

\switchcolumn

\begin{greek}[variant=ancient]%
οὐδέν ἐστιν ἡ μάχαιρα· οὐκ ἂν πριαίμην οὐδὲ μιᾶς μάρκης.!

\end{greek}%
\switchcolumn*

Ich habe selbst seiner Zeit 3 Mark dafür gegeben. 

\switchcolumn

\begin{greek}[variant=ancient]%
αὐτὸς ἀντέδωκα τούτου ποτὲ τρεῖς μάρκας.

\end{greek}%
\switchcolumn*

Ich verdiene nichts daran. 

\switchcolumn

\begin{greek}[variant=ancient]%
οὐδέν μοι περιγίγνεται.

\end{greek}%
\switchcolumn*

Wirklich? 

\switchcolumn

\begin{greek}[variant=ancient]%
ἄληθες;

\end{greek}%
\switchcolumn*

Schwören Sie einmal! 

\switchcolumn

\begin{greek}[variant=ancient]%
ὄμοσον!

\end{greek}%
\switchcolumn*

Bei Gott! 

\switchcolumn

\begin{greek}[variant=ancient]%
οὐ μὰ τοὺς θεούς!

\end{greek}%
\switchcolumn*

Verkaufen Sie es an einen Andern! 

\switchcolumn

\begin{greek}[variant=ancient]%
πώλει τοῦτο ἄλλῳ τινί!

\end{greek}%
\switchcolumn*

Ich will es Ihnen \emph{abkaufen}. 

\switchcolumn

\begin{greek}[variant=ancient]%
ὠνήσομαί \emph{σοι} ἐγώ.

\end{greek}%
\switchcolumn*

Da haben Sie das Geld. 

\switchcolumn

\begin{greek}[variant=ancient]%
ἔχε δὴ τἀργύριον.

\end{greek}%
\switchcolumn*

Das wäre abgemacht. 

\switchcolumn

\begin{greek}[variant=ancient]%
\emph{ταῦτα} δή.

\end{greek}%
\switchcolumn*

Ich habe 3 Mark \emph{dafür} bezahlt. 

\switchcolumn

\begin{greek}[variant=ancient]%
ἀπέδοκα \emph{ὀφείλων} τρεῖς μάρκας.

\end{greek}%
\switchcolumn*

In Leipzig verkauft man das Dutzend für 20 Mark. 

\switchcolumn

\begin{greek}[variant=ancient]%
ἐν Λειψίᾳ{*} πωλοῦνται κατὰ δώδεκα εἴκοσι μαρκῶν.

\end{greek}%
\switchcolumn*

Das hier hat er für 1 Mark verkauft. 

\switchcolumn

\begin{greek}[variant=ancient]%
τοδὶ ἀπέδοτο μιᾶς μάρκης.

\end{greek}%
\switchcolumn*[


\section{Beim Schneider}

]Guten Tag! 

\switchcolumn

\begin{greek}[variant=ancient]%
χαῖρε!

\end{greek}%
\switchcolumn*

Guten Tag, mein Herr! 

\switchcolumn

\begin{greek}[variant=ancient]%
χαῖρε καὶ σύ!

\end{greek}%
\switchcolumn*

Womit kann ich dienen? 

\switchcolumn

\begin{greek}[variant=ancient]%
ἥκεις δὲ κατὰ τί;

\end{greek}%
\switchcolumn*

Was wünschen Sie? 

\switchcolumn

\begin{greek}[variant=ancient]%
τοῦ δέει;

\end{greek}%
\switchcolumn*

Ich brauche Rock und Hose. 

\switchcolumn

\begin{greek}[variant=ancient]%
δέομαι ἱμαίου τε καὶ βρακῶν.

\end{greek}%
\switchcolumn*

Das Hemd. 

\switchcolumn

\begin{greek}[variant=ancient]%
ὁ χιτών.

\end{greek}%
\switchcolumn*

Der Hut. 

\switchcolumn

\begin{greek}[variant=ancient]%
ὁ πῖλος.

\end{greek}%
\switchcolumn*

Der Überrock. 

\switchcolumn

\begin{greek}[variant=ancient]%
τὸ ἐπάνω ἱμάτιον.

\end{greek}%
\switchcolumn*

Die Stiefel. 

\switchcolumn

\begin{greek}[variant=ancient]%
τὰ ὑποδήματα.

\end{greek}%
\switchcolumn*

Der Strumpf. 

\switchcolumn

\begin{greek}[variant=ancient]%
ἡ περικνημίς.

\end{greek}%
\switchcolumn*

Das Taschentuch. 

\switchcolumn

\begin{greek}[variant=ancient]%
τὸ ῥινόμακτρον.

\end{greek}%
\switchcolumn*

Was soll ich \emph{dafür} zahlen? 

\switchcolumn

\begin{greek}[variant=ancient]%
τί τελῶ ταῦτα \emph{ὠνούμενος;}

\end{greek}%
\switchcolumn*

50 Mark für einen Rock und 20 Mark für die Beinkleider. 

\switchcolumn

\begin{greek}[variant=ancient]%
πεντήκοντα μάρκας{*} εἰς ἱμάτιον, εἴκοσι δ' εἰς βράκας.

\end{greek}%
\switchcolumn*

Hier ist ein sehr schöner Rock nebst Beinkleidern. 

\switchcolumn

\begin{greek}[variant=ancient]%
κάλλιστον τοδὶ ἱμάτιον μετὰ βρακῶν.

\end{greek}%
\switchcolumn*

Wird er mir passen? 

\switchcolumn

\begin{greek}[variant=ancient]%
ἆρ' \emph{ἁρμόσει} μοι;

\end{greek}%
\switchcolumn*

Legen Sie gefälligst ab! 

\switchcolumn

\begin{greek}[variant=ancient]%
κατάθου δῆτα τὸ ἐπάνω ἱμάτιον.

\end{greek}%
\switchcolumn*

\vspace{0.5em}
Bitte, ziehen Sie einmal den Rock aus! 

\switchcolumn

\begin{tabular}{ll}
\ldelim\{{2}{1em}[] & \begin{greek}[variant=ancient]%
ἀπόδυθι, ἀντιβολῶ, θοἰμάτιον!\end{greek}%
\tabularnewline
 & \begin{greek}[variant=ancient]%
βούλει ἀποδύεσθαι θοἰμάτιον;\end{greek}%
\tabularnewline
\end{tabular}

\switchcolumn*

Sie haben keinen neuen Rock an. 

\switchcolumn

\begin{greek}[variant=ancient]%
οὐ καινὸν ἀμπέχει ἱμάτιον.

\end{greek}%
\switchcolumn*

Nein, der alte Rock hat Löcher. 

\switchcolumn

\begin{greek}[variant=ancient]%
οὐ μὰ Δί'· ἀλλ' ὀπὰς ἔχει τὸ τριβώνιον.

\end{greek}%
\switchcolumn*

Was Sie nun für einen schönen Anzug haben! 

\switchcolumn

\begin{greek}[variant=ancient]%
ποίαν ἤδη ἔχεις σκευήν!

\end{greek}%
\switchcolumn*

Der neue Rock sitzt vortrefflich. 

\switchcolumn

\begin{greek}[variant=ancient]%
ἄριστ' ἔχει τὸ καινὸν ἱμάτιον!

\end{greek}%
\switchcolumn*

Haben Sie etwas daran aus\textcompwordmark{}zusetzen?

\switchcolumn

\begin{greek}[variant=ancient]%
ἔχεις τι ψέγειν τούτου;

\end{greek}%
\switchcolumn*

Er steht mir nicht. 

\switchcolumn

\begin{greek}[variant=ancient]%
οὐ \emph{πρέπει} μοι.

\end{greek}%
\switchcolumn*[


\section{Schuhwerk}

]Die Stiefel fehlen noch. 

\switchcolumn

\begin{greek}[variant=ancient]%
ὑποδημάτων δεῖ.

\end{greek}%
\switchcolumn*

Nimm hier meine! 

\switchcolumn

\begin{greek}[variant=ancient]%
τἀμὰ ταυτὶ λάμβανε!

\end{greek}%
\switchcolumn*

Erst zieh' diesen an! 

\switchcolumn

\begin{greek}[variant=ancient]%
τοῦτο πρῶτον ὑποδύου.

\end{greek}%
\switchcolumn*

Zieh' endlich die Stiefel an! 

\switchcolumn

\begin{greek}[variant=ancient]%
ἄνυσον ὑποδυσάμενος!

\end{greek}%
\switchcolumn*

Zieh' die Stiefeletten aus! 

\switchcolumn

\begin{greek}[variant=ancient]%
ἀποδύου τὰς ἐμβάδας (τὰ ἐμβάδια).

\end{greek}%
\switchcolumn*

Zieh' diese hier an! 

\switchcolumn

\begin{greek}[variant=ancient]%
ὑπόδυθι τάσδε.

\end{greek}%
\switchcolumn*

Passen sie? 

\switchcolumn

\begin{greek}[variant=ancient]%
ἆρ' ἁρμόττουσιν.

\end{greek}%
\switchcolumn*

Ja, sie sitzen vortrefflich. 

\switchcolumn

\begin{greek}[variant=ancient]%
νὴ Δί', ἀλλ' ἄριστ' ἔχει.

\end{greek}%
\switchcolumn*

Wo haben Sie das Paar Stiefeletten gekauft, das Sie anhaben? 

\switchcolumn

\begin{greek}[variant=ancient]%
πόθεν πριάμενος τὸ ζεῦγος ἐμβάδων τουτὶ φορεῖς;

\end{greek}%
\switchcolumn*

Auf \emph{dem} Markte. 

\switchcolumn

\begin{greek}[variant=ancient]%
ἐν ἀγορᾷ.

\end{greek}%
\switchcolumn*

Für wieviel? 

\switchcolumn

\begin{greek}[variant=ancient]%
καὶ πόσου;

\end{greek}%
\switchcolumn*

Für 16 Mark. 

\switchcolumn

\begin{greek}[variant=ancient]%
ἑκκαίδεκα μαρκῶν{*}.

\end{greek}%
\switchcolumn*[


\section{Vom Obstmarkt}

]Ich muß auf \emph{den} Markt gehen. 

\switchcolumn

\begin{greek}[variant=ancient]%
εἰς ἀγορὰν βαδιστέον μοι.

\end{greek}%
\switchcolumn*

Wes\textcompwordmark{}halb?

\switchcolumn

\begin{greek}[variant=ancient]%
τίνος ἕνεκα;

\end{greek}%
\switchcolumn*

Sie geht auf den Markt, um Trauben zu holen. 

\switchcolumn

\begin{greek}[variant=ancient]%
χωρεῖ εἰς ἀγορὰν ἐπὶ βότρυς.

\end{greek}%
\switchcolumn*

Ich will sie kaufen, wenn du mir das Geld giebst. 

\switchcolumn

\begin{greek}[variant=ancient]%
ὠνήσομαι, ἐὰν σύ μοι δῷς τἀργύριον.

\end{greek}%
\switchcolumn*

Da hast du ein paar Groschen! 

\switchcolumn

\begin{greek}[variant=ancient]%
ἰδοὺ λαβὲ μικρὸν ἀργυρίδιον!

\end{greek}%
\switchcolumn*

Was soll ich kaufen? 

\switchcolumn

\begin{greek}[variant=ancient]%
τί βούλει με πρίασθαι;

\end{greek}%
\switchcolumn*

Wir wollen für dieses Geld Pfirsiche kaufen. 

\switchcolumn

\begin{greek}[variant=ancient]%
ὠνησόμεθα περσικὰ τούτου τοῦ ἀργυρίου.

\end{greek}%
\switchcolumn*\bgroup

\myafterpagetrue\mysetaligntext{german}{Kaufe mir{ }}\mysetalign{german}Äpfel.

\egroup\switchcolumn\bgroup

\mysetaligntext{greek}{\textgreek[variant=ancient]{ἀγόρασόν μοι}{ }}\mysetalign*{greek}\textgreek[variant=ancient]{μῆλα.}

\egroup\switchcolumn*\bgroup

\mysetalign{german}Aprikosen.

\egroup\switchcolumn\bgroup

\begin{greek}[variant=ancient]%
\mysetalign*{greek}ἀρμενιακά (μῆλα).

\end{greek}%
\egroup\switchcolumn*\bgroup

\mysetalign{german}Birnen. 

\egroup\switchcolumn\bgroup

\begin{greek}[variant=ancient]%
\mysetalign*{greek}ἄπια.

\end{greek}%
\egroup\switchcolumn*\bgroup

\mysetalign{german}Erdbeeren. 

\egroup\switchcolumn\bgroup

\begin{greek}[variant=ancient]%
\mysetalign*{greek}χαμοκέρασα{*}.

\end{greek}%
\egroup\switchcolumn*\bgroup

\mysetalign{german}Gemüse. 

\egroup\switchcolumn\bgroup

\begin{greek}[variant=ancient]%
\mysetalign*{greek}λάχανα.

\end{greek}%
\egroup\switchcolumn*\bgroup

\mysetalign{german}Kastanien. 

\egroup\switchcolumn\bgroup

\begin{greek}[variant=ancient]%
\mysetalign*{greek}κάστανα.

\end{greek}%
\egroup\switchcolumn*\bgroup

\mysetalign{german}Kirschen. 

\egroup\switchcolumn\bgroup

\begin{greek}[variant=ancient]%
\mysetalign*{greek}κεράσια.

\end{greek}%
\egroup\switchcolumn*\bgroup

\mysetalign{german}Wallnüsse. 

\egroup\switchcolumn\bgroup

\begin{greek}[variant=ancient]%
\mysetalign*{greek}κάρυα.

\end{greek}%
\egroup\switchcolumn*\bgroup

\mysetalign{german}Haselnüsse. 

\egroup\switchcolumn\bgroup

\begin{greek}[variant=ancient]%
\mysetalign*{greek}λεπτοκάρυα.

\end{greek}%
\egroup\switchcolumn*\bgroup

\mysetalign{german}Pfirsiche. 

\egroup\switchcolumn\bgroup

\begin{greek}[variant=ancient]%
\mysetalign*{greek}περσικά (μῆλα).

\end{greek}%
\egroup\switchcolumn*\bgroup

\mysetalign{german}Pflaumen. 

\egroup\switchcolumn\bgroup

\begin{greek}[variant=ancient]%
\mysetalign*{greek}κοκκύμηλα (\textgerman[spelling=old,babelshorthands=true]{Kuckucks\textcompwordmark{}äpfel}).

\end{greek}%
\egroup\switchcolumn*\bgroup

\mysetalign{german}Apfelsinen 

\egroup\switchcolumn\bgroup

\begin{greek}[variant=ancient]%
\mysetalign*{greek}πορτοκάλια{*}. (\textgerman[spelling=old,babelshorthands=true]{Früchte
aus Portugal.})

\end{greek}%
\egroup\switchcolumn*\bgroup

\mysetalign{german}Johannis\textcompwordmark{}beeren. 

\egroup\switchcolumn\bgroup

\begin{greek}[variant=ancient]%
\mysetalign*{greek}φραγγοστάφυλα{*}.

\end{greek}%
\egroup\switchcolumn*\bgroup

\mysetalign{german}Radies\textcompwordmark{}chen. 

\egroup\switchcolumn\bgroup

\begin{greek}[variant=ancient]%
\mysetalign*{greek}ῥαφανίδια.

\end{greek}%
\egroup\switchcolumn*\bgroup

\mysetalign{german}Alles Mögliche. 

\egroup\switchcolumn

\begin{greek}[variant=ancient]%
\mysetalign*{greek}\emph{πάντα.}

\end{greek}%
\switchcolumn*

Wieviel geben Sie für's Geld? 

\switchcolumn

\begin{greek}[variant=ancient]%
πόσον δίδως δῆτα τἀργυρίου;

\end{greek}%
\switchcolumn*

Die Mandel für eine Mark. 

\switchcolumn

\begin{greek}[variant=ancient]%
πεντεκαίδεκα τῆς μάρκες.

\end{greek}%
\switchcolumn*

Was kostet jetzt die Butter? 

\switchcolumn

\begin{greek}[variant=ancient]%
πῶς ὁ βούτυρος (τὸ βούτυρον) το\footnote{\begin{latin}%
? sic. \textgreek[variant=ancient]{οὐκ οἶδα τί τὸ λέξις ἐστί.}\end{latin}%
} νῦν ὤνιος.

\end{greek}%
\switchcolumn*

Sie ist wohlfeil. 

\switchcolumn

\begin{greek}[variant=ancient]%
εὐτελής ἐστιν.

\end{greek}%
\switchcolumn*

Wir müssen sie theuer kaufen. 

\switchcolumn

\begin{greek}[variant=ancient]%
δεῖ τίμιον πρίασθαι αὐτόν.

\end{greek}%
\switchcolumn*

Frische Butter, friesches Fleisch. 

\switchcolumn

\begin{greek}[variant=ancient]%
\emph{χλωρὸς} βούτυρος, χλωρὸν κρέας.

\end{greek}%
\switchcolumn*

Ich habe noch nichts eingekauft. 

\switchcolumn

\begin{greek}[variant=ancient]%
οὐδὲν ἠμπόληκά πω.

\end{greek}%
\switchcolumn*

Wir haben etwas eingekauft und wollen nun nach Hause gehen. 

\switchcolumn

\begin{greek}[variant=ancient]%
οἴκαδ' ἴμεν ἐμπολήσαντές τι.

\end{greek}%
\switchcolumn*

Der Preis. 

\switchcolumn

\begin{greek}[variant=ancient]%
ἡ τιμή.

\end{greek}%

	\switchcolumn*[


\part{In Gesellschaft.}


\section{Tanz}

]Sie tanzt gut; \emph{nicht wahr?} 

\switchcolumn

\begin{greek}[variant=ancient]%
καλῶς ὀρχεῖται· ἦ γάρ;

\end{greek}%
\switchcolumn*

Allerdings. 

\switchcolumn

\begin{greek}[variant=ancient]%
μάλιστα.

\end{greek}%
\switchcolumn*

Ich bin ent\textcompwordmark{}zückt.

\switchcolumn

\begin{greek}[variant=ancient]%
κεκήλημαι ἔγωγε.

\end{greek}%
\switchcolumn*

Ich werde Polka mit ihr tanzen (Schottisch, Walzer, Française). 

\switchcolumn

\begin{greek}[variant=ancient]%
ὀρχήσομαι μετ' αὐτῆς τὸ Πολωνικόν (τὸ Καληδονικόν, τὸ Γερμανικόν,
τὸ Γαλλικόν).

\end{greek}%
\switchcolumn*

Erlauben Sie mir diesen Tanz, gnädige Frau? (— Fräulein?) 

\switchcolumn

\begin{greek}[variant=ancient]%
δὸς ὀρχεῖσθαι τοῦτο μετὰ σοῦ, ὦ γύναι! (--- ὦ κόρη!)

\end{greek}%
\switchcolumn*

Recht gern! 

\switchcolumn

\begin{greek}[variant=ancient]%
\emph{φθόνος οὐδείς.}

\end{greek}%
\switchcolumn*

Bitte, hören Sie auf, ich kann nicht mehr. 

\switchcolumn

\begin{greek}[variant=ancient]%
παῦε δῆτ' ὀρχούμενος, !

\end{greek}%
\switchcolumn*

Ich bin müde. 

\switchcolumn

\begin{greek}[variant=ancient]%
κέκμηκα.

\end{greek}%
\switchcolumn*

Nur dies \emph{eine} Mal erlauben Sie mir noch!

\switchcolumn

\begin{greek}[variant=ancient]%
ἓν μὲν οὖν τουτί μ' ἔασον ὀρχήσασθαι.

\end{greek}%
\switchcolumn*

Nun denn noch dies \emph{eine} Mal und nicht weiter! 

\switchcolumn

\begin{greek}[variant=ancient]%
τοῖτό νυν καὶ μηκέτ' ἄλλο μηδέν.

\end{greek}%
\switchcolumn*

Das ist eine Lust, mit Ihnen zu tanzen! 

\switchcolumn

\begin{greek}[variant=ancient]%
ὡς ἡδὺ μετὰ σοῦ ὀρχεῖσθαι!

\end{greek}%
\switchcolumn*

Wer ist eigentlich der Herr dort, der hierher sieht? der an der Thür
steht? 

\switchcolumn

\begin{greek}[variant=ancient]%
τίς ποθ' ὅδεὁ δεῦρο βλέπων; \emph{ὁ ἐπὶ} ταῖς θύραις;

\end{greek}%
\switchcolumn*

Es ist mein Mann. 

\switchcolumn

\begin{greek}[variant=ancient]%
ἐστὶν οὑμὸς ἀνήρ.

\end{greek}%
\switchcolumn*

Warum macht er ein so verdrießliches Gesicht? 

\switchcolumn

\begin{greek}[variant=ancient]%
τί σκυθρωπάζει;

\end{greek}%
\switchcolumn*

Er ist sehr eifersüchtig. 

\switchcolumn

\begin{greek}[variant=ancient]%
σφόδρα ζηλότυπός ἐστιν.

\end{greek}%
\switchcolumn*

Wir wollen gar nicht thun, als sähen wir ihn. 

\switchcolumn

\begin{greek}[variant=ancient]%
μὴ ὁρᾶν δοκῶμεν αὐτόν.

\end{greek}%
\switchcolumn*

Ich werde mich hüten! 

\switchcolumn

\begin{greek}[variant=ancient]%
φυλάξομαι\footnote{\begin{latin}%
orig. \textgreek[variant=ancient]{φυλάξομαί}\end{latin}%
}!

\end{greek}%
\switchcolumn*

Den Männern ist ja nicht zu trauen! 

\switchcolumn

\begin{greek}[variant=ancient]%
οὐδὲν γὰρ πιστὸν τοῖς ἀνδράσιν.

\end{greek}%
\switchcolumn*

Sie ist erst 3 Monate verheirathet. 

\switchcolumn

\begin{greek}[variant=ancient]%
νύμφη ἐστὶ τρεῖς μῆνας.

\end{greek}%
\switchcolumn*

Der Tanzlehrer. 

\switchcolumn

\begin{greek}[variant=ancient]%
ὁ ὀρχηστοδιδάσκαλος.

\end{greek}%
\switchcolumn*

In die Tanzstunde. 

\switchcolumn

\begin{greek}[variant=ancient]%
εἰς τὸ ὀρχηστοδιδασκαλεῖον.

\end{greek}%
\switchcolumn*[


\section{Eine Geschichte}

]Hören Sie einmal zu, gnädige Frau, ich will Ihnen eine hübsche Geschichte
erzählen. 

\switchcolumn

\begin{greek}[variant=ancient]%
ἄκουσον, ὦ γύναι, λόγον σοι βούλομαι λέξαι χαρίεντα.

\end{greek}%
\switchcolumn*

Nur zu, erzählen Sie! 

\switchcolumn

\begin{greek}[variant=ancient]%
ἴθι\footnote{\begin{german}[spelling=old,babelshorthands=true]%
orig. \textgreek[variant=ancient]{ιθι}\end{german}%
} δὴ, λέξον.

\end{greek}%
\switchcolumn*

Ist das wahr? 

\switchcolumn

\begin{greek}[variant=ancient]%
τί λέγεις;

\end{greek}%
\switchcolumn*

Sie wundern sich? 

\switchcolumn

\begin{greek}[variant=ancient]%
ἐθαύ\textit{μασας;}

\end{greek}%
\switchcolumn*

Sie erzählen mir (erfundene) Geschichten! 

\switchcolumn

\begin{greek}[variant=ancient]%
μύθους μοι λέγεις!

\end{greek}%
\switchcolumn*

Die Wahrheit wollen Sie doch nicht sagen! 

\switchcolumn

\begin{greek}[variant=ancient]%
τἀληθὲς γὰρ οὐκ ἐθέλεις φράσαι.

\end{greek}%
\switchcolumn*

Wenn Sie wirklich die Wahrheit sprechen, so weiß ich nicht was ich
sagen soll. 

\switchcolumn

\begin{greek}[variant=ancient]%
εἴπερ ὄντως σὺ\footnote{\begin{german}[spelling=old,babelshorthands=true]%
orig. \textgreek[variant=ancient]{συ}\end{german}%
} ταῦτ' ἀληθῆ λέγεις, οὐδὲν ἔχω εἰπεῖν.

\end{greek}%
\switchcolumn*

Nach dem, was Sie sagen, muß man sie bewundern. 

\switchcolumn

\begin{greek}[variant=ancient]%
κατὰ τὸν λόγον, ὃν σὺ λέγεις, ἀξία ἐστὶ θαυμάσαι.

\end{greek}%
\switchcolumn*

Reden Sie mit ihr \emph{von} der Sache! 

\switchcolumn

\begin{greek}[variant=ancient]%
λέγ' αὐτῇ τὸ πρᾶγμα.

\end{greek}%
\switchcolumn*

Sagen = angeben.

\switchcolumn

\begin{greek}[variant=ancient]%
φράζειν.

\end{greek}%
\switchcolumn*

Was hat sie darauf erwidert? 

\switchcolumn

\begin{greek}[variant=ancient]%
τί πρὸς ταῦτα εἶπεν;

\end{greek}%
\switchcolumn*

Sie macht Aus\textcompwordmark{}flüchte.

\switchcolumn

\begin{greek}[variant=ancient]%
προφασίζεσται.

\end{greek}%
\switchcolumn*

Ich will euch ein Märchen erzählen nämlich — 

\switchcolumn

\begin{greek}[variant=ancient]%
μῦθον ὑμῖν βούλομαι λέξαι οὕτως\footnote{\begin{german}[spelling=old,babelshorthands=true]%
orig. \textgreek[variant=ancient]{ουτως}\end{german}%
}.

\end{greek}%
\switchcolumn*[


\section{Ich weiß nicht}

]Ich weiß es nicht. 

\switchcolumn

\begin{greek}[variant=ancient]%
οὐκ οἶδα.

\end{greek}%
\switchcolumn*

Ich kann es nicht sagen. 

\switchcolumn

\begin{greek}[variant=ancient]%
οὐκ ἔχω φράσαι.

\end{greek}%
\switchcolumn*

Worauf soll man rathen? 

\switchcolumn

\begin{greek}[variant=ancient]%
ποῖ τις ἂν τράποιτο;

\end{greek}%
\switchcolumn*

Ich will es schon heraus\textcompwordmark{}bekommen.

\switchcolumn

\begin{greek}[variant=ancient]%
γνώσομαι ἔγωγε.

\end{greek}%
\switchcolumn*

Ich weiß es nicht genau. 

\switchcolumn

\begin{greek}[variant=ancient]%
οὐκ οἶδ' ἀκριβῶς.

\end{greek}%
\switchcolumn*

Nein, soviel ich weiß. 

\switchcolumn

\begin{greek}[variant=ancient]%
οὐχ, ὅσον γέ μ' εἰδέναι.

\end{greek}%
\switchcolumn*

Ich weiß nicht sicher, wie es steht. 

\switchcolumn

\begin{greek}[variant=ancient]%
οὐ σάφ' οἶδα, ὅπως ἔχει.

\end{greek}%
\switchcolumn*

Ich kann es nicht glauben. 

\switchcolumn

\begin{greek}[variant=ancient]%
οὐ πείθομαι.

\end{greek}%
\switchcolumn*

Ich weiß es ja. 

\switchcolumn

\begin{greek}[variant=ancient]%
οἶδά τοι.

\end{greek}%
\switchcolumn*

Ist mir bekannt! 

\switchcolumn

\begin{greek}[variant=ancient]%
μεμνήμεθα!

\end{greek}%
\switchcolumn*

Freilich weiß ich es! 

\switchcolumn

\begin{greek}[variant=ancient]%
οἶδα μέντοι!

\end{greek}%
\switchcolumn*

Da Sie es denn zu wissen verlangen, so will ich es sagen. 

\switchcolumn

\begin{greek}[variant=ancient]%
εἰ δὴ ἐπιθυμεῖς εἰδέναι, φράσω.

\end{greek}%
\switchcolumn*

Wär's möglich? 

\switchcolumn

\begin{greek}[variant=ancient]%
τί φής!

\end{greek}%
\switchcolumn*

Ich habe es aus bester Quelle. 

\switchcolumn

\begin{greek}[variant=ancient]%
πέπυσμαι τοῦτο τῶν σάφ' εἰδότων.

\end{greek}%
\switchcolumn*

Haben Sie bereits etwas von der Sache gehört? 

\switchcolumn

\begin{greek}[variant=ancient]%
ἆρ' ἀκήκοάς τι τοῦ πράγματος;

\end{greek}%
\switchcolumn*

Das mußte ich (bis\textcompwordmark{}her noch) nicht.

\switchcolumn

\begin{greek}[variant=ancient]%
τοῦτ' οὐκ ᾔδειν ἐγώ.

\end{greek}%
\switchcolumn*

\emph{O, dann begreife ich, daß} Sie verstimmt sind.

\switchcolumn

\begin{greek}[variant=ancient]%
\emph{οὐκ ἐτὸς ἄρα} λυπεῖ.

\end{greek}%
\switchcolumn*[


\section{Die Schöne und die Häßliche}

]Sehen Sie die hier an, wie \emph{schön} sie ist! 

\switchcolumn

\begin{greek}[variant=ancient]%
ὅρα ταυτηνὶ, ὡς καλή!

\end{greek}%
\switchcolumn*

Wer ist wohl dort die Dame? 

\switchcolumn

\begin{greek}[variant=ancient]%
τίς ποθ' αὑτηί;

\end{greek}%
\switchcolumn*

Die in dem grauen Kleide? 

\switchcolumn

\begin{greek}[variant=ancient]%
ἡ τὸ φαιὸν ἔνδυμα ἀμπεχομένη;

\end{greek}%
\switchcolumn*

Sie ist die schönste (= blühendste) von allen. 

\switchcolumn

\begin{greek}[variant=ancient]%
πασῶν \emph{ὡραιοτάτη} ἐστίν.

\end{greek}%
\switchcolumn*

Wer mag sie nur sein? 

\switchcolumn

\begin{greek}[variant=ancient]%
τίς καί ἐστί ποτε;

\end{greek}%
\switchcolumn*

Kennt sie Jemand von Ihnen? 

\switchcolumn

\begin{greek}[variant=ancient]%
\emph{γιγνώσκει} τις ὑμῶν;

\end{greek}%
\switchcolumn*

Ja, ich. 

\switchcolumn

\begin{greek}[variant=ancient]%
νὴ Δία ἔγωγε.

\end{greek}%
\switchcolumn*

Es ist meine Cousine. 

\switchcolumn

\begin{greek}[variant=ancient]%
ἐστὶν ἀνεψιά μου.

\end{greek}%
\switchcolumn*

Wie schön sie aus\textcompwordmark{}sieht! 

\switchcolumn

\begin{greek}[variant=ancient]%
οἷον τὸ κάλλος αὐτῆς φαίνεται!

\end{greek}%
\switchcolumn*

Sie hat sehr gesunde Farbe. 

\switchcolumn

\begin{greek}[variant=ancient]%
ὡς εὐχροεῖ!

\end{greek}%
\switchcolumn*

Sie hat ein sanftes, schönes Auge. 

\switchcolumn

\begin{greek}[variant=ancient]%
καὶ τὸ βλέμμα ἔχει μαλακὸν καὶ καλόν.

\end{greek}%
\switchcolumn*

Und allerliebste Hände hat sie. 

\switchcolumn

\begin{greek}[variant=ancient]%
καὶ \emph{τὰς} χεῖρας παγκάλας ἔχει.

\end{greek}%
\switchcolumn*

Sie lacht \emph{gern.}

\switchcolumn

\begin{greek}[variant=ancient]%
καὶ ἡδέως γελᾷ.

\end{greek}%
\switchcolumn*

Ich bin in das Mädchen (die Dame) verliebt. 

\switchcolumn

\begin{greek}[variant=ancient]%
ἔρως με εἴληφε τῆς κόρης ταύτης.

\end{greek}%
\switchcolumn*

Aber sie hat wohl nichts? 

\switchcolumn

\begin{greek}[variant=ancient]%
ἀλλ' ἔχει οὐδέν;

\end{greek}%
\switchcolumn*

O nein, sie ist reich; sie hat ein respectables Vermögen. 

\switchcolumn

\begin{greek}[variant=ancient]%
πλουτεῖ \emph{μὲν οὖν·} οὐσίαν γὰρ ἔχει συχνήν.

\end{greek}%
\switchcolumn*

Weißt du, wem sie ganz ähnlich sieht? Der A. 

\switchcolumn

\begin{greek}[variant=ancient]%
οὖσθ' ᾗ μάλιστ' ἔοικεν; τῇ Ἀ.

\end{greek}%
\switchcolumn*

Dort ist ein schönes Mädchen! (Mädel!) 

\switchcolumn

\begin{greek}[variant=ancient]%
ἐνταῦθα μείραξ ὡραία ἐστίν.

\end{greek}%
\switchcolumn*

Wer ist denn die hinter ihr? 

\switchcolumn

\begin{greek}[variant=ancient]%
τίς γάρ ἐσθ' ἡ ὄπισθεν αὐτῆς.

\end{greek}%
\switchcolumn*

Wer die ist? Frau Schulze. 

\switchcolumn

\begin{greek}[variant=ancient]%
ἥτις ἐστίν; Σχουλζίου γυνή.

\end{greek}%
\switchcolumn*

Die Andere interessirt mich weniger. 

\switchcolumn

\begin{greek}[variant=ancient]%
τῆς ἑτέρας μοι ἧττον μέλει.

\end{greek}%
\switchcolumn*

Sie ist häßlich. 

\switchcolumn

\begin{greek}[variant=ancient]%
αἰσχρὰ γάρ ἐστιν.

\end{greek}%
\switchcolumn*

Und hat eine stumpfe (kolbige) Nase. 

\switchcolumn

\begin{greek}[variant=ancient]%
καὶ σιμή (ἐστιν).

\end{greek}%
\switchcolumn*

Sie ist geschminkt. 

\switchcolumn

\begin{greek}[variant=ancient]%
καὶ καταπεπλασμένη (ἐστίν).

\end{greek}%
\switchcolumn*

sie riecht nach Pomade. 

\switchcolumn

\begin{greek}[variant=ancient]%
ὄζει δὲ μύρου.

\end{greek}%
\switchcolumn*

Riechst du etwas? 

\switchcolumn

\begin{greek}[variant=ancient]%
ὀσφραίνει τι;

\end{greek}%
\switchcolumn*

Die Pomade riecht nicht gut. 

\switchcolumn

\begin{greek}[variant=ancient]%
οὐχ ἡδὺ τὸ μύρον τουτί.

\end{greek}%
\switchcolumn*[


\section{Herr Schulze}

]Schulze heißt er? \emph{Was ist das für ein} Schulze? 

\switchcolumn

\begin{greek}[variant=ancient]%
Σχούλζιος αὐτῷ ὄνομα; ποῖος οὗτος ὁ Σχούλζιος;

\end{greek}%
\switchcolumn*

Kennen Sie ihn nicht? 

\switchcolumn

\begin{greek}[variant=ancient]%
οὐκ οἶσθα αὐτόν;

\end{greek}%
\switchcolumn*

Nein, ich bin fremd hier und erst eben angekommen. 

\switchcolumn

\begin{greek}[variant=ancient]%
οὐ μὰ Δία ἔγωγε, ξένος γάρ εἰμι ἀρτίως ἀφιγμένος.

\end{greek}%
\switchcolumn*

Er spielt die erste Rolle in der Stadt. 

\switchcolumn

\begin{greek}[variant=ancient]%
πράττει τὰ μέγιστα ἐν τῇ πόλει.

\end{greek}%
\switchcolumn*

Er hat einen \emph{großen} Bart. 

\switchcolumn

\begin{greek}[variant=ancient]%
ἔχει δὲ πώγωνα.

\end{greek}%
\switchcolumn*

Und graues Haar? 

\switchcolumn

\begin{greek}[variant=ancient]%
καὶ πολιός ἐστιν;

\end{greek}%
\switchcolumn*

Wovon lebt er? 

\switchcolumn

\begin{greek}[variant=ancient]%
πόθεν διαζῇ;

\end{greek}%
\switchcolumn*

Der Mann ist schnell reich geworben. 

\switchcolumn

\begin{greek}[variant=ancient]%
ταχέως ὁ ἀνὴρ γεγένηται πλούσιος.

\end{greek}%
\switchcolumn*

Wodurch? 

\switchcolumn

\begin{greek}[variant=ancient]%
τί δρῶν;

\end{greek}%
\switchcolumn*

Er hat ursprünglich ein Handwerk gelernt, dann wurde er Landwirth
und jetzt ist er Kaufmann. 

\switchcolumn

\begin{greek}[variant=ancient]%
πρῶτον μὲν γὰρ τέχνην τιν' ἔμαθεν· εἶτα γεωργὸς ἐγένετο, νῦν δὲ ἔμπορός
ἐστιν.

\end{greek}%
\switchcolumn*\bgroup

\myafterpagetrue\mysetaligntext{german}{Es ist{ }}\mysetalign{german}Fabrikant.

\egroup\switchcolumn\bgroup

\begin{greek}[variant=ancient]%
ἐργαστήριον ἔχει.

\end{greek}%
\egroup\switchcolumn*\bgroup

\mysetalign{german}Arbeiter.

\egroup\switchcolumn\bgroup

\begin{greek}[variant=ancient]%
ἐργάτης

\end{greek}%
\egroup\switchcolumn*\bgroup

\mysetalign{german}(Amts- etc.) Richter. 

\egroup\switchcolumn\bgroup

\begin{greek}[variant=ancient]%
δικαστής.

\end{greek}%
\egroup\switchcolumn*\bgroup

\mysetalign{german}Unterbeamter. 

\egroup\switchcolumn\bgroup

\begin{greek}[variant=ancient]%
ὑπάλληλος.

\end{greek}%
\egroup\switchcolumn*\bgroup

\mysetalign{german}Rechts\textcompwordmark{}anwalt. 

\egroup\switchcolumn\bgroup

\begin{greek}[variant=ancient]%
σύνδικος.

\end{greek}%
\egroup\switchcolumn*\bgroup

\mysetalign{german}Apotheker. 

\egroup\switchcolumn\bgroup

\begin{greek}[variant=ancient]%
φαρμακοπώλης.

\end{greek}%
\egroup\switchcolumn*\bgroup

\mysetalign{german}Banquier. 

\egroup\switchcolumn\bgroup

\begin{greek}[variant=ancient]%
τραπεζίτης.

\end{greek}%
\egroup\switchcolumn*\bgroup

\mysetalign{german}Officier. 

\egroup\switchcolumn\bgroup

\begin{greek}[variant=ancient]%
ἀξιωματικός.

\end{greek}%
\egroup\switchcolumn*\bgroup

\mysetalign{german}Schüler. 

\egroup\switchcolumn\bgroup

\begin{greek}[variant=ancient]%
μαθητής.

\end{greek}%
\egroup\switchcolumn*\bgroup

\mysetalign{german}Student. 

\egroup\switchcolumn\bgroup

\begin{greek}[variant=ancient]%
φοιτητής.

\end{greek}%
\egroup\switchcolumn*\bgroup

\mysetalign{german}Lehrer. 

\egroup\switchcolumn\bgroup

\begin{greek}[variant=ancient]%
διδάσκαλος.

\end{greek}%
\egroup\switchcolumn*\bgroup

\mysetalign{german}Professor. 

\egroup\switchcolumn

\begin{greek}[variant=ancient]%
καθηγητής.

\end{greek}%
\switchcolumn*

Er ist vom Lande. 

\switchcolumn

\begin{greek}[variant=ancient]%
ἐκ τῶν ἀγρῶν ἐστιν.

\end{greek}%
\switchcolumn*

Er ist aus der Nachbarschaft. 

\switchcolumn

\begin{greek}[variant=ancient]%
ἐκ τῶν γειτόνων ἐστίν.

\end{greek}%
\switchcolumn*

Mir ist er langweilig. 

\switchcolumn

\begin{greek}[variant=ancient]%
ἄχθομαι αὐτῷ συνὼν ἔγωγε.

\end{greek}%
\switchcolumn*

Er ist nicht schlecht von Charakter. 

\switchcolumn

\begin{greek}[variant=ancient]%
οὐ πονηρός ἐστι τοὺς τρόπους.

\end{greek}%
\switchcolumn*

(Seht nur) wie protzig er hereingekommen ist! 

\switchcolumn

\begin{greek}[variant=ancient]%
\emph{ὡς σοβαρὸς} εἰσελήλυθεν!

\end{greek}%
\switchcolumn*

Es scheint mir nicht guter Ton zu sein, sich so zu betragen. 

\switchcolumn

\begin{greek}[variant=ancient]%
οὐκ ἀστεῖόν μοι δοκεῖ εἶναι τοιτοῦτον ἑαυτὸν παρέχειν.

\end{greek}%
\switchcolumn*

Aber N. N. ist wirklich ein Gentleman. 

\switchcolumn

\begin{greek}[variant=ancient]%
ὁ δὲ Ν. Ν. νὴ Δία γεννάδας ἀνήρ!

\end{greek}%
\switchcolumn*[


\section{Wie alt?}

]Er hat nur eine einzige Tochter. 

\switchcolumn

\begin{greek}[variant=ancient]%
θυγάτηρ αὐτῷ μόνη οὖσα τυγχάνει.

\end{greek}%
\switchcolumn*

Wie alt ist sie? 

\switchcolumn

\begin{greek}[variant=ancient]%
πηλίκη ἐστίν;

\end{greek}%
\switchcolumn*

Sie ist über ein Jahr älter als du. 

\switchcolumn

\begin{greek}[variant=ancient]%
πλεῖν ἢ 'νιαυτῷ σου πρεσβυτέρα ἐστίν.

\end{greek}%
\switchcolumn*

Über 20 Jahre \emph{alt}. 

\switchcolumn

\begin{greek}[variant=ancient]%
ὑπὲρ εἴκοσιν ἔτη \emph{γεγονυῖα.}

\end{greek}%
\switchcolumn*

Du bist ein junger Mann \emph{von} 19 Jahren.

\switchcolumn

\begin{greek}[variant=ancient]%
σὺ δὲ ἀνὴρ νέος εἶ ἐννεακαίδεκα ἐτῶν.

\end{greek}%
\switchcolumn*

Du mußt mit denen unter zwanzig tanzen. 

\switchcolumn

\begin{greek}[variant=ancient]%
δεῖ οὖν ὀρχεῖσθαί σε μετὰ τῶν \emph{ἐντὸς} εἴκοσιν.

\end{greek}%
\switchcolumn*

Sie sitzt dort bei den älteren Damen. 

\switchcolumn

\begin{greek}[variant=ancient]%
ἐνταῦτα κάθηται παρὰ ταῖς πρεσβυτέραις γυναιξίν.

\end{greek}%
\switchcolumn*

Wo? zeig' einmal! 

\switchcolumn

\begin{greek}[variant=ancient]%
τοῦ; δεῖξον!

\end{greek}%
\switchcolumn*

Was hat sie für Toilette? 

\switchcolumn

\begin{greek}[variant=ancient]%
ποίαν τιν' ἔχει σκευήν;

\end{greek}%
\switchcolumn*

Ihre Mutter ist \emph{seit} 10 Jahren todt.

\switchcolumn

\begin{greek}[variant=ancient]%
τέθνηκεν ἡ μήτηρ αὐτῆς ἔτη δέκα.

\end{greek}%
\switchcolumn*

Ihr Vater ist ein Sechziger. 

\switchcolumn

\begin{greek}[variant=ancient]%
ἑξηκοντέτης ἐστὶν αὐτῆς ὁ πατήρ.

\end{greek}%
\switchcolumn*

Die Familie. 

\switchcolumn

\begin{greek}[variant=ancient]%
ὁ οἶκος.

\end{greek}%

	\switchcolumn*[


\part{Liebes\textcompwordmark{}glück und Liebes\textcompwordmark{}meh.}


\section{Liebes\textcompwordmark{}sehnsucht}

]Wie denken Sie über das Mädel? 

\switchcolumn

\begin{greek}[variant=ancient]%
τί οὖν\footnote{\begin{german}[spelling=old,babelshorthands=true]%
orig. \textgreek[variant=ancient]{ουν}\end{german}%
} ἐρεῖς περὶ τῆς μείρακος;

\end{greek}%
\switchcolumn*

Alles nichts gegen meine Anna! 

\switchcolumn

\begin{greek}[variant=ancient]%
λῆρός ἐστι τἆλλα πρὸς Ἄνναν.

\end{greek}%
\switchcolumn*

Die Sehnsucht nach Anna quält mich. 

\switchcolumn

\begin{greek}[variant=ancient]%
ἵμερός με (\textgerman[spelling=old,babelshorthands=true]{od.} πόθος
με) διαλυμαίνεται Ἄννης.

\end{greek}%
\switchcolumn*

Im Ernst? 

\switchcolumn

\begin{greek}[variant=ancient]%
ὢ τί λέτεις;

\end{greek}%
\switchcolumn*

Du wunderst dich? 

\switchcolumn

\begin{greek}[variant=ancient]%
ἐθαύ\emph{μασας};

\end{greek}%
\switchcolumn*

Warum wunderst du dich? 

\switchcolumn

\begin{greek}[variant=ancient]%
τί ἐθαύμασας;

\end{greek}%
\switchcolumn*

Wie schmerzlich für mich, daß sie nicht da ist! 

\switchcolumn

\begin{greek}[variant=ancient]%
ὡς ἄχθομαι αὐτῆς ἀπούσης!

\end{greek}%
\switchcolumn*

Sei kein Thor! 

\switchcolumn

\begin{greek}[variant=ancient]%
μὴ ἄφρων γένῃ!

\end{greek}%
\switchcolumn*

Die Zeit wird mir lang, weil ich das herrliche Mädchen nicht sehe. 

\switchcolumn

\begin{greek}[variant=ancient]%
πάνυ πολύς μοι δοκεῖ εἶναι χρόνος, ὅτι οὐχ ὁρῶ αὑτὴν τοιαύτην οὖσαν.

\end{greek}%
\switchcolumn*

Sie ist nicht hier. 

\switchcolumn

\begin{greek}[variant=ancient]%
οὐκ ἐνθάδε ἐστίν.

\end{greek}%
\switchcolumn*

Aber sie ist schon auf dem Wege. 

\switchcolumn

\begin{greek}[variant=ancient]%
ἀλλ' ἔρχεται.

\end{greek}%
\switchcolumn*

Da kommt sie! 

\switchcolumn

\begin{greek}[variant=ancient]%
ἡδὶ προσέρχεται!

\end{greek}%
\switchcolumn*

Jetzt sehe ich sie endlich. 

\switchcolumn

\begin{greek}[variant=ancient]%
νῦν\footnote{\begin{german}[spelling=old,babelshorthands=true]%
orig. \textgreek[variant=ancient]{νὖν}\end{german}%
} γε ἤδη καθορῶ αὐτήν.

\end{greek}%
\switchcolumn*

Sie ist schon ziemlich lange da. 

\switchcolumn

\begin{greek}[variant=ancient]%
\emph{ἥκει} ἐπιεικῶς πάλαι.

\end{greek}%
\switchcolumn*

Das ist unerhört! 

\switchcolumn

\begin{greek}[variant=ancient]%
ἄτοπον τουτὶ πρᾶγμα!

\end{greek}%
\switchcolumn*

Was fällt dir ein? 

\switchcolumn

\begin{greek}[variant=ancient]%
τί πάσχεις;

\end{greek}%
\switchcolumn*

Siehst du nicht? N. lauft ihr nach. Er begrüßt sie angelegentlich! 

\switchcolumn

\begin{greek}[variant=ancient]%
οὐχ ὁρᾶς; Ν. ἀκολουθεῖ κατόπιν αὐτῆς καὶ ἀσπάζεται!

\end{greek}%
\switchcolumn*

Das interessirt mich wenig. 

\switchcolumn

\begin{greek}[variant=ancient]%
ὀλίγον μοι μέλει.

\end{greek}%
\switchcolumn*

Sie reicht ihm die Hand! 

\switchcolumn

\begin{greek}[variant=ancient]%
ἡ δὲ δεξιοῦται αὐτόν.

\end{greek}%
\switchcolumn*

Ach, ich Ärmster! 

\switchcolumn

\begin{greek}[variant=ancient]%
οἴμοι κακοδαίμων.

\end{greek}%
\switchcolumn*

Sie scheint dich nicht zu sehen. 

\switchcolumn

\begin{greek}[variant=ancient]%
οὐ δοκεῖ ὁρᾶν σε.

\end{greek}%
\switchcolumn*

Sie hat ihm die Hand gegeben. 

\switchcolumn

\begin{greek}[variant=ancient]%
ἐνέβελε τὴν δεξιάν.

\end{greek}%
\switchcolumn*

Kümmere dich nicht weiter um sie! 

\switchcolumn

\begin{greek}[variant=ancient]%
ταύτην μὲν \emph{ἔα χαίρειν}!

\end{greek}%
\switchcolumn*

Ich gehe. Ich will meine Tante begrüßen. 

\switchcolumn

\begin{greek}[variant=ancient]%
ἀλλ' εἶμι· προσερῶ γὰρ τὴν τεθίδα.

\end{greek}%
\switchcolumn*

Ich habe sie bereits begrüßt. 

\switchcolumn

\begin{greek}[variant=ancient]%
ἐγὼ δὲ προσείρηκα αὐτήν.

\end{greek}%
\switchcolumn*

Das ist gar \emph{nicht schön} von Ihnen, \emph{daß} Sie mich nicht
begrüßt haben.

\switchcolumn

\begin{greek}[variant=ancient]%
καλῶς γε οὐ προσεῖπάς με! \textgerman[spelling=old,babelshorthands=true]{(ironisch.)}

\end{greek}%
\switchcolumn*[


\section{Soll ich?}

]Was gedenken Sie zu thun? 

\switchcolumn

\begin{greek}[variant=ancient]%
τί ποιεῖν διανοεῖ;

\end{greek}%
\switchcolumn*

Was haben Sie vor? 

\switchcolumn

\begin{greek}[variant=ancient]%
τί μέλλεις δρᾶν;

\end{greek}%
\switchcolumn*

Geben Sie mir einen guten Rath! 

\switchcolumn

\begin{greek}[variant=ancient]%
χρηστόν τι συμβούλευσον!

\end{greek}%
\switchcolumn*

Was soll ich machen? 

\switchcolumn

\begin{greek}[variant=ancient]%
τί ποιήσω;

\end{greek}%
\switchcolumn*

Ich fürchte, Sie werden es bereuen. 

\switchcolumn

\begin{greek}[variant=ancient]%
\emph{οἶμαί} σοι τοῦτο μεταμελήσειν.

\end{greek}%
\switchcolumn*

Sehen Sie sich vor, daß sie Ihnen nicht entgeht. 

\switchcolumn

\begin{greek}[variant=ancient]%
εὐλαβοῦ, μὴ ἐκφύγῃ σ' ἐκείνη.

\end{greek}%
\switchcolumn*

Jetzt ist es an Ihnen, das Weitere zu thun. 

\switchcolumn

\begin{greek}[variant=ancient]%
\emph{σὸν ἔργον} τἆλλα ποιεῖν.

\end{greek}%
\switchcolumn*

Was soll ich also? 

\switchcolumn

\begin{greek}[variant=ancient]%
τί οὖν κελεύεις δρᾶν με;

\end{greek}%
\switchcolumn*

Sie müssen mit ihr sprechen, sobald sich Gelegenheit bietet. 

\switchcolumn

\begin{greek}[variant=ancient]%
δεῖ διαλέγεσθαι αὐτῇ, ὅταν τύχῃς.

\end{greek}%
\switchcolumn*

Gerade das will ich ja! 

\switchcolumn

\begin{greek}[variant=ancient]%
τοῦτ' αὐτὸ γὰρ καὶ βούλομαι.

\end{greek}%
\switchcolumn*

Aber soweit ist die Sache noch nicht. 

\switchcolumn

\begin{greek}[variant=ancient]%
ἀλλ' οὐκ ἔστι πω ἐν τούτῳ τὰ πράγματα.

\end{greek}%
\switchcolumn*

Die Sache hat einen Haken. 

\switchcolumn

\begin{greek}[variant=ancient]%
ἔνι κίνδυνος ἐν τῷ πράγματι.

\end{greek}%
\switchcolumn*

Ein schwieriger Punkt! 

\switchcolumn

\begin{greek}[variant=ancient]%
χαλεπὸν τὸ πρᾶγμα!

\end{greek}%
\switchcolumn*

Machen Sie sich keine Sorge! 

\switchcolumn

\begin{greek}[variant=ancient]%
μὴ φροντίσῃς.

\end{greek}%
\switchcolumn*

Nur nicht ängstlich! 

\switchcolumn

\begin{greek}[variant=ancient]%
μὴ δέδιθι.

\end{greek}%
\switchcolumn*

Haben Sie keine Angst, mein Bester! 

\switchcolumn

\begin{greek}[variant=ancient]%
μηδὲν δέδιτι, ὦ τᾶν\footnote{\begin{german}[spelling=old,babelshorthands=true]%
orig. \textgreek[variant=ancient]{τάν}\end{german}%
}.

\end{greek}%
\switchcolumn*

Es wird Ihnen nichts passiren. 

\switchcolumn

\begin{greek}[variant=ancient]%
οὐδὲν (γὰρ) πείσει.

\end{greek}%
\switchcolumn*

An mir soll es nicht liegen. 

\switchcolumn

\begin{greek}[variant=ancient]%
οὐ τοὐμὸν ἐμποδὼν ἔσται, ὦ τᾶν\footnote{\begin{german}[spelling=old,babelshorthands=true]%
orig. \textgreek[variant=ancient]{τάν}\end{german}%
}.

\end{greek}%
\switchcolumn*

Das will ich schon besorgen. 

\switchcolumn

\begin{greek}[variant=ancient]%
μελήσει μοι τοῦτό γε.

\end{greek}%
\switchcolumn*[


\section{Nur Muth!}

]Beeilen Sie sich! 

\switchcolumn

\begin{greek}[variant=ancient]%
σπεῦδέ νυν! ἔπειγέ νυν!

\end{greek}%
\switchcolumn*

So beeilen Sie sich doch! 

\switchcolumn

\begin{greek}[variant=ancient]%
οὔκουν ἐπείξει;

\end{greek}%
\switchcolumn*

Zögern Sie nicht! 

\switchcolumn

\begin{greek}[variant=ancient]%
μὴ βράδυνε!

\end{greek}%
\switchcolumn*

Machen Sie schnell! 

\switchcolumn

\begin{greek}[variant=ancient]%
\emph{ἄνυε!}

\end{greek}%
\switchcolumn*

So machen Sie doch schnell! 

\switchcolumn

\begin{greek}[variant=ancient]%
οὐκ ἀνύσεις;

\end{greek}%
\switchcolumn*

Sie dürfen nicht zögern. 

\switchcolumn

\begin{greek}[variant=ancient]%
οὐ μέλλειν χρή σε.

\end{greek}%
\switchcolumn*

Wir wollen uns nicht aufhalten. 

\switchcolumn

\begin{greek}[variant=ancient]%
μὴ διατρίβωμεν.

\end{greek}%
\switchcolumn*

So halten Sie sich doch nicht auf! 

\switchcolumn

\begin{greek}[variant=ancient]%
οὐ μὴ διατρίψεις;

\end{greek}%
\switchcolumn*

Jetzt gilt es! 

\switchcolumn

\begin{greek}[variant=ancient]%
νῦν ὁ καιρός!

\end{greek}%
\switchcolumn*

Nun so versuchen Sie es doch wenigstens! 

\switchcolumn

\begin{greek}[variant=ancient]%
ἀλλ᾽ οὖν πεπειράσθω γε.

\end{greek}%
\switchcolumn*

Auf Ihre Verantwortung hin will ich's thun. 

\switchcolumn

\begin{greek}[variant=ancient]%
δράσω τοίνυν σοὶ πίσυνος.

\end{greek}%
\switchcolumn*

Ich will es versuchen. 

\switchcolumn

\begin{greek}[variant=ancient]%
πειράσομαι.

\end{greek}%
\switchcolumn*

Und wenn es den Kopf kostet! 

\switchcolumn

\begin{greek}[variant=ancient]%
κἂν δέῃ μ᾽ ἀποθανεῖν!

\end{greek}%
\switchcolumn*

Ich bin schon darüber. 

\switchcolumn

\begin{greek}[variant=ancient]%
ἀλλὰ δρῶ τοῦτο.

\end{greek}%
\switchcolumn*

Endlich ist es so weit! 

\switchcolumn

\begin{greek}[variant=ancient]%
ἤδη ᾽στὶ τοῦτ᾽ ἐκεῖνο!

\end{greek}%
\switchcolumn*

Und wenn sie Nein sagt und nicht will? 

\switchcolumn

\begin{greek}[variant=ancient]%
κἂν μὴ φῇ μηδ᾽ ἐθελήσῃ;

\end{greek}%
\switchcolumn*

Wir werden gleich sehen. 

\switchcolumn

\begin{greek}[variant=ancient]%
εἰσόμεθ᾽ αὐτίκα.

\end{greek}%
\switchcolumn*

Ich will gleich einmal sehen. 

\switchcolumn

\begin{greek}[variant=ancient]%
ἐγὼ \emph{εἴσομαι.}

\end{greek}%
\switchcolumn*[


\section{Liebes\textcompwordmark{}glück}

]Ich verehre Sie. 

\switchcolumn

\begin{greek}[variant=ancient]%
ἐραστής εἰμι σός.

\end{greek}%
\switchcolumn*

Ist das wahr? 

\switchcolumn

\begin{greek}[variant=ancient]%
τί λέγεις;

\end{greek}%
\switchcolumn*

Warum sagen Sie das? 

\switchcolumn

\begin{greek}[variant=ancient]%
τί τοῦτο λέγεις;

\end{greek}%
\switchcolumn*

Weil ich Sie liebe. 

\switchcolumn

\begin{greek}[variant=ancient]%
ὁτιὴ φιλῶ λέγεις;

\end{greek}%
\switchcolumn*

Wenn Sie mich wirklich von Herzen lieben, so sprechen Sie mit meiner
Mutter. 

\switchcolumn

\begin{greek}[variant=ancient]%
εἴπερ ὄντως ἐκ τῆς καρδίας με φιλεῖς, πρόσειπε τἡν μητέρα μου.

\end{greek}%
\switchcolumn*

Erlauben Sie mir einen Kuß! 

\switchcolumn

\begin{greek}[variant=ancient]%
\emph{δός μοι} κύσαι. (δὸς κύσαι.)

\end{greek}%
\switchcolumn*

Geben Sie mir einen Kuß! Bitte bitte! 

\switchcolumn

\begin{greek}[variant=ancient]%
κύσον με, ἀντιβολῶ!

\end{greek}%
\switchcolumn*

Einen Kuß! 

\switchcolumn

\begin{greek}[variant=ancient]%
φέρε, σε κύσω!

\end{greek}%
\switchcolumn*

Ich weiß zwar gewiß, daß die Mutter darüber böse sein wird, aber Ihnen
zu Gefallen will ich es thun. 

\switchcolumn

\begin{greek}[variant=ancient]%
οἶδα μὲν σαφῶς, ὅτι ἡ μήτηρ ἀχθέσεται, σοῦ ἕνεκα τοῦτο δράσω.

\end{greek}%
\switchcolumn*

Hören Sie auf! 

\switchcolumn

\begin{greek}[variant=ancient]%
παῦε! παῦε!

\end{greek}%
\switchcolumn*

Wie glücklich bin ich! 

\switchcolumn

\begin{greek}[variant=ancient]%
ὡς ἥδομαι!

\end{greek}%
\switchcolumn*

Ach, daß mich nur dir Mutter nicht sieht! 

\switchcolumn

\begin{greek}[variant=ancient]%
οἴμοι, ἡ μήτηρ ὅπως μή μ᾽ ὄψεται!

\end{greek}%
\switchcolumn*

Wir sind ja allein (unter uns). 

\switchcolumn

\begin{greek}[variant=ancient]%
\emph{αὐτοὶ} γάρ ἐσμεν.

\end{greek}%
\switchcolumn*

Pst! Seien Sie still! 

\switchcolumn

\begin{greek}[variant=ancient]%
ἤ ἤ· σιώπα.

\end{greek}%
\switchcolumn*

Geben Sie mir die Hand! 

\switchcolumn

\begin{greek}[variant=ancient]%
δός μοι τὴν χεῖρα τὴν δεξιάν.

\end{greek}%
\switchcolumn*

Ich schwöre Ihnen ewige Treue! 

\switchcolumn

\begin{greek}[variant=ancient]%
οὐδέποτέ σ᾽ ἀπολείψειν φημί!

\end{greek}%
\switchcolumn*[


\section{Die Schwiegermutter}

]Was geht da vor? — Was ist das? 

\switchcolumn

\begin{greek}[variant=ancient]%
τί τὸ πρᾶγμα; — τουτὶ τί ἐστιν;

\end{greek}%
\switchcolumn*

Allmächtiger Gott! 

\switchcolumn

\begin{greek}[variant=ancient]%
ὦ Ζεῦ βασιλεῦ!

\end{greek}%
\switchcolumn*

Verwünscht! 

\switchcolumn

\begin{greek}[variant=ancient]%
οἴμοι κακοδαίμων!

\end{greek}%
\switchcolumn*

Wir sind verrathen! 

\switchcolumn

\begin{greek}[variant=ancient]%
προδεδόμεθα! 

\end{greek}%
\switchcolumn*

Hier ist der schändliche Mensch! 

\switchcolumn

\begin{greek}[variant=ancient]%
οὗτος ὁ πανοῦργος!

\end{greek}%
\switchcolumn*

Sind Sie verrückt? 

\switchcolumn

\begin{greek}[variant=ancient]%
τί ποιεῖς;

\end{greek}%
\switchcolumn*

Was fällt Ihnen ein? 

\switchcolumn

\begin{greek}[variant=ancient]%
τί πάσχεις;

\end{greek}%
\switchcolumn*

O Sie Abscheulicher! 

\switchcolumn

\begin{greek}[variant=ancient]%
ὦ βδελυρὲ σύ!

\end{greek}%
\switchcolumn*

Ereifern Sie sich nicht! 

\switchcolumn

\begin{greek}[variant=ancient]%
μὴ πρὸς ὀργήν!

\end{greek}%
\switchcolumn*

Das ist eine Sünde und Schande! 

\switchcolumn

\begin{greek}[variant=ancient]%
ἀνόσια ἐπάθομεν!

\end{greek}%
\switchcolumn*

\emph{Nein,} über diese Unverschämtheit!

\switchcolumn

\begin{greek}[variant=ancient]%
ἆρ᾽ οὐχ ὕβρις ταῦτ᾽ ἐστὶ πολλή;

\end{greek}%
\switchcolumn*

Hören Sie auf! 

\switchcolumn

\begin{greek}[variant=ancient]%
παῦε!

\end{greek}%
\switchcolumn*

Gehen Sie Ihrer Wege! 

\switchcolumn

\begin{greek}[variant=ancient]%
ἄπιθ᾽ ἐκποδών!

\end{greek}%
\switchcolumn*

Machen Sie, daß Sie hinaus\textcompwordmark{}kommen! 

\switchcolumn

\begin{greek}[variant=ancient]%
οὐκ εἶ θύραζε;

\end{greek}%
\switchcolumn*

Entfernen Sie sich doch! 

\switchcolumn

\begin{greek}[variant=ancient]%
οὐκ ἄπει δῆτα ἐκποδών;

\end{greek}%
\switchcolumn*

Gehen Sie zum Teufel! 

\switchcolumn

\begin{greek}[variant=ancient]%
ἐς κόρακας!

\end{greek}%
\switchcolumn*

Fort mit Ihnen! 

\switchcolumn

\begin{greek}[variant=ancient]%
ἄπερρε!

\end{greek}%
\switchcolumn*

Der Teufel soll Sie holen! 

\switchcolumn

\begin{greek}[variant=ancient]%
ἀπολεῖ κάκιστα!

\end{greek}%
\switchcolumn*

So gehen Sie doch zum Teufel! 

\switchcolumn

\begin{greek}[variant=ancient]%
οὐκ ἐς κόρακας;

\end{greek}%
\switchcolumn*

\vspace{0.5em}
Sie sind verrückt, Madame! 

\switchcolumn

\begin{tabular}{ll}
\rdelim\{{2}{1em}[] & \begin{greek}[variant=ancient]%
παραπαίεις, ὦ γύναι.\end{greek}%
\tabularnewline
 & \begin{greek}[variant=ancient]%
ὦ γύναι, ὡς παραπαίεις! \end{greek}%
\tabularnewline
\end{tabular}

\switchcolumn*

Sie beleidigen mich! 

\switchcolumn

\begin{greek}[variant=ancient]%
οἴμοι, ὡς ὑβρίζεις!

\end{greek}%
\switchcolumn*

Pfui! 

\switchcolumn

\begin{greek}[variant=ancient]%
αἰβοί!

\end{greek}%
\switchcolumn*

Das soll Ihnen nicht so hingehen! 

\switchcolumn

\begin{greek}[variant=ancient]%
οὔτοι καταπροίξει (τοῦτο δρῶν)!

\end{greek}%
\switchcolumn*

Das soll Ihnen schlecht bekommen! 

\switchcolumn

\begin{greek}[variant=ancient]%
οὐ χαιρήσεις.

\end{greek}%
\switchcolumn*

Das will ich Ihnen anstreichen! 

\switchcolumn

\begin{greek}[variant=ancient]%
ἐγώ σε παύσω τοῦ θράσους.

\end{greek}%
\switchcolumn*

Nun, so mäßigen Sie sich doch! 

\switchcolumn

\begin{greek}[variant=ancient]%
ἀλλ᾽ ἀνάσχου!

\end{greek}%
\switchcolumn*

Ist es nicht arg, daß Sie das thun? 

\switchcolumn

\begin{greek}[variant=ancient]%
οὐ δεινὸν δῆτά σε τοῦτο δράσαι;

\end{greek}%
\switchcolumn*

Das ist empörend! 

\switchcolumn

\begin{greek}[variant=ancient]%
οὐκ ἀνασχετὸν τοῦτο!

\end{greek}%
\switchcolumn*

Verwünscht! was soll ich thun? 

\switchcolumn

\begin{greek}[variant=ancient]%
οἴμοι, τί δράσω;

\end{greek}%
\switchcolumn*

Sehen Sie, was Sie gethan haben? 

\switchcolumn

\begin{greek}[variant=ancient]%
ὁρᾷς, ἃ δέδρακας;

\end{greek}%
\switchcolumn*

Sie sind schuld daran! 

\switchcolumn

\begin{greek}[variant=ancient]%
σὺ τούτων αἴτιος\footnote{\begin{latin}%
orig. \textgreek[variant=ancient]{αἰτιος}\end{latin}%
}.

\end{greek}%
\switchcolumn*[


\section{Wie ärgerlich!}

]Was hängst du den Kopf? 

\switchcolumn

\begin{greek}[variant=ancient]%
τί κύπτεις;

\end{greek}%
\switchcolumn*

Ich schäme mich. 

\switchcolumn

\begin{greek}[variant=ancient]%
αἰσχύνομαι.

\end{greek}%
\switchcolumn*

Die Frau hat dich in der That sehr schlecht behandelt. 

\switchcolumn

\begin{greek}[variant=ancient]%
αἴσχιστά τοί σ᾽ εἰργάσατο ἡ γυνή.

\end{greek}%
\switchcolumn*

Sie ist sehr böse auf uns. 

\switchcolumn

\begin{greek}[variant=ancient]%
ὀργὴν ἡμῖν ἔχει πολλήν.

\end{greek}%
\switchcolumn*

Das ist höchst ärgerlich für uns. 

\switchcolumn

\begin{greek}[variant=ancient]%
τοῦτ᾽ ἔστ᾽ ἄλγιστον ἡμῖν.

\end{greek}%
\switchcolumn*

Ich ärgere mich immer wieder, daß ich das gethan habe. 

\switchcolumn

\begin{greek}[variant=ancient]%
πόλλ᾽ ἄχθομαι, ὅτι ἔδρασα τοῦτο.

\end{greek}%
\switchcolumn*

Das hatte ich nicht erwartet. 

\switchcolumn

\begin{greek}[variant=ancient]%
τουτὶ μὰ Δί᾽ οὐδέποτ᾽ ἤλπισα.

\end{greek}%
\switchcolumn*

Knirsche nicht mit den Zähnen! 

\switchcolumn

\begin{greek}[variant=ancient]%
μὴ πρῖε τοὺς ὀδόντας!

\end{greek}%
\switchcolumn*

Das läßt sich nicht ändern. 

\switchcolumn

\begin{greek}[variant=ancient]%
ταῦτα μὲν δὴ ταῦτα.

\end{greek}%
\switchcolumn*

Sei nicht rachsüchtig! 

\switchcolumn

\begin{greek}[variant=ancient]%
μὴ μνησικακήσῃς.

\end{greek}%
\switchcolumn*

Es ist am besten, wir bleiben ruhig. 

\switchcolumn

\begin{greek}[variant=ancient]%
ἡσυχίαν ἄγειν βέλτιστόν ἐστιν.

\end{greek}%
\switchcolumn*

Das war ein Fehler von uns. 

\switchcolumn

\begin{greek}[variant=ancient]%
ἡμάρτομεν ταῦτα.

\end{greek}%
\switchcolumn*

Sei nicht böse, mein Lieber! 

\switchcolumn

\begin{greek}[variant=ancient]%
μὴ ἀγανάκτει, ὦ ᾽γαθέ.

\end{greek}%
\switchcolumn*

Aber ich kann unmöglich schweigen. 

\switchcolumn

\begin{greek}[variant=ancient]%
ἀλλ᾽ \emph{οὐκ ἔσθ᾽ ὅπως} σιγήσομαι.

\end{greek}%
\switchcolumn*

Daran bist du ganz allein schuld. 

\switchcolumn

\begin{greek}[variant=ancient]%
αἴτιος μέντοι σὺ τούτων εἶ μόνος.

\end{greek}%
\switchcolumn*

Es war \emph{nicht richtig, daß} du das thatest.

\switchcolumn

\begin{greek}[variant=ancient]%
οὐκ ὀρθῶς τοῦτ᾽ ἔδρασας!

\end{greek}%
\switchcolumn*

Was geht das \emph{dich} an?

\switchcolumn

\begin{greek}[variant=ancient]%
τί δὲ σοὶ τοῦτο;

\end{greek}%
\switchcolumn*

Was fiel dir denn ein, daß du das thatest? 

\switchcolumn

\begin{greek}[variant=ancient]%
τί δὴ \emph{μαθὼν} τοῦτ᾽ ἐποίησας;

\end{greek}%
\switchcolumn*

O über die Thorheit! 

\switchcolumn

\begin{greek}[variant=ancient]%
τῆς μωρίας!

\end{greek}%
\switchcolumn*

Wie \emph{unrecht} du gehandelt hast!

\switchcolumn

\begin{greek}[variant=ancient]%
ὡς οὐκ ὀρθῶς τοῦτ᾽ ἔδρασας!

\end{greek}%
\switchcolumn*

Das war Unrecht von dir. 

\switchcolumn

\begin{greek}[variant=ancient]%
τοῦτ᾽ οὐκ ὀρθῶς ἐποίησας.

\end{greek}%
\switchcolumn*

\emph{Das ist es, was} du mir zum Vorwurf machst?

\switchcolumn

\begin{greek}[variant=ancient]%
ταῦτ᾽ ἐπικαλεῖς;

\end{greek}%
\switchcolumn*

Aber es ging nicht anders. 

\switchcolumn

\begin{greek}[variant=ancient]%
ἀλλ᾽ οὐκ ἦν παρὰ ταῦτ᾽ ἄλλα.

\end{greek}%
\switchcolumn*

Gieb mir keine guten Lehren, sondern — 

\switchcolumn

\begin{greek}[variant=ancient]%
μὴ νουθέτει με, ἀλλὰ —

\end{greek}%
\switchcolumn*

Über dich kann man sich krank ärgern. 

\switchcolumn

\begin{greek}[variant=ancient]%
ἀπολεῖς με!

\end{greek}%
\switchcolumn*

Aber soviel sage ich dir: 

\switchcolumn

\begin{greek}[variant=ancient]%
ἓν δέ σοι λέγω·

\end{greek}%
\switchcolumn*

Mir \emph{thut} das Fräulein \emph{leid.}

\switchcolumn

\begin{greek}[variant=ancient]%
περὶ τῆς κόρης ἀνιῶμαι.

\end{greek}%
\switchcolumn*[


\section{Keine schlechten Witze!}

]Wie komisch sich das aus\textcompwordmark{}nahm! 

\switchcolumn

\begin{greek}[variant=ancient]%
ὡς καταγέλαστον ἐφάνη τὸ πρᾶγμα!

\end{greek}%
\switchcolumn*

Das ist ein Hauptwitz! 

\switchcolumn

\begin{greek}[variant=ancient]%
τοῦτο πάνυ γελοῖον!

\end{greek}%
\switchcolumn*

Das geht auf mich! 

\switchcolumn

\begin{greek}[variant=ancient]%
πρὸς \emph{ἐμὲ} ταῦτ᾽ ἐστίν.

\end{greek}%
\switchcolumn*

Er macht schlechte Witze. 

\switchcolumn

\begin{greek}[variant=ancient]%
σκώπτει.

\end{greek}%
\switchcolumn*

Mach' keine schlechten Witze! 

\switchcolumn

\begin{greek}[variant=ancient]%
μὴ σκῶπτε!

\end{greek}%
\switchcolumn*

Mach' keine schlechten Witze \emph{über mich!}

\switchcolumn

\begin{greek}[variant=ancient]%
μὴ σκῶπτέ με!

\end{greek}%
\switchcolumn*

Du machst doch nicht etwa des\textcompwordmark{}wegen schlechte Witze
über mich? 

\switchcolumn

\begin{greek}[variant=ancient]%
μῶν με σκώπτεις ὁρῶν τοῦτο;

\end{greek}%
\switchcolumn*

Laß dich doch nicht aus\textcompwordmark{}lachen! 

\switchcolumn

\begin{greek}[variant=ancient]%
καταγέλαστος εἶ.

\end{greek}%
\switchcolumn*

Wir lachen nicht über dich. 

\switchcolumn

\begin{greek}[variant=ancient]%
οὐ σοῦ καταγελῶμεν.

\end{greek}%
\switchcolumn*

Nun, worüber denn? 

\switchcolumn

\begin{greek}[variant=ancient]%
ἀλλὰ τοῦ;

\end{greek}%
\switchcolumn*

Worüber lachst du? 

\switchcolumn

\begin{greek}[variant=ancient]%
ἐπὶ τῷ γελᾶς;

\end{greek}%
\switchcolumn*

Hör' auf! — Schweig'! 

\switchcolumn

\begin{greek}[variant=ancient]%
παῦε! — σιώπα!

\end{greek}%
\switchcolumn*

Sei so gut und rede nicht mehr mit mir! 

\switchcolumn

\begin{greek}[variant=ancient]%
\emph{βούλει} μὴ προσαγορεύειν ἐμέ;

\end{greek}%
\switchcolumn*[


\section{Ende gut, Alles gut!}

]Vielleicht kann es noch gut werden! 

\switchcolumn

\begin{greek}[variant=ancient]%
ἴσως ἂν εὖ γένοιτο.

\end{greek}%
\switchcolumn*

\vspace{0.5em}
So Gott will. 

\switchcolumn

\begin{tabular}{ll}
\rdelim\{{2}{1em}[] & \begin{greek}[variant=ancient]%
σὺν θεῷ δ᾽ εἰρήσεται.\end{greek}%
\tabularnewline
 & \begin{greek}[variant=ancient]%
ἢν θεοὶ θέλωσιν.\end{greek}%
\tabularnewline
\end{tabular}

\switchcolumn*

Wer bürgt dir dafür? 

\switchcolumn

\begin{greek}[variant=ancient]%
καὶ τίς ἐγγυητής ἐστι τούτου;

\end{greek}%
\switchcolumn*

Wenn es uns gelingt, so will ich Gott innig denken. 

\switchcolumn

\begin{greek}[variant=ancient]%
ἢν \emph{κατορθώσωμεν, ἐπαινέσομαι} τὸν θεὸν πάνυ σφόδρα.

\end{greek}%
\switchcolumn*

Wie es sich gehört. 

\switchcolumn

\begin{greek}[variant=ancient]%
ὥσπερ εἰκός ἐστιν.

\end{greek}%
\switchcolumn*

In Gottes Namen! 

\switchcolumn

\begin{greek}[variant=ancient]%
τυχαγαθῇ:

\end{greek}%
\switchcolumn*

Wenn es uns aber mißlingt? 

\switchcolumn

\begin{greek}[variant=ancient]%
ἢν δὲ \emph{σφαλῶμεν;}

\end{greek}%
\switchcolumn*

Hurrah! (Freudenruf.) 

\switchcolumn

\begin{greek}[variant=ancient]%
ἀλαλαί!

\end{greek}%
\switchcolumn*

Was du für \emph{Glück hast!} 

\switchcolumn

\begin{greek}[variant=ancient]%
ὡς εὐτυχὴς εἶ!

\end{greek}%
\switchcolumn*

Er hat großes Glück. 

\switchcolumn

\begin{greek}[variant=ancient]%
εὐτυχέστατα πέπραγεν.

\end{greek}%
\switchcolumn*

Inwiefern? 

\switchcolumn

\begin{greek}[variant=ancient]%
τίνι τρόπῳ;

\end{greek}%
\switchcolumn*

Er hat ein ganz junges Mädchen geheirathet. 

\switchcolumn

\begin{greek}[variant=ancient]%
παῖδα κόρην γεγάμηκεν.

\end{greek}%
\switchcolumn*

Er ist ein reicher Mann geworden. 

\switchcolumn

\begin{greek}[variant=ancient]%
πλούσιος γεγένηται.

\end{greek}%
\switchcolumn*

Er kann das Leben genießen. 

\switchcolumn

\begin{greek}[variant=ancient]%
ἔχει τῆς ἥβης ἀπολαῦσαι.

\end{greek}%
\switchcolumn*

Wenn's weiter nichts ist! 

\switchcolumn

\begin{greek}[variant=ancient]%
εἶτα τί τοῦτο;

\end{greek}%
\switchcolumn*

Seine Freunde vermissen ihn schmerzlich. 

\switchcolumn

\begin{greek}[variant=ancient]%
ποθεινός ἐστι τοῖς φίλοις.

\end{greek}%
\switchcolumn*

Er ist ein Freund von mir. 

\switchcolumn

\begin{greek}[variant=ancient]%
ἐστὶ \emph{τῶν} φίλων.

\end{greek}%

	\switchcolumn*[


\part{Im Hause.}


\section{Da wohnt er}

]Werden Sie mir wohl sagen können, wo hier Herr M. wohnt?

\switchcolumn

\begin{greek}[variant=ancient]%
ἔχοις ἂν φράσαι μοι (τόν κύριον{*}) Μύλλερον, ὅπου ἐνθάδε οἰκεῖ;

\end{greek}%
\switchcolumn*

Ich möchte gern erfahren, wo Müller wohnt.

\switchcolumn

\begin{greek}[variant=ancient]%
ἡδέως ἂν μάθοιμι, ποῦ Μύλλερος οἰκεῖ.

\end{greek}%
\switchcolumn*

Das möchte ich gern wissen.

\switchcolumn

\begin{greek}[variant=ancient]%
τοῦτ᾽ με δίδαξον!

\end{greek}%
\switchcolumn*

In der Leipziger Straße.

\switchcolumn

\begin{greek}[variant=ancient]%
ἐν τῇ Λειψιανῇ{*} ὁδῷ.

\end{greek}%
\switchcolumn*

Er zieht aus.

\switchcolumn

\begin{greek}[variant=ancient]%
μετοικίζεται.

\end{greek}%
\switchcolumn*

Er ist aus\textcompwordmark{}gezogen.

\switchcolumn

\begin{greek}[variant=ancient]%
φροῦδός ἐστιν ἐξῳκισμένος.

\end{greek}%
\switchcolumn*

Da sieht er zum Fenster heraus!

\switchcolumn

\begin{greek}[variant=ancient]%
ὁδὶ ἐκ θυρίδος παρακύπτει.

\end{greek}%
\switchcolumn*

Das ist er.

\switchcolumn

\begin{greek}[variant=ancient]%
\emph{οὗτός ἐστ’ ἐκεῖνος.}

\end{greek}%
\switchcolumn*

Wer klopft?

\switchcolumn

\begin{greek}[variant=ancient]%
τίς ἐσθ’ ὁ τὴν θύραν κόπτων;

\end{greek}%
\switchcolumn*

Mach' die Thür auf!

\switchcolumn

\begin{greek}[variant=ancient]%
ἄνοιγε τὴν θύραν!

\end{greek}%
\switchcolumn*

Mach' doch auf!

\switchcolumn

\begin{greek}[variant=ancient]%
οὐκ ἀνοίξεις;

\end{greek}%
\switchcolumn*

Mach' endlich die Thür auf!

\switchcolumn

\begin{greek}[variant=ancient]%
ἄνοιγ’ \emph{ἀνύσας} τὴν θύραν.

\end{greek}%
\switchcolumn*

Wer ist da?

\switchcolumn

\begin{greek}[variant=ancient]%
τίς οὗτος;

\end{greek}%
\switchcolumn*

Melden Sie mich!

\switchcolumn

\begin{greek}[variant=ancient]%
εἰσάγγειλον.

\end{greek}%
\switchcolumn*

Ich weiß Ihren Namen nicht genau.

\switchcolumn

\begin{greek}[variant=ancient]%
οὐκ οἶδ’ ἀκριβῶς σου τοὔνομα.

\end{greek}%
\switchcolumn*

Ist Müller zu Hause?

\switchcolumn

\begin{greek}[variant=ancient]%
\emph{ἔνδον} ἐστὶ Μύλλερος;

\end{greek}%
\switchcolumn*

Nein, er ist nicht zu Hause.

\switchcolumn

\begin{greek}[variant=ancient]%
οὐκ ἔνδον ἐστίν.

\end{greek}%
\switchcolumn*

\emph{Augenblicklich} ist er nicht zu Hause.

\switchcolumn

\begin{greek}[variant=ancient]%
οὐκ ἔνδον ὢν τυγχάνει.

\end{greek}%
\switchcolumn*

Er ist spazieren.

\switchcolumn

\begin{greek}[variant=ancient]%
περίπατον ποιεῖται.

\end{greek}%
\switchcolumn*

So?

\switchcolumn

\begin{greek}[variant=ancient]%
ἄληρες;

\end{greek}%
\switchcolumn*

Er steht an der Thür.

\switchcolumn

\begin{greek}[variant=ancient]%
ἐπὶ ταῖς θύραις ἕστηκεν.

\end{greek}%
\switchcolumn*

Er ist im Begriff aus\textcompwordmark{}zugehen.

\switchcolumn

\begin{greek}[variant=ancient]%
μέλλει \emph{θύραζε βαδίζειν.}

\end{greek}%
\switchcolumn*[


\section{Am Morgen}

]Er ist im Schlafzimmer.

\switchcolumn

\begin{greek}[variant=ancient]%
ἐστὶν ἐν τῷ δωματίῳ.

\end{greek}%
\switchcolumn*

Das Bett.

\switchcolumn

\begin{greek}[variant=ancient]%
τὰ \emph{στρώματα.}

\end{greek}%
\switchcolumn*

Im Bette.

\switchcolumn

\begin{greek}[variant=ancient]%
ἐν τοῖς στρώμασιν.

\end{greek}%
\switchcolumn*

Er schläft eben.

\switchcolumn

\begin{greek}[variant=ancient]%
ἀρτίως εὕδει.

\end{greek}%
\switchcolumn*

Du, wach' auf!

\switchcolumn

\begin{greek}[variant=ancient]%
οὗτος, ἐγείρου!

\end{greek}%
\switchcolumn*

Steh' auf!

\switchcolumn

\begin{greek}[variant=ancient]%
ἀνίστασο!

\end{greek}%
\switchcolumn*

Zünde Licht an!

\switchcolumn

\begin{greek}[variant=ancient]%
ἅπτε λύχνον!

\end{greek}%
\switchcolumn*

Sehr wohl.

\switchcolumn

\begin{greek}[variant=ancient]%
ταῦτα.

\end{greek}%
\switchcolumn*

Hast du dich gewaschen?

\switchcolumn

\begin{greek}[variant=ancient]%
ἆρ᾽ ἀπονένιψαι;

\end{greek}%
\switchcolumn*

Kannst du \emph{ohne} Handtuch zurechtkommen?

\switchcolumn

\begin{greek}[variant=ancient]%
ἀνύτεις χειρόμακτρον οὐκ ἔχων;

\end{greek}%
\switchcolumn*

Du siehst schrecklich schmutzig aus.

\switchcolumn

\begin{greek}[variant=ancient]%
αὐχμεῖς αἰσχρῶς.

\end{greek}%
\switchcolumn*

Er hat sich nicht gebadet.

\switchcolumn

\begin{greek}[variant=ancient]%
οὐκ ἐλούσατο.

\end{greek}%
\switchcolumn*

Wisch' den Tisch ab!

\switchcolumn

\begin{greek}[variant=ancient]%
ἀποκάθαιρε τὴν τράπεζαν!

\end{greek}%
\switchcolumn*

Ich will zu hause bleiben.

\switchcolumn

\begin{greek}[variant=ancient]%
οἴκοι μενῶ.

\end{greek}%
\switchcolumn*

Wir wollen zu Hause bei mir studiren.

\switchcolumn

\begin{greek}[variant=ancient]%
ἔνδον παρ᾽ ἐμοὶ διατρίψομεν (περὶ τὰ μαθήματα).

\end{greek}%
\switchcolumn*

Bei dir?

\switchcolumn

\begin{greek}[variant=ancient]%
παρὰ σοί;

\end{greek}%
\switchcolumn*

Ganz recht.

\switchcolumn

\begin{greek}[variant=ancient]%
πάνυ.

\end{greek}%
\switchcolumn*

Du warst gestern bei mir.

\switchcolumn

\begin{greek}[variant=ancient]%
παρ᾽ ἐμοὶ χθὲς ἦσθα.

\end{greek}%
\switchcolumn*

Kommt heute in meine Wohnung!

\switchcolumn

\begin{greek}[variant=ancient]%
ἥκετ᾽ \emph{εἰς ἐμοῦ} τήμερον!

\end{greek}%
\switchcolumn*[


\section{Sitzen. Stehen}

]Leg' ab!

\switchcolumn

\begin{greek}[variant=ancient]%
ἀποδύου!

\end{greek}%
\switchcolumn*

Ich ziehe mich schon aus.

\switchcolumn

\begin{greek}[variant=ancient]%
καὶ δὴ ἐκδύομαι.

\end{greek}%
\switchcolumn*

Wohin wollen wir uns setzen?

\switchcolumn

\begin{greek}[variant=ancient]%
\emph{ποῦ} καθιζησόμεθα;

\end{greek}%
\switchcolumn*

Nehmt Platz!

\switchcolumn

\begin{greek}[variant=ancient]%
κάθησθε!

\end{greek}%
\switchcolumn*

\begin{tabular}{lc}
Setzen Sie sich! & \ldelim\}{2}{1em}[]\tabularnewline
Setz' dich nieder! & \tabularnewline
\end{tabular}

\switchcolumn

\begin{greek}[variant=ancient]%
\vspace{0.5em}


κάθιζε!

\end{greek}%
\switchcolumn*

Wenn du erlaubst!

\switchcolumn

\begin{greek}[variant=ancient]%
εἰ ταῦτα \emph{δοκεῖ!}

\end{greek}%
\switchcolumn*

So, ich sitze.

\switchcolumn

\begin{greek}[variant=ancient]%
ἰδού· κάθημαι.

\end{greek}%
\switchcolumn*

Ich sitze schon!

\switchcolumn

\begin{greek}[variant=ancient]%
κάθημαι ᾽γὼ πάλαι.

\end{greek}%
\switchcolumn*

Du hast keinen guten Platz.

\switchcolumn

\begin{greek}[variant=ancient]%
οὐ καθίζεις ἐν καλῷ.

\end{greek}%
\switchcolumn*

Hast du nichts zu essen?

\switchcolumn

\begin{greek}[variant=ancient]%
οὐκ ἔχεις καταφαγεῖν;

\end{greek}%
\switchcolumn*

\emph{Darf ich} dir ein Abendbrot vorsetzen?

\switchcolumn

\begin{greek}[variant=ancient]%
βούλει παραθῶ σοι δόρπον.

\end{greek}%
\switchcolumn*

Ich bitte nur um ein Stück Brot und Fleisch.

\switchcolumn

\begin{greek}[variant=ancient]%
αἰτῶ λαβεῖν τιν᾽ ἄρτον καὶ κρέας.!

\end{greek}%
\switchcolumn*

Ich habe mir zu trinken \emph{mitgebracht}.

\switchcolumn

\begin{greek}[variant=ancient]%
\emph{ἥκω φέρων} πιεῖν.

\end{greek}%
\switchcolumn*

Gieb mir einmal zu trinken!

\switchcolumn

\begin{greek}[variant=ancient]%
δός μοι πιεῖν.

\end{greek}%
\switchcolumn*

Gleich.

\switchcolumn

\begin{greek}[variant=ancient]%
ἰδού.

\end{greek}%
\switchcolumn*

Es ist unrecht, daß du hier sitzest.

\switchcolumn

\begin{greek}[variant=ancient]%
ἀδικεῖς ἐνθάδε καθήμενος.

\end{greek}%
\switchcolumn*

Steh' \emph{wieder} auf!

\switchcolumn

\begin{greek}[variant=ancient]%
ἀνίστασο!

\end{greek}%
\switchcolumn*

So steh' doch schnell auf, ehe dich jemand sieht!

\switchcolumn

\begin{greek}[variant=ancient]%
ούκουν ἀναστήσει ταχύ, πρίν τινά σ᾽ ἰδεῖν;

\end{greek}%
\switchcolumn*

Steh' gerade!

\switchcolumn

\begin{greek}[variant=ancient]%
ἀνίστασο ὀρθός.

\end{greek}%
\switchcolumn*

\emph{Bleib'} stehen!

\switchcolumn

\begin{greek}[variant=ancient]%
στῆθι.

\end{greek}%
\switchcolumn*

Zu Befehl, Herr Hauptmann!

\switchcolumn

\begin{greek}[variant=ancient]%
ταῦτα, ὦ λοχαγέ!

\end{greek}%
\switchcolumn*[


\section{Frau und Kinder}

]Sie hat einen kleinen Jungen bekommen.

\switchcolumn

\begin{greek}[variant=ancient]%
ἄρρεν ἔτεκε παιδίον.

\end{greek}%
\switchcolumn*

Er hat viele kleine Kinder zu ernähren.

\switchcolumn

\begin{greek}[variant=ancient]%
βόσκει μικρὰ πολλὰ παιδία.

\end{greek}%
\switchcolumn*

Wo sind die Kinder?

\switchcolumn

\begin{greek}[variant=ancient]%
ποῦ τὰ παιδία;

\end{greek}%
\switchcolumn*

Wo ist meine Frau hin?

\switchcolumn

\begin{greek}[variant=ancient]%
ποῖ ἡ γυνὴ φρούδη ᾽στίν;

\end{greek}%
\switchcolumn*

Wer kann mir sagen, wo meine Frau ist?

\switchcolumn

\begin{greek}[variant=ancient]%
τίς ἂν φράσειε, ποῦ ᾽στι ἡ γυνή;

\end{greek}%
\switchcolumn*

Sie wäscht und päppelt das Kind.

\switchcolumn

\begin{greek}[variant=ancient]%
λούει καὶ ψωμίζει τὸ παιδίον.

\end{greek}%
\switchcolumn*

Die Kinder sind gewaschen.

\switchcolumn

\begin{greek}[variant=ancient]%
ἀπονενιμμένα ἐστὶ τὰ παιδία.

\end{greek}%
\switchcolumn*

Sie bringt die Kinder zu Bette.

\switchcolumn

\begin{greek}[variant=ancient]%
κατακλίνει τὰ παιδία.

\end{greek}%
\switchcolumn*

Es ist \emph{höchste} Zeit. 

\switchcolumn

\begin{greek}[variant=ancient]%
καιρὸς δέ.

\end{greek}%
\switchcolumn*

Ihr habt lange genug gespielt.

\switchcolumn

\begin{greek}[variant=ancient]%
ἱκανὸν κρόνον ἐπαίζετε.

\end{greek}%
\switchcolumn*

Sie würfeln. — Um was?

\switchcolumn

\begin{greek}[variant=ancient]%
κυβεύουσιν. — περὶ τοῦ;

\end{greek}%
\switchcolumn*

Sei artig!

\switchcolumn

\begin{greek}[variant=ancient]%
κοσμίως ἔχε!

\end{greek}%
\switchcolumn*

Thu' das ja nicht!

\switchcolumn

\begin{greek}[variant=ancient]%
μηδαμῶς τοῦτ᾽ ἐργάσῃ!

\end{greek}%
\switchcolumn*

Da, schau' einmal!

\switchcolumn

\begin{greek}[variant=ancient]%
ἰδού· θέασαι!

\end{greek}%
\switchcolumn*

Der Onkel hat hübsche Geschenke mitgebracht.

\switchcolumn

\begin{greek}[variant=ancient]%
ὁ θεῖος ἥκει φέρων δῶρα χαρίεντα.!

\end{greek}%
\switchcolumn*

Lies\textcompwordmark{}chen klatscht vor Freude in die Hände.

\switchcolumn

\begin{greek}[variant=ancient]%
Λουίσιον{*} τὼ χεῖρ᾽ ἀνακροτεῖ ὑφ᾽ ἡδονῆς.!

\end{greek}%
\switchcolumn*

Meine Frau ist nicht zu sehen.

\switchcolumn

\begin{greek}[variant=ancient]%
ἡ δὲ γυνὴ \emph{φαίνεται.}

\end{greek}%
\switchcolumn*

Suchst du mich etwa?

\switchcolumn

\begin{greek}[variant=ancient]%
μῶν ἐμὲ ζητεῖς;

\end{greek}%
\switchcolumn*

Komm her, mein goldiger Schatz!

\switchcolumn

\begin{greek}[variant=ancient]%
δεῦρό νυν, ὦ χρυσίον.

\end{greek}%
\switchcolumn*[


\section{Kinderkrawall}

]Das ist Unrecht von dir.

\switchcolumn

\begin{greek}[variant=ancient]%
ταῦτ’ οὐκ ὀρθῶς ποιεῖς.

\end{greek}%
\switchcolumn*

Das ist unrecht, daß du mir das thust.

\switchcolumn

\begin{greek}[variant=ancient]%
ἀδικεῖς γέ με τοῦτο ποιῶν.

\end{greek}%
\switchcolumn*

Wenn du mich ärgern willst, so soll dir's schlecht gehen!

\switchcolumn

\begin{greek}[variant=ancient]%
ἤν τι λυπῇς με, οὐ χαιρήσεις!

\end{greek}%
\switchcolumn*

Gieb mir's wieder!

\switchcolumn

\begin{greek}[variant=ancient]%
ἀλλ’ ἀπόδος αὐτό!

\end{greek}%
\switchcolumn*

Oder du sollst sehen (= ich ergreife andere Maßregeln)!

\switchcolumn

\begin{greek}[variant=ancient]%
ἢ τἀπὶ τούτοις δρῶ.

\end{greek}%
\switchcolumn*

Soll ich dir eine Ohrfeige geben?

\switchcolumn

\begin{greek}[variant=ancient]%
τὴν γνάθον βούλει θένω;

\end{greek}%
\switchcolumn*

Das sollst du nicht umsonst gesagt haben!

\switchcolumn

\begin{greek}[variant=ancient]%
οὐ μὰ ∆ία σὺ καταπροίξει τοῦτο \emph{λέγων!}

\end{greek}%
\switchcolumn*

Was hast du vor?

\switchcolumn

\begin{greek}[variant=ancient]%
τί μέλλεις δρᾶν;

\end{greek}%
\switchcolumn*

Du sollst gehörige Prügel bekommen.

\switchcolumn

\begin{greek}[variant=ancient]%
κλαύσει μακρά.

\end{greek}%
\switchcolumn*

(Daß du berstest!) Hol' dich der Kuckuck!

\switchcolumn

\begin{greek}[variant=ancient]%
διαρραγείης!

\end{greek}%
\switchcolumn*

Da hast du eine Backpfeife!

\switchcolumn

\begin{greek}[variant=ancient]%
οὑτοσί σοι κόνδυλος!

\end{greek}%
\switchcolumn*

Zum Donnerwetter!Immer hau' ihn!

\switchcolumn

\begin{greek}[variant=ancient]%
ἐς κόρακας!

\end{greek}%
\switchcolumn*

Wart', ich will dir's weisen!

\switchcolumn

\begin{greek}[variant=ancient]%
παῖε παῖε!

\end{greek}%
\switchcolumn*

Kommt mir nicht zu nahe!

\switchcolumn

\begin{greek}[variant=ancient]%
μὴ πρόσιτε.

\end{greek}%
\switchcolumn*

Hurrah!

\switchcolumn

\begin{greek}[variant=ancient]%
ἀλαλαί!

\end{greek}%
\switchcolumn*

Jetzt haben wir ihn!

\switchcolumn

\begin{greek}[variant=ancient]%
νῦν ἔχεται μέσος!

\end{greek}%
\switchcolumn*

Wollt ihr weg!

\switchcolumn

\begin{greek}[variant=ancient]%
οὐχὶ σοῦσθε;

\end{greek}%
\switchcolumn*

Wir sollt ihr nicht wieder kommen!

\switchcolumn

\begin{greek}[variant=ancient]%
οὐδὲν ἄν με φλαῦρον ἔτι ἐργάσαισθε.

\end{greek}%
\switchcolumn*[


\section{Kinderzucht}

]Was ist das für ein Lärm da drin?

\switchcolumn

\begin{greek}[variant=ancient]%
τίς οὗτος ὁ ἔνδον θόρυβος;

\end{greek}%
\switchcolumn*

Schreit nicht so!

\switchcolumn

\begin{greek}[variant=ancient]%
μὴ βοᾶτε! — μὴ βοᾶτε μηδαμῶς! — μὴ κεκράγατε!

\end{greek}%
\switchcolumn*

So hört doch endlich!

\switchcolumn

\begin{greek}[variant=ancient]%
οὐκ ἄκούσεσθε ἐτεόν;

\end{greek}%
\switchcolumn*

Was giebt's?

\switchcolumn

\begin{greek}[variant=ancient]%
τί ἔστιν;

\end{greek}%
\switchcolumn*

Was ist los? Um was handelt es sich?

\switchcolumn

\begin{greek}[variant=ancient]%
τί τὸ πρᾶγμα;

\end{greek}%
\switchcolumn*

Wer schreit \emph{nach mir?}

\switchcolumn

\begin{greek}[variant=ancient]%
τίς ὁ βοῶν με;

\end{greek}%
\switchcolumn*

Soll ich's sagen?

\switchcolumn

\begin{greek}[variant=ancient]%
εἴπω;

\end{greek}%
\switchcolumn*

Erzähle es mir!

\switchcolumn

\begin{greek}[variant=ancient]%
κάτειπέ μοι.

\end{greek}%
\switchcolumn*

Karl hat uns geprügelt.

\switchcolumn

\begin{greek}[variant=ancient]%
Κάρολος πληγὰς ἡμῖν ἐνέβαλλεν.

\end{greek}%
\switchcolumn*

Ist's möglich?

\switchcolumn

\begin{greek}[variant=ancient]%
τί φής!

\end{greek}%
\switchcolumn*

Und was war dir Ursache davon?

\switchcolumn

\begin{greek}[variant=ancient]%
ἡ δ’ αἰτία τίς ἦν;

\end{greek}%
\switchcolumn*

Warum?

\switchcolumn

\begin{greek}[variant=ancient]%
τιή;

\end{greek}%
\switchcolumn*

So hitzig?

\switchcolumn

\begin{greek}[variant=ancient]%
ὡς ὀξύθυμος!

\end{greek}%
\switchcolumn*

Das ist immer so deine Art!

\switchcolumn

\begin{greek}[variant=ancient]%
οὗτος ὁ τρόπος πανταχοῦ!

\end{greek}%
\switchcolumn*

Ich bin nicht schuld daran.

\switchcolumn

\begin{greek}[variant=ancient]%
οὐκ ἐγὼ τούτων αἴτιος.

\end{greek}%
\switchcolumn*

Ja mit mir hat er es ebenso gemacht.

\switchcolumn

\begin{greek}[variant=ancient]%
νὴ ∆ία, κἀμὲ τοῦτ’ ἔδρασε ταὐτόν.

\end{greek}%
\switchcolumn*

Du willst es in Abrede stellen?

\switchcolumn

\begin{greek}[variant=ancient]%
ἀρνεῖ;

\end{greek}%
\switchcolumn*

Nicht gemuckst!

\switchcolumn

\begin{greek}[variant=ancient]%
μὴ γρύξῃς!

\end{greek}%
\switchcolumn*

Daß du mir keine Lügen sagst!

\switchcolumn

\begin{greek}[variant=ancient]%
ὅπως ἐρεῖς μηδὲν ψεῦδος!

\end{greek}%
\switchcolumn*

Du verdienst Schläge.

\switchcolumn

\begin{greek}[variant=ancient]%
ἄξιος εἶ πληγὰς λαβεῖν.

\end{greek}%
\switchcolumn*

Du, halt' einmal! Wo rennst du hin?

\switchcolumn

\begin{greek}[variant=ancient]%
ἐπίσχες, οὗτος! ποῖ θεῖς;

\end{greek}%
\switchcolumn*

Sei nicht böse, lieber Vater!

\switchcolumn

\begin{greek}[variant=ancient]%
μηδὲν ἀγανάακτει, ὦ πάτερ!

\end{greek}%
\switchcolumn*

Man muß sich todtärgern!

\switchcolumn

\begin{greek}[variant=ancient]%
οἴμοι, διαρραγήσομαι.

\end{greek}%

	\switchcolumn*[


\part{Aus dem politischen Leben.}


\section{Parteibewegung}

]Eugen ist da?

\switchcolumn

\begin{greek}[variant=ancient]%
ὁ Εὐγενὴς ἐπιδεδήμεκεν;

\end{greek}%
\switchcolumn*

Schon seit vorgestern.

\switchcolumn

\begin{greek}[variant=ancient]%
τρίτην ἤδη ἡμέραν.

\end{greek}%
\switchcolumn*

Er wird doch wohl eine Rede halten?

\switchcolumn

\begin{greek}[variant=ancient]%
\emph{οὐκοῦν} δημηγορήσει;

\end{greek}%
\switchcolumn*

Versteht sich! Heute Abend.

\switchcolumn

\begin{greek}[variant=ancient]%
εὖ ἴσθ᾽ ὅτι εἰς ἑσπέραν.

\end{greek}%
\switchcolumn*

Worüber? Über alles Mögliche.

\switchcolumn

\begin{greek}[variant=ancient]%
περὶ τοῦ; περὶ \emph{ἁπάντων} πραγμάτων.

\end{greek}%
\switchcolumn*

Ich will Sie mit in die Versammlung nehmen.

\switchcolumn

\begin{greek}[variant=ancient]%
\emph{ἄξω} σε μετ᾽ ἐμαυτοῦ εἰς τὸν σύλλογον.

\end{greek}%
\switchcolumn*

Ich danke, ich weiß den Weg.

\switchcolumn

\begin{greek}[variant=ancient]%
καλῶς· ἀλλ᾽ οἶδα τὴν ὁδόν.

\end{greek}%
\switchcolumn*

Nun, so machen Sie denn, daß Sie \emph{auch} hinkommen und bringen
Sie noch ein paar Andere mit!

\switchcolumn

\begin{greek}[variant=ancient]%
ἀλλ᾽ ὅπως παρέσει καὶ αὐτὸς καὶ ἄλλους \emph{ἄξεις!}

\end{greek}%
\switchcolumn*

Die Fort\textcompwordmark{}schrittler.

\switchcolumn

\begin{greek}[variant=ancient]%
οἱ καινοτομοῦντες.

\end{greek}%
\switchcolumn*

Die Conservativen.

\switchcolumn

\begin{greek}[variant=ancient]%
οἱ συντηρητικοί.{*}

\end{greek}%
\switchcolumn*

Die Rothen.

\switchcolumn

\begin{greek}[variant=ancient]%
οἱ δημοκρατικοί.

\end{greek}%
\switchcolumn*

Das Parlament.

\switchcolumn

\begin{greek}[variant=ancient]%
ἡ βουλή.

\end{greek}%
\switchcolumn*

Die Commission.

\switchcolumn

\begin{greek}[variant=ancient]%
οἱ ἐπίτροποι.

\end{greek}%
\switchcolumn*

Der Abgeordnete.

\switchcolumn

\begin{greek}[variant=ancient]%
ὁ βουλευτής.

\end{greek}%
\switchcolumn*

Der Wahlkandidat.

\switchcolumn

\begin{greek}[variant=ancient]%
ὁ ὑπόψηφος.

\end{greek}%
\switchcolumn*

Die Majorität.

\switchcolumn

\begin{greek}[variant=ancient]%
οἱ πλείονες.

\end{greek}%
\switchcolumn*

Die Minorität.

\switchcolumn

\begin{greek}[variant=ancient]%
οἱ μείονες.

\end{greek}%
\switchcolumn*

Die Präsident.

\switchcolumn

\begin{greek}[variant=ancient]%
ὁ πρόεδρος.

\end{greek}%
\switchcolumn*

Wer hat die meisten (wenigsten) \emph{Stimmen?}

\switchcolumn

\begin{greek}[variant=ancient]%
τίνι πλεῖσται (ἐλάχισται) γεγόνασιν;

\end{greek}%
\switchcolumn*

Abgeordneter ist, wer die meisten \emph{Stimmen} bekommen hat. 

\switchcolumn

\begin{greek}[variant=ancient]%
βουλευτής ἐστιν, ᾧ ἂν πλεῖσται γένωνται.

\end{greek}%
\switchcolumn*

Ist A.\ gewählt?

\switchcolumn

\begin{greek}[variant=ancient]%
πότερον Ἀ.\ ᾑρέθη;

\end{greek}%
\switchcolumn*

Leider nicht!

\switchcolumn

\begin{greek}[variant=ancient]%
εἰ γὰρ ὤφερε!

\end{greek}%
\switchcolumn*[


\section{Opposition}

]Wir brauchen keine neuen \emph{Steuern!} 

\switchcolumn

\begin{greek}[variant=ancient]%
οὐ δεόμεθα καινῶν δασμῶν!

\end{greek}%
\switchcolumn*

Wir \emph{brauchen} keine neuen Steuern! 

\switchcolumn

\begin{greek}[variant=ancient]%
καινῶν δασμῶν οὐ δεόμεθα!

\end{greek}%
\switchcolumn*

Das wird uns ruiniren!

\switchcolumn

\begin{greek}[variant=ancient]%
τοῦθ᾽ ἡμᾶς ἐπιτρίψει!

\end{greek}%
\switchcolumn*

Ich denke, es giebt einen Mittelweg.

\switchcolumn

\begin{greek}[variant=ancient]%
ἀλλ᾽ εἶναί τί μοι δοκεῖ μέση τούτων ὁδός.

\end{greek}%
\switchcolumn*

Jetzt ist Schonung der Steuerkraft nöthig!

\switchcolumn

\begin{greek}[variant=ancient]%
νῦν \emph{ἔργον} εὐτελείας!

\end{greek}%
\switchcolumn*

Die Kolonialpolitik bringt keinen Nutzen.

\switchcolumn

\begin{greek}[variant=ancient]%
τί \emph{πλέον ἐστὶν} ἔξω ἐποικεῖν;

\end{greek}%
\switchcolumn*

Das gefällt mir nicht!

\switchcolumn

\begin{greek}[variant=ancient]%
τοῦτό μ᾽ οὐκ ἀρέσκει!

\end{greek}%
\switchcolumn*

Dahinter steckt etwas!

\switchcolumn

\begin{greek}[variant=ancient]%
ἔστιν ἐνταῦθά τι κακόν!

\end{greek}%
\switchcolumn*

Was hat man davon?

\switchcolumn

\begin{greek}[variant=ancient]%
τί κέρδος;

\end{greek}%
\switchcolumn*

Was werden wir davon haben?

\switchcolumn

\begin{greek}[variant=ancient]%
τί κερδανοῦμεν;

\end{greek}%
\switchcolumn*

Was kann das nützen?

\switchcolumn

\begin{greek}[variant=ancient]%
πῶς ξυνοίσει ταῦτα;

\end{greek}%
\switchcolumn*

Ich weiß schon, wo man hinaus\textcompwordmark{}will!

\switchcolumn

\begin{greek}[variant=ancient]%
οἶδα τὸν νοῦν!

\end{greek}%
\switchcolumn*

Fort mit Bis\textcompwordmark{}marck! 

\switchcolumn

\begin{greek}[variant=ancient]%
Βίσμαρκ ἐρρέτω!

\end{greek}%
\switchcolumn*

Bravo! Bravo!

\switchcolumn

\begin{greek}[variant=ancient]%
εὖγε! εὖγε!

\end{greek}%
\switchcolumn*

Wie gut ist es, einen so vortrefflichen Abgeordneten zu haben!

\switchcolumn

\begin{greek}[variant=ancient]%
ὡς ἀγαθὸν τοιοῦτον ἔχειν βουλευτήν!

\end{greek}%
\switchcolumn*

Unsinn!

\switchcolumn

\begin{greek}[variant=ancient]%
οὐδὲν λέγεις!

\end{greek}%
\switchcolumn*

Wir hängen diese Tiraden zum Halse heraus!

\switchcolumn

\begin{greek}[variant=ancient]%
πάνυ μοι ἤδη ταῦτ᾽ ἐστὶ χολή.

\end{greek}%
\switchcolumn*

Still!

\switchcolumn

\begin{greek}[variant=ancient]%
σίγα!

\end{greek}%
\switchcolumn*[


\section{Zum Schlutz}

]Wer wünscht das Wort?

\switchcolumn

\begin{greek}[variant=ancient]%
τίς ἀγορεύειν βούλεται;

\end{greek}%
\switchcolumn*

Ich.

\switchcolumn

\begin{greek}[variant=ancient]%
ἐγώ.

\end{greek}%
\switchcolumn*

Ist noch Jemand, der zu sprechen wünscht?

\switchcolumn

\begin{greek}[variant=ancient]%
ἔσθ᾽ ὅστις ἕτερος βούλεται λέγειν;

\end{greek}%
\switchcolumn*

Es wird wohl Niemand dagegen stimmen.

\switchcolumn

\begin{greek}[variant=ancient]%
οὐ δεὶς ἀντιχειροτονήσειεν ἄν.

\end{greek}%
\switchcolumn*

Ich stimme dagegen.

\switchcolumn

\begin{greek}[variant=ancient]%
ἐγὼ τἀναντία ψηφίζομαι.

\end{greek}%
\switchcolumn*

So ist's recht.

\switchcolumn

\begin{greek}[variant=ancient]%
καλῶς γε ποιῶν.

\end{greek}%
\switchcolumn*

Thu', was du \emph{denkst!} 

\switchcolumn

\begin{greek}[variant=ancient]%
ποίει, ὅτι ἄν σοι δοκῇ.

\end{greek}%
\switchcolumn*

Was ist heute berathen worden?

\switchcolumn

\begin{greek}[variant=ancient]%
τί βεβούλεται τήμερον;

\end{greek}%
\switchcolumn*

Was hat man denn beschlossen?

\switchcolumn

\begin{greek}[variant=ancient]%
τί δῆτ᾽ ἔδοξεν;

\end{greek}%
\switchcolumn*

Noch nichts; es war \emph{Stimmen}gleichheit.

\switchcolumn

\begin{greek}[variant=ancient]%
οὐδέν πω· ἶσαι γὰρ ἐγένοντο.

\end{greek}%
\switchcolumn*

Eine so unsinnige Versammlung habe ich noch nicht erlebt.

\switchcolumn

\begin{greek}[variant=ancient]%
τοιοῦτον σύλλογον οὔπω \emph{ὄπωπα.}

\end{greek}%

	\switchcolumn*[


\part{Beim Skat\textcompwordmark{}spiel.}


\section{Ein Spiel mit Redens\textcompwordmark{}arten}

]Wollen wir nicht ein Spielchen machen?

\switchcolumn

\begin{greek}[variant=ancient]%
βούλεσθε παιδιὰν παίζωμεν;

\end{greek}%
\switchcolumn*

Meinetwegen.

\switchcolumn

\begin{greek}[variant=ancient]%
\emph{οὐδὲν κωλύει.}

\end{greek}%
\switchcolumn*

Was wollen wir spielen?

\switchcolumn

\begin{greek}[variant=ancient]%
παιδιὰν τίνα;

\end{greek}%
\switchcolumn*

Einen Skat wollen wir machen.

\switchcolumn

\begin{greek}[variant=ancient]%
(σκατιούμεθα).

\end{greek}%
\switchcolumn*

Wer giebt?

\switchcolumn

\begin{greek}[variant=ancient]%
τίς ὁ διαδώσων;

\end{greek}%
\switchcolumn*

Ich frage.

\switchcolumn

\begin{greek}[variant=ancient]%
ἐμὸν τὸ ἐρωτᾶν.

\end{greek}%
\switchcolumn*

Eichel, Grün, Roth, Schellen.

\switchcolumn

\begin{greek}[variant=ancient]%
τὰ βαλάνια, τὰ φυλλεῖα, τὰ ἐρυθρά, τά κρόταλα.

\end{greek}%
\switchcolumn*

Eichel sticht.

\switchcolumn

\begin{greek}[variant=ancient]%
κρατεῖ τὰ βαλάνια.

\end{greek}%
\switchcolumn*

Geben Sie Grün zu!

\switchcolumn

\begin{greek}[variant=ancient]%
ἀπόδος φυλλεῖα!

\end{greek}%
\switchcolumn*

Ich?

\switchcolumn

\begin{greek}[variant=ancient]%
ἐγώ;

\end{greek}%
\switchcolumn*

Freilich (Sie)!

\switchcolumn

\begin{greek}[variant=ancient]%
σὺ μέντοι!

\end{greek}%
\switchcolumn*

Was habe ich davon?

\switchcolumn

\begin{greek}[variant=ancient]%
τί κερδανῶ;

\end{greek}%
\switchcolumn*

Was ich für ein Pech habe!

\switchcolumn

\begin{greek}[variant=ancient]%
ὡς δυστυκής εἰμι!

\end{greek}%
\switchcolumn*

Nur nicht ängstlich!

\switchcolumn

\begin{greek}[variant=ancient]%
μὴ δέδιθι!

\end{greek}%
\switchcolumn*

Sehen Sie sich vor, daß Ihnen der rothe Wenzel nicht entgeht!

\switchcolumn

\begin{greek}[variant=ancient]%
εὐλαβοῦ, μὴ ἐκφύγῃ σε τῶν ἐρυθρῶν ὁ κράτιστος!

\end{greek}%
\switchcolumn*

Jetzt ist's an Ihnen, zu sehen, wie wir gewinnen!

\switchcolumn

\begin{greek}[variant=ancient]%
σὸν \emph{ἔργον} φροντίζειν, ὅπως κρατήσομεν.

\end{greek}%
\switchcolumn*

Jetzt gilt es!

\switchcolumn

\begin{greek}[variant=ancient]%
νῶν ὁ καιρός!

\end{greek}%
\switchcolumn*

Jetzt haben wir ihn! 

\switchcolumn

\begin{greek}[variant=ancient]%
νῶν ἔχεται μέσος!

\end{greek}%
\switchcolumn*

Hau' ihm, Lucas!

\switchcolumn

\begin{greek}[variant=ancient]%
παῖε, παῖε τὸν πανοῦργον!

\end{greek}%
\switchcolumn*

Das soll Ihnen schlecht bekommen, daß Sie das rothe Daus gestochen
haben!

\switchcolumn

\begin{greek}[variant=ancient]%
οὔ τοι μὰ Δία χαιρήσεις, ὁτιὴ τοῦτ᾽ ἔδρασας.!

\end{greek}%
\switchcolumn*

Verwünscht! Das ist zum Haaraus\textcompwordmark{}raufen!

\switchcolumn

\begin{greek}[variant=ancient]%
οἴμοι, διαρραγήσομαι!

\end{greek}%
\switchcolumn*

Ich weiß schon, wie Sie es machen.

\switchcolumn

\begin{greek}[variant=ancient]%
τοὺς τρόπους σου ἐπίσταμαι.

\end{greek}%
\switchcolumn*

Feine Rase!

\switchcolumn

\begin{greek}[variant=ancient]%
εὖ γε ξυνέβαλες!

\end{greek}%
\switchcolumn*

Du wunderst dich?

\switchcolumn

\begin{greek}[variant=ancient]%
ἐθαύμασας;!

\end{greek}%
\switchcolumn*

Darin bin ich Meister.

\switchcolumn

\begin{greek}[variant=ancient]%
ταύτεῃ κράτιστός εἰμι.

\end{greek}%
\switchcolumn*

Sie spielen falsch!

\switchcolumn

\begin{greek}[variant=ancient]%
ἀδικεῖς!

\end{greek}%
\switchcolumn*

Du hast die Mogelei nicht bemerkt.

\switchcolumn

\begin{greek}[variant=ancient]%
τὸ πραττόμενόν σε λέληθεν.

\end{greek}%
\switchcolumn*

Ist das wahr?

\switchcolumn

\begin{greek}[variant=ancient]%
τί λέγεις;

\end{greek}%
\switchcolumn*

Ent\textcompwordmark{}schuldigen Sie!

\switchcolumn

\begin{greek}[variant=ancient]%
σύγγνωθί μοι!

\end{greek}%
\switchcolumn*

Kellner, zünden Sie Licht an!

\switchcolumn

\begin{greek}[variant=ancient]%
ἅπτε, παῖ, λύχνον!

\end{greek}%
\switchcolumn*

Was fällt Ihnen denn ein, daß Sie die Zehn aus\textcompwordmark{}spielen? 

\switchcolumn

\begin{greek}[variant=ancient]%
τί δὴ μαθὼν τοῦτο ποιεῖς;

\end{greek}%
\switchcolumn*

Die Noth zwingt mich dazu.

\switchcolumn

\begin{greek}[variant=ancient]%
ἡ ἀνάγκη με πιέζει.

\end{greek}%
\switchcolumn*

Verwünscht! was soll ich thun?

\switchcolumn

\begin{greek}[variant=ancient]%
οἴ μοι, τί δράςω;

\end{greek}%
\switchcolumn*

Geben Sie mir einen guten Rath!

\switchcolumn

\begin{greek}[variant=ancient]%
χρηστόν τι συμβούλευσον.

\end{greek}%
\switchcolumn*

Er will's gewinnen.

\switchcolumn

\begin{greek}[variant=ancient]%
ἐθέλει οὗτος κρατῆσαι.

\end{greek}%
\switchcolumn*

Geben Sie sich keine vergebliche Mühe!

\switchcolumn

\begin{greek}[variant=ancient]%
λίθον ἕψεις!

\end{greek}%
\switchcolumn*

Hilf Himmel!

\switchcolumn

\begin{greek}[variant=ancient]%
Ἄπολλον ἀποτρόπαιε!

\end{greek}%
\switchcolumn*

O weh! Jetzt geht's uns (zweien) schlecht!

\switchcolumn

\begin{greek}[variant=ancient]%
ἒ ἔ, παρὰ νῷν στενάζειν!

\end{greek}%
\switchcolumn*

Gerade das will ich ja!

\switchcolumn

\begin{greek}[variant=ancient]%
τοῦτ᾽ αὐτὸ γὰρ καὶ βούλομαι!

\end{greek}%
\switchcolumn*

Zähle einmal!

\switchcolumn

\begin{greek}[variant=ancient]%
λόγισαι!

\end{greek}%
\switchcolumn*

Wir haben verspielt!

\switchcolumn

\begin{greek}[variant=ancient]%
\emph{ἀπολώλαμεν} ἡμεῖς.

\end{greek}%
\switchcolumn*

Bitte, bezahlen Sie!

\switchcolumn

\begin{greek}[variant=ancient]%
ἀπότισον \emph{δῆτα!}

\end{greek}%
\switchcolumn*

Mein Geld ist futsch!

\switchcolumn

\begin{greek}[variant=ancient]%
φροῦδα τὰ χρήματα!

\end{greek}%
\switchcolumn*

Es steht schlecht mit mir.

\switchcolumn

\begin{greek}[variant=ancient]%
φαῦλόν ἐστι τὸ ἐμὸν πρᾶγμα.

\end{greek}%
\switchcolumn*

Wir machen miserable Geschäfte.

\switchcolumn

\begin{greek}[variant=ancient]%
ἀθλίως πεπράγαμεν.

\end{greek}%
\switchcolumn*[


\section{Ein Grand}

](Ein Grand.)

\switchcolumn

\begin{greek}[variant=ancient]%
(τὸ παμμέγιστον.)

\end{greek}%
\switchcolumn*

A. Wer giebt denn?

\switchcolumn

\begin{greek}[variant=ancient]%
τίς ὁ διαδώσων;

\end{greek}%
\switchcolumn*

B. Du selbst.

\switchcolumn

\begin{greek}[variant=ancient]%
αὐτὸς σύ.

\end{greek}%
\switchcolumn*

C. Immer, wer fragt.

\switchcolumn

\begin{greek}[variant=ancient]%
ὁ ἀεὶ ἐρωτήσας.

\end{greek}%
\switchcolumn*

B. Nun gieb mir aber einmal anständige Karten; ich habe den ganzen
Abend noch kein Spiel gehabt!

\switchcolumn

\begin{greek}[variant=ancient]%
δός τι δῆτ᾽ ἐμοὶ· οὐδὲν γὰρ πώποτ᾽ ἔλαβον ἔγωγε τῇδε τῇ\footnote{\begin{latin}%
\textgreek[variant=ancient]{τὸ ῥῆμα οὐ δύναμαι διαγνῶναι.}\end{latin}%
} ἑσπέρᾳ!

\end{greek}%
\switchcolumn*

C. Ich frage. Grün Solo!

\switchcolumn

\begin{greek}[variant=ancient]%
ἐμὸν τὸ ἐρωτᾶν. τὰ φυλλεῖα αὐτὰ\footnote{\begin{latin}%
\textgreek[variant=ancient]{τὸ ῥῆμα οὐ δύναμαι διαγνῶναι.}\end{latin}%
} καθ᾽ αὑτά!

\end{greek}%
\switchcolumn*

B. Das halt' ich!

\switchcolumn

\begin{greek}[variant=ancient]%
ἔχω ἔγωγε!

\end{greek}%
\switchcolumn*

C. Null?

\switchcolumn

\begin{greek}[variant=ancient]%
τὸ μηδέν;

\end{greek}%
\switchcolumn*

B. Auch das.

\switchcolumn

\begin{greek}[variant=ancient]%
καὶ τοῦτό γε.

\end{greek}%
\switchcolumn*

C. Passe.

\switchcolumn

\begin{greek}[variant=ancient]%
παραχωρῶ ἔγωγε.

\end{greek}%
\switchcolumn*

A. Ich auch.

\switchcolumn

\begin{greek}[variant=ancient]%
κἀγώ.

\end{greek}%
\switchcolumn*

B. Grand.

\switchcolumn

\begin{greek}[variant=ancient]%
τὸ παμμέγιστον.

\end{greek}%
\switchcolumn*

B. Ich spiele selbst aus. Hier! Wenzel 'raus!

\switchcolumn

\begin{greek}[variant=ancient]%
ἐμὸν τὸ ἐξάρχειν. ἰδού. ἀπόδοτε δὴ τοὺς κρατίστους!

\end{greek}%
\switchcolumn*

C. Ja, den kann ich nicht!

\switchcolumn

\begin{greek}[variant=ancient]%
οὐ δυνατὸς ἐγὼ μὰ Δία ὑπὲρ τοῦτον.

\end{greek}%
\switchcolumn*

A. Nanu?!

\switchcolumn

\begin{greek}[variant=ancient]%
τί φής;

\end{greek}%
\switchcolumn*

B. Hurrah! Der Alte liegt im Skat! Hier!

\switchcolumn

\begin{greek}[variant=ancient]%
βαβαιάξ! ἀπόκειται ὁ παγκράτιστος! ἰδού

\end{greek}%
\switchcolumn*

C. Himmeldonnerwetter!

\switchcolumn

\begin{greek}[variant=ancient]%
ἐς κόρακας!

\end{greek}%
\switchcolumn*

A. Kreuzmillionen . . .!

\switchcolumn

\begin{greek}[variant=ancient]%
Ἄπολλον ἀποτρόπαιε!

\end{greek}%
\switchcolumn*

C. Ih, da soll doch der Deiwel 'reinfahren!

\switchcolumn

\begin{greek}[variant=ancient]%
οἴμοι κακοδαίμων!

\end{greek}%
\switchcolumn*

A. Heiliges Gewitter! Hast du denn gar nichts?

\switchcolumn

\begin{greek}[variant=ancient]%
ὦ Ζεῦ βασιλεῦ! οὐκ ἄρ᾽ ἔχεις\footnote{\begin{latin}%
\textgreek[variant=ancient]{τὸ ῥῆμα οὐ δύναμαι διαγνῶναι.}\end{latin}%
} οὐδέν;

\end{greek}%
\switchcolumn*

C. Dieser ist unser! 'rin, was Beine hat!

\switchcolumn

\begin{greek}[variant=ancient]%
ἀλλὰ τοῦτό γε γίγνεται ἥμῖν. νῦν ὁ καιρὸς ἐπιδοῦναι!

\end{greek}%
\switchcolumn*

B. Halt! Gesprochen wird nicht beim Spiel!

\switchcolumn

\begin{greek}[variant=ancient]%
μὴ δῆτα — οὐ γὰρ ἔστι λαλεῖν τῷ παίζοντι!

\end{greek}%
\switchcolumn*

C. So, das ist auch unser!\\
Gottlob! Aus dem Schneider wären wir!

\switchcolumn

\begin{greek}[variant=ancient]%
ἰδοὺ καὶ τοῦτο ἡμῖν!

τὸ μέσον καλῶς τετμήκαμεν!

\end{greek}%
\switchcolumn*

A. Oh, wir kriegen noch viel mehr!

\switchcolumn

\begin{greek}[variant=ancient]%
ἕξομεν ἔτι πολλῷ πλέον, ὦ τάν.

\end{greek}%
\switchcolumn*

B. Keinen Stich! Der Rest ist mein!

\switchcolumn

\begin{greek}[variant=ancient]%
οὐκ ἄλλ᾽ οὐδὲ ἕν. ἐμὰ γὰρ τὰ λοιπά!

\end{greek}%
\switchcolumn*

A. u. C. Oho! — Wahrhaftig!

\switchcolumn

\begin{greek}[variant=ancient]%
οὐδὲν λέγεις! — μὰ τὸν Δί᾽ οὐ τοίνυν!

\end{greek}%
\switchcolumn*

A. Ja wie konntest du aber auch \emph{die} Farbe spielen? Wir mußten
ja dicke gewinnen! 

\switchcolumn

\begin{greek}[variant=ancient]%
πῶς ἄρ᾽ οὖν ἐπὶ ταῦτα ἦλθες; ἐμέλλομεν γάρ τοι σφοδρῶς ὑπερέχειν!

\end{greek}%
\switchcolumn*

Ich sitze hier mit der ganzen Grün.

\switchcolumn

\begin{greek}[variant=ancient]%
ἐγὼ δὲ κάθημαι οὕτω πάντα τὰ φυλλεῖα ἔχων.

\end{greek}%
\switchcolumn*

C. So? Warum stichst du denn nicht? Ich habe ganz richtig aus\textcompwordmark{}gespielt.
— du bist schuld!

\switchcolumn

\begin{greek}[variant=ancient]%
ἄληθες; τί δὴ παθὴν οὐχ ὑπερέβαλες\footnote{\begin{latin}%
orig. \textgreek[variant=ancient]{ὑπερ-|έβαλες}\end{latin}%
} σύ; εὖ γὰρ ἐμοίησα ἔγωγε. — σὺ δὲ τούτου αἴτιος!

\end{greek}%
\switchcolumn*

B. Das war Grand mit Vieren! Sechzig. Wer giebt?

\switchcolumn

\begin{greek}[variant=ancient]%
παμμέγιστον τοῦτ᾽ ἦν μετὰ τεσσάρων! ἑξήκοντα. τίς ὁ διαδώσων;

\end{greek}%

	\switchcolumn*[{


\part[Sprichwörtliches aus der Umgangs\textcompwordmark{}sprache Altgriechische
Bezeichnungen für moderne Begriffe]{Sprichwörtliches aus der Umgangs\textcompwordmark{}sprache.}

}]Mensch, ärgere dich nicht!

\switchcolumn

\begin{greek}[variant=ancient]%
μὴ σεαυτὸν ἔσθιε, ὦ ᾽γαθέ!

\end{greek}%
\switchcolumn*

Eines Mannes Rede ist keine Rede.

\switchcolumn

\begin{greek}[variant=ancient]%
πρὶν ἂν ἀμφοῖν μῦθον ἀκούσῃς, οὐκ ἂν δικάσαις.

\end{greek}%
\switchcolumn*

Das hieße Eulen nach Athen tragen.

\switchcolumn

\begin{greek}[variant=ancient]%
τίς γλαῦκ᾽ Ἀθήναζε ἄγαγεν;

\end{greek}%
\switchcolumn*

Vorsicht ist die Mutter der Weis\textcompwordmark{}heit. 

\switchcolumn

\begin{greek}[variant=ancient]%
ἡ (γὰρ) εὐλάβεια πάντα σώζει.

\end{greek}%
\switchcolumn*

Eine Schwalbe macht noch keinen Sommer.

\switchcolumn

\begin{greek}[variant=ancient]%
μία χελιδὼν ἔαρ οὐ ποιεῖ.

\end{greek}%
\switchcolumn*

Menge dich nicht in meine Sachen!

\switchcolumn

\begin{greek}[variant=ancient]%
μὴ τὸν ἐμὸν οἴκει οἶκον!!

\end{greek}%
\switchcolumn*

Der reine Menschenfeind (Timon)!

\switchcolumn

\begin{greek}[variant=ancient]%
Τίμων \emph{καθαρός!}

\end{greek}%
\switchcolumn*

Immer das alte Lied!

\switchcolumn

\begin{greek}[variant=ancient]%
ὁ Διὸς Κόρινθος!

\end{greek}%
\switchcolumn*

\begin{latin}%
Hic Rhodus, hic salta!

\end{latin}%
\switchcolumn

\begin{greek}[variant=ancient]%
ἰδοὺ ἡ Ῥόδος\footnote{\begin{latin}%
orig. \textgreek[variant=ancient]{ʼΡόδος}\end{latin}%
}, ἰδοὺ καὶ τὸ πήδημα!

\end{greek}%
\switchcolumn*

Ein trauriger Peter (Japper)!

\switchcolumn

\begin{greek}[variant=ancient]%
Μυσῶν ἔσχατος!

\end{greek}%
\switchcolumn*

Das Gute ist rar.

\switchcolumn

\begin{greek}[variant=ancient]%
ὀλίγον τὸ χρηστόν ἐστιν.

\end{greek}%
\switchcolumn*

Es ist kein Vorwärts\textcompwordmark{}kommen (für uns).

\switchcolumn

\begin{greek}[variant=ancient]%
οὔτε θέομεν οὔτ᾽ ἐλαύνομεν.!

\end{greek}%
\switchcolumn*

Geld regiert die Welt.

\switchcolumn

\begin{greek}[variant=ancient]%
ἅπαντα (γὰρ) τῷ πλουτεῖν ὑπήκοα.!

\end{greek}%
\switchcolumn*

\begin{latin}%
Donec eris felix, multos numerabis amicos.

\end{latin}%
\switchcolumn

\begin{greek}[variant=ancient]%
ζεῖ χύτρα, ζῇ φιλία.!

\end{greek}%
\switchcolumn*

Durch Schaden wird man klug!

\switchcolumn

\begin{greek}[variant=ancient]%
\quotedblbase παθὼν δέ τε νήπιος ἔγνω.``

\end{greek}%
\switchcolumn*

\begin{latin}%
Tempi passati!

\end{latin}%
\switchcolumn

\begin{greek}[variant=ancient]%
πάλαι ποτ᾽ ἦσαν ἄλκιμοι Μιλήσιοι.

\end{greek}%
\switchcolumn*

\begin{latin}%
Ubi bene, ibi patria!

\end{latin}%
\switchcolumn

\begin{greek}[variant=ancient]%
πατρὶς γάρ ἐστι πᾶσ᾽, ἵν ἂν πράττῃ τις εὖ.

\end{greek}%
\switchcolumn*

Er ist der beste Bruder auch nicht!

\switchcolumn

\begin{greek}[variant=ancient]%
ἐστὶ τοῦ πονηροῦ κόμματος.

\end{greek}%
\switchcolumn*

\begin{latin}%
Parturiunt montes etc.

\end{latin}%
\switchcolumn

\begin{greek}[variant=ancient]%
ὤδινεν ὄρος, εἶτα μῦν ἀπέτεκεν.

\end{greek}%
\switchcolumn*

Du giebst dir vergebliche Mühe.

\switchcolumn

\begin{greek}[variant=ancient]%
λίθον ἕψεις.

\end{greek}%
\switchcolumn*

Das Übel ärger machen.

\switchcolumn

\begin{greek}[variant=ancient]%
πλέον θάτερον ποιεῖν.

\end{greek}%
\switchcolumn*

Eile mit Weile.

\switchcolumn

\begin{greek}[variant=ancient]%
σπεῦδε βραδέως!\textgerman[spelling=old,babelshorthands=true]{ (Wahlspruch
des Kaisers Augustus.)}

\end{greek}%
\switchcolumn*

Laß dir genügen!

\switchcolumn

\begin{greek}[variant=ancient]%
πλέον ἥμισυ παντός!

\end{greek}%

	\switchcolumn[0]*[


\section*{Altgriechische (auch neue\textrm{\textmd{{*}}} gutgebildete) Bezeichnungen
für moderne Begriffe aus dem Neugriechischen.\addcontentsline{toc}{section}{aus dem Neugriechischen}}

]Der Reichs\textcompwordmark{}tag 

\switchcolumn

\begin{greek}[variant=ancient]%
ἡ βουλή.

\end{greek}%
\switchcolumn*

Der Abgeordnete.

\switchcolumn

\begin{greek}[variant=ancient]%
ὁ βουλευτής.

\end{greek}%
\switchcolumn*

Das Heer.

\switchcolumn

\begin{greek}[variant=ancient]%
ὁ στρατός.

\end{greek}%
\switchcolumn*

Der Bürgermeister.

\switchcolumn

\begin{greek}[variant=ancient]%
ὁ δήμαρχος.

\end{greek}%
\switchcolumn*

Das Bureau.

\switchcolumn

\begin{greek}[variant=ancient]%
τὸ γραφεῖον.

\end{greek}%
\switchcolumn*

Die orientalische Frage.

\switchcolumn

\begin{greek}[variant=ancient]%
τὸ ζήτημα τὸ ἀνατολικόν.

\end{greek}%
\switchcolumn*

Das Gericht.

\switchcolumn

\begin{greek}[variant=ancient]%
τὸ δικαστήριον.

\end{greek}%
\switchcolumn*

Die Partei.

\switchcolumn

\begin{greek}[variant=ancient]%
τὸ κόμμα.

\end{greek}%
\switchcolumn*

conservativ.

\switchcolumn

\begin{greek}[variant=ancient]%
συντηρητικός.

\end{greek}%
\switchcolumn*

liberal.

\switchcolumn

\begin{greek}[variant=ancient]%
φιλελεύθερος.

\end{greek}%
\switchcolumn*

Der (Wahl-)Candidat.

\switchcolumn

\begin{greek}[variant=ancient]%
ὁ ὑπόψηφος.

\end{greek}%
\switchcolumn*

Der Minister.

\switchcolumn

\begin{greek}[variant=ancient]%
ὁ ὑπουργός.

\end{greek}%
\switchcolumn*

Das Ministerium des Aus\textcompwordmark{}wärtigen. 

\switchcolumn

\begin{greek}[variant=ancient]%
τὸ ἡπουργεῖον{*} τῶν ἐξωτερικῶν.

\end{greek}%
\switchcolumn*

\qquad{}des Innern.

\switchcolumn

\qquad{}\textgreek[variant=ancient]{τῶν ἐσωτερικῶν.}

\switchcolumn*

\qquad{}der Finanzen.

\switchcolumn

\qquad{}\textgreek[variant=ancient]{τῶν οἰκονομικῶν.}

\switchcolumn*

\qquad{}der Justiz.

\switchcolumn

\qquad{}\textgreek[variant=ancient]{τῆς δικαιοσύνης.}

\switchcolumn*

\qquad{}des Krieges.

\switchcolumn

\qquad{}\textgreek[variant=ancient]{τῶν στρατιωτικῶν.}

\switchcolumn*

\qquad{}des Kultus.

\switchcolumn

\qquad{}\textgreek[variant=ancient]{τῶν ἐκκλησιαστικῶν.}

\switchcolumn*

\qquad{}des öffentlichen Unterrichts.

\switchcolumn

\qquad{}\textgreek[variant=ancient]{τῆς δημοσίας ἐκπαιδεύσεως.}

\switchcolumn*

Der Landrath, Amts\textcompwordmark{}hauptmann.

\switchcolumn

\begin{greek}[variant=ancient]%
ὁ ἔπαρχος.

\end{greek}%
\switchcolumn*

Der Präsident.

\switchcolumn

\begin{greek}[variant=ancient]%
ὁ πρόεδρος.

\end{greek}%
\switchcolumn*

Die Regierung.

\switchcolumn

\begin{greek}[variant=ancient]%
ἡ κυβέρνησις.

\end{greek}%
\switchcolumn*

Die Regierungs\textcompwordmark{}partei.

\switchcolumn

\begin{greek}[variant=ancient]%
τὸ κυβερνητικὸν κόμμα.

\end{greek}%
\switchcolumn*

Die Zeitung.

\switchcolumn

\begin{greek}[variant=ancient]%
ἡ ἐφημερίς.

\end{greek}%
\switchcolumn*

Die Times.

\switchcolumn

\begin{greek}[variant=ancient]%
οἰ καιροί\footnote{\begin{latin}%
orig. \textgreek[variant=ancient]{Καιροί}\end{latin}%
}.

\end{greek}%
\switchcolumn[0]*[\StarOrnament]

Das Dampf\textcompwordmark{}schiff. 

\switchcolumn

\begin{greek}[variant=ancient]%
τὸ ἀτμόπλοιον.{*}

\end{greek}%
\switchcolumn*

Das Segelschiff.

\switchcolumn

\begin{greek}[variant=ancient]%
τὸ ἱστιοφόρον.

\end{greek}%
\switchcolumn*

Der Bahnhof.

\switchcolumn

\begin{greek}[variant=ancient]%
ὁ σταθμός.

\end{greek}%
\switchcolumn*

Der Bahnzug.

\switchcolumn

\begin{greek}[variant=ancient]%
ἡ ἁμαξοστοιχία.{*}

\end{greek}%
\switchcolumn*

Die Eisenbahn.

\switchcolumn

\begin{greek}[variant=ancient]%
ὁ\footnote{\begin{latin}%
orig. \textgreek[variant=ancient]{ο}\end{latin}%
} σιδηρόδρομος.{*}

\end{greek}%
\switchcolumn*

Der Gasthof, das Hotel.

\switchcolumn

\begin{greek}[variant=ancient]%
τὸ ξενοδοχεῖον.

\end{greek}%
\switchcolumn*

Der Omnibus.

\switchcolumn

\begin{greek}[variant=ancient]%
τὸ λεωφορεῖον.

\end{greek}%
\switchcolumn*

Der Fahrplan. 

\switchcolumn

\begin{greek}[variant=ancient]%
τὸ δρομολόγιον.

\end{greek}%
\switchcolumn[0]*[\StarOrnament]

Der Apotheker.

\switchcolumn

\begin{greek}[variant=ancient]%
ὁ φαρμακοπώλης.

\end{greek}%
\switchcolumn*

Der Arbeiter.

\switchcolumn

\begin{greek}[variant=ancient]%
ὁ ἐργάτης.

\end{greek}%
\switchcolumn*

Der Streik.

\switchcolumn

\begin{greek}[variant=ancient]%
ἡ ἀπεργία.{*}

\end{greek}%
\switchcolumn*

Der Barbier.

\switchcolumn

\begin{greek}[variant=ancient]%
ὁ κουρεύς.

\end{greek}%
\switchcolumn*

Der Baumeister.

\switchcolumn

\begin{greek}[variant=ancient]%
ὁ ἀρχιτέκτων.

\end{greek}%
\switchcolumn*

Der Briefträger.

\switchcolumn

\begin{greek}[variant=ancient]%
ὁ γραμματοφόρος.

\end{greek}%
\switchcolumn*

Der Buchbinder.

\switchcolumn

\begin{greek}[variant=ancient]%
ὁ βιβλιοδέτης.{*}

\end{greek}%
\switchcolumn*

Der Buchdrucker.

\switchcolumn

\begin{greek}[variant=ancient]%
ὁ τυπογράφος.{*}

\end{greek}%
\switchcolumn*

Der Buchhändler.

\switchcolumn

\begin{greek}[variant=ancient]%
ὁ βιβλιοπώλης.

\end{greek}%
\switchcolumn*

Der Droschkenkutscher.

\switchcolumn

\begin{greek}[variant=ancient]%
ὁ ἁμαξηλάτης.{*}

\end{greek}%
\switchcolumn*

Der Handwerker.

\switchcolumn

\begin{greek}[variant=ancient]%
ὁ τεχνίτης.

\end{greek}%
\switchcolumn*

Der Ingenieur.

\switchcolumn

\begin{greek}[variant=ancient]%
ὁ μηχανικός.

\end{greek}%
\switchcolumn*

Der Journalist.

\switchcolumn

\begin{greek}[variant=ancient]%
ὁ ἐφημεριδογράφος.{*}

\end{greek}%
\switchcolumn*

Der Handels\textcompwordmark{}mann. 

\switchcolumn

\begin{greek}[variant=ancient]%
ὁ παντοπώλης.

\end{greek}%
\switchcolumn*

Der Lehrer.

\switchcolumn

\begin{greek}[variant=ancient]%
ὁ διδάσκαλος.

\end{greek}%
\switchcolumn*

Der Offizier.

\switchcolumn

\begin{greek}[variant=ancient]%
ὁ ἀξιωματικός.

\end{greek}%
\switchcolumn*

Der Photograph.

\switchcolumn

\begin{greek}[variant=ancient]%
ὁ φωτογράφος.{*}

\end{greek}%
\switchcolumn*

Der Professor.

\switchcolumn

\begin{greek}[variant=ancient]%
ὁ καθηγητής.

\end{greek}%
\switchcolumn*

Der Redacterur.

\switchcolumn

\begin{greek}[variant=ancient]%
ὁ συντάκτης.{*}

\end{greek}%
\switchcolumn*

Der Gerichts\textcompwordmark{}rath. 

\switchcolumn

\begin{greek}[variant=ancient]%
ὁ δικαστής.

\end{greek}%
\switchcolumn*

Der Schrift\textcompwordmark{}setzer. 

\switchcolumn

\begin{greek}[variant=ancient]%
ὁ τυποθέτης.{*}

\end{greek}%
\switchcolumn*

Der Wichsier.

\switchcolumn

\begin{greek}[variant=ancient]%
ὁ καθαριστής.

\end{greek}%
\switchcolumn*

Der Student.

\switchcolumn

\begin{greek}[variant=ancient]%
ὁ φοιτητής.

\end{greek}%
\switchcolumn*

Der Tabaks\textcompwordmark{}händler.\footnote{\begin{latin}%
orig. \textgerman[spelling=old,babelshorthands=true]{Tabaks\textcompwordmark{}händler..}\end{latin}%
}

\switchcolumn

\begin{greek}[variant=ancient]%
ὁ καπνοπώλης.{*}

\end{greek}%
\switchcolumn*

Der Uhrmacher.

\switchcolumn

\begin{greek}[variant=ancient]%
ὁ ὡρολογοποιός.{*}

\end{greek}%
\switchcolumn[0]*[\StarOrnament]

Die Apotheke.

\switchcolumn

\begin{greek}[variant=ancient]%
τὸ φαρμακοπωλεῖον.

\end{greek}%
\switchcolumn*

Das Café.

\switchcolumn

\begin{greek}[variant=ancient]%
τὸ καφενεῖον.{*}

\end{greek}%
\switchcolumn*

Die Droschke.

\switchcolumn

\begin{greek}[variant=ancient]%
ἡ ἅμαξα.

\end{greek}%
\switchcolumn*

Der Kirchhof.

\switchcolumn

\begin{greek}[variant=ancient]%
τὸ κοιμητήριον.

\end{greek}%
\switchcolumn*

Der Klub.

\switchcolumn

\begin{greek}[variant=ancient]%
ἡ λέσχη.

\end{greek}%
\switchcolumn*

Das Lesezimmer.

\switchcolumn

\begin{greek}[variant=ancient]%
τὸ ἀναγνωστήριον.

\end{greek}%
\switchcolumn*

Das Concert.

\switchcolumn

\begin{greek}[variant=ancient]%
ἡ συμφωνία.

\end{greek}%
\switchcolumn*

Das Schloß.

\switchcolumn

\begin{greek}[variant=ancient]%
τὰ ἀνάκτορα.

\end{greek}%
\switchcolumn*

Das Herrenhaus.

\switchcolumn

\begin{greek}[variant=ancient]%
ἡ ἔπαυλις.

\end{greek}%
\switchcolumn*

Das Trottoir.

\switchcolumn

\begin{greek}[variant=ancient]%
τὸ πεζοδρόμιον.{*}

\end{greek}%
\switchcolumn*

Die Post.

\switchcolumn

\begin{greek}[variant=ancient]%
τὸ ταχυδρομεῖον.

\end{greek}%
\switchcolumn*

Die Freimarke.

\switchcolumn

\begin{greek}[variant=ancient]%
τὸ γραμματόσημον.

\end{greek}%
\switchcolumn*

Die Postkarte.

\switchcolumn

\begin{greek}[variant=ancient]%
τὸ ἐπιστολικὸν δελτάριον.

\end{greek}%
\switchcolumn*

Die Promenade.

\switchcolumn

\begin{greek}[variant=ancient]%
ὁ περίπατος.

\end{greek}%
\switchcolumn*

Das Rathhaus.

\switchcolumn

\begin{greek}[variant=ancient]%
τὸ δημαρχεῖον.

\end{greek}%
\switchcolumn*

Die Straße.

\switchcolumn

\begin{greek}[variant=ancient]%
ἡ ὁδός.

\end{greek}%
\switchcolumn*

Die Vorstadt.

\switchcolumn

\begin{greek}[variant=ancient]%
τὸ προάστειον.

\end{greek}%
\switchcolumn*

Die Universität.

\switchcolumn

\begin{greek}[variant=ancient]%
τὸ πανεπιστήμιον.{*}

\end{greek}%
\switchcolumn*

Der Briefkasten.

\switchcolumn

\begin{greek}[variant=ancient]%
τὸ γραμματοκιβώτιον.{*}

\end{greek}%
\switchcolumn*

Das Löschpapier.

\switchcolumn

\begin{greek}[variant=ancient]%
τὸ στουπόχαρτον.{*}

\end{greek}%
\switchcolumn*

Das Telegramm.

\switchcolumn

\begin{greek}[variant=ancient]%
τὸ τηλεγράφημα.{*}

\end{greek}%
\switchcolumn*

telegraphisch.

\switchcolumn

\begin{greek}[variant=ancient]%
τηλεγραφικῶς.{*}

\end{greek}%
\switchcolumn*

Die Tinte.

\switchcolumn

\begin{greek}[variant=ancient]%
(ἡ μελάνη) τὸ μέλαν.

\end{greek}%
\switchcolumn*

Das Tintenfaß.

\switchcolumn

\begin{greek}[variant=ancient]%
τὸ μελανοδοχεῖον.

\end{greek}%
\switchcolumn*

Der umschlag (Kouvert). 

\switchcolumn

\begin{greek}[variant=ancient]%
τὸ περικάλυμμα.

\end{greek}%
\switchcolumn[0]*[\StarOrnament]

Die Bürste.

\switchcolumn

\begin{greek}[variant=ancient]%
ἡ ψήκτρα.

\end{greek}%
\switchcolumn*

Das Faß.

\switchcolumn

\begin{greek}[variant=ancient]%
ὁ κάδος.

\end{greek}%
\switchcolumn*

Das Fenster.

\switchcolumn

\begin{greek}[variant=ancient]%
τὸ παραθύριον.

\end{greek}%
\switchcolumn*

Die Glocke, Klingel.

\switchcolumn

\begin{greek}[variant=ancient]%
τὸ κωδώνιον.

\end{greek}%
\switchcolumn*

klingeln.

\switchcolumn

\begin{greek}[variant=ancient]%
κωδωνίζειν.

\end{greek}%
\switchcolumn*

Holz, Kohlen.

\switchcolumn

\begin{greek}[variant=ancient]%
ξύλα, ἄνθρακες.

\end{greek}%
\switchcolumn*

Die Möbel.

\switchcolumn

\begin{greek}[variant=ancient]%
τὰ ἔπιπλα.

\end{greek}%
\switchcolumn*

Der Ofen.

\switchcolumn

\begin{greek}[variant=ancient]%
ἡ ἑστία.

\end{greek}%
\switchcolumn*

Das Pianoforte.

\switchcolumn

\begin{greek}[variant=ancient]%
τὸ κλειδοκύμβαλον.

\end{greek}%
\switchcolumn*

Der Saal.

\switchcolumn

\begin{greek}[variant=ancient]%
ἡ αἴθουσα.

\end{greek}%
\switchcolumn*

Das Schlafzimmer.

\switchcolumn

\begin{greek}[variant=ancient]%
ὁ κοιτών.

\end{greek}%
\switchcolumn*

Der Schrank.

\switchcolumn

\begin{greek}[variant=ancient]%
ἡ σκευοθήκη.

\end{greek}%
\switchcolumn*

Der Kleiderschrank.

\switchcolumn

\begin{greek}[variant=ancient]%
ἡ ἱματιοθήκη.

\end{greek}%
\switchcolumn*

Der Schreibtisch.

\switchcolumn

\begin{greek}[variant=ancient]%
τὸ γραφεῖον.

\end{greek}%
\switchcolumn*

Die Schwefelhölzchen.

\switchcolumn

\begin{greek}[variant=ancient]%
τὰ θειαφοκέρια.{*}

\end{greek}%
\switchcolumn*

Die Seife.

\switchcolumn

\begin{greek}[variant=ancient]%
ὁ σάπων.

\end{greek}%
\switchcolumn*

Das Sopha.

\switchcolumn

\begin{greek}[variant=ancient]%
τὸ ἀνάκλιντρον.

\end{greek}%
\switchcolumn*

Die Treppe.

\switchcolumn

\begin{greek}[variant=ancient]%
ἡ κλῖμαξ, τὸ ἀνάβαθρον.

\end{greek}%
\switchcolumn*

Die Gardinen.

\switchcolumn

\begin{greek}[variant=ancient]%
τὸ παραπέτασμα.

\end{greek}%
\switchcolumn*

Das Waschbecken.

\switchcolumn

\begin{greek}[variant=ancient]%
ἡ λεκάνη.

\end{greek}%
\switchcolumn*

Der Waschtisch.

\switchcolumn

\begin{greek}[variant=ancient]%
ὁ νιπτήρ.

\end{greek}%
\switchcolumn*

Das Zimmer.

\switchcolumn

\begin{greek}[variant=ancient]%
τὸ δωμάτιον.

\end{greek}%
\switchcolumn*

Der Uhrschlüssel.

\switchcolumn

\begin{greek}[variant=ancient]%
τὸ κλειδίον.

\end{greek}%
\switchcolumn*

Der Zahnstocher. 

\switchcolumn

\begin{greek}[variant=ancient]%
ἡ ὀδοντογλυφίς.

\end{greek}%
\switchcolumn[0]*[\StarOrnament]

Der Keiser.

\switchcolumn

\begin{greek}[variant=ancient]%
ὁ αὐτοκράτως.

\end{greek}%
\switchcolumn*

Deutschland.

\switchcolumn

\begin{greek}[variant=ancient]%
Γερμανία.

\end{greek}%
\switchcolumn*

Die Deutschen.

\switchcolumn

\begin{greek}[variant=ancient]%
οἱ Γερμανοί.

\end{greek}%
\switchcolumn*

Österreich.

\switchcolumn

\begin{greek}[variant=ancient]%
Αὐστρία.{*}

\end{greek}%
\switchcolumn*

Ungarn.

\switchcolumn

\begin{greek}[variant=ancient]%
Οὐγγαρία.{*}

\end{greek}%
\switchcolumn*

England.

\switchcolumn

\begin{greek}[variant=ancient]%
Ἀγγλία.{*}

\end{greek}%
\switchcolumn*

Die Engländer.

\switchcolumn

\begin{greek}[variant=ancient]%
οἱ Ἄγγλοι.

\end{greek}%
\switchcolumn*

Rußland.

\switchcolumn

\begin{greek}[variant=ancient]%
Ρωσία.{*}\footnote{\begin{latin}%
sic.\textgreek[variant=ancient]{ «ʼΡωσία» φαίνεταί μοι βέλτιον ἤ «Ρωσία».}\end{latin}%
}

\end{greek}%
\switchcolumn*

Die Russen.

\switchcolumn

\begin{greek}[variant=ancient]%
οἱ Ρῶσοι.{*}\footnote{\begin{latin}%
sic.\textgreek[variant=ancient]{ «ʼΡῶσοι» φαίνεταί μοι βέλτιον ἤ «Ρῶσοι».}\end{latin}%
}

\end{greek}%
\switchcolumn*

Frankreich.

\switchcolumn

\begin{greek}[variant=ancient]%
Γαλλία.

\end{greek}%
\switchcolumn*

Die Franzosen.

\switchcolumn

\begin{greek}[variant=ancient]%
οἱ Γάλλοι.

\end{greek}%
\switchcolumn*

Dänemark.

\switchcolumn

\begin{greek}[variant=ancient]%
Δανία.{*}

\end{greek}%
\switchcolumn*

Italien.

\switchcolumn

\begin{greek}[variant=ancient]%
Ἰταλία.

\end{greek}%
\switchcolumn*

Spanien.

\switchcolumn

\begin{greek}[variant=ancient]%
Ἱσπανία.

\end{greek}%
\switchcolumn*

Türkei.

\switchcolumn

\begin{greek}[variant=ancient]%
Τουρκία.{*}

\end{greek}%
\switchcolumn*

Berlin.

\switchcolumn

\begin{greek}[variant=ancient]%
Βερόλινον.{*}

\end{greek}%
\switchcolumn*

Wien.

\switchcolumn

\begin{greek}[variant=ancient]%
Βιέννη.{*}

\end{greek}%
\switchcolumn*

Peters\textcompwordmark{}burg. 

\switchcolumn

\begin{greek}[variant=ancient]%
Πετρούπολις.{*}

\end{greek}%
\switchcolumn*

Paris.

\switchcolumn

\begin{greek}[variant=ancient]%
Παρίσιοι.{*}

\end{greek}%
\switchcolumn*

London.

\switchcolumn

\begin{greek}[variant=ancient]%
Λόνδινον.{*}

\end{greek}%
\switchcolumn*

Der Congreß.

\switchcolumn

\begin{greek}[variant=ancient]%
τὸ συνέδριον.

\end{greek}%
\switchcolumn*

Die Commission.

\switchcolumn

\begin{greek}[variant=ancient]%
ἡ ἐπιτροπή.

\end{greek}%
\switchcolumn*

Fürst Bismarck.

\switchcolumn

\begin{greek}[variant=ancient]%
ὁ πρίγκιψ Βίσμαρκ.

\end{greek}%
\switchcolumn*

Er lebe hoch!

\switchcolumn

\begin{greek}[variant=ancient]%
ζέτω!

\end{greek}%
\switchcolumn[0]*[


\subsection*{\emph{Die Wochentage} heißen neugriechisch:}

]Sonntag.

\switchcolumn

\begin{greek}[variant=ancient]%
(ἡ\footnote{\begin{latin}%
orig. \textgreek[variant=ancient]{ὴ}\end{latin}%
}) κυριακή.

\end{greek}%
\switchcolumn*

Montag.

\switchcolumn

\begin{greek}[variant=ancient]%
ἡ δευτέρα.

\end{greek}%
\switchcolumn*

Dienstag.

\switchcolumn

\begin{greek}[variant=ancient]%
ἡ τρίτη.

\end{greek}%
\switchcolumn*

Mittwoch.

\switchcolumn

\begin{greek}[variant=ancient]%
ἡ τετάρτη.

\end{greek}%
\switchcolumn*

Donners\textcompwordmark{}tag.

\switchcolumn

\begin{greek}[variant=ancient]%
ἡ πέμπτη.

\end{greek}%
\switchcolumn*

Freitag.

\switchcolumn

\begin{greek}[variant=ancient]%
(ἡ) παρασκευή \textgerman[spelling=old,babelshorthands=true]{(Küst\textcompwordmark{}tag)}.

\end{greek}%
\switchcolumn*

Sonnabend (Samstag).

\switchcolumn

\begin{greek}[variant=ancient]%
(τὸ) σάββατον.

\end{greek}%

\end{paracol}

\section*{Zum Merken und Citiren.\addcontentsline{toc}{section}{Allerlei zum Merken und Citiren}}

\noindent \begin{center}
\emph{Die neun Musen:}
\par\end{center}
\begin{greek}[variant=ancient]%
\begin{quote}
Κλειώ τ᾽ Εὐτέρπη τε Θάλειά τε Μελπονένη τε\\
Τερψιχόρη τ᾽ Ἐρατώ τε Πολύμνιά τ᾽ Οὐρανίη τε,\\
Καλλιόπη θ᾽· ἡ δὲ προφερεστάτη ἐστὶν ἁπασέων.

\source{\textgerman[spelling=old,babelshorthands=true]{Lateinisches
Merkwort: }\textlatin{TUM PECCET. (Hesiod. Theog. 77.)}}
\end{quote}
\end{greek}%
\noindent \begin{center}
\emph{Die drei Grazien:}
\par\end{center}
\begin{greek}[variant=ancient]%
\begin{quote}
Ἀγλαΐη τε καὶ Εὐφροσύνη Θαλίη τ᾽ ἐρατείνη. 

\source{\textlatin{(Hesiod. Theog. 909.)}}
\end{quote}
\end{greek}%
\noindent \begin{center}
\emph{Die drei Parzen:}
\par\end{center}
\begin{greek}[variant=ancient]%
\begin{quote}
Κλωθώ τε Λάχεσίς τε καὶ Ἄτροπος, αἵ τε διδοῦσι\\
θνητοῖς ἀνθρώποισιν ἔχειν ἀγαθόν τε κακόν τε. 

\source{\textlatin{(Hesiod. Theog. 905.)}}
\end{quote}
\end{greek}%
\noindent \begin{center}
\emph{Die drei Gorgonen:}
\par\end{center}
\begin{greek}[variant=ancient]%
\begin{quote}
Σθεινώ τ᾽ Εὐρυάλη τε Μέδουσά τε λυγρὰ παθοῦσα.

\source{\textlatin{(Hesiod. Theog. 276.)}}
\end{quote}
\end{greek}%
\noindent \begin{center}
\emph{Scipio bei Numantia über Gracchus:}
\par\end{center}
\begin{greek}[variant=ancient]%
\begin{quote}
ὥς ἀπόλοιτο καὶ ἄλλος, ὅτις τοιαῦτά γε ῥέζοι. 

\source{\textlatin{(Hom. Od. 1, 47.)}}
\end{quote}
\end{greek}%
\noindent \begin{center}
\emph{Cicero's Wahlspruch:}
\par\end{center}
\begin{greek}[variant=ancient]%
\begin{quote}
αἲεν ἀριστεύειν καὶ ὑπείροχον ἔμμεναι ἄλλων. 

\source{\textlatin{(Hom. Il. 6, 208.)}}
\end{quote}
\end{greek}%
\noindent \begin{center}
\emph{Hector's Wahlspruch:}
\par\end{center}
\begin{greek}[variant=ancient]%
\begin{quote}
εἷς οἰωνὸς ἄριστος, ἀμύνεσθαι περὶ πάτρης. 

\source{\textlatin{(Hom. Il. 12, 243.)}}
\end{quote}
\end{greek}%
\noindent \begin{center}
\emph{Alexander's des Großen Wahlspruch:}
\par\end{center}
\begin{greek}[variant=ancient]%
\begin{quote}
ἄμφότερον, βασιλεύς τ᾽ ἀγαθός κρατερός τ᾽ αἰχμητής. 

\source{\textlatin{(Hom. Il. 3, 197.)}}
\end{quote}
\end{greek}%
\noindent \begin{center}
\emph{Scipio auf den Trümmern Karthago's.}
\par\end{center}
\begin{greek}[variant=ancient]%
\begin{quote}
ἔσσεται ἦμαρ, ὅτ᾽ ἄν ποτ᾽ ὀλώλῃ Ἴλιος ἱρή\\
καὶ Πρίαμος καὶ λαὸς ἐϋμμελίω Πριάμοιο. 

\source{\textlatin{(Hom. Il. 6, 448.)}}
\end{quote}
\end{greek}%
\noindent \begin{center}
\emph{Die sieben Weisen:}
\par\end{center}
\begin{greek}[variant=ancient]%
\begin{quote}
Ἑπτὰ σοφῶν, Κλεόβουλε, σὲ μὲν τεκνώσατο Λίνδος·\\
φατὶ δὲ Συσιφία χθὼν Περίανδρον ἔχειν·\\
Πιττακὸν ἁ Μυτιλάνα· Βίαντα δὲ δῖα Πριήνη·\\
Μίλητος δὲ Θαλῆν, ἄκρον ἔρεισμα Δίκας·\\
ἁ Σπάρτα Χίλωνα· Σόλωνα δὲ Κεκροπὶς αἶα.\\
πάντας ἀριζάλου σωφροσύνας φύλακας. 
\end{quote}
\end{greek}%
\noindent \begin{center}
Die Aus\textcompwordmark{}sprüche der sieben weisen (nach Diogenes
Laërtius):
\par\end{center}

Thales: \textgreek[variant=ancient]{γνῶθι σαυτόν!} (Erkenne dich selbst!)

Solon: \textgreek[variant=ancient]{μηδὲν ἄγαν!} (Nichts übertreiben!)

Chilon: \textgreek[variant=ancient]{ἐγγύα πάρα δ᾽ ἄτα!} (Bürgen thut
würgen In Geldsachen hört die Gemüthlichkeit auf.)

Pittacus: \textgreek[variant=ancient]{καιρὸν γνῶθι!} (Nimm den Augenblick
wahr!)

Bias: \textgreek[variant=ancient]{οἱ πλεῖστοι κακοί.} (Viele Köche
verderben den Brei.)

Kleobulus: \textgreek[variant=ancient]{μέτρον ἄριστον.} (Maßhalten
ist gut.)

Periander: \textgreek[variant=ancient]{μελέτη τὸ πᾶν.} (Übung macht
den Meister.)

Das (angeblich) delphische Orakel über Sokrates:
\begin{greek}[variant=ancient]%
\begin{quote}
Σοφὸς Σοφοκλῆς, σοφώτερος δ᾽ Εὐριπίδης,\\
Ἀνδρῶν δὲ πάντων Σωκράτης σοφώτατος. 

\source{\textlatin{(Schol. Aristoph. Nub. v. 144.)}}
\end{quote}
\end{greek}%
\noindent \begin{center}
\emph{Die Worte des Archimedes:}
\par\end{center}
\begin{enumerate}
\item \textgreek[variant=ancient]{Εὕρηκα!}
\item \textgreek[variant=ancient]{δός μοι ποῦ στῶ καὶ τὰν γᾶν κινασῶ! }
\item \textlatin{noli istud disturbare!}
\end{enumerate}
\noindent \begin{center}
\emph{Kaiser Augustus auf dem Sterbebette:}
\par\end{center}
\begin{greek}[variant=ancient]%
\begin{quote}
— — εἰ δὲ πᾶν ἔχει καλῶς, τῷ παιγνίῳ\\
Δότε κρότον καὶ πάντες ὑμεῖς μετὰ χαρᾶς κτυπήσατε! 

\source{\textlatin{(Sueton. Octav. 99.)}}
\end{quote}
\end{greek}%
\noindent \begin{center}
\emph{Die spartanische Mutter zu ihrem Sohne:}
\par\end{center}
\begin{greek}[variant=ancient]%
\begin{quote}
Τέκνον, ἢ τὰν ἢ ἐπὶ τᾶς!

\source{\textlatin{(Plutarch. }Λακαινῶν ἀποφθέγματα.\textlatin{)}}
\end{quote}
\end{greek}%
\noindent \begin{center}
\emph{Weg mit den sorgen!}
\par\end{center}
\begin{greek}[variant=ancient]%
\begin{quote}
τὸ σήμερον μέλει μοι,\\
τὸ δ᾽ αὔριον τίς οἶδεν; 

\source{\textgerman[spelling=old,babelshorthands=true]{(Anakreon)}}
\end{quote}
\end{greek}%
\noindent \begin{center}
\emph{Griechische Tages\textcompwordmark{}eintheilung:}\\
6 Stunden für die Arbeit, 4 Stunden für den Lebens\textcompwordmark{}genuß:
\par\end{center}
\begin{greek}[variant=ancient]%
\begin{quote}
ἓξ ὧραι μόχθοις ἱκανώταται· αἱ δὲ μετ᾽ αὐτὰς\\
γράμμασι δεικνύμεναι ζῆθι λέγουσι βροτοῖς.

1— 6: αʹ. βʹ. γʹ. δʹ. εʹ. ϛʹ.

7—10: ζʹ. ηʹ. θʹ. ιʹ. 

\source{\textgerman[spelling=old,babelshorthands=true]{(Alter Spruch.)}}
\end{quote}
\end{greek}%


\noindent \begin{center}
Druck von Hesse \& Becker in Leipzig.
\par\end{center}

\pagebreak{}

\begin{center}
{\Huge{}(Das originale Buch hat Ankündigungen hier.)}\\

\par\end{center}{\Huge \par}



\cleardoublepage{}


\part*{Redaktionelle Hinweise\protect \\
zur Digitalisierung und Setzung\protect \\
des Buches}

Der originale Text hat \emph{keine} Fußnote, aber der Digitalsetzer
fügt \emph{alle} die Fußnoten ein.


\section*{Buch\textcompwordmark{}staben}

Aa Bb Cc Dd Ee Ff Gg Hh Ii Jj Kk Ll Mm Nn Oo Pp Qq Rr Sss Tt Uu Vv
Ww Xx Yy Zz Ää Öö Üü ß

\begin{latin}%
A\textgerman[spelling=old,babelshorthands=true]{Aa }U\textgerman[spelling=old,babelshorthands=true]{Uu;
}C\textgerman[spelling=old,babelshorthands=true]{Cc }E\textgerman[spelling=old,babelshorthands=true]{Ee
}S\textgerman[spelling=old,babelshorthands=true]{Sss }G\textgerman[spelling=old,babelshorthands=true]{Gg;
}K\textgerman[spelling=old,babelshorthands=true]{Kk }H\textgerman[spelling=old,babelshorthands=true]{Hh;
}N\textgerman[spelling=old,babelshorthands=true]{Nn }R\textgerman[spelling=old,babelshorthands=true]{Rr
}X\textgerman[spelling=old,babelshorthands=true]{Xx; }M\textgerman[spelling=old,babelshorthands=true]{Mm
}W\textgerman[spelling=old,babelshorthands=true]{Ww }V\textgerman[spelling=old,babelshorthands=true]{Vv
}B\textgerman[spelling=old,babelshorthands=true]{Bb }Y\textgerman[spelling=old,babelshorthands=true]{Yy;
}O\textgerman[spelling=old,babelshorthands=true]{Oo }Q\textgerman[spelling=old,babelshorthands=true]{Qq
}P\textgerman[spelling=old,babelshorthands=true]{Pp }D\textgerman[spelling=old,babelshorthands=true]{Dd;
}T\textgerman[spelling=old,babelshorthands=true]{Tt }L\textgerman[spelling=old,babelshorthands=true]{Ll;
}IJ\textgerman[spelling=old,babelshorthands=true]{Iij }F\textgerman[spelling=old,babelshorthands=true]{Ff;
}Z\textgerman[spelling=old,babelshorthands=true]{Zz }ß\textgerman[spelling=old,babelshorthands=true]{ß}

\end{latin}%
\begin{greek}[variant=ancient]%
\begin{comment}
Α\,α Β\,ββ Γ\,γ Δ\,δ Ε\,ε Ζ\,ζ Η\,η Θ\,θθ Ι\,ι Κ\,κ Λ\,λ
Μ\,μ Ν\,ν Ξ\,ξ Ο\,ο Π\,ππ Ρ\,ρρ Σ\,σς Τ\,τ Υ\,υ Φ\,φφ Χ\,χ
Ψ\,ψ Ω\,ω; στ
\end{comment}


\end{greek}%

\section*{Buch\textcompwordmark{}stabenverbund}

Diese Buch\textcompwordmark{}stabenkombinationen werden Buch\textcompwordmark{}stabenverbund
(auch oder Ligatur), wenn es keine Grenze zwischen die Buch\textcompwordmark{}staben
ist: ch, ck, ſt, tz; ff, fi, fl, ft, ll, ſi, ſſ; (seit Anfang des 20.
Jh.) ſch. \quotedblbase Eins`` also \quotedblbase Einsatz``, und
\quotedblbase Wachstube`` (eng. \textenglish{guardhouse}, lat. \textlatin{commissarius})
also \quotedblbase Wachs\textcompwordmark{}tube`` (eng. \textenglish{wax
tube}, lat. \textlatin{tubus cerae}).

Laut Wikipädia (\url{http://de.wikipedia.org/wiki/Fraktursatz}),
\quotedblbase ch``, \quotedblbase ck``, \quotedblbase ſt`` und
\quotedblbase tz`` werden im Sperrsatz nicht aufgelöst, also alle
anderen Ligaturen werden aufgelöst und gesperrt: \emph{ch, ck, ſt,
tz; ff, fi, fl, ft, ll, ſi, ſſ;} \emph{ſch}.


\section*{Wörter}
\begin{itemize}
\item Verb, dessen Ende \quotedblbase -ieren`` im neudeutschen Sprache
ist, wird \quotedblbase -iren``.
\item \quotedblbase gibt`` wird \quotedblbase giebt``.
\item \quotedblbase C``, die in \quotedblbase K`` verändert wird, bleibt
weiterhin bestehen. z.\,B.: Object \quotedblbase Objekt``, activ
\quotedblbase aktiv``, corrigirt \quotedblbase korrigiert``.
\end{itemize}
Dieses Dokument, dessen ursprüngliche Buch (\url{https://archive.org/details/sprechensieatti00johngoog})
im \quotedblbase Internet Archive`` erhältlich ist, wurde mit \textlatin{\LaTeX{}}
gesetzt. Sein Quelltext ist online erhältlich: \url{https://github.com/na4zagin3/Sprechen-Sie-attisch}.

Nachdem Zagin (@na4zagin3) des Zirkels \quotedblbase Hyalinios``
digitalisierte das Buch, veröffentlichte er am 31.\ Dezember 2015
es, um auf den 89.\ \emph{Comic Market} zu bringen.
\end{document}
