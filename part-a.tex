\switchcolumn*[

\part{%
	\trDE{Allgemeinen Inhalts.}%
	\trJA{一般的事項}%
}

]
\switchcolumn*[
\section{%
	\trDE{Guten Tag!}%
	\trJA{こんにちは!}%
}

]%
\indent
\trDE{Ah! Guten Tag!}%
\trJA{おお、こんにちは!}

\switchcolumn

\begin{greek}[variant=ancient]%
ὦ χαῖρε!

\end{greek}%
\switchcolumn*

\trDE{Guten Morgen, Karl!}%
\trJA{おはやう、カール\footnote{ΚάρολοςはKarlのラテン語形Carolusをギリシャ語化した形のやうである。}!}

\switchcolumn

\begin{greek}[variant=ancient]%
χαῖρ' ὦ Κάρολε!

\end{greek}%
\switchcolumn*

\trDE{Guten Morgen, Gustav! (Erwiderung)}%
\trJA{おはやう、グスタフ!(返答)}

\switchcolumn

\begin{greek}[variant=ancient]%
καὶ σύγε ὦ Γούσταβε!

\end{greek}%
\switchcolumn*

\trDE{Seien Sie mir schön willkommen!}%
\trJA{Seien Sie mir schön willkommen!}% ToDo

\switchcolumn

\begin{greek}[variant=ancient]%
ὦ χαῖρε, φίλτατε!

\end{greek}%
\switchcolumn*

\trDE{Ah! freue mich außerordentlich!}%
\trJA{Ah! freue mich außerordentlich!}% ToDo

\switchcolumn

\begin{greek}[variant=ancient]%
ἀσπάζομαι!

\end{greek}%
\switchcolumn*

\trDE{Freue mich außerordentlich, Herr Müller!}%
\trJA{ようこそ、ミュラーさん。歓迎します。}% ToDo

\switchcolumn

\begin{greek}[variant=ancient]%
Μύλλερον ἀσπάζομαι!

\end{greek}%
\switchcolumn*

\trDE{Ganz auf meiner Seite!}%
\trJA{私もです。}% ToDo

\switchcolumn

\begin{greek}[variant=ancient]%
κἄγογέ σε!

\end{greek}%
\switchcolumn*

\trDE{Guten Tag! Guten Tag! Wie freue ich mich, daß Sie gekommen sind, Verehrtester!}%
\trJA{今日は、今日は、友よ。お越し下さって、嬉しく思います。}% ToDo


\switchcolumn

\begin{greek}[variant=ancient]%
χαῖρε, χαῖρε, ὡς ἀσμένῳ μοι ἦλθες, ὦ φίλτατε!

\end{greek}%
\switchcolumn*

\trDE{Ah! Guten Tag! Was bringen Sie?}%
\trJA{おお、こんにちは。何を持って来たのですか。}% ToDo

\switchcolumn

\begin{greek}[variant=ancient]%
ὦ χαῖρε, τί φέρεις!

\end{greek}%
\switchcolumn*

\trDE{Ah! Guten Tag, Perikles; was steht zu Diensten?}%
\trJA{おお、今日は、ペリクレースよ。何を持って来たのですか。}% ToDo

\switchcolumn

\begin{greek}[variant=ancient]%
ὦ χαῖρε, Περίκλεις, τί ἔστιν;

\end{greek}%
\switchcolumn*

\trDE{Giebt's was Neues?}%
\trJA{何か新しいことはありませんか?}% ToDo

\switchcolumn

\begin{greek}[variant=ancient]%
λέγεται τί καινόν; (νεώτερον, \textgerman[spelling=old,babelshorthands=true]{Schlimmes})

\end{greek}%
\switchcolumn*

\trDE{Guten Abend, meine Herren (meine Damen)! Meine (jungen) Damen!}%
\trJA{今晩は。諸君、(御夫人方)、御令嬢!}% ToDo

\switchcolumn

\begin{greek}[variant=ancient]%
χαίρετε, ὦ φίλοι (ὦ δέσποιναι)! ὦ κόραι!

\end{greek}%
\switchcolumn*

\trDE{Paul läßt Sie grüßen.}%
\trJA{パウロに挨拶するやう言われました。}% ToDo

\switchcolumn

\begin{greek}[variant=ancient]%
Παῦλος ἐπέστειλε φράσαι χαίρειν σοι.

\end{greek}%
\switchcolumn*

\trDE{Mein lieber Herr!}%
\trJA{おお、友(男)よ。}% ToDo

\switchcolumn

\begin{greek}[variant=ancient]%
ὦ φίλ' ἄνερ!

\end{greek}%
\switchcolumn*[


\section{%
\trDE{Wie geht's?}%
\trJA{元気?}% ToDo
}

]\indent %
\begin{tabular}{lc}
\trDE{Wie geht es Ihnen?}%
\trJA{御機嫌如何ですか?}% ToDo
	& \ldelim\}{2}{1em}[]\tabularnewline
\trDE{Was machen Sie?}%
\trJA{何をされているのですか?}% ToDo
	& \tabularnewline
\end{tabular}

\switchcolumn

\begin{greek}[variant=ancient]%
\vspace{0.5em}
τί πράττεις;

\end{greek}%
\switchcolumn*

\trDE{Danke, es geht mir ganz wohl.}%
\trJA{頗る良好です。}% ToDo

\switchcolumn

\begin{greek}[variant=ancient]%
πάντ' ἀγαθὰ πράττω, ὦ φίλε.

\end{greek}%
\switchcolumn*

\trDE{Ich bin besser daran, als gestern.}%
\trJA{昨日より良くなりました。}% ToDo

\switchcolumn

\begin{greek}[variant=ancient]%
ἄμεινον πράττω ἢ χθές.

\end{greek}%
\switchcolumn*

\trDE{Wie geht es Ihrem Vater?}%
\trJA{父君はお元気ですか?}% ToDo

\switchcolumn

\begin{greek}[variant=ancient]%
τί πράττει ὁ πατήρ σου;

\end{greek}%
\switchcolumn*

\trDE{Es geht ihm recht gut.}%
\trJA{よくやっております。}% ToDo

\switchcolumn

\begin{greek}[variant=ancient]%
εὐδαιμόνως πράττει.

\end{greek}%
\switchcolumn*

\trDE{Wie steht es sonst bei euch?}%
\trJA{他に何かございましたか?}% ToDo

\switchcolumn

\begin{greek}[variant=ancient]%
τί δ' ἄλλο παρ' ὑμῖν;

\end{greek}%
\switchcolumn*

\trDE{Wie befinden Sie sich?}%
\trJA{如何なされましたか?}% ToDo

\switchcolumn

\begin{greek}[variant=ancient]%
πῶς ἔχεις;

\end{greek}%
\switchcolumn*

\trDE{Schlecht.}%
\trJA{悪うございます。}% ToDo

\switchcolumn

\begin{greek}[variant=ancient]%
ἔχω κακῶς.

\end{greek}%
\switchcolumn*

\trDE{Ich habe keine Freude mehr am Leben.}%
\trJA{生まれてこの方、喜びなど一つもありはしませんでした。}% ToDo

\switchcolumn

\begin{greek}[variant=ancient]%
οὐδεμίαν ἔχω τῷ βίῳ χάριν.

\end{greek}%
\switchcolumn*

\trDE{Es geht mir (wirthschaftlich) nicht gut.}%
\trJA{(経済的に)よくありません。}% ToDo

\switchcolumn

\begin{greek}[variant=ancient]%
κακῶς πράττω.

\end{greek}%
\switchcolumn*

\trDE{Es steht schlecht mit mir.}%
\trJA{あまりよくありません。}% ToDo

\switchcolumn

\begin{greek}[variant=ancient]%
φαῦλόν ἐστι τὸ ἐμὸν πρᾶγμα.

\end{greek}%
\switchcolumn*

\trDE{Wie lebt sich's in Leipzig?}%
\trJA{ライプツィヒでの生活はいかがですか。}% ToDo

\switchcolumn

\begin{greek}[variant=ancient]%
τίς ἐσθ' ὁ ἐν Λειψίᾳ{*} βίος;

\end{greek}%
\switchcolumn*

\trDE{Ganz hübsch.}%
\trJA{誠に宜しゅうございます。}% ToDo

\switchcolumn

\begin{greek}[variant=ancient]%
οὐκ ἄχαρις.

\end{greek}%
\switchcolumn*[


\section{%
	\trDE{Was fehlt Ihnen?}%
	\trJA{何か悪い所はありますか?}%
}

]\indent %
\begin{tabular}{lc}
\trDE{Was fehlt Ihnen?}%
\trJA{何か悪い所はありますか?}%
& \ldelim\}{2}{1em}[]\tabularnewline
\trDE{Was ist mit Ihnen?}%
\trJA{何かありましたか?}%
& \tabularnewline
\end{tabular}

\switchcolumn

\begin{greek}[variant=ancient]%
\vspace{0.5em}
τί πράττεις;

\end{greek}%
\switchcolumn*

\trDE{Es geht mir merkwürdig.}%
\trJA{著しく(良い)。}%

\switchcolumn

\begin{greek}[variant=ancient]%
πάσχω θαυμαστόν.

\end{greek}%
\switchcolumn*

\trDE{Was haben Sie für Schmerzen.}%
\trJA{何処か痛みますか。}

\switchcolumn

\begin{greek}[variant=ancient]%
τί κάμνεις.

\end{greek}%
\switchcolumn*

\trDE{Was ist Ihnen zugestoßen?}%
\trJA{何があったのですか?}%

\switchcolumn

\begin{greek}[variant=ancient]%
τί πέπονθας.

\end{greek}%
\switchcolumn*

\trDE{Wie ist es Ihnen ergangen?}%
\trJA{何があったのですか?}%

\switchcolumn

\begin{greek}[variant=ancient]%
τί ἔπαθες.

\end{greek}%
\switchcolumn*

\trDE{Warum seufzen Sie?}%
\trJA{何故嘆いているのか?}%

\switchcolumn

\begin{greek}[variant=ancient]%
τί στένεις.

\end{greek}%
\switchcolumn*

\trDE{Warum sind Sie so verstimmt?}%
\trJA{何故苛立っているのか?}%

\switchcolumn

\begin{greek}[variant=ancient]%
τί δυσφορεῖς.

\end{greek}%
\switchcolumn*

\trDE{Sieh nicht so finster aus, mein Lieber!}%
\trJA{そう陰気になるな、我が子よ!}%

\switchcolumn

\begin{greek}[variant=ancient]%
μὴ σκυθρώπαζε, ὦ τέκνον!

\end{greek}%
\switchcolumn*

\trDE{Ich langweile mich hier.}%
\trJA{ここに居るのは飽きた。}%

\switchcolumn

\begin{greek}[variant=ancient]%
ἄχθομαι ἐνθάδε παρών.

\end{greek}%
\switchcolumn*

\trDE{Sie scheinen mir zu frieren.}%
\trJA{君は凍えているやうだ。}%

\switchcolumn

\begin{greek}[variant=ancient]%
ῥιγῶν μοι δοκεῖς.

\end{greek}%
\switchcolumn*

\trDE{Mir ist schwindlig.}%
\trJA{眩暈がする。}%

\switchcolumn

\begin{greek}[variant=ancient]%
ἰλιγγιῶ.

\end{greek}%
\switchcolumn*

\trDE{Ich habe Kopf\textcompwordmark{}schmerz.}%
\trJA{頭痛がする。}%

\switchcolumn

\begin{greek}[variant=ancient]%
ἀλγῶ τὴν κεφαλήν\footnote{\begin{latin}%
orig. \textgreek[variant=ancient]{κεφαλὴν}\end{latin}%
}!

\end{greek}%
\switchcolumn*

\trDE{Sie haben jedenfalls Katzenjammer.}%
\trJA{確実に、君は二日酔いではないな。}%

\switchcolumn

\begin{greek}[variant=ancient]%
οὐκ ἔσθ' ὅπως οὐ κραιπαλᾷς.

\end{greek}%
\switchcolumn*

\trDE{An welcher Krankheit leben Sie?}%
\trJA{何の病気を患っているのか?}%

\switchcolumn

\begin{greek}[variant=ancient]%
τίνα νόσον νοσεῖς;

\end{greek}%
\switchcolumn*

\trDE{Sie haben doch wohl die Seekrankheit.}%
\trJA{たぶん、君は船酔いだ。}%

\switchcolumn

\begin{greek}[variant=ancient]%
ναυτιᾷς δήπου.

\end{greek}%
\switchcolumn*

\trDE{Du bekommst den Schnupfen.}%
\trJA{君は(鼻)風邪を引いている。}%

\switchcolumn

\begin{greek}[variant=ancient]%
κόρυζά σε λαμβάνει.

\end{greek}%
\switchcolumn*

\trDE{Ich leide an den Augen.}%
\trJA{眼炎を患っている。}%

\switchcolumn

\begin{greek}[variant=ancient]%
ὀφθαλμιῶ.

\end{greek}%
\switchcolumn*

\trDE{Bist du müde?}%
\trJA{もう疲れた?}%

\switchcolumn

\begin{greek}[variant=ancient]%
ἆρα κέκμηκας;

\end{greek}%
\switchcolumn*

\trDE{Mir thun die Beine weh von dem weiten Wege.}%
\trJA{長い旅路を来たので脚が痛い。}%

\switchcolumn

\begin{greek}[variant=ancient]%
ἀλγῶ τὰ σκέλη μακρὰν ὁδὸν διεληλυθώς.

\end{greek}%
\switchcolumn*

\trDE{Du bist besser zu Fuße als ich.}%
\trJA{歩くことにかけては、君は僕より強い。}%

\switchcolumn

\begin{greek}[variant=ancient]%
κρείττων εἶ μου σὺ βαδίζειν.

\end{greek}%
\switchcolumn*

\trDE{Sie wird ohnmächtig.}%
\trJA{彼女\footnote{唐突に現れた彼女は一体誰なんだろうか。}は気を失いつつある。}% TODO

\switchcolumn

\begin{greek}[variant=ancient]%
ὡρακιᾷ.

\end{greek}%
\switchcolumn*[


\section{%
	\trDE{Leben Sie wohl!}%
	\trJA{お元気で!}%
}

]

\indent
\trDE{Leben Sie wohl!}%
\trJA{お元気で!}%

\switchcolumn

\begin{greek}[variant=ancient]%
ὑγίαινε!

\end{greek}%
\switchcolumn*

\trDE{Ich will gehen, leben Sie wohl!}%
\trJA{さて、私は行く。お元気で!}%

\switchcolumn

\begin{greek}[variant=ancient]%
ἀλλ' εἶμι, σὺ δ' ὑγίαινε!

\end{greek}%
\switchcolumn*

\trDE{Leben Sie wohl (Erwiderung)!}%
\trJA{君も!}%

\switchcolumn

\begin{greek}[variant=ancient]%
καὶ σύγε!

\end{greek}%
\switchcolumn*

\trDE{Leben Sie recht wohl!}%
\trJA{お幸せに!}%

\switchcolumn

\begin{greek}[variant=ancient]%
χαῖρε πολλά!

\end{greek}%
\switchcolumn*

\trDE{Geben Sie mir eine Hand!}%
\trJA{右手\footnote{δεξιά, ἡは右手の意であるが、歓迎や挨拶の時にはδεξιὰν διδόναι [προτείνειν, ἐμβάλλειν]するものらしい。ホメーロスでは、差し出すものはδεξιάであってχείρは使われていなかったらしい。LSJ δεξιάより。}を差し出して下さい!}%

\switchcolumn

\begin{greek}[variant=ancient]%
ἔμβαλέ μοι τὴν δεξιάν!

\end{greek}%
\switchcolumn*

\trDE{Nun so leben Sie denn wohl und behalten Sie mich in gutem Andenken!}%
\trJA{では、さらば。私のことを憶えておいてくれ。}%

\switchcolumn

\begin{greek}[variant=ancient]%
ἀλλὰ χαῖρε πολλὰ καὶ μέμνησό μου!

\end{greek}%
\switchcolumn*

\trDE{Auf Wiedersehen!}%
\trJA{ではまた!}%

\switchcolumn

\begin{greek}[variant=ancient]%
εἰς αὖθις!

\end{greek}%
\switchcolumn*

\trDE{Viel Vergnügen!}%
\trJA{楽しんで来てね!}%

\switchcolumn

\begin{greek}[variant=ancient]%
ἴθι χαίρων!

\end{greek}%
\switchcolumn*

\trDE{Gute Nacht!}%
\trJA{お休み!}%

\switchcolumn

\begin{greek}[variant=ancient]%
ὑγίαινε!\textgerman[spelling=old,babelshorthands=true]{ (Auch am Morgen
beim Abschied).}

\end{greek}%
\switchcolumn*[


\section{%
	\trDE{Ich bitte}%
	\trJA{すみません!}%
}

]
\indent
\trDE{Verzeihen Sie!}%
\trJA{すみません!}%

\switchcolumn

\begin{greek}[variant=ancient]%
συγγνώμην ἔχε!

\end{greek}%
\switchcolumn*

\trDE{Ent\textcompwordmark{}schuldigen Sie!}%
\trJA{ごめんなさい!}%

\switchcolumn

\begin{greek}[variant=ancient]%
σύγγνωθί μοι.

\end{greek}%
\switchcolumn*

\trDE{Es ist meins. Geben Sie mir es, bitte!}%
\trJA{それは私のものだ。こちらに渡してくださいませんか。}%

\switchcolumn

\begin{greek}[variant=ancient]%
ἐστι τὸ ἐμόν. ἀλλὰ δός μοι, ἀντιβολῶ!

\end{greek}%
\switchcolumn*

\trDE{Ich bitte Sie, geben Sie es mir!}%
\trJA{どうか、それを私にくださいませんか?}%

\switchcolumn

\begin{greek}[variant=ancient]%
δός μοι πρὸς τῶν θεῶν!

\end{greek}%
\switchcolumn*

\trDE{Ich bitte Sie inständigst.}%
\trJA{どうか、よろしくお願いします。}%

\switchcolumn

\begin{greek}[variant=ancient]%
πρὸς τοῦ Διός, ἀντιβολῶ σε.

\end{greek}%
\switchcolumn*

\trDE{Ich bitte um Himmels\textcompwordmark{}willen!.}%
\trJA{天地神明に誓って。}%

\switchcolumn

\begin{greek}[variant=ancient]%
πρὸς πάντων θεῶν!

\end{greek}%
\switchcolumn*

\trDE{Thun Sie mir den Gefallen!}%
\trJA{頼み事があるのだが。}%

\switchcolumn

\begin{greek}[variant=ancient]%
χάρισαί μοι!

\end{greek}%
\switchcolumn*

\trDE{Nun, so thun Sie uns denn den Gefallen.}%
\trJA{我々の為に頼まれてくれないか。}%

\switchcolumn

\begin{greek}[variant=ancient]%
ἀλλὰ χάρισαι ἡμῖν!

\end{greek}%
\switchcolumn*

\trDE{Thun Sie mir einen kleinen Gefallen!}%
\trJA{ちょっとした頼み事があるのだが。}%

\switchcolumn

\begin{greek}[variant=ancient]%
χάρισαι βραχύ τί μοι!

\end{greek}%
\switchcolumn*

\trDE{Was soll ich Ihnen zu Gefallen thun?}%
\trJA{何をしたら良い?}%

\switchcolumn

\begin{greek}[variant=ancient]%
τί σοι χαρίσωμαι.

\end{greek}%
\switchcolumn*

\trDE{Sei so gut und gieb mir's.}%
\trJA{私にそれをくれないか?(どれくらい叮嚀なんだろ……?)}%

\switchcolumn

\begin{greek}[variant=ancient]%
βούλει μοι δοῦναι;

\end{greek}%
\switchcolumn*

\emph{Den} Gefallen will ich Ihnen thun.

\switchcolumn

\begin{greek}[variant=ancient]%
χαριοῦμαί σοι τοῦτο.

\end{greek}%
\switchcolumn*

Gleich!

\switchcolumn

\begin{greek}[variant=ancient]%
ταῦτα!

\end{greek}%
\switchcolumn*

Recht gern!

\switchcolumn

\begin{greek}[variant=ancient]%
φθόνος οὐδείς!

\end{greek}%
\switchcolumn*

Sagen Sie es doch gefälligst den Anderen!

\switchcolumn

\begin{greek}[variant=ancient]%
οὐ δῆτα γενναίως τοῖς ἄλλοις ἐρεῖς;

\end{greek}%
\switchcolumn*

Bitte, sag' es ihm doch!

\switchcolumn

\begin{greek}[variant=ancient]%
εἰπὲ δῆτα αὐτῷ πρὸς τῶν θεῶν!

\end{greek}%
\switchcolumn*

\emph{Darf ich mir erlauben} Ihnen einzuschenken?

\switchcolumn

\begin{greek}[variant=ancient]%
\emph{βούλει} ἐγχέω σοι πιεῖν;

\end{greek}%
\switchcolumn*[


\section{Ich danke}

]\indent Ich danke!

\switchcolumn

\begin{greek}[variant=ancient]%
ἐπαινῶ.

\end{greek}%
\switchcolumn*

Ich danke Ihnen!

\switchcolumn

\begin{greek}[variant=ancient]%
ἐπαινῶ τὸ σόν!

\end{greek}%
\switchcolumn*

Ich danke Ihnen für Ihre freundliche Gesinnung.

\switchcolumn

\begin{greek}[variant=ancient]%
ἐπαινῶ τὴν σὴν πρόνοιαν.

\end{greek}%
\switchcolumn*

Haben Sie vielen Dank dafür!

\switchcolumn

\begin{greek}[variant=ancient]%
εὖ γ' ἐμοίησας!

\end{greek}%
\switchcolumn*

Sie sind sehr gütig.

\switchcolumn

\begin{greek}[variant=ancient]%
γενναῖος εἶ.

\end{greek}%
\switchcolumn*

Ich werde Ihnen nur dankbar sein, wenn Sie das thun.

\switchcolumn

\begin{greek}[variant=ancient]%
χάριν γε εἴσομαι, ἐὰν τοῦτο ποιῇς.

\end{greek}%
\switchcolumn*

Ich bin Ihnen zu Danke verpflichtet.

\switchcolumn

\begin{greek}[variant=ancient]%
κεκάρισαί μοι.

\end{greek}%
\switchcolumn*

Der Himmel segne Sie tausendmal!

\switchcolumn

\begin{greek}[variant=ancient]%
πόλλ' ἀγαθὰ γένοιό σοι!

\end{greek}%
\switchcolumn*

Danke schön! (auch ablehnend.)

\switchcolumn

\begin{greek}[variant=ancient]%
καλῶς!

\end{greek}%
\switchcolumn*

Ich danke bestens! (des\textcompwordmark{}gl.)

\switchcolumn

\begin{greek}[variant=ancient]%
κάλλιστα· ἐπαινῶ.

\end{greek}%
\switchcolumn*

Bravo! Bravo!

\switchcolumn

\begin{greek}[variant=ancient]%
εὖγε! εὖγι.

\end{greek}%
\switchcolumn*

Wie herrlich!

\switchcolumn

\begin{greek}[variant=ancient]%
ὡς ἡδύ!

\end{greek}%
\switchcolumn*

Hurrah! (Freudenruf.)

\switchcolumn

\begin{greek}[variant=ancient]%
ἀλαλαί!

\end{greek}%
\switchcolumn*

Das macht nichts. Das ist einerlei.

\switchcolumn

\begin{greek}[variant=ancient]%
οὐδὲν διαφέρει.

\end{greek}%
\switchcolumn*

\begin{tabular}{lc}
Das kümmert mich wenig. & \ldelim\}{2}{1em}[]\tabularnewline
Daran liegt mir wenig. & \tabularnewline
\end{tabular}

\switchcolumn

\begin{greek}[variant=ancient]%
\vspace{0.5em}
ὀλίγον μέλει μοι.

\end{greek}%
\switchcolumn*

Was geht das \emph{mich} an?

\switchcolumn

\begin{greek}[variant=ancient]%
τί δ' ἐμοὶ ταῦτα;

\end{greek}%
\switchcolumn*

Was geht \emph{Sie} das an?.

\switchcolumn

\begin{greek}[variant=ancient]%
τί δ' σοὶ τοῦτο;

\end{greek}%
\switchcolumn*

Sie interessirt es wahrscheinlich nicht.

\switchcolumn

\begin{greek}[variant=ancient]%
σοὶ δ' ἴσως οὐδὲν μέλει.

\end{greek}%
\switchcolumn*

Da sieh \emph{du} zu!

\switchcolumn

\begin{greek}[variant=ancient]%
αὐτὸς σκόπει σύ!

\end{greek}%
\switchcolumn*

Es ist einmal so Sitte.

\switchcolumn

\begin{greek}[variant=ancient]%
νόμος γάρ ἐστιν.

\end{greek}%
\switchcolumn*[


\section{Können Sie Griechisch?}

]\indent Können Sie Griechisch?

\switchcolumn

\begin{greek}[variant=ancient]%
ἐπίστασαι ἑλληνίζειν;

\end{greek}%
\switchcolumn*

Ein wenig.

\switchcolumn

\begin{greek}[variant=ancient]%
ὀλίγον τι.

\end{greek}%
\switchcolumn*

Natürlich!

\switchcolumn

\begin{greek}[variant=ancient]%
εἰκότως γε!

\end{greek}%
\switchcolumn*

Ja freilich!

\switchcolumn

\begin{greek}[variant=ancient]%
μάλιστα!

\end{greek}%
\switchcolumn*

Ja gewiß!

\switchcolumn

\begin{greek}[variant=ancient]%
ἔγωγε νὴ Δία!

\end{greek}%
\switchcolumn*

Darin bin ich stark.

\switchcolumn

\begin{greek}[variant=ancient]%
ταύτῃ κράτιστός εἰμι.

\end{greek}%
\switchcolumn*

Schön!

\switchcolumn

\begin{greek}[variant=ancient]%
καλῶς!

\end{greek}%
\switchcolumn*

Da wollen wir einmal Griechisch mit einander sprechen!

\switchcolumn

\begin{greek}[variant=ancient]%
διαλεχθῶμεν οὖν ἑλληνικῶς!

\end{greek}%
\switchcolumn*

Meinetwegen.

\switchcolumn

\begin{greek}[variant=ancient]%
οὐδὲν κωλύει.

\end{greek}%
\switchcolumn*

Was meinen sie?

\switchcolumn

\begin{greek}[variant=ancient]%
τί λέγεις;

\end{greek}%
\switchcolumn*

Verstehen Sie, was ich meine?

\switchcolumn

\begin{greek}[variant=ancient]%
ξυνίης τὰ λεγόμενα;

\end{greek}%
\switchcolumn*

Haben Sie verstanden, was ich meine?

\switchcolumn

\begin{greek}[variant=ancient]%
ξυνῆκας, ὃ λέγω;!

\end{greek}%
\switchcolumn*

Nein, ich verstehe es nicht.

\switchcolumn

\begin{greek}[variant=ancient]%
οὐ ξυνίημι μὰ Δία.

\end{greek}%
\switchcolumn*

Wiederholen Sie es gefälligst noch einmal!

\switchcolumn

\begin{greek}[variant=ancient]%
αὖθις ἐξ ἀρχῆς λέγε, ἀντιβολῶ!

\end{greek}%
\switchcolumn*

Seien Sie so gut und sprechen sie langsamer!

\switchcolumn

\begin{greek}[variant=ancient]%
βούλει σχολαίτερον λέγειν;

\end{greek}%
\switchcolumn*[


\section{Fragen}

]\indent Was giebt's?

\switchcolumn

\begin{greek}[variant=ancient]%
τί δ' ἔστιν;

\end{greek}%
\switchcolumn*

Wie?

\switchcolumn

\begin{greek}[variant=ancient]%
τί λέγεις;

\end{greek}%
\switchcolumn*

\emph{Was} denn?

\switchcolumn

\begin{greek}[variant=ancient]%
τί δή;

\end{greek}%
\switchcolumn*

Was \emph{denn}?

\switchcolumn

\begin{greek}[variant=ancient]%
τί δαί;

\end{greek}%
\switchcolumn*

\emph{Wie} denn?

\switchcolumn

\begin{greek}[variant=ancient]%
πῶς δή;

\end{greek}%
\switchcolumn*

Wie \emph{denn}?

\switchcolumn

\begin{greek}[variant=ancient]%
πῶς δαί;

\end{greek}%
\switchcolumn*

Warum denn?

\switchcolumn

\begin{greek}[variant=ancient]%
ὁτιὴ τί δή; τιὴ τί δή;

\end{greek}%
\switchcolumn*

Wes\textcompwordmark{}halb?

\switchcolumn

\begin{greek}[variant=ancient]%
τίνος ἕνεκα;

\end{greek}%
\switchcolumn*

In wiefern?

\switchcolumn

\begin{greek}[variant=ancient]%
τίνι τρόπῳ;

\end{greek}%
\switchcolumn*

Wieso denn?

\switchcolumn

\begin{greek}[variant=ancient]%
πῶς δή;

\end{greek}%
\switchcolumn*

Bitte, wo?

\switchcolumn

\begin{greek}[variant=ancient]%
ποῦ δῆτα;

\end{greek}%
\switchcolumn*

Wohin? Woher?

\switchcolumn

\begin{greek}[variant=ancient]%
ποῖ; πόθεν;

\end{greek}%
\switchcolumn*

Wann?

\switchcolumn

\begin{greek}[variant=ancient]%
πηνίκα;

\end{greek}%
\switchcolumn*

Er straft ihn.

\switchcolumn

\begin{greek}[variant=ancient]%
κολάζει αὐτόν.

\end{greek}%
\switchcolumn*

Wofür?

\switchcolumn

\begin{greek}[variant=ancient]%
τί δράσαντα;

\end{greek}%
\switchcolumn*

Wodurch?

\switchcolumn

\begin{greek}[variant=ancient]%
τί δρῶν;

\end{greek}%
\switchcolumn*

Zu welchem Zwecke denn?

\switchcolumn

\begin{greek}[variant=ancient]%
ἵνα δὴ τί;

\end{greek}%
\switchcolumn*

Um was handelt es sich?

\switchcolumn

\begin{greek}[variant=ancient]%
τί τὸ πρᾶγμα;

\end{greek}%
\switchcolumn*

Meinen Sie nicht auch?

\switchcolumn

\begin{greek}[variant=ancient]%
οὐ καὶ σοὶ δοκεῖ;

\end{greek}%
\switchcolumn*

Wär's möglich?

\switchcolumn

\begin{greek}[variant=ancient]%
πῶς φής;

\end{greek}%
\switchcolumn*

Wo blieb' \emph{ich}?

\switchcolumn

\begin{greek}[variant=ancient]%
τί ἐγὼ δέ;

\end{greek}%
\switchcolumn*

Laß doch einmal sehen!

\switchcolumn

\begin{greek}[variant=ancient]%
φέρ' ἴδω!

\end{greek}%
\switchcolumn*

Nun, machen sie Fort\textcompwordmark{}schritte?

\switchcolumn

\begin{greek}[variant=ancient]%
τί δέ, ἐπιδώσεις λαμβάνεις;

\end{greek}%
\switchcolumn*[


\section{Wie heißen Sie?}

]\indent Wie heißen Sie?

\switchcolumn

\begin{greek}[variant=ancient]%
ὄνομά σοι τί ἐστιν;

\end{greek}%
\switchcolumn*

Wie heißen Sie mit Vor- und Zunamen?

\switchcolumn

\begin{greek}[variant=ancient]%
τίνα σοι ὀνόματα.

\end{greek}%
\switchcolumn*

Wie heißen Sie eigentlich?

\switchcolumn

\begin{greek}[variant=ancient]%
τί σοί ποτ' ἔστ' ὄνομα;

\end{greek}%
\switchcolumn*

Wer \emph{sind} Sie?

\switchcolumn

\begin{greek}[variant=ancient]%
σὺ δὲ τίς εἰ;

\end{greek}%
\switchcolumn*

Wer sind \emph{Sie}?

\switchcolumn

\begin{greek}[variant=ancient]%
τίς δ' εἶ σύ;

\end{greek}%
\switchcolumn*

Wer sind Sie eigentlich?

\switchcolumn

\begin{greek}[variant=ancient]%
σὺ δ' εἶ τίς ἐτεόν;

\end{greek}%
\switchcolumn*

Ich heiße Müller.

\switchcolumn

\begin{greek}[variant=ancient]%
ὄνομά μοι Μύλλερος.

\end{greek}%
\switchcolumn*

Wer ist eigentlich der hier?

\switchcolumn

\begin{greek}[variant=ancient]%
τίς ποθ' ὅδε;

\end{greek}%
\switchcolumn*

Wer muß das nur sein?

\switchcolumn

\begin{greek}[variant=ancient]%
τίς ἄρα ποτ' ἐστίν;

\end{greek}%
\switchcolumn*

Und wo sind Sie her?

\switchcolumn

\begin{greek}[variant=ancient]%
καὶ ποδαπός;

\end{greek}%
\switchcolumn*

Wo wohnen Sie?

\switchcolumn

\begin{greek}[variant=ancient]%
ποῦ κατοικεῖς;

\end{greek}%
\switchcolumn*

Ich wohne ganz in der Nähe.

\switchcolumn

\begin{greek}[variant=ancient]%
ἐγγύτατα οἰκῶ.

\end{greek}%
\switchcolumn*

Ich wohne weit.

\switchcolumn

\begin{greek}[variant=ancient]%
τηλοῦ οἰκῶ.

\end{greek}%
\switchcolumn*

Nennen Sie mich nicht bei Namen!

\switchcolumn

\begin{greek}[variant=ancient]%
μὴ κάλει μου τοὔνομα!

\end{greek}%
\switchcolumn*

So rufen Sie mich doch nicht, ich bitte Sie!

\switchcolumn

\begin{greek}[variant=ancient]%
οὐ μὴ καλεῖς με; ἱκετεύω!

\end{greek}%
\switchcolumn*[


\section{Wieviel Uhr ist es?}

]\indent Wie viel Uhr ist es?

\switchcolumn

\begin{greek}[variant=ancient]%
τίς ὥρα ἐστίν;

\end{greek}%
\switchcolumn*

Wie spät ist es am Tage?

\switchcolumn

\begin{greek}[variant=ancient]%
πηνίκ' ἐστὶ τῆς ἡμέρας;

\end{greek}%
\switchcolumn*

Es ist um Eins.

\switchcolumn

\begin{greek}[variant=ancient]%
ἐσὶ μία ὥρα.

\end{greek}%
\switchcolumn*

Es ist um Zwei (Dri, Vier).

\switchcolumn

\begin{greek}[variant=ancient]%
εἰσὶ δύο (τρεῖς, τέσσαρες) ὦραι.!

\end{greek}%
\switchcolumn*

Es ist ½2 Uhr.

\switchcolumn

\begin{greek}[variant=ancient]%
ἐστὶ μία ὥρα καὶ ἡμίσεια.

\end{greek}%
\switchcolumn*

Um welche Zeit?

\switchcolumn

\begin{greek}[variant=ancient]%
πηνίκα;

\end{greek}%
\switchcolumn*

Um ein Uur.

\switchcolumn

\begin{greek}[variant=ancient]%
τῇ πρώτῃ ὥρᾳ.

\end{greek}%
\switchcolumn*

Um zwei.

\switchcolumn

\begin{greek}[variant=ancient]%
τῇ δευτέρᾳ (ὥρα).

\end{greek}%
\switchcolumn*

Es ist noch weiter (später).

\switchcolumn

\begin{greek}[variant=ancient]%
περαιτέρω ἐστίν.

\end{greek}%
\switchcolumn*

Es ist ein Viertel nach Sieben.

\switchcolumn

\begin{greek}[variant=ancient]%
εἰσὶν ἑπτὰ ὦραι καὶ τέταρτον.

\end{greek}%
\switchcolumn*

Es ist drei Viertel auf Eins.

\switchcolumn

\begin{greek}[variant=ancient]%
εἰσὶ δώδεκα (ὦραι) καὶ τρία τέταρτα.

\end{greek}%
\switchcolumn*

Um die dritte Stunde.

\switchcolumn

\begin{greek}[variant=ancient]%
περὶ τὴν τρίτην ὥραν.!

\end{greek}%
\switchcolumn*

Gegen halb fünf.

\switchcolumn

\begin{greek}[variant=ancient]%
περὶ τὴν τετάρτην καὶ ἡμίσειαν!

\end{greek}%
\switchcolumn*

Ich werde um ¾11 Uhr kommen.

\switchcolumn

\begin{greek}[variant=ancient]%
ἥξω εἰς τὴν δεκάτην καὶ τρία τέταρτα.

\end{greek}%
\switchcolumn*[


\section{Tages\textcompwordmark{}zeiten}

]\indent Zu Mittag.

\switchcolumn

\begin{greek}[variant=ancient]%
ἐν μεσημβρίᾳ.

\end{greek}%
\switchcolumn*

Vormittags.

\switchcolumn

\begin{greek}[variant=ancient]%
πρὸ μεσημβρίας.

\end{greek}%
\switchcolumn*

Nachmittags.

\switchcolumn

\begin{greek}[variant=ancient]%
μετὰ μεσημβρίαν.

\end{greek}%
\switchcolumn*

Es ist hell.

\switchcolumn

\begin{greek}[variant=ancient]%
φῶς ἐστιν.

\end{greek}%
\switchcolumn*

Es ist (wird) dunkel.

\switchcolumn

\begin{greek}[variant=ancient]%
σκότος γίγνεται.

\end{greek}%
\switchcolumn*

Im Finstern.

\switchcolumn

\begin{greek}[variant=ancient]%
ἐν (τῷ) σκότῳ.

\end{greek}%
\switchcolumn*

Abends.

\switchcolumn

\begin{greek}[variant=ancient]%
\emph{τῆς ἑσπέρας.}

\end{greek}%
\switchcolumn*

Gestern Abend.

\switchcolumn

\begin{greek}[variant=ancient]%
\emph{ἑσπέρας.}

\end{greek}%
\switchcolumn*

Heute Abend. (künstig.)

\switchcolumn

\begin{greek}[variant=ancient]%
\emph{εἰς ἑσπέραν.}

\end{greek}%
\switchcolumn*

Abends spät.

\switchcolumn

\begin{greek}[variant=ancient]%
νύκτωρ ὀψέ.

\end{greek}%
\switchcolumn*

Den Tag über.

\switchcolumn

\begin{greek}[variant=ancient]%
δι' ἡμέρας.

\end{greek}%
\switchcolumn*

Die ganze Nacht hindurch.

\switchcolumn

\begin{greek}[variant=ancient]%
ὅλην τὴν νύκτα.

\end{greek}%
\switchcolumn*

Vom frühen Morgen an.

\switchcolumn

\begin{greek}[variant=ancient]%
ἐξ ἑωθινοῦ.

\end{greek}%
\switchcolumn*

Von früh an.

\switchcolumn

\begin{greek}[variant=ancient]%
ἐξ ἕω.

\end{greek}%
\switchcolumn*

Gleich von früh an.

\switchcolumn

\begin{greek}[variant=ancient]%
ἕωθεν εὐθύς.

\end{greek}%
\switchcolumn*

Heute Morgens.

\switchcolumn

\begin{greek}[variant=ancient]%
ἕωθεν.

\end{greek}%
\switchcolumn*

Morgen früh.

\switchcolumn

\begin{greek}[variant=ancient]%
αὔριον ἕωθεν.

\end{greek}%
\switchcolumn*

Heute.

\switchcolumn

\begin{greek}[variant=ancient]%
τῇδε τῇ ἡμέρᾳ. --- τήμερον\footnote{\begin{latin}%
\textgreek[variant=ancient]{ὁ τυπογράφος ἔγραψα τὸν οὐ γεγραμμένον
τόνον.}\end{latin}%
}.

\end{greek}%
\switchcolumn*

Gestern.

\switchcolumn

\begin{greek}[variant=ancient]%
χθές. ἐχθές.

\end{greek}%
\switchcolumn*

Morgen.

\switchcolumn

\begin{greek}[variant=ancient]%
αὔριον.

\end{greek}%
\switchcolumn*

Übermorgen.

\switchcolumn

\begin{greek}[variant=ancient]%
ἕνης. εἰς ἕνηςν

\end{greek}%
\switchcolumn*

Vorgestern.

\switchcolumn

\begin{greek}[variant=ancient]%
τρίτην ἡμέραν. (\textgerman[spelling=old,babelshorthands=true]{auch}
νεωστί).

\end{greek}%
\switchcolumn*[


\section{Jetzt\textcompwordmark{}zeit. Feste}

]\indent In der jetzigen Zeit.

\switchcolumn

\begin{greek}[variant=ancient]%
ἐν τῷ νῦν χρόνῳ.

\end{greek}%
\switchcolumn*

Gerade wie früher.

\switchcolumn

\begin{greek}[variant=ancient]%
ὥσπερ καὶ πρὸ τοῦ.

\end{greek}%
\switchcolumn*

Auf welchen Tag?

\switchcolumn

\begin{greek}[variant=ancient]%
ἐς\footnote{\begin{latin}%
\textgreek[variant=ancient]{ὁ τυπογράφος θαυμάζω τοῦ ἑνεκα γέγραφε
οὐ «εἰς» ὃν Ἀττικοὶ ἔχραον, ἀλλὰ «ἐς» ὃν Ἰωνικοί.}\end{latin}%
} τίνα ἡμέραν.

\end{greek}%
\switchcolumn*

Für sogleich.

\switchcolumn

\begin{greek}[variant=ancient]%
ἐς αὐτίκα μάλα.

\end{greek}%
\switchcolumn*

Vor Kurzem.

\switchcolumn

\begin{greek}[variant=ancient]%
τὸ ἔναγχος.

\end{greek}%
\switchcolumn*

Lange genug.

\switchcolumn

\begin{greek}[variant=ancient]%
ἰκανὸν χρόνον.

\end{greek}%
\switchcolumn*

Heute über 14 Tage.

\switchcolumn

\begin{greek}[variant=ancient]%
μεθ' ἡμέρας \emph{μεντεκαὶδεκα} ἀπὸ τῆς τήμερον.

\end{greek}%
\switchcolumn*

Heuer.

\switchcolumn

\begin{greek}[variant=ancient]%
τῆτες.

\end{greek}%
\switchcolumn*

Vor'm Jahr.

\switchcolumn

\begin{greek}[variant=ancient]%
πέρυσιν.

\end{greek}%
\switchcolumn*

Über's Jahr.

\switchcolumn

\begin{greek}[variant=ancient]%
εἰς νέωτα.

\end{greek}%
\switchcolumn*

Alle vier Jahre.

\switchcolumn

\begin{greek}[variant=ancient]%
δι' ἔτους \emph{πέμπτου.}

\end{greek}%
\switchcolumn*

Monatlich.

\switchcolumn

\begin{greek}[variant=ancient]%
κατὰ μῆνα.

\end{greek}%
\switchcolumn*

Der Frühling. Der Sommer.

\switchcolumn

\begin{greek}[variant=ancient]%
τὸ ἔαρ. τὸ θέρος.

\end{greek}%
\switchcolumn*

Der Herbst. Der Winter.

\switchcolumn

\begin{greek}[variant=ancient]%
τὸ φθινόπωρον. ὁ χειμών.

\end{greek}%
\switchcolumn*

Zur Winters\textcompwordmark{}zeit.

\switchcolumn

\begin{greek}[variant=ancient]%
χειμῶνος ὄντος.

\end{greek}%
\switchcolumn*

Das Fest.

\switchcolumn

\begin{greek}[variant=ancient]%
ἡ ἑορτή.

\end{greek}%
\switchcolumn*

Weihnachten.

\switchcolumn

\begin{greek}[variant=ancient]%
τὰ Χριστούγεννα.{*}

\end{greek}%
\switchcolumn*

Neujahr.

\switchcolumn

\begin{greek}[variant=ancient]%
ἡ πρώτη τοῦ ἔτους.

\end{greek}%
\switchcolumn*

Fastnacht.

\switchcolumn

\begin{greek}[variant=ancient]%
αἱ ἀπόκρεω.{*}

\end{greek}%
\switchcolumn*

Charfreitag.

\switchcolumn

\begin{greek}[variant=ancient]%
ἡ μεγάλη παρασκευή.{*}

\end{greek}%
\switchcolumn*

Ostern.

\switchcolumn

\begin{greek}[variant=ancient]%
τὸ πάσχα.{*}

\end{greek}%
\switchcolumn*

Pfingsten.

\switchcolumn

\begin{greek}[variant=ancient]%
ἡ πεντηκοστή.

\end{greek}%
\switchcolumn*

Geburts\textcompwordmark{}tag.

\switchcolumn

\begin{greek}[variant=ancient]%
τὸ γενέθλια.

\end{greek}%
\switchcolumn*

Jahres\textcompwordmark{}tag (Stiftungs\textcompwordmark{}fest)..

\switchcolumn

\begin{greek}[variant=ancient]%
ἡ ἐπέτειος ἑορτή.

\end{greek}%
\switchcolumn*[\centering\rule{1.5in}{1pt}]

Die Monate:

\switchcolumn

\begin{greek}[variant=ancient]%
οἱ μῆνες: Ἰανουάριος. Φεβρουάριος. Μάρτιος. Ἀπρίλιος. Μάϊος. Ἰούνιος.
Ἰούλιος. Αὔγουστος. Σεπτέμβριος. Ὀκτώβριος. Νοεμβριος. Δεκέμβριος.

\end{greek}%
\switchcolumn*[


\section{Das Wetter}

]\indent Was haben wir für Wetter?

\switchcolumn

\begin{greek}[variant=ancient]%
ποῖος ὁ \emph{ἀὴρ} τό νῦν;

\end{greek}%
\switchcolumn*

Das Wetter ist schön.

\switchcolumn

\begin{greek}[variant=ancient]%
εὐδία ἐστίν.

\end{greek}%
\switchcolumn*

Es ist herrliches Wetter.

\switchcolumn

\begin{greek}[variant=ancient]%
εὐδία ἐστὶν ἡδίστη.

\end{greek}%
\switchcolumn*

Die Sonne scheint.

\switchcolumn

\begin{greek}[variant=ancient]%
ἐξέχει εἵλη ἔχομεν ἥλιον. φαίνεται ὁ ἥλιος. ἥλιος λάμπει.

\end{greek}%
\switchcolumn*

Es ist warm.

\switchcolumn

\begin{greek}[variant=ancient]%
θάλμος ἐστίν.

\end{greek}%
\switchcolumn*

Es ist windig. (Der Wind geht.)

\switchcolumn

\begin{greek}[variant=ancient]%
ἄνεμος γίγνεται.

\end{greek}%
\switchcolumn*

Es weht ein starker Wind.

\switchcolumn

\begin{greek}[variant=ancient]%
ἄνεμος πνεῖ \emph{μέγας.}

\end{greek}%
\switchcolumn*

Wir haben Nord-, Süd-, Ost-, Westwind.

\switchcolumn

\begin{greek}[variant=ancient]%
ἄνεμος γίγνεται βόρειος, νότιος\footnote{\begin{latin}%
\textgreek[variant=ancient]{ὁ τυπογράφος ἔγραψα τὸν οὐ γεγραμμένον
τόνον.}\end{latin}%
}, ἀνατολικός, δυτικός.!

\end{greek}%
\switchcolumn*

Es umwölkt sich.

\switchcolumn

\begin{greek}[variant=ancient]%
ξυννεφεῖ.

\end{greek}%
\switchcolumn*

Es sprüht.

\switchcolumn

\begin{greek}[variant=ancient]%
ψακάζει.

\end{greek}%
\switchcolumn*

Es regnet.

\switchcolumn

\begin{greek}[variant=ancient]%
ὕει.

\end{greek}%
\switchcolumn*

Es gießt sehr.

\switchcolumn

\begin{greek}[variant=ancient]%
ὄμβρος πολὺς γίγνεται.

\end{greek}%
\switchcolumn*

Es donnert.

\switchcolumn

\begin{greek}[variant=ancient]%
βροντᾷ.

\end{greek}%
\switchcolumn*

Wir haben ein Gewitter.

\switchcolumn

\begin{greek}[variant=ancient]%
βρονταὶ γίγνονται καὶ κεραυνοί.

\end{greek}%
\switchcolumn*

Es blitzt stark.

\switchcolumn

\begin{greek}[variant=ancient]%
ἀστράπτει πολὺ νὴ Δία.

\end{greek}%
\switchcolumn*

Es hat eingeschlagen.

\switchcolumn

\begin{greek}[variant=ancient]%
ἔπεσε σκηπτός. ἔπεσε κεραυνός.

\end{greek}%
\switchcolumn*

Es ist kalt. (sehr kalt.)

\switchcolumn

\begin{greek}[variant=ancient]%
ψῦχός ἐστιν. (ψ. ἐστι μέγεστον.)

\end{greek}%
\switchcolumn*

Es schneit! hu!

\switchcolumn

\begin{greek}[variant=ancient]%
νίφει· βαβαιάξ!

\end{greek}%
\switchcolumn*

Es schneit sehr.

\switchcolumn

\begin{greek}[variant=ancient]%
χιών γίγνεται πολλή.

\end{greek}%
\switchcolumn*

Es friet.

\switchcolumn

\begin{greek}[variant=ancient]%
χρύος γίγνεται.

\end{greek}%
\switchcolumn*

Warum machst du den (Sonnen-)Schirm zu?.

\switchcolumn

\begin{greek}[variant=ancient]%
τί πάλιν ξυνάγεις τὸ σκιάδειον;

\end{greek}%
\switchcolumn*

Mach' ihn wieder auf!

\switchcolumn

\begin{greek}[variant=ancient]%
ἐκπέτασον αὐτό!

\end{greek}%
\switchcolumn*

Her mit dem Schirm!

\switchcolumn

\begin{greek}[variant=ancient]%
φέρε τὸ σκιάδειον!

\end{greek}%
\switchcolumn*

Halte den Schirm über mich!

\switchcolumn

\begin{greek}[variant=ancient]%
ὑπέρεχέ μου τὸ σκιάδειον.

\end{greek}%
\switchcolumn*

Nimm dich hier vor dem Schmutze in Acht!

\switchcolumn

\begin{greek}[variant=ancient]%
τὸν πηλὸν τουτονὶ φύλαξαι!

\end{greek}%
\switchcolumn*[


\section{Abreise}

]\indent Wann reisen Sie nach Berlin?

\switchcolumn

\begin{greek}[variant=ancient]%
πότε \emph{ἄπει} εἰς Βερόλινον{*} (Λόνδινον, Βιέννην{*} \textgerman[spelling=old,babelshorthands=true]{Wien,}
Γαστάϊν{*} , Παρισίους, Πετρούπολιν{*}, εἰς Ἑλβητίαν, Κίσσιγγεν{*},
Δρέσδην{*}, Βρυξέλας{*}, Μόναχον \textgerman[spelling=old,babelshorthands=true]{München});

\end{greek}%
\switchcolumn*

Um 12. November.

\switchcolumn

\begin{greek}[variant=ancient]%
τῇ δωδεκάτῃ Νοεμβρίου.

\end{greek}%
\switchcolumn*

Nach Leipzig sind Sie bis\textcompwordmark{}her noch nicht gekommen.

\switchcolumn

\begin{greek}[variant=ancient]%
εἰς Λειψίαν{*} οὔπω ἐλήλυθας.

\end{greek}%
\switchcolumn*

In den Ferien hätte ich Lust auf's Land zu gehen.

\switchcolumn

\begin{greek}[variant=ancient]%
ἐν τῷ ἀναπαύλης χρόνῳ ἐπιθυμῶ ἐλθεῖν εἰς ἀγρόν.

\end{greek}%
\switchcolumn*

Mit welcher Gelegenheit wollen Sie reisen?

\switchcolumn

\begin{greek}[variant=ancient]%
τίς σοι γενήσεται πόρος τῆς ὁδοῦ;

\end{greek}%
\switchcolumn*

Um vier Uhr mit dem Bahnzuge.

\switchcolumn

\begin{greek}[variant=ancient]%
τῇ τετάρτῃ ὥρᾳ χρώμενος τῇ ἁμαξοστοιχίᾳ\footnote{\begin{latin}%
\textgreek[variant=ancient]{τῷ τυπογράφῳ ἄσκοπος τὸ γράμμα «ἁ» ἦν.}\end{latin}%
}.{*}

\end{greek}%
\switchcolumn*

O, dann ist es Zeit zu gehen.

\switchcolumn

\begin{greek}[variant=ancient]%
ὥρα βαδίζειν ἄρ' ἐστίν.

\end{greek}%
\switchcolumn*

Es ist Zeit auf den Bahnhof zu gehen.

\switchcolumn

\begin{greek}[variant=ancient]%
ὥρα ἐστὶν εἰς τὸν (σιδηροδρομικὸν{*}) σταθμὸν βαδίζειν.

\end{greek}%
\switchcolumn*

Es wäre längst Zeit gewesen!

\switchcolumn

\begin{greek}[variant=ancient]%
ὥρα\footnote{\begin{latin}%
\textgreek[variant=ancient]{ὁ τυπογράφος ἔγραψα τὸν οὐ γεγραμμένον
τόνον.}\end{latin}%
} ἦν πάλαι.

\end{greek}%
\switchcolumn*

Nun, so reisen Sie glücklich!

\switchcolumn

\begin{greek}[variant=ancient]%
ἀλλ' ἴθι χαίρων!

\end{greek}%
\switchcolumn*

Adieu!

\switchcolumn

\begin{greek}[variant=ancient]%
χαῖρε καὶ σύ!

\end{greek}%
\switchcolumn*

Er ist abgereist.

\switchcolumn

\begin{greek}[variant=ancient]%
οἴχεται.

\end{greek}%
\switchcolumn*

Mein Bruder ist seit 5 Monaten fort.

\switchcolumn

\begin{greek}[variant=ancient]%
ὁ ἐμὸς ἀδελφὸς πέντε μῆνας ἄπεστιν.

\end{greek}%
\switchcolumn*

Er ist auf der Reise.

\switchcolumn

\begin{greek}[variant=ancient]%
ἀποδημῶν ἐστιν.

\end{greek}%
\switchcolumn*[


\section{Gehen. Weg.}

]\indent Kommen Sie mit!

\switchcolumn

\begin{greek}[variant=ancient]%
ἕπου!

\end{greek}%
\switchcolumn*

Kommen Sie mit mir!

\switchcolumn

\begin{greek}[variant=ancient]%
ἕπου μετ' ἐμοῦ!

\end{greek}%
\switchcolumn*

Der Bahnhof ist nicht weit.

\switchcolumn

\begin{greek}[variant=ancient]%
ἔστ' οὐ μεκρὰν ἄποθεν ὁ σταθμός.

\end{greek}%
\switchcolumn*

Nun, so wollen wir gehen.

\switchcolumn

\begin{greek}[variant=ancient]%
ἄγε νυν ἴωμεν.

\end{greek}%
\switchcolumn*

Wir wollen fortgehen

\switchcolumn

\begin{greek}[variant=ancient]%
ἀπίωμεν.

\end{greek}%
\switchcolumn*

Wir wollen weitergehen.

\switchcolumn

\begin{greek}[variant=ancient]%
χωρῶμεν.

\end{greek}%
\switchcolumn*

Vorwärts!

\switchcolumn

\begin{greek}[variant=ancient]%
χώρει!

\end{greek}%
\switchcolumn*

Wir wollen Euch voraus\textcompwordmark{}gehen.

\switchcolumn

\begin{greek}[variant=ancient]%
προίωμεν ὑμῶν.

\end{greek}%
\switchcolumn*

Ich werde eine Droschke nehmen.

\switchcolumn

\begin{greek}[variant=ancient]%
ἁμάξῃ χρήσομαι.

\end{greek}%
\switchcolumn*

Ich werde vielmehr den Omnibus benutzen.

\switchcolumn

\begin{greek}[variant=ancient]%
ἐγὼ μὲν οὖν χρήσομαι τῷ λεωφορείῳ{*}.

\end{greek}%
\switchcolumn*

Ich meinerseits gehe zu Fuße.

\switchcolumn

\begin{greek}[variant=ancient]%
βαδίζω ἔγωγε.

\end{greek}%
\switchcolumn*

Du reitest.

\switchcolumn

\begin{greek}[variant=ancient]%
ὀχεῖ!

\end{greek}%
\switchcolumn*

Sagen Sie, auf welchem Wege kommen wir am schnellsten nach dem Bahnhofe?

\switchcolumn

\begin{greek}[variant=ancient]%
φράζε, ὅπῃ τάχιστα ἀφιξόμεθα εἰς τὸν σταθμόν;

\end{greek}%
\switchcolumn*

Wir können den Weg nicht finden.

\switchcolumn

\begin{greek}[variant=ancient]%
οὐ δυνάμεθα ἐξευρεῖν τὴν ὁδόν.

\end{greek}%
\switchcolumn*

Ich weiß nicht mehr, wo wir sind.

\switchcolumn

\begin{greek}[variant=ancient]%
οὐκέτι οἶδα, ποῖ γῆς ἐσμεν..

\end{greek}%
\switchcolumn*

Sie haben den Weg verfehlt.

\switchcolumn

\begin{greek}[variant=ancient]%
τῆς ὁδοῦ ἡμάρτηκας.

\end{greek}%
\switchcolumn*

Ach, du mein Gott!

\switchcolumn

\begin{greek}[variant=ancient]%
ὦ φίλιο θεοί!

\end{greek}%
\switchcolumn*

Gehen Sie die Straße hier, so werden Sie sogleich auf den Marktplatz
kommen.

\switchcolumn

\begin{greek}[variant=ancient]%
ἴθι τὴν ὁδὸν ταυτηνί καὶ τὐθὺς ἐπὶ τὴν ἀγορὰν ἥξεις.

\end{greek}%
\switchcolumn*

Und was dann?

\switchcolumn

\begin{greek}[variant=ancient]%
εἶτα τί;

\end{greek}%
\switchcolumn*

Dann müssen Sie rechts (links) gehen.

\switchcolumn

\begin{greek}[variant=ancient]%
εἶτα βαδιστέα σοι ἐπὶ δεξιά (ἐπ' ἀριστερά).

\end{greek}%
\switchcolumn*

Gerade aus!

\switchcolumn

\begin{greek}[variant=ancient]%
ὀρθήν!

\end{greek}%
\switchcolumn*

Wie weit ist es etwa?

\switchcolumn

\begin{greek}[variant=ancient]%
πόση τις ἡ ὁδός;

\end{greek}%
\switchcolumn*

Danke.

\switchcolumn

\begin{greek}[variant=ancient]%
καλῶς.

\end{greek}%
\switchcolumn*

Nun, da wollen wir uns beeilen.

\switchcolumn

\begin{greek}[variant=ancient]%
ἀλλὰ σπεύδωμεν.

\end{greek}%
\switchcolumn*

Gehen Sie zu!

\switchcolumn

\begin{greek}[variant=ancient]%
χώρει!

\end{greek}%
\switchcolumn*

Wir sind erst nach dem zweiten Läuten gekommen.

\switchcolumn

\begin{greek}[variant=ancient]%
ὕστερον ἤλθομεν τοῦ δευτέρου σημείου.

\end{greek}%
\switchcolumn*[


\section{Warte!}

]\indent Du, halt einmal!

\switchcolumn

\begin{greek}[variant=ancient]%
ἐπίσχες, οὗτος!

\end{greek}%
\switchcolumn*

Warte einmal!

\switchcolumn

\begin{greek}[variant=ancient]%
ἔχε νυν ἥσυχος!

\end{greek}%
\switchcolumn*

Halt! Bleib' stehen!

\switchcolumn

\begin{greek}[variant=ancient]%
μέν' ἥσυχος! στῆθι!

\end{greek}%
\switchcolumn*

Nicht von der Stelle!

\switchcolumn

\begin{greek}[variant=ancient]%
ἔχ' ἀστέμας αὐτοῦ!

\end{greek}%
\switchcolumn*

So warte doch!

\switchcolumn

\begin{greek}[variant=ancient]%
\emph{οὐ μενεῖς;}

\end{greek}%
\switchcolumn*

Warte eine Weile auf mich!

\switchcolumn

\begin{greek}[variant=ancient]%
ἐπανάμεινον μ' ὀλίγον χρόνον.

\end{greek}%
\switchcolumn*

Ich werde gleich wiederkommen.

\switchcolumn

\begin{greek}[variant=ancient]%
ἀλλ' ἥξω ταχέως.

\end{greek}%
\switchcolumn*

Wo soll ich dich erwarten?

\switchcolumn

\begin{greek}[variant=ancient]%
ποῦ ἀναμεῶ;

\end{greek}%
\switchcolumn*

Komm' nur schnell wieder!

\switchcolumn

\begin{greek}[variant=ancient]%
ἧκέ νυν ταχύ!

\end{greek}%
\switchcolumn*

Da bin ich wieder.

\switchcolumn

\begin{greek}[variant=ancient]%
ἰδού, πάρειμι.

\end{greek}%
\switchcolumn*

Bist du wieder da?

\switchcolumn

\begin{greek}[variant=ancient]%
ἥκεις;

\end{greek}%
\switchcolumn*

Ich bin dir doch nicht zu lange gewesen?

\switchcolumn

\begin{greek}[variant=ancient]%
μῶν ἐπισχεῖν σοι δοκῶ;

\end{greek}%
\switchcolumn*

Wo bist du nur so lange geblieben?

\switchcolumn

\begin{greek}[variant=ancient]%
ποῦ ποτ' ἦσθα ἀπ' ἐμοῦ (ἀφ' ἡμῶν) τὸν πολὺν τοῦτον χρόνον;

\end{greek}%
\switchcolumn*[


\section{Komm her!}

]\indent Komm her!

\switchcolumn

\begin{greek}[variant=ancient]%
δεῦρ' ἐλθέ!

\end{greek}%
\switchcolumn*

Komm hierher!

\switchcolumn

\begin{greek}[variant=ancient]%
ἐλθὲ δεῦρο!

\end{greek}%
\switchcolumn*

Geh' her!

\switchcolumn

\begin{greek}[variant=ancient]%
χώρει δεῦρο!

\end{greek}%
\switchcolumn*

Geh' hierher, zu mir!

\switchcolumn

\begin{greek}[variant=ancient]%
βάδιζε δεῦρο, ὡς ἐμέ!

\end{greek}%
\switchcolumn*

Du kommst wie gerusen.

\switchcolumn

\begin{greek}[variant=ancient]%
ἥκεις ὥσπερ κατὰ θεῖον.

\end{greek}%
\switchcolumn*

Woher kommst du?

\switchcolumn

\begin{greek}[variant=ancient]%
πόθεν βαδίζεις;

\end{greek}%
\switchcolumn*

Aber wo kommst du eigentlich her?

\switchcolumn

\begin{greek}[variant=ancient]%
ἀτὰρ πόθεν ἥκεις ἐτεόν;

\end{greek}%
\switchcolumn*

Ich komme von Müllers.

\switchcolumn

\begin{greek}[variant=ancient]%
\emph{ἐκ Μυλλέρου} ἔρχομαι.

\end{greek}%
\switchcolumn*

Geh' mit mir hinein!

\switchcolumn

\begin{greek}[variant=ancient]%
εἴσιθι αμ'\footnote{ὁ τυπογράφος ἔγραψα τὸν οὐ γεγραμμένον τόνον.}
ἐμοί.

\end{greek}%
\switchcolumn*

Ich bitte dich, noch bei uns zu bleiben.

\switchcolumn

\begin{greek}[variant=ancient]%
δέομαί σου περαμεῖναι ἡμῖν.

\end{greek}%
\switchcolumn*

Das geht nicht!

\switchcolumn

\begin{greek}[variant=ancient]%
ἀλλ' οὐχ οἷόν τε!

\end{greek}%
\switchcolumn*

Wohin gehst du?

\switchcolumn

\begin{greek}[variant=ancient]%
ποῖ βαδίζεις;

\end{greek}%
\switchcolumn*

So bleib' doch da!

\switchcolumn

\begin{greek}[variant=ancient]%
οὐ παραμενεῖς;

\end{greek}%
\switchcolumn*

Wir lassen dich nicht fort.

\switchcolumn

\begin{greek}[variant=ancient]%
οὔ σ' ἀφήσομεν.

\end{greek}%
\switchcolumn*

Ich will zum Friseur.

\switchcolumn

\begin{greek}[variant=ancient]%
\emph{βούλομαι εἰς} τὸ κουρεῖον.

\end{greek}%
\switchcolumn*

Wir lassen dich durchaus nicht fort.

\switchcolumn

\begin{greek}[variant=ancient]%
οὐκ ἀφήσομέν σε μά δία οὐδέποτε!

\end{greek}%
\switchcolumn*

Laßt mich los!

\switchcolumn

\begin{greek}[variant=ancient]%
μέθεσθέ μου!

\end{greek}%
\switchcolumn*

Kommt schnell zu mir her!

\switchcolumn

\begin{greek}[variant=ancient]%
ἴτε δεῦρ' ὡς ἐμὲ ταχέως.

\end{greek}%
\switchcolumn*

Heute Abend will ich kommen.

\switchcolumn

\begin{greek}[variant=ancient]%
εἰς ἑσπέραν ἥξω.

\end{greek}%
\switchcolumn*

Weg ist er!

\switchcolumn

\begin{greek}[variant=ancient]%
φροῦδός ἐστιν!

\end{greek}%
\switchcolumn*

Wo ist er denn hin?

\switchcolumn

\begin{greek}[variant=ancient]%
ποῖ γὰρ οἴχεται;

\end{greek}%
\switchcolumn*

Er ist fort zum Friseur.

\switchcolumn

\begin{greek}[variant=ancient]%
εἰς τὸ κουρεῖον οἴχεται.

\end{greek}%
\switchcolumn*

Er geht heim.

\switchcolumn

\begin{greek}[variant=ancient]%
οἴκαδ' ἔρχεται.

\end{greek}%
\switchcolumn*

Wir wollen wieder heimgehen.

\switchcolumn

\begin{greek}[variant=ancient]%
ἀπίωμεν οἴκαδ' αὖθις.

\end{greek}%
\switchcolumn*

Er will ihnen entgegen gehen.

\switchcolumn

\begin{greek}[variant=ancient]%
ἀπαντῆσαι αὐτοῖς βούλεται.

\end{greek}%
\switchcolumn*

Er ist ihr begegnet.

\switchcolumn

\begin{greek}[variant=ancient]%
ξυνήντησεν αὐτῇ.

\end{greek}%
\switchcolumn*

Wo wollen wir uns treffen?

\switchcolumn

\begin{greek}[variant=ancient]%
ποῖ ἀπαντησόμεθα;

\end{greek}%
\switchcolumn*

Hier.

\switchcolumn

\begin{greek}[variant=ancient]%
ἐνθάδε.

\end{greek}%
\switchcolumn*[


\section{Bier her!}

]\indent Kellner! Kellner!

\switchcolumn

\begin{greek}[variant=ancient]%
παῖ! παῖ!

\end{greek}%
\switchcolumn*

Wo steckt denn die Bedienung?

\switchcolumn

\begin{greek}[variant=ancient]%
οὐ περιδραμεῖταί τις δεῦρο τῶν παίδων;

\end{greek}%
\switchcolumn*

Sie da, Kellner, wohin laufen Sie? --- Nach Gläsern.

\switchcolumn

\begin{greek}[variant=ancient]%
οὗτος σὺ, παῖ, ποῖ θεῖς; --- Ἐπ' ἐκπώματα.

\end{greek}%
\switchcolumn*

Kommen Sie hierher!

\switchcolumn

\begin{greek}[variant=ancient]%
ἐλθὲ δεῦρο!

\end{greek}%
\switchcolumn*

Bringen Sie mir einmal schnell Bier und Hasenbraten!

\switchcolumn

\begin{greek}[variant=ancient]%
ἔνεγκέ μοι ταχέως ζῦθον καὶ λαγῷα.

\end{greek}%
\switchcolumn*

Ganz wohl, mein Herr!

\switchcolumn

\begin{greek}[variant=ancient]%
\emph{ταῦτα,} ὦ δέσποτα.

\end{greek}%
\switchcolumn*

So, da bringe ich Alles.

\switchcolumn

\begin{greek}[variant=ancient]%
ἰδού, ἅπαντ' ἐγὼ φέρω.

\end{greek}%
\switchcolumn*

Das Bier schmeckt gut!

\switchcolumn

\begin{greek}[variant=ancient]%
ὡς ἡδὺς ὁ ζῦθος!

\end{greek}%
\switchcolumn*

Es schmeckt mir nicht.

\switchcolumn

\begin{greek}[variant=ancient]%
οὐκ ἀρέσκει με.

\end{greek}%
\switchcolumn*

Das Bier \emph{schmeckt} sehr stark nach Pech.

\switchcolumn

\begin{greek}[variant=ancient]%
\emph{ὄζει} πίττης ὁ ζῦθος ὀξύτατον.!

\end{greek}%
\switchcolumn*

Bier her, Kellner! --- Schleunigst!

\switchcolumn

\begin{greek}[variant=ancient]%
πέρε σὺ ζῦθον ὁ παῖς! --- πάσῃ τέχνῃ!

\end{greek}%
\switchcolumn*

So beeilen Sie sich doch!

\switchcolumn

\begin{greek}[variant=ancient]%
οὐ θᾶττον ἐγκονήσεις;

\end{greek}%
\switchcolumn*

Sie sorgen schlecht für uns.

\switchcolumn

\begin{greek}[variant=ancient]%
κακῶς ἐπιμελεῖ ἡμῶν!

\end{greek}%
\switchcolumn*

Kellner, schenken Sie mir noch einmal ein!

\switchcolumn

\begin{greek}[variant=ancient]%
παῖ, ἕτερον ἔγχεον!

\end{greek}%
\switchcolumn*

Schenken Sie mir auch ein!

\switchcolumn

\begin{greek}[variant=ancient]%
ἔγχει κἀμοί!

\end{greek}%
\switchcolumn*

Heute Abend wollen wir nach langer Zeit wieder einmal gehörig zechen.

\switchcolumn

\begin{greek}[variant=ancient]%
εἰς ἑσπέραν μεθυσθῶμεν διὰ χρόνου.

\end{greek}%
\switchcolumn*

Das Kneipen taugt nichts.

\switchcolumn

\begin{greek}[variant=ancient]%
κακὸν τὸ πίνειν!

\end{greek}%
\switchcolumn*

Man bekommt Katzenjammer von dem Bier.

\switchcolumn

\begin{greek}[variant=ancient]%
κραιπάλη γίγνεται ἀπὸ τοῖ ζύθου.

\end{greek}%
\switchcolumn*

Ich will Bier \emph{holen}.

\switchcolumn

\begin{greek}[variant=ancient]%
ἐπὶ ζῦθον εἶμι.

\end{greek}%
\switchcolumn*

Ich werde Sie nöthigenfalls rufen.

\switchcolumn

\begin{greek}[variant=ancient]%
καλέσω σε, εἴ τι δέοι.

\end{greek}%
\switchcolumn*

Ich gehe und hole mir noch eins.

\switchcolumn

\begin{greek}[variant=ancient]%
ἕτερον ἰὼν κομιοῦμαι.

\end{greek}%
\switchcolumn*

Hier haben Sie es!

\switchcolumn

\begin{greek}[variant=ancient]%
ἰδού, τουτὶ λαβέ.

\end{greek}%
\switchcolumn*

Schön. Sie sollen ein Trinkgeld von mir bekommen.

\switchcolumn

\begin{greek}[variant=ancient]%
καλῶς. εὐεργετήσω σε.

\end{greek}%
\switchcolumn*

Ich bin nicht im Stande hier zu bleiben.

\switchcolumn

\begin{greek}[variant=ancient]%
οὐχ οἷός τ' εἰμὶ ἐυθάδε μένειν.

\end{greek}%
\switchcolumn*

Der Rauch beißt mich in die Augen.

\switchcolumn

\begin{greek}[variant=ancient]%
ὁ καπνὸς δάκνει τὰ βλέφαρά μου.

\end{greek}%
\switchcolumn*

Komm', geh' mit!

\switchcolumn

\begin{greek}[variant=ancient]%
ἕπου μετ' ἐμοῦ.

\end{greek}%
\switchcolumn*

Der Rauch vertreibt mich.

\switchcolumn

\begin{greek}[variant=ancient]%
ὁ καπνός μ' ἐκπέμπει.

\end{greek}%
\switchcolumn*

Kellner, rechnen Sie einmal die Zeche zusammen!

\switchcolumn

\begin{greek}[variant=ancient]%
παῖ, λόγισαι ταῦτα.

\end{greek}%
\switchcolumn*

Sie hatten 6 Bier, Hasenbraten, Brot, macht 2½ Mark.

\switchcolumn

\begin{greek}[variant=ancient]%
εἴχετε ζύθου ἕξ (ποτήρια) καὶ λαγῷα καὶ ἄρτον· γίγνονται οὖν ἡμῖν
δύο μάρκαι{*} καὶ ἡμίσεια.

\end{greek}%
\switchcolumn*

Hier haben Sie!

\switchcolumn

\begin{greek}[variant=ancient]%
ἰδού, λαβέ.

\end{greek}%
\switchcolumn*

Ich taumele beim Gehen.

\switchcolumn

\begin{greek}[variant=ancient]%
σφαλλόμενος ἔρχομαι.

\end{greek}%
\switchcolumn*[


\section{Mich hungert}

]\indent Ich bekomme Hunger.

\switchcolumn

\begin{greek}[variant=ancient]%
λιμός με λαμβάνει.

\end{greek}%
\switchcolumn*

Ich habe nichts zu essen.

\switchcolumn

\begin{greek}[variant=ancient]%
οὐκ ἔχω καταφαγεῖν.

\end{greek}%
\switchcolumn*

Er hat einen Bärenhunger.

\switchcolumn

\begin{greek}[variant=ancient]%
βουλιμιᾷ.

\end{greek}%
\switchcolumn*

Ich komme vor Hunger um..

\switchcolumn

\begin{greek}[variant=ancient]%
ἀπόλωλα ὑπὸ λιμοῦ.

\end{greek}%
\switchcolumn*

Soll ich Ihnen etwas zu essen (zu trinken) geben?

\switchcolumn

\begin{greek}[variant=ancient]%
φέρε τί σοι δῶ \emph{φαγεῖν;} (πιεῖν;)

\end{greek}%
\switchcolumn*

Geben Sie mir etwas zu essen!

\switchcolumn

\begin{greek}[variant=ancient]%
δός μοι φαγεῖν!

\end{greek}%
\switchcolumn*

Ich will zu Tische gehen.

\switchcolumn

\begin{greek}[variant=ancient]%
βαδιοῦμαι ἐπὶ δεῖπνον.

\end{greek}%
\switchcolumn*

Sie haben noch nicht zu Mittag gegessen?

\switchcolumn

\begin{greek}[variant=ancient]%
οὔπω δεδείπνηκας;

\end{greek}%
\switchcolumn*

Nein!

\switchcolumn

\begin{greek}[variant=ancient]%
μὰ Δί' ἐγὼ μὲν οὔ.

\end{greek}%
\switchcolumn*

Ich muß fort zu Tische.

\switchcolumn

\begin{greek}[variant=ancient]%
δεῖ με χωρεῖν ἐπὶ δεῖπνον.

\end{greek}%
\switchcolumn*

Nun, so gehen Sie schnell zum Essen!

\switchcolumn

\begin{greek}[variant=ancient]%
ἀλλ' ἐπὶ δεῖπνον ταχὺ βάδιζε!

\end{greek}%
\switchcolumn*

Er kommt zu Tische.

\switchcolumn

\begin{greek}[variant=ancient]%
ἐπὶ δεῖπνον ἔρχεται.

\end{greek}%
\switchcolumn*

Der Tisch ist gedeckt.

\switchcolumn

\begin{greek}[variant=ancient]%
τὸ δεῖπνόν ἐστ' ἐπεσκευασμένον.!

\end{greek}%
\switchcolumn*

Die Tasse.

\switchcolumn

\begin{greek}[variant=ancient]%
τὸ κύπελλον.

\end{greek}%
\switchcolumn*

Der Teller.

\switchcolumn

\begin{greek}[variant=ancient]%
τὸ λεκάνιον.

\end{greek}%
\switchcolumn*

Die Schüssel.

\switchcolumn

\begin{greek}[variant=ancient]%
τὸ τρυβλίον.

\end{greek}%
\switchcolumn*

Das Messer.

\switchcolumn

\begin{greek}[variant=ancient]%
τὸ μαχαίριον.

\end{greek}%
\switchcolumn*

Die Gabel.

\switchcolumn

\begin{greek}[variant=ancient]%
τὸ πειρούνιον.{*}

\end{greek}%
\switchcolumn*

Die Serviette.

\switchcolumn

\begin{greek}[variant=ancient]%
τὸ χειρόμακτρον.

\end{greek}%
\switchcolumn*[


\section{Mahlzeit}

]\indent Ich lade dich zum Frühstück ein.

\switchcolumn

\begin{greek}[variant=ancient]%
ἐπ' ἄριστον καλῶ σε.

\end{greek}%
\switchcolumn*

Er hat mich zum Frühstück geladen.

\switchcolumn

\begin{greek}[variant=ancient]%
ἐπ' ἄριστον μ' ἐκάλεσεν.!

\end{greek}%
\switchcolumn*

Wir werden gut essen und trinken.

\switchcolumn

\begin{greek}[variant=ancient]%
εὐωχησόμεθα ἡμεῖς γε.

\end{greek}%
\switchcolumn*

Ich rechnete darauf, daß Sie kommen mürden.

\switchcolumn

\begin{greek}[variant=ancient]%
ἐλογιζόμην\footnote{\begin{latin}%
\textgreek[variant=ancient]{τῷ τυπογράφῳ ἄσκοπος τὸ γράμμα «ό» ἦν.}\end{latin}%
} ἐγώ σε παρέσεσθαι.

\end{greek}%
\switchcolumn*

Er frühstückt.

\switchcolumn

\begin{greek}[variant=ancient]%
ἀριστᾷ.

\end{greek}%
\switchcolumn*

\myafterpagetrue\mysetaligntext{german}{Es giebt{ }}\mysetalign{german}Braten. 

\switchcolumn

\begin{greek}[variant=ancient]%
\mysetaligntext{greek}{πάρεστι{ }}\mysetalign*{greek}κρέα ὠπτημένα.

\end{greek}%
\switchcolumn*\bgroup\mysetalign{german} Kalbs\textcompwordmark{}braten.

\egroup\switchcolumn\bgroup

\begin{greek}[variant=ancient]%
\mysetalign*{greek}(κρέα) μόσχεια.\mysetaligntext{greek}{πάρεστι
(κρέα){ }}

\end{greek}%
\egroup\switchcolumn*\bgroup

\mysetalign{german}Kinderbraten. 

\egroup\switchcolumn\bgroup

\begin{greek}[variant=ancient]%
\mysetalign*{greek}βόεια.

\end{greek}%
\egroup\switchcolumn*\bgroup

\mysetalign{german}Schweinebraten. 

\egroup\switchcolumn\bgroup

\begin{greek}[variant=ancient]%
\mysetalign*{greek}χοίρεια.

\end{greek}%
\egroup\switchcolumn*\bgroup

\mysetalign{german}Hammelbraten. 

\egroup\switchcolumn\bgroup

\begin{greek}[variant=ancient]%
\mysetalign*{greek}ἄρνεια.

\end{greek}%
\egroup\myafterpagefalse\switchcolumn*\bgroup

\mysetalign{german}Ziegenbraten. 

\egroup\switchcolumn\bgroup

\begin{greek}[variant=ancient]%
\mysetalign*{greek}ἐρίφεια.

\end{greek}%
\egroup\myafterpagefalse\switchcolumn*\bgroup

\mysetalign{german}Keule, Schinken. 

\egroup\switchcolumn\bgroup

\begin{greek}[variant=ancient]%
\mysetalign*{greek}κωλῆ.

\end{greek}%
\egroup\myafterpagefalse\switchcolumn*\bgroup

\mysetalign{german}Hasenbraten. 

\egroup\switchcolumn\bgroup

\begin{greek}[variant=ancient]%
\mysetalign*{greek}λαγῷα.

\end{greek}%
\egroup\myafterpagefalse\switchcolumn*\bgroup

\mysetalign{german}Geflügel. 

\egroup\switchcolumn\bgroup

\begin{greek}[variant=ancient]%
\mysetalign*{greek}ὀρνίθεια.

\end{greek}%
\egroup\myafterpagefalse\switchcolumn*\bgroup

\mysetalign{german}Aal. 

\egroup\switchcolumn\bgroup

\begin{greek}[variant=ancient]%
\mysetalign*{greek}ἐγχέλεια.

\end{greek}%
\egroup\switchcolumn*

Aal habe ich nicht gern; lieber äße ich Geflügel.

\switchcolumn

\begin{greek}[variant=ancient]%
οὐ χαίρω ἐγχέλεσιν, ἀλλ' \emph{ἥδιον}\footnote{\begin{latin}%
\textgreek[variant=ancient]{ὁ τυπογράφος ἔγραψα τὸν οὐ γεγραμμένον
ἦχον καὶ τόνον.}\end{latin}%
} ἂν φάγοιμι ὀρνίθεια.

\end{greek}%
\switchcolumn*

Das esse ich am liebsten.

\switchcolumn

\begin{greek}[variant=ancient]%
ταῦτα γὰρ \emph{ἥδιστ'} ἐσθίω.

\end{greek}%
\switchcolumn*

Das habe ich gestern gegessen.

\switchcolumn

\begin{greek}[variant=ancient]%
τοῦτο χθὲς ἔφαγον.

\end{greek}%
\switchcolumn*

Bringen Sie Krammets\textcompwordmark{}vögel für mich her!

\switchcolumn

\begin{greek}[variant=ancient]%
φέρε δεῦρο κίχλας ἐμοί!

\end{greek}%
\switchcolumn*

Kosten Sie einmal davon!

\switchcolumn

\begin{greek}[variant=ancient]%
γεῦσαι λαβών!

\end{greek}%
\switchcolumn*

Essen Sie einmal dies!

\switchcolumn

\begin{greek}[variant=ancient]%
φάγε τουτί!

\end{greek}%
\switchcolumn*

Nein, das bekommt mir gar nicht gut.

\switchcolumn

\begin{greek}[variant=ancient]%
μὰ τὸν Δία, οὐ γὰρ οὐδαμῶς μοι ξύμφορον.

\end{greek}%
\switchcolumn*

Knus\textcompwordmark{}pern Sie einmal dies!

\switchcolumn

\begin{greek}[variant=ancient]%
ἔντραγε τουτί!

\end{greek}%
\switchcolumn*

Genöthigt wird principiell nicht.

\switchcolumn

\begin{greek}[variant=ancient]%
οὐ προσαναγκάζομεν οὐδαμῶς.

\end{greek}%
\switchcolumn*

Das Fleisch schmeckt sehr gut.

\switchcolumn

\begin{greek}[variant=ancient]%
τὰ κρέα ἥδιστά ἐστιν.

\end{greek}%
\switchcolumn*

Das schmeckt gut.

\switchcolumn

\begin{greek}[variant=ancient]%
ὡς ἡδύ!

\end{greek}%
\switchcolumn*

Die Sause schmeckt sehr gut.

\switchcolumn

\begin{greek}[variant=ancient]%
ὡς ἡδὺ τὸ κατάχυσμα!

\end{greek}%
\switchcolumn*

Eins vermisse ich noch.

\switchcolumn

\begin{greek}[variant=ancient]%
ἕν ἔτι ποθῶ.

\end{greek}%
\switchcolumn*

Geben Sie mir doch ein Stück Brot!

\switchcolumn

\begin{greek}[variant=ancient]%
δός μοι δῆτα ὀλίγον τι ἄρτου!

\end{greek}%
\switchcolumn*

Und ein Stück Wurst\\
und Erbsenbrei.

\switchcolumn

\begin{greek}[variant=ancient]%
καὶ χορδῆς τι\\
καὶ ἔτνος\footnote{\begin{latin}%
\textgreek[variant=ancient]{ὁ τυπογράφος ἔγραψα τὸν οὐ γεγραμμένον
ἦχον καὶ τόνον.}\end{latin}%
} πίσινον.

\end{greek}%
\switchcolumn*

Der Nachtisch.

\switchcolumn

\begin{greek}[variant=ancient]%
τὸ ἐπίδειπνον.

\end{greek}%
\switchcolumn*

Was wollen wir zum Dessert essen?

\switchcolumn

\begin{greek}[variant=ancient]%
τί ἐπιδειπνήσομεν;

\end{greek}%
\switchcolumn*

Bringen Sie noch etwas Weißbrot mit Schweizerkäse!

\switchcolumn

\begin{greek}[variant=ancient]%
παράθες ἔτι ὀλίγον τι ἄρτου πυρίνου μετὰ τυροῦ ἑλβητικοῦ!

\end{greek}%
\switchcolumn*

Es wird Kuchen gebacken.

\switchcolumn

\begin{greek}[variant=ancient]%
πόπανα πέττεται.

\end{greek}%
\switchcolumn*

Da haben Sie auch ein Stück Speckkuchen.

\switchcolumn

\begin{greek}[variant=ancient]%
λαβὲ καὶ πλακοῦντος πίονος τόμον.

\end{greek}%
\switchcolumn*

Ich danke bestens! (Nein!)

\switchcolumn

\begin{greek}[variant=ancient]%
κάλλιστα· ἐπαινῶ.

\end{greek}%
\switchcolumn*

Auch ich habe genug.

\switchcolumn

\begin{greek}[variant=ancient]%
κἀμοί γ' ἅλις.

\end{greek}%
\switchcolumn*

Bringen Sie Wein! (Weiß-, Roth-.)

\switchcolumn

\begin{greek}[variant=ancient]%
φέρ' οἶνον (λευκόν, ἐρυθρόν).

\end{greek}%
\switchcolumn*

Der Wein hat Bouquet.

\switchcolumn

\begin{greek}[variant=ancient]%
ὀσμὴν ἔχει ὁ οἶνος ὁδί.

\end{greek}%
\switchcolumn*

Ich trinke diesen Wein hier gern.

\switchcolumn

\begin{greek}[variant=ancient]%
ἡδέως\footnote{\begin{latin}%
\textgreek[variant=ancient]{τῷ τυπογράφῳ ἄσκοπος τὸ γράμμα «ἡ» ἦν.}\end{latin}%
} πίνω τὸν οἶνον τονδί.

\end{greek}%
\switchcolumn*

Es ist noch Wein übrig geblieben.

\switchcolumn

\begin{greek}[variant=ancient]%
οἶνός ἐστι περιλελειμμένος.

\end{greek}%
\switchcolumn*

Wie viel etwa?

\switchcolumn

\begin{greek}[variant=ancient]%
πόσον τι;

\end{greek}%
\switchcolumn*

Über die Hälfte.

\switchcolumn

\begin{greek}[variant=ancient]%
ὑπὲρ ἥμισυ.

\end{greek}%
\switchcolumn*

Was soll ich damit machen?

\switchcolumn

\begin{greek}[variant=ancient]%
τί χρήσομαι τούτῳ;

\end{greek}%
