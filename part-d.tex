\switchcolumn*[


\part{In Gesellschaft.}


\section{Tanz}

]Sie tanzt gut; \emph{nicht wahr?} 

\switchcolumn

\begin{greek}[variant=ancient]%
καλῶς ὀρχεῖται· ἦ γάρ;

\end{greek}%
\switchcolumn*

Allerdings. 

\switchcolumn

\begin{greek}[variant=ancient]%
μάλιστα.

\end{greek}%
\switchcolumn*

Ich bin ent\textcompwordmark{}zückt.

\switchcolumn

\begin{greek}[variant=ancient]%
κεκήλημαι ἔγωγε.

\end{greek}%
\switchcolumn*

Ich werde Polka mit ihr tanzen (Schottisch, Walzer, Française). 

\switchcolumn

\begin{greek}[variant=ancient]%
ὀρχήσομαι μετ' αὐτῆς τὸ Πολωνικόν (τὸ Καληδονικόν, τὸ Γερμανικόν,
τὸ Γαλλικόν).

\end{greek}%
\switchcolumn*

Erlauben Sie mir diesen Tanz, gnädige Frau? (— Fräulein?) 

\switchcolumn

\begin{greek}[variant=ancient]%
δὸς ὀρχεῖσθαι τοῦτο μετὰ σοῦ, ὦ γύναι! (--- ὦ κόρη!)

\end{greek}%
\switchcolumn*

Recht gern! 

\switchcolumn

\begin{greek}[variant=ancient]%
\emph{φθόνος οὐδείς.}

\end{greek}%
\switchcolumn*

Bitte, hören Sie auf, ich kann nicht mehr. 

\switchcolumn

\begin{greek}[variant=ancient]%
παῦε δῆτ' ὀρχούμενος, !

\end{greek}%
\switchcolumn*

Ich bin müde. 

\switchcolumn

\begin{greek}[variant=ancient]%
κέκμηκα.

\end{greek}%
\switchcolumn*

Nur dies \emph{eine} Mal erlauben Sie mir noch!

\switchcolumn

\begin{greek}[variant=ancient]%
ἓν μὲν οὖν τουτί μ' ἔασον ὀρχήσασθαι.

\end{greek}%
\switchcolumn*

Nun denn noch dies \emph{eine} Mal und nicht weiter! 

\switchcolumn

\begin{greek}[variant=ancient]%
τοῖτό νυν καὶ μηκέτ' ἄλλο μηδέν.

\end{greek}%
\switchcolumn*

Das ist eine Lust, mit Ihnen zu tanzen! 

\switchcolumn

\begin{greek}[variant=ancient]%
ὡς ἡδὺ μετὰ σοῦ ὀρχεῖσθαι!

\end{greek}%
\switchcolumn*

Wer ist eigentlich der Herr dort, der hierher sieht? der an der Thür
steht? 

\switchcolumn

\begin{greek}[variant=ancient]%
τίς ποθ' ὅδεὁ δεῦρο βλέπων; \emph{ὁ ἐπὶ} ταῖς θύραις;

\end{greek}%
\switchcolumn*

Es ist mein Mann. 

\switchcolumn

\begin{greek}[variant=ancient]%
ἐστὶν οὑμὸς ἀνήρ.

\end{greek}%
\switchcolumn*

Warum macht er ein so verdrießliches Gesicht? 

\switchcolumn

\begin{greek}[variant=ancient]%
τί σκυθρωπάζει;

\end{greek}%
\switchcolumn*

Er ist sehr eifersüchtig. 

\switchcolumn

\begin{greek}[variant=ancient]%
σφόδρα ζηλότυπός ἐστιν.

\end{greek}%
\switchcolumn*

Wir wollen gar nicht thun, als sähen wir ihn. 

\switchcolumn

\begin{greek}[variant=ancient]%
μὴ ὁρᾶν δοκῶμεν αὐτόν.

\end{greek}%
\switchcolumn*

Ich werde mich hüten! 

\switchcolumn

\begin{greek}[variant=ancient]%
φυλάξομαι\footnote{\begin{latin}%
orig. \textgreek[variant=ancient]{φυλάξομαί}\end{latin}%
}!

\end{greek}%
\switchcolumn*

Den Männern ist ja nicht zu trauen! 

\switchcolumn

\begin{greek}[variant=ancient]%
οὐδὲν γὰρ πιστὸν τοῖς ἀνδράσιν.

\end{greek}%
\switchcolumn*

Sie ist erst 3 Monate verheirathet. 

\switchcolumn

\begin{greek}[variant=ancient]%
νύμφη ἐστὶ τρεῖς μῆνας.

\end{greek}%
\switchcolumn*

Der Tanzlehrer. 

\switchcolumn

\begin{greek}[variant=ancient]%
ὁ ὀρχηστοδιδάσκαλος.

\end{greek}%
\switchcolumn*

In die Tanzstunde. 

\switchcolumn

\begin{greek}[variant=ancient]%
εἰς τὸ ὀρχηστοδιδασκαλεῖον.

\end{greek}%
\switchcolumn*[


\section{Eine Geschichte}

]Hören Sie einmal zu, gnädige Frau, ich will Ihnen eine hübsche Geschichte
erzählen. 

\switchcolumn

\begin{greek}[variant=ancient]%
ἄκουσον, ὦ γύναι, λόγον σοι βούλομαι λέξαι χαρίεντα.

\end{greek}%
\switchcolumn*

Nur zu, erzählen Sie! 

\switchcolumn

\begin{greek}[variant=ancient]%
ἴθι\footnote{\begin{german}[spelling=old,babelshorthands=true]%
orig. \textgreek[variant=ancient]{ιθι}\end{german}%
} δὴ, λέξον.

\end{greek}%
\switchcolumn*

Ist das wahr? 

\switchcolumn

\begin{greek}[variant=ancient]%
τί λέγεις;

\end{greek}%
\switchcolumn*

Sie wundern sich? 

\switchcolumn

\begin{greek}[variant=ancient]%
ἐθαύ\textit{μασας;}

\end{greek}%
\switchcolumn*

Sie erzählen mir (erfundene) Geschichten! 

\switchcolumn

\begin{greek}[variant=ancient]%
μύθους μοι λέγεις!

\end{greek}%
\switchcolumn*

Die Wahrheit wollen Sie doch nicht sagen! 

\switchcolumn

\begin{greek}[variant=ancient]%
τἀληθὲς γὰρ οὐκ ἐθέλεις φράσαι.

\end{greek}%
\switchcolumn*

Wenn Sie wirklich die Wahrheit sprechen, so weiß ich nicht was ich
sagen soll. 

\switchcolumn

\begin{greek}[variant=ancient]%
εἴπερ ὄντως σὺ\footnote{\begin{german}[spelling=old,babelshorthands=true]%
orig. \textgreek[variant=ancient]{συ}\end{german}%
} ταῦτ' ἀληθῆ λέγεις, οὐδὲν ἔχω εἰπεῖν.

\end{greek}%
\switchcolumn*

Nach dem, was Sie sagen, muß man sie bewundern. 

\switchcolumn

\begin{greek}[variant=ancient]%
κατὰ τὸν λόγον, ὃν σὺ λέγεις, ἀξία ἐστὶ θαυμάσαι.

\end{greek}%
\switchcolumn*

Reden Sie mit ihr \emph{von} der Sache! 

\switchcolumn

\begin{greek}[variant=ancient]%
λέγ' αὐτῇ τὸ πρᾶγμα.

\end{greek}%
\switchcolumn*

Sagen = angeben.

\switchcolumn

\begin{greek}[variant=ancient]%
φράζειν.

\end{greek}%
\switchcolumn*

Was hat sie darauf erwidert? 

\switchcolumn

\begin{greek}[variant=ancient]%
τί πρὸς ταῦτα εἶπεν;

\end{greek}%
\switchcolumn*

Sie macht Aus\textcompwordmark{}flüchte.

\switchcolumn

\begin{greek}[variant=ancient]%
προφασίζεσται.

\end{greek}%
\switchcolumn*

Ich will euch ein Märchen erzählen nämlich — 

\switchcolumn

\begin{greek}[variant=ancient]%
μῦθον ὑμῖν βούλομαι λέξαι οὕτως\footnote{\begin{german}[spelling=old,babelshorthands=true]%
orig. \textgreek[variant=ancient]{ουτως}\end{german}%
}.

\end{greek}%
\switchcolumn*[


\section{Ich weiß nicht}

]Ich weiß es nicht. 

\switchcolumn

\begin{greek}[variant=ancient]%
οὐκ οἶδα.

\end{greek}%
\switchcolumn*

Ich kann es nicht sagen. 

\switchcolumn

\begin{greek}[variant=ancient]%
οὐκ ἔχω φράσαι.

\end{greek}%
\switchcolumn*

Worauf soll man rathen? 

\switchcolumn

\begin{greek}[variant=ancient]%
ποῖ τις ἂν τράποιτο;

\end{greek}%
\switchcolumn*

Ich will es schon heraus\textcompwordmark{}bekommen.

\switchcolumn

\begin{greek}[variant=ancient]%
γνώσομαι ἔγωγε.

\end{greek}%
\switchcolumn*

Ich weiß es nicht genau. 

\switchcolumn

\begin{greek}[variant=ancient]%
οὐκ οἶδ' ἀκριβῶς.

\end{greek}%
\switchcolumn*

Nein, soviel ich weiß. 

\switchcolumn

\begin{greek}[variant=ancient]%
οὐχ, ὅσον γέ μ' εἰδέναι.

\end{greek}%
\switchcolumn*

Ich weiß nicht sicher, wie es steht. 

\switchcolumn

\begin{greek}[variant=ancient]%
οὐ σάφ' οἶδα, ὅπως ἔχει.

\end{greek}%
\switchcolumn*

Ich kann es nicht glauben. 

\switchcolumn

\begin{greek}[variant=ancient]%
οὐ πείθομαι.

\end{greek}%
\switchcolumn*

Ich weiß es ja. 

\switchcolumn

\begin{greek}[variant=ancient]%
οἶδά τοι.

\end{greek}%
\switchcolumn*

Ist mir bekannt! 

\switchcolumn

\begin{greek}[variant=ancient]%
μεμνήμεθα!

\end{greek}%
\switchcolumn*

Freilich weiß ich es! 

\switchcolumn

\begin{greek}[variant=ancient]%
οἶδα μέντοι!

\end{greek}%
\switchcolumn*

Da Sie es denn zu wissen verlangen, so will ich es sagen. 

\switchcolumn

\begin{greek}[variant=ancient]%
εἰ δὴ ἐπιθυμεῖς εἰδέναι, φράσω.

\end{greek}%
\switchcolumn*

Wär's möglich? 

\switchcolumn

\begin{greek}[variant=ancient]%
τί φής!

\end{greek}%
\switchcolumn*

Ich habe es aus bester Quelle. 

\switchcolumn

\begin{greek}[variant=ancient]%
πέπυσμαι τοῦτο τῶν σάφ' εἰδότων.

\end{greek}%
\switchcolumn*

Haben Sie bereits etwas von der Sache gehört? 

\switchcolumn

\begin{greek}[variant=ancient]%
ἆρ' ἀκήκοάς τι τοῦ πράγματος;

\end{greek}%
\switchcolumn*

Das mußte ich (bis\textcompwordmark{}her noch) nicht.

\switchcolumn

\begin{greek}[variant=ancient]%
τοῦτ' οὐκ ᾔδειν ἐγώ.

\end{greek}%
\switchcolumn*

\emph{O, dann begreife ich, daß} Sie verstimmt sind.

\switchcolumn

\begin{greek}[variant=ancient]%
\emph{οὐκ ἐτὸς ἄρα} λυπεῖ.

\end{greek}%
\switchcolumn*[


\section{Die Schöne und die Häßliche}

]Sehen Sie die hier an, wie \emph{schön} sie ist! 

\switchcolumn

\begin{greek}[variant=ancient]%
ὅρα ταυτηνὶ, ὡς καλή!

\end{greek}%
\switchcolumn*

Wer ist wohl dort die Dame? 

\switchcolumn

\begin{greek}[variant=ancient]%
τίς ποθ' αὑτηί;

\end{greek}%
\switchcolumn*

Die in dem grauen Kleide? 

\switchcolumn

\begin{greek}[variant=ancient]%
ἡ τὸ φαιὸν ἔνδυμα ἀμπεχομένη;

\end{greek}%
\switchcolumn*

Sie ist die schönste (= blühendste) von allen. 

\switchcolumn

\begin{greek}[variant=ancient]%
πασῶν \emph{ὡραιοτάτη} ἐστίν.

\end{greek}%
\switchcolumn*

Wer mag sie nur sein? 

\switchcolumn

\begin{greek}[variant=ancient]%
τίς καί ἐστί ποτε;

\end{greek}%
\switchcolumn*

Kennt sie Jemand von Ihnen? 

\switchcolumn

\begin{greek}[variant=ancient]%
\emph{γιγνώσκει} τις ὑμῶν;

\end{greek}%
\switchcolumn*

Ja, ich. 

\switchcolumn

\begin{greek}[variant=ancient]%
νὴ Δία ἔγωγε.

\end{greek}%
\switchcolumn*

Es ist meine Cousine. 

\switchcolumn

\begin{greek}[variant=ancient]%
ἐστὶν ἀνεψιά μου.

\end{greek}%
\switchcolumn*

Wie schön sie aus\textcompwordmark{}sieht! 

\switchcolumn

\begin{greek}[variant=ancient]%
οἷον τὸ κάλλος αὐτῆς φαίνεται!

\end{greek}%
\switchcolumn*

Sie hat sehr gesunde Farbe. 

\switchcolumn

\begin{greek}[variant=ancient]%
ὡς εὐχροεῖ!

\end{greek}%
\switchcolumn*

Sie hat ein sanftes, schönes Auge. 

\switchcolumn

\begin{greek}[variant=ancient]%
καὶ τὸ βλέμμα ἔχει μαλακὸν καὶ καλόν.

\end{greek}%
\switchcolumn*

Und allerliebste Hände hat sie. 

\switchcolumn

\begin{greek}[variant=ancient]%
καὶ \emph{τὰς} χεῖρας παγκάλας ἔχει.

\end{greek}%
\switchcolumn*

Sie lacht \emph{gern.}

\switchcolumn

\begin{greek}[variant=ancient]%
καὶ ἡδέως γελᾷ.

\end{greek}%
\switchcolumn*

Ich bin in das Mädchen (die Dame) verliebt. 

\switchcolumn

\begin{greek}[variant=ancient]%
ἔρως με εἴληφε τῆς κόρης ταύτης.

\end{greek}%
\switchcolumn*

Aber sie hat wohl nichts? 

\switchcolumn

\begin{greek}[variant=ancient]%
ἀλλ' ἔχει οὐδέν;

\end{greek}%
\switchcolumn*

O nein, sie ist reich; sie hat ein respectables Vermögen. 

\switchcolumn

\begin{greek}[variant=ancient]%
πλουτεῖ \emph{μὲν οὖν·} οὐσίαν γὰρ ἔχει συχνήν.

\end{greek}%
\switchcolumn*

Weißt du, wem sie ganz ähnlich sieht? Der A. 

\switchcolumn

\begin{greek}[variant=ancient]%
οὖσθ' ᾗ μάλιστ' ἔοικεν; τῇ Ἀ.

\end{greek}%
\switchcolumn*

Dort ist ein schönes Mädchen! (Mädel!) 

\switchcolumn

\begin{greek}[variant=ancient]%
ἐνταῦθα μείραξ ὡραία ἐστίν.

\end{greek}%
\switchcolumn*

Wer ist denn die hinter ihr? 

\switchcolumn

\begin{greek}[variant=ancient]%
τίς γάρ ἐσθ' ἡ ὄπισθεν αὐτῆς.

\end{greek}%
\switchcolumn*

Wer die ist? Frau Schulze. 

\switchcolumn

\begin{greek}[variant=ancient]%
ἥτις ἐστίν; Σχουλζίου γυνή.

\end{greek}%
\switchcolumn*

Die Andere interessirt mich weniger. 

\switchcolumn

\begin{greek}[variant=ancient]%
τῆς ἑτέρας μοι ἧττον μέλει.

\end{greek}%
\switchcolumn*

Sie ist häßlich. 

\switchcolumn

\begin{greek}[variant=ancient]%
αἰσχρὰ γάρ ἐστιν.

\end{greek}%
\switchcolumn*

Und hat eine stumpfe (kolbige) Nase. 

\switchcolumn

\begin{greek}[variant=ancient]%
καὶ σιμή (ἐστιν).

\end{greek}%
\switchcolumn*

Sie ist geschminkt. 

\switchcolumn

\begin{greek}[variant=ancient]%
καὶ καταπεπλασμένη (ἐστίν).

\end{greek}%
\switchcolumn*

sie riecht nach Pomade. 

\switchcolumn

\begin{greek}[variant=ancient]%
ὄζει δὲ μύρου.

\end{greek}%
\switchcolumn*

Riechst du etwas? 

\switchcolumn

\begin{greek}[variant=ancient]%
ὀσφραίνει τι;

\end{greek}%
\switchcolumn*

Die Pomade riecht nicht gut. 

\switchcolumn

\begin{greek}[variant=ancient]%
οὐχ ἡδὺ τὸ μύρον τουτί.

\end{greek}%
\switchcolumn*[


\section{Herr Schulze}

]Schulze heißt er? \emph{Was ist das für ein} Schulze? 

\switchcolumn

\begin{greek}[variant=ancient]%
Σχούλζιος αὐτῷ ὄνομα; ποῖος οὗτος ὁ Σχούλζιος;

\end{greek}%
\switchcolumn*

Kennen Sie ihn nicht? 

\switchcolumn

\begin{greek}[variant=ancient]%
οὐκ οἶσθα αὐτόν;

\end{greek}%
\switchcolumn*

Nein, ich bin fremd hier und erst eben angekommen. 

\switchcolumn

\begin{greek}[variant=ancient]%
οὐ μὰ Δία ἔγωγε, ξένος γάρ εἰμι ἀρτίως ἀφιγμένος.

\end{greek}%
\switchcolumn*

Er spielt die erste Rolle in der Stadt. 

\switchcolumn

\begin{greek}[variant=ancient]%
πράττει τὰ μέγιστα ἐν τῇ πόλει.

\end{greek}%
\switchcolumn*

Er hat einen \emph{großen} Bart. 

\switchcolumn

\begin{greek}[variant=ancient]%
ἔχει δὲ πώγωνα.

\end{greek}%
\switchcolumn*

Und graues Haar? 

\switchcolumn

\begin{greek}[variant=ancient]%
καὶ πολιός ἐστιν;

\end{greek}%
\switchcolumn*

Wovon lebt er? 

\switchcolumn

\begin{greek}[variant=ancient]%
πόθεν διαζῇ;

\end{greek}%
\switchcolumn*

Der Mann ist schnell reich geworben. 

\switchcolumn

\begin{greek}[variant=ancient]%
ταχέως ὁ ἀνὴρ γεγένηται πλούσιος.

\end{greek}%
\switchcolumn*

Wodurch? 

\switchcolumn

\begin{greek}[variant=ancient]%
τί δρῶν;

\end{greek}%
\switchcolumn*

Er hat ursprünglich ein Handwerk gelernt, dann wurde er Landwirth
und jetzt ist er Kaufmann. 

\switchcolumn

\begin{greek}[variant=ancient]%
πρῶτον μὲν γὰρ τέχνην τιν' ἔμαθεν· εἶτα γεωργὸς ἐγένετο, νῦν δὲ ἔμπορός
ἐστιν.

\end{greek}%
\switchcolumn*\bgroup

\myafterpagetrue\mysetaligntext{german}{Es ist{ }}\mysetalign{german}Fabrikant.

\egroup\switchcolumn\bgroup

\begin{greek}[variant=ancient]%
ἐργαστήριον ἔχει.

\end{greek}%
\egroup\switchcolumn*\bgroup

\mysetalign{german}Arbeiter.

\egroup\switchcolumn\bgroup

\begin{greek}[variant=ancient]%
ἐργάτης

\end{greek}%
\egroup\switchcolumn*\bgroup

\mysetalign{german}(Amts- etc.) Richter. 

\egroup\switchcolumn\bgroup

\begin{greek}[variant=ancient]%
δικαστής.

\end{greek}%
\egroup\switchcolumn*\bgroup

\mysetalign{german}Unterbeamter. 

\egroup\switchcolumn\bgroup

\begin{greek}[variant=ancient]%
ὑπάλληλος.

\end{greek}%
\egroup\switchcolumn*\bgroup

\mysetalign{german}Rechts\textcompwordmark{}anwalt. 

\egroup\switchcolumn\bgroup

\begin{greek}[variant=ancient]%
σύνδικος.

\end{greek}%
\egroup\switchcolumn*\bgroup

\mysetalign{german}Apotheker. 

\egroup\switchcolumn\bgroup

\begin{greek}[variant=ancient]%
φαρμακοπώλης.

\end{greek}%
\egroup\switchcolumn*\bgroup

\mysetalign{german}Banquier. 

\egroup\switchcolumn\bgroup

\begin{greek}[variant=ancient]%
τραπεζίτης.

\end{greek}%
\egroup\switchcolumn*\bgroup

\mysetalign{german}Officier. 

\egroup\switchcolumn\bgroup

\begin{greek}[variant=ancient]%
ἀξιωματικός.

\end{greek}%
\egroup\switchcolumn*\bgroup

\mysetalign{german}Schüler. 

\egroup\switchcolumn\bgroup

\begin{greek}[variant=ancient]%
μαθητής.

\end{greek}%
\egroup\switchcolumn*\bgroup

\mysetalign{german}Student. 

\egroup\switchcolumn\bgroup

\begin{greek}[variant=ancient]%
φοιτητής.

\end{greek}%
\egroup\switchcolumn*\bgroup

\mysetalign{german}Lehrer. 

\egroup\switchcolumn\bgroup

\begin{greek}[variant=ancient]%
διδάσκαλος.

\end{greek}%
\egroup\switchcolumn*\bgroup

\mysetalign{german}Professor. 

\egroup\switchcolumn

\begin{greek}[variant=ancient]%
καθηγητής.

\end{greek}%
\switchcolumn*

Er ist vom Lande. 

\switchcolumn

\begin{greek}[variant=ancient]%
ἐκ τῶν ἀγρῶν ἐστιν.

\end{greek}%
\switchcolumn*

Er ist aus der Nachbarschaft. 

\switchcolumn

\begin{greek}[variant=ancient]%
ἐκ τῶν γειτόνων ἐστίν.

\end{greek}%
\switchcolumn*

Mir ist er langweilig. 

\switchcolumn

\begin{greek}[variant=ancient]%
ἄχθομαι αὐτῷ συνὼν ἔγωγε.

\end{greek}%
\switchcolumn*

Er ist nicht schlecht von Charakter. 

\switchcolumn

\begin{greek}[variant=ancient]%
οὐ πονηρός ἐστι τοὺς τρόπους.

\end{greek}%
\switchcolumn*

(Seht nur) wie protzig er hereingekommen ist! 

\switchcolumn

\begin{greek}[variant=ancient]%
\emph{ὡς σοβαρὸς} εἰσελήλυθεν!

\end{greek}%
\switchcolumn*

Es scheint mir nicht guter Ton zu sein, sich so zu betragen. 

\switchcolumn

\begin{greek}[variant=ancient]%
οὐκ ἀστεῖόν μοι δοκεῖ εἶναι τοιτοῦτον ἑαυτὸν παρέχειν.

\end{greek}%
\switchcolumn*

Aber N. N. ist wirklich ein Gentleman. 

\switchcolumn

\begin{greek}[variant=ancient]%
ὁ δὲ Ν. Ν. νὴ Δία γεννάδας ἀνήρ!

\end{greek}%
\switchcolumn*[


\section{Wie alt?}

]Er hat nur eine einzige Tochter. 

\switchcolumn

\begin{greek}[variant=ancient]%
θυγάτηρ αὐτῷ μόνη οὖσα τυγχάνει.

\end{greek}%
\switchcolumn*

Wie alt ist sie? 

\switchcolumn

\begin{greek}[variant=ancient]%
πηλίκη ἐστίν;

\end{greek}%
\switchcolumn*

Sie ist über ein Jahr älter als du. 

\switchcolumn

\begin{greek}[variant=ancient]%
πλεῖν ἢ 'νιαυτῷ σου πρεσβυτέρα ἐστίν.

\end{greek}%
\switchcolumn*

Über 20 Jahre \emph{alt}. 

\switchcolumn

\begin{greek}[variant=ancient]%
ὑπὲρ εἴκοσιν ἔτη \emph{γεγονυῖα.}

\end{greek}%
\switchcolumn*

Du bist ein junger Mann \emph{von} 19 Jahren.

\switchcolumn

\begin{greek}[variant=ancient]%
σὺ δὲ ἀνὴρ νέος εἶ ἐννεακαίδεκα ἐτῶν.

\end{greek}%
\switchcolumn*

Du mußt mit denen unter zwanzig tanzen. 

\switchcolumn

\begin{greek}[variant=ancient]%
δεῖ οὖν ὀρχεῖσθαί σε μετὰ τῶν \emph{ἐντὸς} εἴκοσιν.

\end{greek}%
\switchcolumn*

Sie sitzt dort bei den älteren Damen. 

\switchcolumn

\begin{greek}[variant=ancient]%
ἐνταῦτα κάθηται παρὰ ταῖς πρεσβυτέραις γυναιξίν.

\end{greek}%
\switchcolumn*

Wo? zeig' einmal! 

\switchcolumn

\begin{greek}[variant=ancient]%
τοῦ; δεῖξον!

\end{greek}%
\switchcolumn*

Was hat sie für Toilette? 

\switchcolumn

\begin{greek}[variant=ancient]%
ποίαν τιν' ἔχει σκευήν;

\end{greek}%
\switchcolumn*

Ihre Mutter ist \emph{seit} 10 Jahren todt.

\switchcolumn

\begin{greek}[variant=ancient]%
τέθνηκεν ἡ μήτηρ αὐτῆς ἔτη δέκα.

\end{greek}%
\switchcolumn*

Ihr Vater ist ein Sechziger. 

\switchcolumn

\begin{greek}[variant=ancient]%
ἑξηκοντέτης ἐστὶν αὐτῆς ὁ πατήρ.

\end{greek}%
\switchcolumn*

Die Familie. 

\switchcolumn

\begin{greek}[variant=ancient]%
ὁ οἶκος.

\end{greek}%
