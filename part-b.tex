\switchcolumn*[


\part{In der Schule.}


\section{In die Schule!}

]\indent Es ist Zeit zu gehen!

\switchcolumn

\begin{greek}[variant=ancient]%
ὥρα προβαίνειν σοί ἐστιν.

\end{greek}%
\switchcolumn*

Es ist Zeit in's Gymnasium zu gehen!

\switchcolumn

\begin{greek}[variant=ancient]%
ὥρα ἐστὶν εἰς τὸ γυμνάσιον βαδίζειν.

\end{greek}%
\switchcolumn*

So mach' doch, daß du in's Gymnasium kommst!

\switchcolumn

\begin{greek}[variant=ancient]%
οὐκ ἂν φθάνοις εἰς τὸ γυνμάσιον ἰών;

\end{greek}%
\switchcolumn*

Halt dich nicht auf! --- Beeile dich!

\switchcolumn

\begin{greek}[variant=ancient]%
μή νυν διάτριβε! --- σπεῦδέ νυν!

\end{greek}%
\switchcolumn*

Du hast keine Zeit mehr zu verlieren.

\switchcolumn

\begin{greek}[variant=ancient]%
ὁ καιρός ἐστι μηκέτι μέλλειν.

\end{greek}%
\switchcolumn*

Mach' dir keine Sorge!

\switchcolumn

\begin{greek}[variant=ancient]%
μὴ φροντίσῃς.

\end{greek}%
\switchcolumn*

Nur nicht ängstlich!

\switchcolumn

\begin{greek}[variant=ancient]%
μηδὲν δείσῃς.

\end{greek}%
\switchcolumn*

Sei unbesorgt!

\switchcolumn

\begin{greek}[variant=ancient]%
μηδὲν φοβηθῆς.

\end{greek}%
\switchcolumn*[


\section{Zu spät gekommen!}

]Wir wollen beten!

\switchcolumn

\begin{greek}[variant=ancient]%
ἀλλ' εὐχώμεθα!

\end{greek}%
\switchcolumn*

Ich bin \emph{doch nicht etwa} zu spät gekommen?

\switchcolumn

\begin{greek}[variant=ancient]%
\emph{μῶν} ὕστερος πάρειμι;

\end{greek}%
\switchcolumn*

Ich bin zu spät gekommen.

\switchcolumn

\begin{greek}[variant=ancient]%
ὕστερος ἦλθον!

\end{greek}%
\switchcolumn*

Hilf Himmel! --- Ach, ich Ärmster!

\switchcolumn

\begin{greek}[variant=ancient]%
Ἄπολλον ἀποτρόπαιε! --- οἴμοι κακοδαίμων!

\end{greek}%
\switchcolumn*

Ich Unglücks\textcompwordmark{}wurm!

\switchcolumn

\begin{greek}[variant=ancient]%
κακοδαίμων ἐγώ!

\end{greek}%
\switchcolumn*

Verwünscht!

\switchcolumn

\begin{greek}[variant=ancient]%
οἴμοι τάλας!

\end{greek}%
\switchcolumn*

Wo kommen Sie denn nur her?

\switchcolumn

\begin{greek}[variant=ancient]%
πόθεν ἥκεις ἐτεόν;

\end{greek}%
\switchcolumn*

Sie sind wieder zu spät gekommen!

\switchcolumn

\begin{greek}[variant=ancient]%
ὕστερον αὖθις ἦλθες!

\end{greek}%
\switchcolumn*

Wes\textcompwordmark{}halb sind Sie jetzt erst gekommen?

\switchcolumn

\begin{greek}[variant=ancient]%
τοῦ ἕνεκα τηνικάδε ἀφίκου;

\end{greek}%
\switchcolumn*

Es hat noch nicht acht geschlagen.

\switchcolumn

\begin{greek}[variant=ancient]%
οὐ γάρ πω ἐσήμηνε τὴν ὀγδόην.

\end{greek}%
\switchcolumn*

Sie sind erst nach dem Läuten gekommen!

\switchcolumn

\begin{greek}[variant=ancient]%
ὕστερος σὺ ἦλθες τοῦ σημείου.

\end{greek}%
\switchcolumn*

Seien Sie nicht böse; meine Uhr geht falsch.

\switchcolumn

\begin{greek}[variant=ancient]%
μὴ ἀγανάκτει· τὸ γὰρ ὡρολόγιόν μου \emph{οὐκ ὀρθῶς χωρεῖ.}

\end{greek}%
\switchcolumn*

Wirklich? Zeigen Sie einmal!

\switchcolumn

\begin{greek}[variant=ancient]%
ἄληθες; ἀλλὰ δεῖξον! (\textgerman[spelling=old,babelshorthands=true]{nicht:}
ἀληθές;)

\end{greek}%
\switchcolumn*

Setzen Sie sich!

\switchcolumn

\begin{greek}[variant=ancient]%
κάθιζε!

\end{greek}%
\switchcolumn*[


\section{Schriftliche Arbeiten}

]Wollen einmal sehen, was Sie geschrieben haben!

\switchcolumn

\begin{greek}[variant=ancient]%
φέρ' ἴδω, τί ουν ἔγραψας.

\end{greek}%
\switchcolumn*

Hier ist es.

\switchcolumn

\begin{greek}[variant=ancient]%
\emph{ἰδού.}

\end{greek}%
\switchcolumn*

Wovon handelt der Aufsatz?

\switchcolumn

\begin{greek}[variant=ancient]%
ἐστὶ δὲ περὶ τοῦ τὰ γεγραμμένα;

\end{greek}%
\switchcolumn*

Geben Sie das Heft her, damit ich es lesen kann.

\switchcolumn

\begin{greek}[variant=ancient]%
\emph{φέρε} τὸ βιβλίον, ἵν' ἀναγνῶ.

\end{greek}%
\switchcolumn*

Wollen einmal sehen, was darin steht!

\switchcolumn

\begin{greek}[variant=ancient]%
φέρ' ἴδω, τί ἔνεστιν.

\end{greek}%
\switchcolumn*

Haben Sie einen Bleistift?

\switchcolumn

\begin{greek}[variant=ancient]%
ἔχεις κυκλομόλυβδον;

\end{greek}%
\switchcolumn*

Das R hier ist miserabel.

\switchcolumn

\begin{greek}[variant=ancient]%
τὸ ῥῶ τουτὶ μοχθηρόν.

\end{greek}%
\switchcolumn*

Was ist denn das eigentlich für ein Buchstabe?

\switchcolumn

\begin{greek}[variant=ancient]%
τουτὶ τί \emph{ποτ'} ἐστὶ γράμμα;

\end{greek}%
\switchcolumn*

Sie geben sich keine Mühe!

\switchcolumn

\begin{greek}[variant=ancient]%
οὐκ ἐπιμελὴς εἶ.

\end{greek}%
\switchcolumn*

Haben Sie das allein gemacht (verfaßt)?

\switchcolumn

\begin{greek}[variant=ancient]%
αὐτὸς δὺ ταῦτα ἔγραφες;

\end{greek}%
\switchcolumn*

Verfaßt ist es von mir, aber von meinem Vater corrigirt.

\switchcolumn

\begin{greek}[variant=ancient]%
συντέταχθαι μὲν ταῦτα ὑπ' ἐμοῦ, διώρθωται δὲ ὑπὸ τοῦ πατρός.

\end{greek}%
\switchcolumn*

Haben Sie alles berührt und nichts übergangen?

\switchcolumn

\begin{greek}[variant=ancient]%
ἦ πάντα ἐπελήλυθας κοὐδὲν παρῆλθες;

\end{greek}%
\switchcolumn*

Ich glaube wenigstens.

\switchcolumn

\begin{greek}[variant=ancient]%
δοκεῖ γοῦν μοι.

\end{greek}%
\switchcolumn*

Das steht nicht darin.

\switchcolumn

\begin{greek}[variant=ancient]%
οὐκ ἔνεστι τοῦτο.

\end{greek}%
\switchcolumn*

Ich habe die Nacht nicht geschlafen, sondern bis zum Morgen an meiner
Rede gearbeitet.

\switchcolumn

\begin{greek}[variant=ancient]%
οὐκ ἐκάθευδον τὴν νύκτα ἀλλὰ\footnote{\begin{latin}%
\textgreek[variant=ancient]{ὁ τυπογράφος ἔγραψα τὸν οὐ γεγραμμένον
τόνον.}\end{latin}%
} διεπονούμην πρὸς φῶς περὶ τὸν λόγον.

\end{greek}%
\switchcolumn*

Ich weiß schon, wie Sie es machen. 

\switchcolumn

\begin{greek}[variant=ancient]%
τούς τρόπους σου ἐπίσταμαι.

\end{greek}%
\switchcolumn*

Hier haben Sie zweimal dasselbe gesagt! 

\switchcolumn

\begin{greek}[variant=ancient]%
ἐνταῦθα δὶς ταὐτὸν εἶπες!

\end{greek}%
\switchcolumn*

Gleich von vornherein haben Sie einen kolossalen Bock gemacht. 

\switchcolumn

\begin{greek}[variant=ancient]%
\emph{εὐθὺς} ἡμάρτηκας θαυμασίως ὡς.

\end{greek}%
\switchcolumn*

Ihre Arbeit enthält 20 Fehler. 

\switchcolumn

\begin{greek}[variant=ancient]%
ἔχει τὸ σὸν εἴκοσιν ἁμαρτίας.

\end{greek}%
\switchcolumn*

Sie wissen von vielen Dingen nichts. 

\switchcolumn

\begin{greek}[variant=ancient]%
πολλά σε λανθάνει.

\end{greek}%
\switchcolumn*[


\section{Grammatisches}

]Weiter nun!

\switchcolumn

\begin{greek}[variant=ancient]%
ἴθι νυν.

\end{greek}%
\switchcolumn*

Ich will Sie einmal examiniren, wie es mit Ihnen im Griechischen steht. 

\switchcolumn

\begin{greek}[variant=ancient]%
βούλομαι λαβεῖν σου πεῖραν, ὅπως ἔχεις περὶ τῶν Ἑλληνικῶν.

\end{greek}%
\switchcolumn*

\emph{Wie} heißt der Genitiv von diesem Wort? 

\switchcolumn

\begin{greek}[variant=ancient]%
ποία \emph{ἐστὶν} ἡ γενικὴ ταύτης τῆς λέξεως;

\end{greek}%
\switchcolumn*

Der Nominativ, Dativ, Accusativ, Vocativ? 

\switchcolumn

\begin{greek}[variant=ancient]%
ἡ ὀνομαστική, δοτική, αἰτιατική, κλητική;

\end{greek}%
\switchcolumn*

Falsch! 

\switchcolumn

\begin{greek}[variant=ancient]%
μὴ δῆτα!

\end{greek}%
\switchcolumn*

Der Genitiv von diesem Worte ist ungebräuchlich. 

\switchcolumn

\begin{greek}[variant=ancient]%
ἡ γενικὴ τῆς λέξεως ταύτης ἄχρηστός ἐστιν.

\end{greek}%
\switchcolumn*

Ganz richtig! 

\switchcolumn

\begin{greek}[variant=ancient]%
ὀρθῶς γε!

\end{greek}%
\switchcolumn*

Wie heißt der Indicativ des Präsens von diesem Verb? 

\switchcolumn

\begin{greek}[variant=ancient]%
ποῖός ἐστιν ὁ ἐνεστὼς (χρόνος) τῆς ὁριστικῆς τοῦ ῥήματος τούτου;!

\end{greek}%
\switchcolumn*

Das will ich mir notiren. 

\switchcolumn

\begin{greek}[variant=ancient]%
μνημόσυνα ταῦτα γράψομαι.

\end{greek}%
\switchcolumn*

Ich schreibe mir das auf. 

\switchcolumn

\begin{greek}[variant=ancient]%
γράφομαι τοῦτο.

\end{greek}%
\switchcolumn*

Der Conjunctiv, Optativ, Imperativ. 

\switchcolumn

\begin{greek}[variant=ancient]%
ἡ ὑποτακτική, εὐκτική, προστακτική.

\end{greek}%
\switchcolumn*

Der Infinitiv, das Particip. 

\switchcolumn

\begin{greek}[variant=ancient]%
ἡ ἀπαρέμφατος, ἡ μετοχή.

\end{greek}%
\switchcolumn*

Das Imperfect, Perfect. 

\switchcolumn

\begin{greek}[variant=ancient]%
ὁ παρατατικός, ὁ παρακείμενος.

\end{greek}%
\switchcolumn*

Plus\textcompwordmark{}quamperfect, Aorist. 

\switchcolumn

\begin{greek}[variant=ancient]%
ὁ ὑπερσυντελικός, ἀόριστος.

\end{greek}%
\switchcolumn*

Futurum. (Erstes, zweites.) 

\switchcolumn

\begin{greek}[variant=ancient]%
ὁ μέλλων. (πρῶτος, δεύτερος.)

\end{greek}%
\switchcolumn*

Das Activ, Passiv. 

\switchcolumn

\begin{greek}[variant=ancient]%
τὸ ἐνεργητικόν, παθητικόν.

\end{greek}%
\switchcolumn*

Sie betonen falsch. 

\switchcolumn

\begin{greek}[variant=ancient]%
οὐκ ὀρθῶς τονοῖς.

\end{greek}%
\switchcolumn*

Der Accent (Acut, Gravis, Circumflex). 

\switchcolumn

\begin{greek}[variant=ancient]%
ἡ κεραία (ἡ ὀξεῖα, βαρεῖα, περισπωμένη).

\end{greek}%
\switchcolumn*

Der Artikel muß stehen. 

\switchcolumn

\begin{greek}[variant=ancient]%
δεῖ τοῦ ἄρθρου.

\end{greek}%
\switchcolumn*[


\section{Verkehrte Antworten}

]Geben Sie Acht! 

\switchcolumn

\begin{greek}[variant=ancient]%
πρόσεχε τὸν νοῦν!

\end{greek}%
\switchcolumn*

Beantworten sie mir, was ich fragen werde. 

\switchcolumn

\begin{greek}[variant=ancient]%
ἀπόκριναι, ἅττ' ἄν ἔρωμαι.

\end{greek}%
\switchcolumn*

Antworten Sie bestimmt! 

\switchcolumn

\begin{greek}[variant=ancient]%
ἀπόκριναι σαφῶς!

\end{greek}%
\switchcolumn*

Reden Sie laut. 

\switchcolumn

\begin{greek}[variant=ancient]%
λέξον \emph{μέγα.}

\end{greek}%
\switchcolumn*

Versuchen Sie etwas recht Scharfsinniges u. Gescheites zu sagen! 

\switchcolumn

\begin{greek}[variant=ancient]%
ἀποκινδύνευε λεπτόν τι καὶ σοφὸν λέγειν.

\end{greek}%
\switchcolumn*

Bitte, sprechen Sie weiter! 

\switchcolumn

\begin{greek}[variant=ancient]%
λέγοις ἂν ἄλλο.

\end{greek}%
\switchcolumn*

Fahren Sie fort! 

\switchcolumn

\begin{greek}[variant=ancient]%
λέγε, ὦ 'γαθέ!

\end{greek}%
\switchcolumn*

Nun, Sie scheinen nicht zu wissen, was Sie sagen sollen. 

\switchcolumn

\begin{greek}[variant=ancient]%
ἀλλ' οὐκ ἔχειν ἔοικας, ὅτι λέγῃς.

\end{greek}%
\switchcolumn*

Warum reden Sie nicht weiter? 

\switchcolumn

\begin{greek}[variant=ancient]%
τί σιωπᾷς;

\end{greek}%
\switchcolumn*

Sagen Sie mir, was Sie meinen! 

\switchcolumn

\begin{greek}[variant=ancient]%
εἰπέ μοι, ὅτι\footnote{\begin{latin}%
\textgreek[variant=ancient]{ὁ τυπογράφος ἔγραψα τὸν οὐ γεγραμμένον
ἦχον.}\end{latin}%
} λέγεις.

\end{greek}%
\switchcolumn*

Was reden Sie da für verkehrtes Zeug? 

\switchcolumn

\begin{greek}[variant=ancient]%
τί ταῦτα ληρεῖς;

\end{greek}%
\switchcolumn*

Sie schwatzen in's Blaue hinein! 

\switchcolumn

\begin{greek}[variant=ancient]%
ἄλλως φλυαρεῖς;

\end{greek}%
\switchcolumn*

Das ist was ganz Anderes! 

\switchcolumn

\begin{greek}[variant=ancient]%
οὐ ταὐτόν, ὦ 'τάν!

\end{greek}%
\switchcolumn*

Nicht darnach frage ich Sie! 

\switchcolumn

\begin{greek}[variant=ancient]%
οὐ τοῦτ' ἐρωτῶ σε.

\end{greek}%
\switchcolumn*

Doch (\textlatin{sc.} abbrechend) antworten Sie einmal auf meine Frage. 

\switchcolumn

\begin{greek}[variant=ancient]%
\emph{καὶ μὴν} ἐπερωτηθεὶς ἀπόκριναί μοι.

\end{greek}%
\switchcolumn*

Sie sprechen in Räthseln! 

\switchcolumn

\begin{greek}[variant=ancient]%
δι' αἰνιγμῶν λέγεις.

\end{greek}%
\switchcolumn*

Ist das Ihr Ernst oder scherzen Sie? 

\switchcolumn

\begin{greek}[variant=ancient]%
σπουδάζεις ταῦτα ἢ παίζεις;

\end{greek}%
\switchcolumn*

Unsinn! 

\switchcolumn

\begin{greek}[variant=ancient]%
οὐδὲν λέγεις!

\end{greek}%
\switchcolumn*

Machen Sie weiter kein Gerede! 

\switchcolumn

\begin{greek}[variant=ancient]%
μὴ λάλει!

\end{greek}%
\switchcolumn*

\vspace{0.5em}
Schweigen Sie! 

\switchcolumn

\begin{tabular}{ll}
\ldelim\{{2}{1em}[] & \begin{greek}[variant=ancient]%
σίγα!\end{greek}%
\tabularnewline
 & \begin{greek}[variant=ancient]%
σιώπα!\end{greek}%
\tabularnewline
\end{tabular}

\switchcolumn*

So schweigen Sie doch! 

\switchcolumn

\begin{greek}[variant=ancient]%
οὐ σιγήσει;

\end{greek}%
\switchcolumn*

O Sie Schwachkopf! 

\switchcolumn

\begin{greek}[variant=ancient]%
ὦ μῶρε σύ!

\end{greek}%
\switchcolumn*[


\section{Abbildungen}

]Ich will Ihnen eine Abbildung zeigen.

\switchcolumn

\begin{greek}[variant=ancient]%
εἰκόνα ὑμῖν ἐπιδείξω.

\end{greek}%
\switchcolumn*

Sehen Sie einmal hinunter! 

\switchcolumn

\begin{greek}[variant=ancient]%
βλέψατε κάτω!

\end{greek}%
\switchcolumn*

Sehen Sie hinauf! 

\switchcolumn

\begin{greek}[variant=ancient]%
βλέψατε ἄνω!

\end{greek}%
\switchcolumn*

Wo sehen Sie hin? 

\switchcolumn

\begin{greek}[variant=ancient]%
ποῖ βλέπεις;

\end{greek}%
\switchcolumn*

Sie sehen wo anders hin. 

\switchcolumn

\begin{greek}[variant=ancient]%
ἑτέρωσε βλέπεις.

\end{greek}%
\switchcolumn*

Sieh einmal hierher! 

\switchcolumn

\begin{greek}[variant=ancient]%
δεῦρο σκεψαι!

\end{greek}%
\switchcolumn*

\myafterpagetrue\mysetaligntext{german}{Ich höre ein Geräusch{ }}\mysetalign{german}dahinten. 

\switchcolumn

\begin{greek}[variant=ancient]%
καὶ μὴν αἰσθάνομαι ψόφου τινός ἐξόπισθεν.

\end{greek}%
\myafterpagefalse\switchcolumn*\bgroup

\mysetalign{german}da vorn. 

\egroup\switchcolumn

\begin{greek}[variant=ancient]%
ἐν τῷ πρόσθεν.

\end{greek}%
\switchcolumn*

Hören Sie auf zu schwatzen! 

\switchcolumn

\begin{greek}[variant=ancient]%
παῦσαι λαλῶν!

\end{greek}%
\switchcolumn*

So schwatzen Sie doch nicht! 

\switchcolumn

\begin{greek}[variant=ancient]%
οὐ μὴ λαλήσετε;

\end{greek}%
\switchcolumn*[


\section{Griechische Dichter}

]Sagen Sie mir nun die schönste Stelle aus der Antigone her! 

\switchcolumn

\begin{greek}[variant=ancient]%
ἐκ τῆς Ἀντιγόνης τὸ νῦν εἰπὲ τὴν καλλίστην ῥῆσιν ἀπολέγων.

\end{greek}%
\switchcolumn*

Den Anfang der Odyssee. 

\switchcolumn

\begin{greek}[variant=ancient]%
τὸ πρῶτον τῆς Ὀδυσσείας.

\end{greek}%
\switchcolumn*

Was bedeutet diese Stelle? 

\switchcolumn

\begin{greek}[variant=ancient]%
τί \emph{νοεῖ} τοῦτο;

\end{greek}%
\switchcolumn*

Sie sind nicht recht bei Troste! 

\switchcolumn

\begin{greek}[variant=ancient]%
κακοδαιμονᾷς.

\end{greek}%
\switchcolumn*

Wie naiv! 

\switchcolumn

\begin{greek}[variant=ancient]%
ὡς εὐηθικῶς!

\end{greek}%
\switchcolumn*

Wo haben Sie Ihren Verstand? 

\switchcolumn

\begin{greek}[variant=ancient]%
ποῦ τὸν νοῦν ἔχεις;

\end{greek}%
\switchcolumn*

Sie sind von Sinnen. 

\switchcolumn

\begin{greek}[variant=ancient]%
παραφρονεῖς!

\end{greek}%
\switchcolumn*

Diese Stelle hat Sophokles nicht so aufgefaßt, wie Sie sie auffassen.
Überlegen Sie es sich besser! 

\switchcolumn

\begin{greek}[variant=ancient]%
τὴν ῥῆσιν ταύτην οὐκ οὕτω Σοφοκλῆς ὑπελάμβανεν, ὡς σὺ ὑπολαμβάνεις.
ὅρα δὴ βέλτιον.

\end{greek}%
\switchcolumn*

Beachten Sie diesen Aus\textcompwordmark{}druck!\_

\switchcolumn

\begin{greek}[variant=ancient]%
σκόπει τὸ ῥῆμα τοῦτο!

\end{greek}%
\switchcolumn*

\textgreek[variant=ancient]{ἥκω} ist gleichbedeutend mit \textgreek[variant=ancient]{κατέρχομαι.} 

\switchcolumn

\begin{greek}[variant=ancient]%
ἥκω ταὐτόν ἐστι τῷ κατέρχομαι.

\end{greek}%
\switchcolumn*

Was soll das bedeuten? 

\switchcolumn

\begin{greek}[variant=ancient]%
τίς ὁ νοῦς.

\end{greek}%
\switchcolumn*

Jetzt sprechen sie vernünftig. 

\switchcolumn

\begin{greek}[variant=ancient]%
τουτὶ φρονίμως ἤδη λέγεις.

\end{greek}%
\switchcolumn*

Sie haben nunmehr den Sinn vollkommen inne. 

\switchcolumn

\begin{greek}[variant=ancient]%
πάντ' ἔχεις ἤδη.

\end{greek}%
\switchcolumn*

Sie haben gut combinirt. 

\switchcolumn

\begin{greek}[variant=ancient]%
εὖ γε ξυνέβαλες!

\end{greek}%
\switchcolumn*

Das ist ohne Zweifel \emph{das Schönste, was} Sophokles gedichtet
hat. 

\switchcolumn

\begin{greek}[variant=ancient]%
τοῦτο δήπου κάλλιστον πεποίηκε Σοφοκλῆς.

\end{greek}%
\switchcolumn*

Sophokles steht über Euripides. 

\switchcolumn

\begin{greek}[variant=ancient]%
Σοφοκλῆς πρότερός ἐστ' Εὐριπίδου.

\end{greek}%
\switchcolumn*

Doch ist dieser ebenfalls ein guter Dichter. 

\switchcolumn

\begin{greek}[variant=ancient]%
ὁ δ' ἀγαθὸς ποιητής ἐστι καὶ αὐτός.

\end{greek}%
\switchcolumn*

Ich bin kein Verehrer des Euripides. 

\switchcolumn

\begin{greek}[variant=ancient]%
οὐκ ἐπαινῶ Εὐριπίδην μὰ Δία.

\end{greek}%
\switchcolumn*

Fällt Ihnen nicht ein Vers des Euripides ein? 

\switchcolumn

\begin{greek}[variant=ancient]%
οὐκ ἀναμιμνήσκει ἴαμβον Εὐριπίδου;

\end{greek}%
\switchcolumn*

Das können sie ziemlich gut. 

\switchcolumn

\begin{greek}[variant=ancient]%
τουτὶ μὲν ἐπιεικῶς σύγ' ἐπίστασαι.

\end{greek}%
\switchcolumn*

Im Euripides sind Sie gut bewandert. 

\switchcolumn

\begin{greek}[variant=ancient]%
Εὐριπίδην πεπάτηκας ἀκριβῶς.

\end{greek}%
\switchcolumn*

Wo haben Sie das so gut gelernt? 

\switchcolumn

\begin{greek}[variant=ancient]%
πόθεν ταῦτ' ἔμαθες οὕτω καλῶς;

\end{greek}%
\switchcolumn*

Ich habe mir viele Stellen von Euripides abgeschrieben. 

\switchcolumn

\begin{greek}[variant=ancient]%
Εὐριπίδου ῥήσεις ἐξεγραψάμην πολλάς.

\end{greek}%
\switchcolumn*

Declamire mir ein Stück von einem neueren Dichter! 

\switchcolumn

\begin{greek}[variant=ancient]%
λέξον τι τῶν νεωτέρων.

\end{greek}%
\switchcolumn*

Sie verdienen es nicht, denn einen originellen Dichter wird man wohl
nicht mehr unter ihnen finden. 

\switchcolumn

\begin{greek}[variant=ancient]%
οὐκ ἔξιοί εἰσι τούτου, γόνιμον γὰρ ποιητὴν οὐκ ἂν ἔτι εὕροις ἐν αὐτοῖς.

\end{greek}%
\switchcolumn*

Welche Ansicht haben Sie über Äschylus? 

\switchcolumn

\begin{greek}[variant=ancient]%
περὶ Αἰσχύλου δὲ τίνα ἔχεις γνώμην;

\end{greek}%
\switchcolumn*

Den Äschylus stelle ich am höchsten unter den Dichtern. 

\switchcolumn

\begin{greek}[variant=ancient]%
Αἰσχύλον νομίζω πρῶτον ἐν ποιηταῖς.

\end{greek}%
\switchcolumn*

Kennen Sie dieses Lied von Simonides? 

\switchcolumn

\begin{greek}[variant=ancient]%
ἐπίστασαι τοῦτο τὸ ἆσμα Σιμωνίδου.

\end{greek}%
\switchcolumn*

Ja! 

\switchcolumn

\begin{greek}[variant=ancient]%
μάλιστα.

\end{greek}%
\switchcolumn*

Ja gewiß! 

\switchcolumn

\begin{greek}[variant=ancient]%
ἔγωγε νὴ Δία\textgerman[spelling=old,babelshorthands=true]{.}

\end{greek}%
\switchcolumn*

Soll ich es ganz hersagen? 

\switchcolumn

\begin{greek}[variant=ancient]%
βούλει πᾶν διεξέλθω;

\end{greek}%
\switchcolumn*

Ist nicht nöthig. 

\switchcolumn

\begin{greek}[variant=ancient]%
οὐδὲν δεῖ.

\end{greek}%
\switchcolumn*

Wie heißen diese Verse? (\textlatin{sc.} mit Namen) 

\switchcolumn

\begin{greek}[variant=ancient]%
ὄνομα δὲ τούτῳ τῷ μέτρῳ τί ἐστιν;

\end{greek}%
\switchcolumn*

Ich kann das Gedicht nicht. 

\switchcolumn

\begin{greek}[variant=ancient]%
τὸ ἆσμα οὐκ ἐπίσταμαι.

\end{greek}%
\switchcolumn*

Doch ich wende mich nun zu dem zweiten Act der Tragödie. 

\switchcolumn

\begin{greek}[variant=ancient]%
καὶ μὴν ἐπὶ τὸ δεύτερον τῆς τραγῳδίας\footnote{\begin{latin}%
\textgreek[variant=ancient]{ὁ τυπογράφος ἔγραψα τὸν οὐ γεγραμμένον
ἴοτα ὑπογραμμένον.}\end{latin}%
} μέρος τρέψομαι.

\end{greek}%
\switchcolumn*[


\section{Übersetzen}

]Suchen Sie in Ihrem Buche den Abschnitt über Sokrates auf! Es ist
Nr.\ 107.

\switchcolumn

\begin{greek}[variant=ancient]%
ζητεῖτε τὸ περὶ Σωκράτους λαβόντες τὸ βιβλίον. ἐστὶ δὲ τὸ ἐκατοστὸν
καὶ ἕβδομον. 

\end{greek}%
\switchcolumn*

Nun, so geben Sie Acht!.

\switchcolumn

\begin{greek}[variant=ancient]%
ἀλλὰ προσέχετε τὸν νοῦν.

\end{greek}%
\switchcolumn*

Wir wollen das (mündlich) in's Griechische übersetzen.

\switchcolumn

\begin{greek}[variant=ancient]%
λέγωμεν ἑλληνικῶς ταῦτα μεταβάλλοντες.

\end{greek}%
\switchcolumn*

Fangen Sie an, N.!

\switchcolumn

\begin{greek}[variant=ancient]%
ἴθι δή\footnote{\begin{latin}%
orig. \textgreek[variant=ancient]{δὴ}\end{latin}%
}, λέγε, ὦ Ν.

\end{greek}%
\switchcolumn*

Ich bin mit Ihrer Übersetzung zufrieden.

\switchcolumn

\begin{greek}[variant=ancient]%
ταῦτα μ' ἤρεσας λέγων.

\end{greek}%
\switchcolumn*

Von wem haben sie Griechisch gelernt?

\switchcolumn

\begin{greek}[variant=ancient]%
τίς σ' ἐδίδαξε τὴν ἑλληνικὴν φωνήν;

\end{greek}%
\switchcolumn*

Fahren Sie \emph{fort!}

\switchcolumn

\begin{greek}[variant=ancient]%
\emph{λέγε.}

\end{greek}%
\switchcolumn*

Das ist wieder ganz geschickt.

\switchcolumn

\begin{greek}[variant=ancient]%
τοῦτ' αὖ δεξιόν.

\end{greek}%
\switchcolumn*

Fahren \emph{Sie} fort!

\switchcolumn

\begin{greek}[variant=ancient]%
λέγε δὴ σύ, ὦ 'γαθέ.

\end{greek}%
\switchcolumn*

Sie übersetzen ungeschickt.

\switchcolumn

\begin{greek}[variant=ancient]%
σκαιῶς ταῦτα λέγεις.

\end{greek}%
\switchcolumn*

Das ist ein Jonisches Wort.

\switchcolumn

\begin{greek}[variant=ancient]%
τοῦτ' ἐστ' Ἰωνικὸν τὸ ῥῆμα.

\end{greek}%
\switchcolumn*

Sie übersetzen in Jonischem Dialekt.

\switchcolumn

\begin{greek}[variant=ancient]%
Ἰωνικῶς λέγεις.

\end{greek}%
\switchcolumn*

Nun, wie wollen Sie übersetzen?

\switchcolumn

\begin{greek}[variant=ancient]%
φέρε δή\footnote{\begin{latin}%
orig. \textgreek[variant=ancient]{δὴ}\end{latin}%
}, τί λέγεις;

\end{greek}%
\switchcolumn*

Machen Sie schnell u.\textgreek[variant=ancient]{}\footnote{\quotedblbase und``}\ übersetzen
Sie!

\switchcolumn

\begin{greek}[variant=ancient]%
ἀλλ' \emph{ἀνύσας} λέγε!

\end{greek}%
\switchcolumn*

Mit Ihnen ist nichts.

\switchcolumn

\begin{greek}[variant=ancient]%
σύγ' οὐδὲν εἶ.

\end{greek}%
\switchcolumn*

Es ist \emph{meine Pflicht,} daß ich Ihnen dies sage.

\switchcolumn

\begin{greek}[variant=ancient]%
δικαίως δὲ τοῦτό σοι λέγω.

\end{greek}%
\switchcolumn*

Sie können ja nicht drei Worte übersetzen, ohne Fehler zu machen.

\switchcolumn

\begin{greek}[variant=ancient]%
σὺ γὰρ οὐδὲ τρία ῥήματα ἑλληνικῶς εἰπεῖν οἷός τ' εἶ πρὶν ἐξαμαρτεῖν.

\end{greek}%
\switchcolumn*

Hören Sie auf!

\switchcolumn

\begin{greek}[variant=ancient]%
\emph{παῦε!}

\end{greek}%
\switchcolumn*

Übersetzen Sie dieses Stück auch schriftlich!.

\switchcolumn

\begin{greek}[variant=ancient]%
καὶ μεταγράφετε αὐτὸ τοῦτο ἑλληνιστί!

\end{greek}%
\switchcolumn*

Verstanden?

\switchcolumn

\begin{greek}[variant=ancient]%
\emph{μανθάνετε;}

\end{greek}%
\switchcolumn*

Ja wohl!

\switchcolumn

\begin{greek}[variant=ancient]%
πάνυ μανθάνομεν.

\end{greek}%
\switchcolumn*

Die Aufgabe.

\switchcolumn

\begin{greek}[variant=ancient]%
τὸ ἔργον.

\end{greek}%
\switchcolumn*

Wie fatal, daß ich das Heft vergessen habe.

\switchcolumn

\begin{greek}[variant=ancient]%
ἐς κόρακας! ὡς ἄχθομαι, ὅτι\footnote{\begin{latin}%
\textgreek[variant=ancient]{τῷ τυπογράφῳ ἄσκοπος τὸ γράμμα «ι» ἦν,
καὶ ὁ τυπογράφος ἔγραψα τὸν οὐ γεγραμμένον τόνον.}\end{latin}%
} ἐπελαθόμην τοὺς χάρτας (τὸ βιβλίον) προσφέρειν.

\end{greek}%
\switchcolumn*

Leih' mir eine Feder und Papier!

\switchcolumn

\begin{greek}[variant=ancient]%
χρῆσόν τί μοι γραφεῖον καὶ χάρτην.

\end{greek}%
\switchcolumn*[


\section{Beschäftigt}

]Jeder geht an seine Arbeit.

\switchcolumn

\begin{greek}[variant=ancient]%
πᾶς χωρεῖ πρὸς ἔργον.

\end{greek}%
\switchcolumn*

Was haben wir (beiden) denn nun weiter zu thun?

\switchcolumn

\begin{greek}[variant=ancient]%
ἄγε δή, τί νῷν ἐντευθενὶ ποιητέον;

\end{greek}%
\switchcolumn*

So, das wäre besorgt.

\switchcolumn

\begin{greek}[variant=ancient]%
ταυτὶ δέδραται.

\end{greek}%
\switchcolumn*

Ich will's besorgen.

\switchcolumn

\begin{greek}[variant=ancient]%
ταῦτα δράσω.

\end{greek}%
\switchcolumn*

Das will ich schon besorgen.

\switchcolumn

\begin{greek}[variant=ancient]%
μελήσει μοι ταῦτα.

\end{greek}%
\switchcolumn*

Da ist Alles, was du brauchst.

\switchcolumn

\begin{greek}[variant=ancient]%
ἰδοὺ πάντα, ὧν δέει.

\end{greek}%
\switchcolumn*

Hast du Alles, was du brauchst?

\switchcolumn

\begin{greek}[variant=ancient]%
ἆρ' ἔχεις ἅπαντα, ἅ δεῖ;

\end{greek}%
\switchcolumn*

Ja, ich habe Alles da, was ich brauche.

\switchcolumn

\begin{greek}[variant=ancient]%
πάντα νὴ Δία πάρεστι μοί, ὅσων δέομαι.

\end{greek}%
\switchcolumn*

Die Sache ist ganz einfach.

\switchcolumn

\begin{greek}[variant=ancient]%
\emph{φαυλότατον} ἔργον.

\end{greek}%
\switchcolumn*

Zu welchem Zwecke thut ihr dies?

\switchcolumn

\begin{greek}[variant=ancient]%
ἵνα δὴ τί τοῦτο δρᾶτε;

\end{greek}%
\switchcolumn*

So geht die Sache viel besser.

\switchcolumn

\begin{greek}[variant=ancient]%
χωρεῖ τὸ πρᾶγμα οὕτω\footnote{\begin{latin}%
\textgreek[variant=ancient]{ὁ τυπογράφος ἔγραψα τὸν οὐ γεγραμμένον
ἦχον και τόνον.}\end{latin}%
} πολλῷ\footnote{\begin{latin}%
\textgreek[variant=ancient]{τῷ τυπογράφῳ ἄσκοπος τὸ γράμμα «ῷ» ἦν.}\end{latin}%
} πᾶλλον.

\end{greek}%
\switchcolumn*

Sei fleißig bei der Arbeit!

\switchcolumn

\begin{greek}[variant=ancient]%
τῷ ἔργῳ πρόσεχε!

\end{greek}%
\switchcolumn*

Mach' es nicht wie die Andern!

\switchcolumn

\begin{greek}[variant=ancient]%
μὴ ποίει, ἅπερ οἱ ἄλλοι δρῶσιν!

\end{greek}%
\switchcolumn*

Die Arbeit geht nicht vorwärts.

\switchcolumn

\begin{greek}[variant=ancient]%
οὐ χωρεῖ τοὖργον.

\end{greek}%
\switchcolumn*

Was wollen Sie \emph{denn }thun?

\switchcolumn

\begin{greek}[variant=ancient]%
τί δαὶ ποιήσεις;

\end{greek}%
\switchcolumn*

Das Weitere ist \emph{Eure} Aufgabe.

\switchcolumn

\begin{greek}[variant=ancient]%
ὑμέτερον ἐντεῦθεν ἔργον.

\end{greek}%
\switchcolumn*

Hilf mir, wenn du (jetzt) keine Abhaltung hast!

\switchcolumn

\begin{greek}[variant=ancient]%
συλλαμβάνου, εἰ μή σέ τι κωλύει!

\end{greek}%
\switchcolumn*

Ich habe keine Zeit.

\switchcolumn

\begin{greek}[variant=ancient]%
\emph{οὐ σχολή} (μοι).

\end{greek}%
\switchcolumn*[


\section{Lob und Tadel}

]Wie denken Sie über diesen Schüler, Herr Rector? 

\switchcolumn

\begin{greek}[variant=ancient]%
τί οὖν ἐρεῖς περὶ τούτου τοῦ μαθητοῦ, ὦ γυμνασίαρχε;

\end{greek}%
\switchcolumn*

Der Mensch ist nicht unbegabt. 

\switchcolumn

\begin{greek}[variant=ancient]%
οὐ σκαιὸς ἄνθρωπος\footnote{orig. ἅνθρωπος}!

\end{greek}%
\switchcolumn*

Er scheint mir nicht unbegabt zu sein. 

\switchcolumn

\begin{greek}[variant=ancient]%
οὐ σκαιός μοι δοκεῖ εἶναι.

\end{greek}%
\switchcolumn*

Nein, er ist (vielmehr) recht befähigt. 

\switchcolumn

\begin{greek}[variant=ancient]%
δεξιὸς \emph{μὲν οὖν} ἐστιν.

\end{greek}%
\switchcolumn*

Und lerneifrig und geweckt. 

\switchcolumn

\begin{greek}[variant=ancient]%
καὶ φιλομαθὴς καὶ ἀγχίνους.

\end{greek}%
\switchcolumn*

Und wie ist der Andere? 

\switchcolumn

\begin{greek}[variant=ancient]%
ὁ δὲ ἕτερος ποῖός τις;

\end{greek}%
\switchcolumn*

Er gehört zur schlechten Sorte. 

\switchcolumn

\begin{greek}[variant=ancient]%
ἐστὶ τοῦ πονηροῦ κόμματος.

\end{greek}%
\switchcolumn*

Nun, mit diesem werde ich später ein Wort reden. 

\switchcolumn

\begin{greek}[variant=ancient]%
ἀλλὰ πρὸς τοῦτον μὲν ὕστερός ἐστί μοι λόγος.

\end{greek}%
\switchcolumn*

Er ist vergeßlich und schwer von Begriffen. 

\switchcolumn

\begin{greek}[variant=ancient]%
ἐπιλήσμων γάρ ἐστι καὶ βραδύς.

\end{greek}%
\switchcolumn*

Und er giebt sich keine Mühe. 

\switchcolumn

\begin{greek}[variant=ancient]%
καὶ οὐκ ἐπιμελής ἐστιν.

\end{greek}%
\switchcolumn*

Er ist der dümmste von allen. 

\switchcolumn

\begin{greek}[variant=ancient]%
ἠλιθιότατός ἐστι πάντων.

\end{greek}%
\switchcolumn*

Er hat sich ganz und gar geändert. 

\switchcolumn

\begin{greek}[variant=ancient]%
πολὺ πάνυ μεθέστηκεν.

\end{greek}%
\switchcolumn*

Ich weiß es wohl. 

\switchcolumn

\begin{greek}[variant=ancient]%
οἶδά τοι.

\end{greek}%
\switchcolumn*

Wir werden entsprechende Maßregeln ergreifen. 

\switchcolumn

\begin{greek}[variant=ancient]%
ποιήσομέν τι τῶν προὔργου.

\end{greek}%
\switchcolumn*

Er ist \quotedblbase dumm, faul und gefräßig.``

\switchcolumn

\begin{greek}[variant=ancient]%
ἠλίθιός τε καὶ ἀργὸς καὶ γάστρις ἐστιν.

\end{greek}%
\switchcolumn*

Er ist ganz verdreht. 

\switchcolumn

\begin{greek}[variant=ancient]%
μεγαγχολᾷ.

\end{greek}%
\switchcolumn*

Wie macht A. seine Sache? 

\switchcolumn

\begin{greek}[variant=ancient]%
ὁ δὲ Ἁ.\ πῶς παρέχει τὰ ἑαυτοῦ;

\end{greek}%
\switchcolumn*

Nach (seinen) Kräften. 

\switchcolumn

\begin{greek}[variant=ancient]%
καθ' ὅσον ἂν σθένῃ!

\end{greek}%
\switchcolumn*

Ziemlich gut. 

\switchcolumn

\begin{greek}[variant=ancient]%
ἐπιεικῶς.

\end{greek}%
\switchcolumn*\bgroup

\myafterpagetrue\mysetaligntext{german}{(Censuren:) { }}\mysetalign{german}1. 

\egroup\switchcolumn\bgroup

\mysetaligntext{greek}{}\mysetalign*{greek}\textgreek[variant=ancient]{εὖγε.}

\egroup\switchcolumn*\bgroup

\mysetalign{german}1b.

\egroup\switchcolumn\bgroup

\mysetalign*{greek}\textgreek[variant=ancient]{καλῶς.}

\egroup\switchcolumn*\bgroup

\mysetalign{german}2a.

\egroup\switchcolumn\bgroup

\mysetalign*{greek}\textgreek[variant=ancient]{ἀκριβῶς.}

\egroup\switchcolumn*\bgroup

\mysetalign{german}2.

\egroup\switchcolumn\bgroup

\mysetalign*{greek}\textgreek[variant=ancient]{ὀρθῶς.}

\egroup\switchcolumn*\bgroup

\mysetalign{german}2b.

\egroup\switchcolumn\bgroup

\mysetalign*{greek}\textgreek[variant=ancient]{ἐπιεικῶς.}

\egroup\switchcolumn*\bgroup

\mysetalign{german}3a.

\egroup\switchcolumn\bgroup

\mysetalign*{greek}\textgreek[variant=ancient]{μετρίως.}

\egroup\switchcolumn*\bgroup

\mysetalign{german}3.

\egroup\switchcolumn\bgroup

\mysetalign*{greek}\textgreek[variant=ancient]{μέσως.}

\egroup\switchcolumn*\bgroup

\mysetalign{german}3b.

\egroup\switchcolumn\bgroup

\mysetalign*{greek}\textgreek[variant=ancient]{φαύλως.}

\egroup\switchcolumn*\bgroup

\mysetalign{german}4.

\egroup\switchcolumn\bgroup

\mysetalign*{greek}\textgreek[variant=ancient]{οὐκ ὀρθῶς.}

\egroup\switchcolumn*[


\section{Singen}

]Singe etwas!

\switchcolumn

\begin{greek}[variant=ancient]%
ᾆδόν τι!

\end{greek}%
\switchcolumn*

Ich kann nicht singen. 

\switchcolumn

\begin{greek}[variant=ancient]%
μελῳδεῖν οὐκ ἐπίσταμαι\footnote{\begin{latin}%
orig. \textgreek[variant=ancient]{επίσταμαι}\end{latin}%
}!

\end{greek}%
\switchcolumn*

Singt einmal ein Lied! 

\switchcolumn

\begin{greek}[variant=ancient]%
μέλος τι ᾄσατε.

\end{greek}%
\switchcolumn*

Was gedenkt Ihr zu singen? 

\switchcolumn

\begin{greek}[variant=ancient]%
τί ἐπινοεῖτε ᾄδειν;

\end{greek}%
\switchcolumn*

Nun, was sollen wir denn singen? 

\switchcolumn

\begin{greek}[variant=ancient]%
ἀλλὰ τί δῆτ' ᾄδωμεν;

\end{greek}%
\switchcolumn*

Sagen Sie nur, was Sie gern hören. 

\switchcolumn

\begin{greek}[variant=ancient]%
εἰπὲ οἶστισι \emph{χαίρεις.}

\end{greek}%
\switchcolumn*

Ein herrliches Lied! 

\switchcolumn

\begin{greek}[variant=ancient]%
ὡς ἡδὺ τὸ μέλος!

\end{greek}%
\switchcolumn*

Wir wollen noch eins singen. 

\switchcolumn

\begin{greek}[variant=ancient]%
ἕτερον ᾀσόμεθα.

\end{greek}%
\switchcolumn*

Erlauben sie, daß ich ein Solo singe! 

\switchcolumn

\begin{greek}[variant=ancient]%
ἔασόν με μονῳδῆσαι.

\end{greek}%
\switchcolumn*

Singe, soviel du willst! 

\switchcolumn

\begin{greek}[variant=ancient]%
ἀλλ' ᾆδ' ὁπόσα βούλει.

\end{greek}%
\switchcolumn*

Hör' auf zu singen! 

\switchcolumn

\begin{greek}[variant=ancient]%
παῦσαι μελῳδῶν!

\end{greek}%
\switchcolumn*

Du singst immer nu \emph{vom} Wein. 

\switchcolumn

\begin{greek}[variant=ancient]%
οὐδὲν γὰρ ᾄδεις πλὴν οἶνον.

\end{greek}%
\switchcolumn*

Das gefällt mir. 

\switchcolumn

\begin{greek}[variant=ancient]%
τουτί μ' ἀρέσκει.

\end{greek}%
\switchcolumn*

Ihnen gefällt das? 

\switchcolumn

\begin{greek}[variant=ancient]%
σὲ δὲ τοῦτ' ἀρέσκει;

\end{greek}%
\switchcolumn*

Was Sie deben gesungen haben, werde ich sicherlich nie vergessen. 

\switchcolumn

\begin{greek}[variant=ancient]%
ὅσα ἄρτι ᾖσας, οὐ μὴ ἐπιλάθωμαί ποτε.!

\end{greek}%
\switchcolumn*

Ich will ein Lied dazu singen. 

\switchcolumn

\begin{greek}[variant=ancient]%
ἐπᾴσομαι μέλος τι.

\end{greek}%
\switchcolumn*[


\section{Sie haben Recht!}

]Sie haben Recht. 

\switchcolumn

\begin{greek}[variant=ancient]%
\emph{εὖ λέγεις.}

\end{greek}%
\switchcolumn*

Sie haben wirklich Recht. 

\switchcolumn

\begin{greek}[variant=ancient]%
εὖ τοι λέγεις.

\end{greek}%
\switchcolumn*

Sie könnten vielleicht Recht haben. 

\switchcolumn

\begin{greek}[variant=ancient]%
ἴσως ἄν τι λέγοις.

\end{greek}%
\switchcolumn*

Sie haben ganz Recht. 

\switchcolumn

\begin{greek}[variant=ancient]%
εὖ πάνυ λέγεις.

\end{greek}%
\switchcolumn*

Sie haben offenbar Recht. 

\switchcolumn

\begin{greek}[variant=ancient]%
εὖ λέγειν σὺ φαίνει.

\end{greek}%
\switchcolumn*

Ich denke, Sie haben Recht. 

\switchcolumn

\begin{greek}[variant=ancient]%
εὖ γέ μοι δοκεῖς λέγειν.

\end{greek}%
\switchcolumn*

Das ist auch meine Ansicht. 

\switchcolumn

\begin{greek}[variant=ancient]%
συνδοκεῖ ταῦτα κἀμοί.

\end{greek}%
\switchcolumn*

Es kommt mir allerdings auch so vor. 

\switchcolumn

\begin{greek}[variant=ancient]%
τοῦτο μὲν κἀμοί δοκεῖ.

\end{greek}%
\switchcolumn*

Das ist ganz klar. 

\switchcolumn

\begin{greek}[variant=ancient]%
τοῦτο περιφανέστατον.

\end{greek}%
\switchcolumn*

Das ist ein billiger Vorschlag. 

\switchcolumn

\begin{greek}[variant=ancient]%
δίκαιος ὁ λόγος.

\end{greek}%
\switchcolumn*

Glaub's gern. 

\switchcolumn

\begin{greek}[variant=ancient]%
\emph{πείθομαι.}

\end{greek}%
\switchcolumn*

Wie es scheint. 

\switchcolumn

\begin{greek}[variant=ancient]%
ὡς ἔοικεν.

\end{greek}%
\switchcolumn*

Dafür giebt es viele Beweise. 

\switchcolumn

\begin{greek}[variant=ancient]%
τούτων τεκμήριά ἐστι πολλά.

\end{greek}%
\switchcolumn*

Ich schließe es aus Thatsachen. 

\switchcolumn

\begin{greek}[variant=ancient]%
ἔργῳ τεκμαίρομαι.

\end{greek}%
\switchcolumn*[


\section{Ja!}

]Ja! (Ohne Zweifel!) 

\switchcolumn

\begin{greek}[variant=ancient]%
νὴ\footnote{\begin{latin}%
\textgreek[variant=ancient]{τυπογράφος ἔγραψα τὸν οὐ γεγραμμένον
τόνον.}\end{latin}%
} Δία!

\end{greek}%
\switchcolumn*

Ja wahrhaftig! 

\switchcolumn

\begin{greek}[variant=ancient]%
νὴ τοὺς θεούς! --- νὴ τὸν Ποσειδῶ!

\end{greek}%
\switchcolumn*

Ganz recht! 

\switchcolumn

\begin{greek}[variant=ancient]%
μάλιστά γε. --- νάνυ!

\end{greek}%
\switchcolumn*

Sehr richtig! 

\switchcolumn

\begin{greek}[variant=ancient]%
κομιδῆ μὲν οὖν!

\end{greek}%
\switchcolumn*

Natürlich! 

\switchcolumn

\begin{greek}[variant=ancient]%
εἰκότως! --- εἰκὸς γάρ!

\end{greek}%
\switchcolumn*

Ja natürlich! 

\switchcolumn

\begin{greek}[variant=ancient]%
εἰκότως γε (νὴ Δία)!

\end{greek}%
\switchcolumn*

Ganz gewiß! 

\switchcolumn

\begin{greek}[variant=ancient]%
εὖ ἴσθ' ὅτι!

\end{greek}%
\switchcolumn*

Ich? Freilich, Sie! 

\switchcolumn

\begin{greek}[variant=ancient]%
ἐγώ; σὺ \emph{μέντοι!}

\end{greek}%
\switchcolumn*

Kann sein! 

\switchcolumn

\begin{greek}[variant=ancient]%
\emph{οὐκ οἶδα.}

\end{greek}%
\switchcolumn*

Kann wohl sein! 

\switchcolumn

\begin{greek}[variant=ancient]%
ἕοικεν!

\end{greek}%
\switchcolumn*

Kein Wunder! 

\switchcolumn

\begin{greek}[variant=ancient]%
κοὐ θαῦμά γε!

\end{greek}%
\switchcolumn*

Und das ist gar kein Wunder! 

\switchcolumn

\begin{greek}[variant=ancient]%
καὶ θαῦμά γ' οὐδέν!

\end{greek}%
\switchcolumn*

Schön! 

\switchcolumn

\begin{greek}[variant=ancient]%
εὖ λέγεις!

\end{greek}%
\switchcolumn*

Du fragst noch? 

\switchcolumn

\begin{greek}[variant=ancient]%
οὐκ\footnote{\begin{latin}%
\textgreek[variant=ancient]{ὁ τυπογράφος ἔγραψα τὸν οὐ γεγραμμένον
ἦχον.}\end{latin}%
} οἶσθα;!

\end{greek}%
\switchcolumn*[


\section{Nein!}

]Nein! 

\switchcolumn

\begin{greek}[variant=ancient]%
\emph{οὐ μὰ Δία!}

\end{greek}%
\switchcolumn*

Nein, ich nicht. 

\switchcolumn

\begin{greek}[variant=ancient]%
μὰ Δί' ἐγὼ μὲν οὔ.

\end{greek}%
\switchcolumn*

Nein, sondern . . .

\switchcolumn

\begin{greek}[variant=ancient]%
οὔκ· ἀλλά . . .

\end{greek}%
\switchcolumn*

Nicht doch! 

\switchcolumn

\begin{greek}[variant=ancient]%
μὴ δῆτα!

\end{greek}%
\switchcolumn*

Thu's nicht! 

\switchcolumn

\begin{greek}[variant=ancient]%
μή νυν ποιήσῃς!

\end{greek}%
\switchcolumn*

Noch nicht! 

\switchcolumn

\begin{greek}[variant=ancient]%
μὴ δῆτά πώ γε.

\end{greek}%
\switchcolumn*

Nicht eher, als bis (dies geschieht) 

\switchcolumn

\begin{greek}[variant=ancient]%
οὔκ, ἢν μὴ (τοῦτο γένηται\footnote{\begin{latin}%
\textgreek[variant=ancient]{ὁ τυπογράφος ἔγραψα τὸν οὐ γεγραμμένον
τόνον.}\end{latin}%
}).

\end{greek}%
\switchcolumn*

Ja nicht! 

\switchcolumn

\begin{greek}[variant=ancient]%
μηδαμῶς!

\end{greek}%
\switchcolumn*

Ist nicht nöthig! 

\switchcolumn

\begin{greek}[variant=ancient]%
οὐδὲν δεῖ!

\end{greek}%
\switchcolumn*

Freilich nicht. 

\switchcolumn

\begin{greek}[variant=ancient]%
μὰ Δί' οὐ μέντοι.

\end{greek}%
\switchcolumn*

(Ich) leider nicht! 

\switchcolumn

\begin{greek}[variant=ancient]%
εἰ γὰρ ὤφελ(ον)!

\end{greek}%
\switchcolumn*

Du bist gescheit! (ironisch ablehnend.) 

\switchcolumn

\begin{greek}[variant=ancient]%
σωφρονεῖς! --- δεξιὸς εἶ!

\end{greek}%
\switchcolumn*

Kein Gedanke! 

\switchcolumn

\begin{greek}[variant=ancient]%
ἥκιστα!

\end{greek}%
\switchcolumn*

Am allerwenigsten! 

\switchcolumn

\begin{greek}[variant=ancient]%
ἥκιστά γε!

\end{greek}%
\switchcolumn*

Um keinen Preis! 

\switchcolumn

\begin{greek}[variant=ancient]%
ἥκιστα πάντων!

\end{greek}%
\switchcolumn*

Nein, und wenn Ihr Euch auf den Kopf stellt! 

\switchcolumn

\begin{greek}[variant=ancient]%
οὐκ ἂν μὰ Δία, εἰ κρέμαισθέ γε ὑμεῖς!

\end{greek}%
\switchcolumn*

Denken Sie, ich sei verrückt? 

\switchcolumn

\begin{greek}[variant=ancient]%
μελαγχολᾶν μ' οὕτως οἴκει;

\end{greek}%
\switchcolumn*

So steht die Sache nicht! 

\switchcolumn

\begin{greek}[variant=ancient]%
οὐχ οὗτος ὁ τρόπος!

\end{greek}%
\switchcolumn*

Wenn zehnmal! 

\switchcolumn

\begin{greek}[variant=ancient]%
ἀλλ' ὅμως!

\end{greek}%
\switchcolumn*

Sie haben \emph{nicht Recht!} 

\switchcolumn

\begin{greek}[variant=ancient]%
\emph{οὐκ ὀρθῶς} λέγεις.

\end{greek}%
\switchcolumn*

Ach was! (Blech!) 

\switchcolumn

\begin{greek}[variant=ancient]%
λῆρος!

\end{greek}%
\switchcolumn*

Das ist Unsinn! 

\switchcolumn

\begin{greek}[variant=ancient]%
οὐδὲν λέγεις!

\end{greek}%
\switchcolumn*

Aber das ist was ganz Anderes! 

\switchcolumn

\begin{greek}[variant=ancient]%
ἀλλ' οὐ ταὐτόν!

\end{greek}%
\switchcolumn*

Aber das gehört ja gar nicht hierher, was Sie sagen! 

\switchcolumn

\begin{greek}[variant=ancient]%
ἀλλ' οὐκ εἶπας ὅμοιον!

\end{greek}%
