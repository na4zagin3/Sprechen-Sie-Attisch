% 1
\section*{Kleine Regeln und Beobachtungen\addcontentsline{toc}{section}{\emph{Kleine Regeln und Beobachtungen}}}

% 2
\setlist[enumerate]{parsep=0pt,itemsep=0pt,partopsep=0pt,topsep=0pt}
\begin{enumerate}[leftmargin=0pt,rightmargin=0pt,listparindent =1cm,labelindent=1cm,labelsep=1ex,labelwidth={*},itemindent={*},align=left]
% 3
\item Nichts erleichtert es so sehr, eine Sprache zu beherrschen, als wenn
man ihre \emph{Schwächen} erspäht. Erst wenn wir ermittelt haben,
was einer Sprache fehlt, verstehen wir recht, warum sie gerade diese
oder jene Wendung vorzieht, diese oder jene Verbindung von Begriffen
liebt, warum sie in dieser oder jener Weise von der Aus\textcompwordmark{}drucks\textcompwordmark{}weise
unserer eigenen Sprache abweicht. Wir erfassen als\textcompwordmark{}dann
ein gutes Theil von ihrem \quotedblbase Geiste``, wie man den Inbegriff
ihrer Besonderheiten so gern nennt.

% 4
Eine bemerkens\textcompwordmark{}werthe Schwäche der griechischen
Sprache nun ist es, daß ihr bei allem Formenreichthum doch ein bequem
zu verwendendes \emph{\,Passivum fehlt}. Die Übereinstimmung eines
großen Theiles der passiven Formen mit den medialen erschwert ihre
Anwendung, weil Deutlichkeit das erste Gesets der Sprache ist, und
vielen Zeitwörtern fehlen überdies die allein dem Passivum eigenen
Formen.

% 5
Um die eigenthümliche Färbung der griechischen Sprache nachzuahmen,
hat man daher zu allererst Folgendes zu beachten:

% 6
\emph{Man meide thunlichst die den medialen gleichlautenden passiven
Formen und achte darauf, wie der Grieche diese zu ersetzen pflegt.}

% 7
Nur die durch den Zusammenhang sofort als solche erkennbaren und
gewisse in häufigen Gebrauch gekommene Passiva der bezeichneten Art
sind unbedenklich anzuwenden.

% 8
Umschreibungen des Passivums geschehen.
\begin{enumerate}
\item durch active Verba, z.\,B.

% 9
\begin{quote}
belehrt werden \textgreek[variant=ancient]{μανθάνειν,}\\
gerühmt werden \textgreek[variant=ancient]{εὐδοκιμεῖν,}\\
geplagt werden \textgreek[variant=ancient]{κάμνειν,}\\
vor Gericht gestellt werden \textgreek[variant=ancient]{εἰσιέναι εἰς
δικαστήριον,}\\
verklagt werden \textgreek[variant=ancient]{φεύγειν,}\\
gehalten werden für \ldots{} \textgreek[variant=ancient]{δοκεῖν,}\\
es wird mir etwas zugefügt \textgreek[variant=ancient]{πάσχω τι,}\\
vertrieben werden \textgreek[variant=ancient]{ἐκπίπτειν,}\\
einer Sache beraubt werden \textgreek[variant=ancient]{ἀπολλύναι τι},\\
getödtet werden \textgreek[variant=ancient]{ἀποθνήσκειν},\\
sie wurden vertrieben \textgreek[variant=ancient]{ἀνέστησαν},\\
es wurde mir geantwortet \textgreek[variant=ancient]{ἤκουσα},\\
es wird mir Gutes erwiesen \textgreek[variant=ancient]{εὖ πάσχω},\\
ich ward durch's Loos gewählt \textgreek[variant=ancient]{ἔλαχον},\\
ich ward freigesprochen \textgreek[variant=ancient]{ἀπέφυγον},\\
ich ward geschmäht \textgreek[variant=ancient]{κακῶς ἤκουσα},\\
ich ward (von Mitleid) ergriffen \textgreek[variant=ancient]{(ἔλεός)
με εἰσῄει}.
\end{quote}
\item vielfach durch \textgreek[variant=ancient]{γέγνεσθαι;} es steht für
gemacht, veranstaltet, bewerkstelligt werden, übertragen, verliehen,
erkauft, erworben werden, verübt w., gefeiert w. (von Festen), geboren
w. und andere Passiva.
\item durch Substantiva mit Verben, z.\,B.

% 10
\begin{quote}
gelobt werden \textgreek[variant=ancient]{ἔπαινον ἔχειν,}\\
es wird (viel) gesprochen \textgreek[variant=ancient]{λόγος ἐστὶ (πολύς),}\\
bestraft werden \textgreek[variant=ancient]{δίκην διδόναι,}\\
es wird gezürnt u. \textgreek[variant=ancient]{ὀργὴ γίγνεται} dgl.\ mehr;
\end{quote}
\item durch Adjektiva mit \textgreek[variant=ancient]{εἶναι,} z.\,B.

% 11
\begin{quote}
gesehen werden \textgreek[variant=ancient]{καταφανῆ εἰναι},\\
es wird dir nicht geglaubt \textgreek[variant=ancient]{ἄπιστος εἶ}
u. dgl. mehr.
\end{quote}
\end{enumerate}
% 12
\item Im Griechischen fehlt die Genauigkeit in der Bezeichnung des Objectes,
wie sie den modernen Sprachen eigen ist. Die letzteren setzen, wenn
zwei verbundene Verba das gleiche Object in verschiedenem Casus erfordern,
zum zweiten Verbum anstatt der Wiederholung des Nomens das persönliche
Pronomen (seiner, ihm, ihn, ihrer, ihr, sie, es, ihnen) als Object,
\emph{der Grieche läßt die Stelle des gemeinsamen Objectes beim zweiten
Verbum unbezeichnet, gleichviel in welchem Casus es stehen müßte.}

% 13
Das dem französischen \textfrench{en} entsprechende Object (welchen,
welche, welches) wird im Griechischen nicht aus\textcompwordmark{}gedrückt,
z.\,B.: Sie werden das Gold aus Lydien holen lassen müssen, wenn
sie welches haben wollen \textgreek[variant=ancient]{ἐκ Λυδίας μεταστέλλεσθαι
τὸ χρυσίον δεήσει αὐτοὺς, ἢν ἐπιθυμήσωσιν.}

% 14
\item Dem Griechen fehlt, wie dem Lateiner, das Mittel zur Hervorhebung
einzelner Satztheile, welches unsere Sprache, ähnlich anderen modernen
Sprachen, darin besitzt, daß sie den hervorzuhebenden Begriff zum
Prädivcte eines neuen Satzes meist mit dem unpersönlichen Subject
es macht, während die übrigen Satztheile in einem abhängigen Satze
vermittelst eines Relativs oder einer Conjunction angefügt werden.
\emph{Im Griechischen muß die der Hervorhebung eines Begriffes dienende
Zerlegung eines Satzes in zwei unterbleiben,} z.\,B.:  Es ist derselbe,
der dies sagt \textgreek[variant=ancient]{ὁ αὐτὸς ταῦτα λέγει}. Wer
ist der Mann, den du rufst? \textgreek[variant=ancient]{τίνα τὸν ἄνδρα
καλεῖς;} Ist es wahr, daß du das gethan hast? \textgreek[variant=ancient]{ἆρ᾽
ἀληθῶς τοῦτ᾽ ἐποίησας;} Wie ist es möglich, daß\ldots{} \textgreek[variant=ancient]{πῶς\ldots{};}
wie kommt es, daß\ldots{} \textgreek[variant=ancient]{πῶς\ldots{}; }
\item Coordinirte Sätze und coordinirte Satztheile kann der Grieche nicht
unverbunden lassen. Asyndetisches Nebeneinanderstellen von Satztheilen
kommt nur selten und zwar als Aus\textcompwordmark{}druck lebhafter
Erregung zur Anwendung.


% 15
In ununterbrochener Rede ist \emph{jeder neue Satz} durch eine passende
Conjunction (\textgreek[variant=ancient]{δέ, καί οὖν, γάρ} etc.) an
das Voraus\textcompwordmark{}gehende \emph{anzuschließen.}


% 16
Der Lernende ist davor zu warnen, \textgreek[variant=ancient]{μέν}
für eine diese Verbin- dung mit dem \emph{Voraus\textcompwordmark{}gehenden}
ersetzende Conjunction zu halten, da es nur zum Hinweis auf das \emph{Folgende}
dient.


% 17
Anfügung ohne Bindewort ist in ununterbrochener Rede nur gestattet:
\begin{enumerate}
\item an den Stellen, wo wir im Deutschen den Doppelpunkt als Interpunctions\textcompwordmark{}zeichen
setzen;
% 18
\item wenn der neue Satz mit stark betontem Demonstrativum oder
% 19
\item wenn der neue Satz mit \textgreek[variant=ancient]{εἶτα} (= und dann)
oder \textgreek[variant=ancient]{ἔπειτα} beginnt;
% 20
\item wo wir im Deutschen mit \emph{\quotedblbase nicht aber``} fortsahren;
es steht dann häufig bloßes \textgreek[variant=ancient]{οὐ} (beziehentlich
\textgreek[variant=ancient]{μή}\,), (weil \textgreek[variant=ancient]{οὐ}
mit \textgreek[variant=ancient]{δέ} \quotedblbase und nicht`` oder
\quotedblbase nicht einmal`` bedeutet), oft jedoch auch \textgreek[variant=ancient]{οὐ
μέντοι.} 
\end{enumerate}
% 21
\item Man merke: Nun so r denn = \textgreek[variant=ancient]{ἀλλά,}


\bgroup\addtolength{\leftskip}{\parindent}\addtolength{\leftskip}{3em}\setlength{\parindent}{-1em}\setlength{\arraycolsep}{0pt}o
dann ... = \textgreek[variant=ancient]{ἄρα,}


da kam, da sagte = \textgreek[variant=ancient]{καὶ ἦλθε, καὶ εἰπεν,}


jedoch = \textgreek[variant=ancient]{μέντοι,}


denn sonst \ldots{} =\textgreek[variant=ancient]{γάρ,}


denn (folgernd), z.B. höre \emph{denn, so} ward er \emph{denn} ..
= \textgreek[variant=ancient]{δή,}


doch wohl (ohne Zweifel) = \textgreek[variant=ancient]{δήπου,}


und schon = \textgreek[variant=ancient]{καὶ δή} (\textgreek[variant=ancient]{δή}
= \textgreek[variant=ancient]{ἤδη}), vgl. \textgreek[variant=ancient]{πάλαι
δή} schon längst, \textgreek[variant=ancient]{νῦν δή} jetzt eben,


wohl aber = \textgreek[variant=ancient]{δὲ,}


\begin{tabular}{lc}
dann erst & \ldelim\}{2}{1em}[]\tabularnewline
erst dann & \tabularnewline
\end{tabular} =\footnote{Ich setzte das Gleichheitszeichen.} \textgreek[variant=ancient]{οὕτω
δή,}


\ldots{} allerdings = \textgreek[variant=ancient]{\ldots{} μήν,}


indessen \ldots{} = \textgreek[variant=ancient]{οὐ μὴν ἀλλά,}


wahrscheinlich (adv.) =\footnote{Ich setzte das Gleichheitszeichen.}
\textgreek[variant=ancient]{ἦ που \ldots{}}


oder (nach Negationen) = \textgreek[variant=ancient]{οὐδέ, μνδέ,}


doch (lat. \textlatin{quaeso}) = \textgreek[variant=ancient]{δῆτα,}


nicht sowohl \textgreek[variant=ancient]{\ldots{}} als vielmehr =
\footnote{Ich habe das geschwungene Klammer gespiegelt.}%
% 22
\begin{tabular}{ll}
\ldelim\{{2}{1em}[] & \begin{greek}[variant=ancient]%
οὐ τοσοῦτον ὅσον \ldots{}\end{greek}%
\tabularnewline
 & \begin{greek}[variant=ancient]%
οὐ τὸ πλέον \ldots{} ἀλλὰ \ldots{}\end{greek}%
\tabularnewline
\end{tabular}


\egroup Aus der Thatsache, daß \quotedblbase o dann ...`` sich
überall passend durch \textgreek[variant=ancient]{ἄρα} geben läßt,
folgt noch keines\textcompwordmark{}wegs, daß umgekehrt \textgreek[variant=ancient]{ἄρα}
sich überall passend durch \quotedblbase o dann \ldots{}`` übersetzen
lasse.

% 23
\item \emph{Großes} Glück \textgreek[variant=ancient]{πολλὴ εὐδαιμονία.}


\begin{continuousitemline}


Großes Mißgeschick \textgreek[variant=ancient]{πολλὴ δυστυχία.}


Großer Überfluß \textgreek[variant=ancient]{πολλὴ ἀφθονία.}


Große Thorheit \textgreek[variant=ancient]{πολλὴ μωρία.}


Große Unwissenheit \textgreek[variant=ancient]{πολλὴ ἀμαθία.}


Große Unvernunft \textgreek[variant=ancient]{πολλὴ ἀλογία.}


Große Geschäftigkeit \textgreek[variant=ancient]{πολλὴ πραγματεία.}


Sehr große Muthlosigkeit \textgreek[variant=ancient]{πλείστη ἀθυμία.}\par\end{continuousitemline}

% 24
\item %
\begin{tabular}[t]{lc}
So ein trefflicher & \rdelim\}{5}{1em}[{\textgreek[variant=ancient]{τοιοῦτος.}}]\tabularnewline
So ein abscheulicher & \tabularnewline
So ein erfahrener & \tabularnewline
So ein beschränkter & \tabularnewline
So ein gefährlicher & \tabularnewline
\end{tabular}


% 25
\begin{continuousitemline}\qquad{}u. s. w.


\begin{tabular}[t]{lc}
So ein trefflicher & \rdelim\}{5}{1em}[{ \textgreek[variant=ancient]{τοιοῦτος.}}]\tabularnewline
So ein abscheulicher & \tabularnewline
So ein erfahrener & \tabularnewline
So ein beschränkter & \tabularnewline
So ein gefährlicher & \tabularnewline
\end{tabular}


% 26
\qquad{}u. s. w.


\begin{tabular}[t]{lc}
So Verwerfliches & \rdelim\}{2}{1em}[{ \textgreek[variant=ancient]{τοιαῦτα.}}]\tabularnewline
So Löbliches & \tabularnewline
\end{tabular}

% 27

\qquad{}u. s. w.


\begin{tabular}[t]{lc}
es klingt schön & \rdelim\}{3}{1em}[{ \textgreek[variant=ancient]{ἡδύ ἐστιν.}}]\tabularnewline
es schmeckt gut & \tabularnewline
es riecht gut & \tabularnewline
\end{tabular}


\begin{tabular}[t]{lc}
(jetzt) so spät & \rdelim\}{2}{1em}[{ \textgreek[variant=ancient]{τηνικάδε.}}]\tabularnewline
(jetzt) so früh & \tabularnewline
\end{tabular}


% 28
\quad{}Der gewöhnliche Aus\textcompwordmark{}druck für


\begin{tabular}[t]{lc}
hoffen & \rdelim\}{2}{1em}[\textgerman{ ist \textgreek[variant=ancient]{οἴεσθαι},}]\tabularnewline
fürchten & \tabularnewline
\end{tabular}


% 29
\begin{tabular}[t]{lc}
versprechen & \rdelim\}{4}{1em}[\textgerman{ ist \textgreek[variant=ancient]{φάναι.}}]\tabularnewline
brohen & \tabularnewline
antworten & \tabularnewline
erwidern & \tabularnewline
\end{tabular}


% 30
\ldots{} fuhr er fort, = \textgreek[variant=ancient]{ἔφη.}\par\end{continuousitemline}

\item Ein Freund \textgreek[variant=ancient]{φίλος \emph{τις.}}


% 31
\begin{continuousitemline}Ein redlicher Freund \textgreek[variant=ancient]{χρηστός
τις \emph{ἄνθρωπος} φίλος.}\par\end{continuousitemline}

% 32
\item Unsere 500 Schüler \textgreek[variant=ancient]{\emph{οἱ} ἡμέτεροι
πεντακόσιοι μαθηταί.}


% 33
\begin{continuousitemline}Meine drei besten Schüler \textgreek[variant=ancient]{\emph{οἱ}
τρεῖς ἄριστοι τῶν μαθητῶν μου.}\par\end{continuousitemline}

% 34
\item Ich verlange kein Geld, sondern Zuneigung (Liebe) \textgreek[variant=ancient]{αἰτῶ
\emph{οὐκ} ἀργύριον, ἀλλ᾽ εὔνοιαν.}
\item Ich \emph{habe} gehabt \textgreek[variant=ancient]{εἶχον,} z. B.
ich habe ebenfalls diese Klasse einmal gehabt \textgreek[variant=ancient]{κἀγὼ
εἶχον τὴν τάξιν ταύτην ποτέ.} Er \emph{ist} gestern bei mir gewesen
\textgreek[variant=ancient]{παρ᾽ ἐμοὶ χθὲς ἦν.}


% 35
Das \emph{Perfectum} von \emph{sein} und \emph{haben} und allen ein
Dauer ausdrückenden Verben wird im Griechischen durch das Imperfectum,
bei den übrigen Verben meist durch den Aorist, seltener durch das
Perfectum wiedergegeben. Läßt sich zu dem Verbum ein Adverb der Vergangenheit
(z. B. damals) hinzudenken, so steht Aorist; läßt sich ein Adverb
der Gegenwart (z. B. nunmehr, bereits) hinzudenken, nur dann steht
Perfectum.


% 36
\begin{continuousexamples}\emph{Hast} du das Geld gefunden? (\textlatin{sc.}
nunmehr) \textgreek[variant=ancient]{ἆρ᾽ εὕρηκας τἀργύριον;}


Ja, ich \emph{habe} es gefunden (\textlatin{sc.} nunmehr) \textgreek[variant=ancient]{εὕρηκανὴ
Δία.}


Wo \emph{hast} du es gefunden? (\textlatin{sc.} damals als du es fandest)
\textgreek[variant=ancient]{ποῦ εὗρες;}


Ich \emph{habe} es (\textlatin{sc.} damals) in dem Garden gefunden
\textgreek[variant=ancient]{ἐν τῷ κήπῳ εὗρον. }\par\end{continuousexamples}

% 37
\item Der Infinitiv Aoristi bezeichnet nach den Verben des Sagens und Meinens
die Vergangenheit, z. B.


% 38
\begin{continuousexamples}\textgreek[variant=ancient]{φησὶν εὑρεῖν}
er behauptet er \emph{habe} gefunden.\par\end{continuousexamples}

\item Bedeutet \emph{daß} soviel wie \emph{mache(t)} daß, so wird es durch
\textgreek[variant=ancient]{ὅπως}\footnote{\begin{latin}%
orig. \textgreek[variant=ancient]{οπως}\end{latin}%
} mit dem \textlatin{Indic. Fut.} ausgedrückt.


% 39
\begin{continuousexamples}Daß es nur kein Mensch erfährt! \textgreek[variant=ancient]{ὅπως
ταῦτα μηδεὶς ἀνθρώπων πεύσεται!}\par\end{continuousexamples}

\item Mit \textgreek[variant=ancient]{ἐξ οὗ} oder \textgreek[variant=ancient]{ἐπεί}
= \emph{seit} verträgt sich kein \textgreek[variant=ancient]{οὐ} oder
\textgreek[variant=ancient]{μή}\footnote{\begin{latin}%
orig. \textgreek[variant=ancient]{μη}\end{latin}%
}\textgreek[variant=ancient]{.}


% 40
\begin{continuousexamples}Seit wir uns \emph{nicht} gesehen, hat
es viel geregnet: \textgreek[variant=ancient]{ἐξ οὗ }oder\textgreek[variant=ancient]{
ἐπεὶ εἴδομεν ἀλλήλους ὕδωρ ἀγένετο πολύ.}\par\end{continuousexamples}

% 41
\item Wo sich statt \emph{sein} denken läßt \emph{gehen,} wird \textgreek[variant=ancient]{παρεῖναι
εἰς} angewandt.


% 42
\begin{continuousexamples}Sind Sie oft im Theater gewesen? \textgreek[variant=ancient]{ἦ
πολλάκις παρῆσθα εἰς τὸ θέατρον;}\par\end{continuousexamples}

% 43
\item Indefinita werden nach Negationen gern negativ, \textgreek[variant=ancient]{πω}
jedoch bleibt unverändert.
\item Ja = doch (franz. \textfrench{si!}) dem Unglauben oder mangelhaften
Glauben versichernd: \textgreek[variant=ancient]{ναί! }
\item \emph{Zu, allzu} bleibt meist unübersetzt; z. B. Wir sind zu wenige
\textgreek[variant=ancient]{ὀλίγοι ἐσμέν,} du hast zu menig geschrieben
\textgreek[variant=ancient]{ὀλίγον ἔγραψας.} \textgreek[variant=ancient]{Τὸ
ὕδωρ ψυχρὸν ὥστε λούσασθαί ἐστιν} (zu kalt). \textgreek[variant=ancient]{Νέοι
ἔτι ἐσμὲν ὥστε τοῦτ᾽ εἰδέναι} (zu jung, als daß wir wissen könnten).


% 44
\emph{Nicht genug} \textgreek[variant=ancient]{ὀλίγος.} Er hat nicht
genug zu leben \textgreek[variant=ancient]{βίον ἔχει ὀλίγον.} Ich
habe nicht genug Geld \textgreek[variant=ancient]{ἀργύριον ἔχω ὀλίγον.}


% 45
\emph{Genug} = ausreichend wird adjectivisch meist durch \textgreek[variant=ancient]{ἱκανός}
ausgedrückt. Geld genug \textgreek[variant=ancient]{ἱκανὸν ἀργύριον.}
Ich denke, zwanzig Schüler sind genug \textgreek[variant=ancient]{ἱκανοὺς
νομίζω μαθηδὰς εἴκοσιν.}


\emph{Genug} = in Menge \textgreek[variant=ancient]{οὐκ ὀλίγος.}

% 46
\item Ein anderer = noch ein weiterer \textgreek[variant=ancient]{ἕτερος;
}ein anderer = irgend welcher andere \textgreek[variant=ancient]{ἄλλος.}


Ich war dort und viele andere \textgreek[variant=ancient]{ἐγὼ παρεγενόμην
καὶ ἕτεροι πολλοί.} Nun, es giebt ja andere gute Bücher genug \textgreek[variant=ancient]{ἀλλ᾽
ἔστιν ἔτερα νὴ Δία χρηστά βιβλία οὐκ ὀλίγα.}


% 47
\begin{continuousitemline}Keine andere Sache \textgreek[variant=ancient]{οὐκ
ἄλλο πρᾶγμα.}


Wer sonst? \textgreek[variant=ancient]{τίς ἄλλος;}\par\end{continuousitemline}

\item Immer noch = \textgreek[variant=ancient]{ἔτι καὶ νῦν,}


% 48
\begin{continuousitemline}noch welches \textgreek[variant=ancient]{ἄλλο,}


noch einige \textgreek[variant=ancient]{ἄλλοι,}


noch irgend einer \textgreek[variant=ancient]{ἄλλος τις.}


Hat er noch (sonstiges) Geld? \textgreek[variant=ancient]{ἆρ᾽ ἔχει
ἀργύριον ἄλλο;}


Er hat \emph{welches} \textgreek[variant=ancient]{ἔχει.}\par\end{continuousitemline}

\item Ihr \emph{beiden} alten Herren \textgreek[variant=ancient]{ὦ \emph{δύο}
πρεσβύτα.}


% 49
\begin{continuousitemline}\emph{Diese beiden} alten Herren hier \textgreek[variant=ancient]{τὼ
πρεσβύτα τώδε.}


\emph{Diese beiden} \textgreek[variant=ancient]{τώδε (ἄμφω).}\par\end{continuousitemline}


% 50
\textgreek[variant=ancient]{ἄμφω} verlangt stets den \emph{Dual}
des beigefügten Substantivs, \textgreek[variant=ancient]{ἀμφότερος}
steht meist mit seinem Substantiv im \emph{Plural.}

\item allein (= allein für sich) \textgreek[variant=ancient]{αὐτός,}


% 51
\begin{continuousitemline}allein (= der einzige) \textgreek[variant=ancient]{μόνος.}


Wir sind allein (unter uns) \textgreek[variant=ancient]{αὐτοί ἐσμεν.}


Wir sind die einzigen \textgreek[variant=ancient]{μόνοι ἐσμέν.}\par\end{continuousitemline}


Ich habe die (schriftliche) Arbeit allein gemacht \textgreek[variant=ancient]{αὐτὸς
ἐγὼ ταῦτα ἔγραψα.} Dagegen \textgreek[variant=ancient]{μόνος ἐγὼ ταῦτα
ἔγραψα} ich bin der Einzige, der diese Arbeit gemacht hat.

% 52
\item Ich habe mehr \emph{von diesen} (z. B. Söhne) wie von jenen (Töchter)
\textgreek[variant=ancient]{πλείους ἔχω τούτους ἢ ἐκείνας} (doch auch
\textgreek[variant=ancient]{ἐκείνους ἢ ταύτας}).
\item Wollen = Lust haben, sich entschließen \textgreek[variant=ancient]{ἐθέλειν.}


% 53
\begin{continuousitemline}Wollen = wünschen \textgreek[variant=ancient]{βούλεσθαι.}


Er hat keine Lust \textgreek[variant=ancient]{οὐκ ἐθέλει.}


(Sehnlich) wünschen \textgreek[variant=ancient]{ἐπιθυμεῖν. }


Wollen = darüber sein \textgreek[variant=ancient]{μέλλειν.}\par\end{continuousitemline}


Wohin eilen sie? Ich will einen Brief zum Briefkasten tragen \textgreek[variant=ancient]{ποῖ
θεῖς; ἐπιστολὴν μέλλω φέρειν εἰς τὸ κιβώτιον (γραμματοκιβώτιον). Ich
will gehen εἶμι oder βαδιοῦμαι.} 


% 54
\begin{continuousitemline}Ich will gehen \textgreek[variant=ancient]{εἶμι}
oder \textgreek[variant=ancient]{βαδιοῦμαι.}\par\end{continuousitemline}

% 55
\item Wo ist dein Bruder? \textgreek[variant=ancient]{ποῦ ᾽σθ᾽ ὁ σὸς ἀδελφός;}
\item Bei = franz. \textfrench{chez} \textgreek[variant=ancient]{παρά} mit
\textlatin{Dat.}


% 56
\begin{continuousitemline}Zu = franz. \textfrench{chez} \textgreek[variant=ancient]{παρά}
mit \textlatin{Acc.}\par\end{continuousitemline}

\item %
\begin{tabular}[t]{ccccc}
Mitnehmen, & ~ & mitbringen & ~ & (von Sachen) \textgreek[variant=ancient]{φέρειν,}\tabularnewline
,, &  & ,, &  & (von Personen) \textgreek[variant=ancient]{ἄγειν.}\tabularnewline
\end{tabular}


% 57
\begin{continuousitemline}Ich will das Buch mitbringen \textgreek[variant=ancient]{οἴσω
τὸ βιβλίον.}


Ich will dich mit (zu ihm) nehmen \textgreek[variant=ancient]{ἄξω
σε παρ᾽ αυτόν.}\par\end{continuousitemline}

% 58
\item Ich gehe (hin) \textgreek[variant=ancient]{βαδίζω,}


\begin{continuousitemline}ich komme (her) \textgreek[variant=ancient]{ἔρχομαι,}


ich bin hergegangen \textgreek[variant=ancient]{ἐλήλυθα,}


ich bin gekommen \textgreek[variant=ancient]{ἥκω,}


ich bin wieder da \textgreek[variant=ancient]{ἥκω,}


bis ich wieder da bin \textgreek[variant=ancient]{μέχρι ἃν ἥκω,}


ich gehe \emph{(weiter)} \textgreek[variant=ancient]{χωρῶ,}


ich will ihn \emph{besuchen} \textgreek[variant=ancient]{εἶμι (εἴσειμι)
ὡς αὐτόν,}


ich werde kommen \textgreek[variant=ancient]{ἥξω.}


Ich will gehen, \emph{um} ihn zu befragen \textgreek[variant=ancient]{εἶμι
ἐρωτήσων αὐτόν.}


Ich komme her, \emph{um} mit\textcompwordmark{}zuspeisen \textgreek[variant=ancient]{ἔρχομαι
δειπνήσων.}


\emph{aus}gehen \textgreek[variant=ancient]{θύραζε ἐξιέναι} oder
\textgreek[variant=ancient]{θ. βαδίζειν.}\par\end{continuousitemline}

% 60
\item Die \emph{guten} Schüler \textgreek[variant=ancient]{οἱ ἀγαθοὶ τῶν
μαθητῶν.}


% 61
\begin{continuousitemline}Die guten \emph{Schüler} \textgreek[variant=ancient]{οἱ
ἀγαθοὶ μαθηταί.}\par\end{continuousitemline}

\item \emph{Da} kommt der junge Mann herbai! \textgreek[variant=ancient]{τὸ
μειράκιον \emph{τοδὶ} (τόδε) προσέρχεται!} 
\item Ich habe \emph{nichts zu} essen \textgreek[variant=ancient]{οὐκ ἔχω
καταφαγεῖν.}
\item \emph{hier,} den Ort des \emph{Sprechenden} bezeichnend, haißt \textgreek[variant=ancient]{ἐνθάδε,}


% 62
\begin{continuousitemline}hier (dem Ort des Sprechenden \emph{nahe})
\textgreek[variant=ancient]{ἐνταῦθα,}


hier (= \emph{an Ort und Stelle,} am Orte selbst) \textgreek[variant=ancient]{αὐτοῦ.}\par\end{continuousitemline}

% 63
\item Jemanden kennen \textgreek[variant=ancient]{γιγνώσκειν τινά.}
\item Zwar nicht groß, aber schön \textgreek[variant=ancient]{μέγας μὲν
οὔ, καλὸς δέ.}
\item Er hat eine breite Stirn \textgreek[variant=ancient]{πλατὺ ἔχει \emph{τὸ}
μέτωπον.}


Sie hat allerliebste Hände \textgreek[variant=ancient]{\emph{τὰς}
χεῖρας ἔχει παγκάλας.}

% 64
\item Beabsichtigen, gedenken \textgreek[variant=ancient]{ἐπινο\emph{εῖν}}
oder \textgreek[variant=ancient]{διανο\emph{εῖσθαι.}}
\item Ich lerne die Gedichte Homers \emph{auswendig} \textgreek[variant=ancient]{μανθάνω
τὰ Ὁμήρου ἔπη.}


\begin{continuousitemline}Ich \emph{kann} die Ilias \emph{auswendig}
\textgreek[variant=ancient]{ἐπίσταμαι Ἰλιάδα.}


Ich könnte die Odyssee \emph{auswendig hersagen} \textgreek[variant=ancient]{δυναίμην
ἂν Ὀδύσσειαν ἀπὸ στόματος εἰπεῖν. }\par\end{continuousitemline}

% 65
\item Mein Vater hat mich gezwungen, die Odyssee auswendig zu lernen \textgreek[variant=ancient]{ὁ
πατὴρ ἠνάγκασέ με Ὀδύσσειαν \emph{μαθεῖν}} = that\textcompwordmark{}sächlich
mit dem Lernen zu Stande zu kommen; \textgreek[variant=ancient]{ἠνάγκασέ
με \emph{μανθάνειν}} bedeutet nur: er zwang mich, mit dem Lernen mich
zu beschäftigen, zu befassen, zu bemühen. 
\item \textgreek[variant=ancient]{Εὖ λέγει }er hat Recht.


\begin{continuousitemline}\textgreek[variant=ancient]{καλῶς λέγει}
er spricht gut.\par\end{continuousitemline}

% 66
\item Ich habe mehr Geld als du, aber Karl hat \emph{das} meiste \textgreek[variant=ancient]{ἐγὼ
μὲν ἀργύριον ἔχω πλέον ἢ σύ, πλεῖστον δὲ Κάρολος.}
\item Der Mann, \emph{dessen} Brief du liest \textgreek[variant=ancient]{ὁ
ἀνήρ, οὖ ἀναγιγνώσκεις \emph{τὴν} ἐπιστολήν.}


\begin{continuousitemline}Wessen Brief liest du? \textgreek[variant=ancient]{\emph{τὴν}
τίνος ἐπιστολὴν ἀναγιγνώσκεις;}\par\end{continuousitemline}

\item Setzest du deinen Hut auf? \textgreek[variant=ancient]{ἦ περιτίθεσαι
\emph{τὸν} πῖλον;}


\begin{continuousitemline}Zieh deine Stiefel aus! \textgreek[variant=ancient]{ἀποδύου
\emph{τὰς} ἐμβάδας!}\par\end{continuousitemline}


Das Possessiv ist durch das Medium bereits ausgedrückt. 

% 67
\item Er wird dich von \emph{deinem} Augenleiden befreien \textgreek[variant=ancient]{ἀπαλλάξει
σε τῆς ὀφθαλμίας.}


\begin{continuousitemline}Ein einziger Tag hat mir \emph{meinen}
ganzen Wohlstand geraubt \textgreek[variant=ancient]{μία ἡμέρα με
\emph{τὸν} πάντα ὄλβον ἀφείλετο.}


Er hat mir \emph{mein} Geld gestahlen \textgreek[variant=ancient]{ὑπείλετό
μου τἀργύρια.}\par\end{continuousitemline}


Bei den Verben \emph{nehmen} und dergl. darf kein Possessiv übersetzt
werden, sobald die durch dasselbe bezeichnete Person bereits genannt
ist.

% 68
\item Brauchst du \emph{etwas?} \textgreek[variant=ancient]{δέει \emph{τίνος;}}


\begin{continuousitemline}Giebt es \emph{was} Neues? \textgreek[variant=ancient]{λέγεται
\emph{τί} καινόν;}\par\end{continuousitemline}

% 69
\item Woher kommst du? \textgreek[variant=ancient]{πόθεν ἥκεις;} Aus dem
Garten \textgreek[variant=ancient]{ἐκ τοῦ κήπου.} Aus welchem? \textgreek[variant=ancient]{ἐκ
\emph{τοῦ} ποίου;}


Wenn \textgreek[variant=ancient]{ποῖος} auf einen mit Artikel versehenen
Gattungsnamen \textlatin{(Substantivum appellativum)} oder einen ihn
vertretenden Satz zurückweist, so nimmt es den Artikel an. Weg bleibt
der Artikel in der Regel nur dann, wenn \textgreek[variant=ancient]{ποῖος}
Prädicat ist. 

% 70
\item \emph{Geld} in kleineren Summen \textgreek[variant=ancient]{ἀργύριον.}


\begin{continuousitemline}Geld = Kapitalien \textgreek[variant=ancient]{χρήματα.
}\par\end{continuousitemline}

\item \textgreek[variant=ancient]{τάχα} ent\textcompwordmark{}spricht
genau dem in unserer Volks\textcompwordmark{}sprache üblichen \emph{am
Ende} (= schließlich, möglicher Weise)


\textgreek[variant=ancient]{ταχύ, ταχέως} schnell, bald,


\textgreek[variant=ancient]{διὰ ταχέων} bald. 

% 71
\item \emph{Unter} = zwischen drin \textgreek[variant=ancient]{ἐν,} z. B.
\textgreek[variant=ancient]{ἐν τοῖς Χριστιανοῖς πολλοί εἰσιν Ἰουδαῖοι.
ἐν νέοις ἀνὴρ γέρων.}
\item \emph{Nicht sonderlich} \textgreek[variant=ancient]{οὐ πάνυ.} Er strengt
sich nicht sonderlich an \textgreek[variant=ancient]{οὐ πάνυ σπουδάζει.}
\item Die natürliche Stellung des Adverbs ist im Griechischen \emph{vor}
dem durch dasselbe zu bestimmenden Begriffe. Abweichung von dieser
Stellung dient zur Hervorhebung des Adverbs. Steht das Adverb mit
Nachdruck zuletzt, so ersetzt diese Stellung das deutsche \emph{und
zwar:} \textgreek[variant=ancient]{χάριν σωθέντες ὑπὸ σοῦ σοὶ ἂν ἔχοιμεν
δικαίως} (und zwar pflichtschuldigst). 
\item Indirecte Ausrufesätze werden in der lateinischen Grammatik den indirecten
Fragesätzten gleichgestellt; im Griechischen unterscheiden sie sich
aber von den indirecten Fragesätzen dadurch, daß diese letzteren mit
dem indirecten oder directen Frageworte beginnen, die Ausrufesätze
hingegen mit dem Relativum, und zwar mit dem \emph{einfachen} Relativum.
% 72
\item Der Deutsche fragt: \emph{Wohin} setzt er sich? der Grieche: \emph{Wo?}
Wohin wollen wir uns setzen? \textgreek[variant=ancient]{ποῦ καθιζησόμεθα;}
\item \emph{Alle Welt} \textfrench{(tout le monde)} heißt \textgreek[variant=ancient]{πάντες
ἄνθρωποι} (ohne Artikel).
\item \emph{Um zu} wird gern durch \textgreek[variant=ancient]{βουλόμενος}
aus|gedrückt.
\item Ich habe bekommen = \textgreek[variant=ancient]{ἔχω,} z. B. ich habe
von meinem Vater 10 Mk. bekommen, \textgreek[variant=ancient]{δέκα
μάρκας ἔχω παρὰ τοῦ πατρός.}
\item \emph{Lieber als} \ldots{} = eher als \ldots{} heißt \textgreek[variant=ancient]{μᾶλλον
ἢ }\ldots{}
\item \emph{Vorhin} heißt \textgreek[variant=ancient]{τότε.} 
\item \textgreek[variant=ancient]{μέν} steht anderen Bindewörtern voran,
also nicht \textgreek[variant=ancient]{πολλοὶ γὰρ μὲν }\ldots{}\textgreek[variant=ancient]{,}
sondern \textgreek[variant=ancient]{πολλοὶ μὲν γὰρ }\ldots{}\textgreek[variant=ancient]{,}
ebenso \textgreek[variant=ancient]{μέν γε, μὲν δή }\ldots{}\textgreek[variant=ancient]{,
μὲν οὖν }\ldots{}\textgreek[variant=ancient]{, μέντοι.}
% 73
\item Den bringlichen Imperativ, welchen wir durch \emph{so} (mach') \emph{doch}
ausdrücken, giebt der Grieche durch (das sehr oft und gern angewendete)
\textgreek[variant=ancient]{οὐ} mit Futurum, z. B. so schweig' doch!
\textgreek[variant=ancient]{οὐ σιγήσει;} Negation ist dabei \textgreek[variant=ancient]{μή,}
z. B. so mach' doch kein Gerede! \textgreek[variant=ancient]{οὐ μὴ
λαλήσεις; }so halte dich doch nicht auf! \textgreek[variant=ancient]{οὐ
μὴ διατρίψεις;}
% 74
\item Satzverbindungen wie folgende: \quotedblbase Wenn ich nach Dresden
komme und über die Brücke gehe, so sehe ich das Denkmal August des
Starken`` werden im Griechischen zerlegt in: \quotedblbase Wenn
ich nach Dresden komme, so sehe ich, wenn ich über die Brücke komme,
das Denkmal.`` Trotzdem gehen die \emph{beiden} Nebensätze dem Hauptsatze
voran. 
% 75
\item Der gewöhnliche Ausdruck für \emph{\quotedblbase ich bitte``} ist
\textgreek[variant=ancient]{πρὸς (τῶν) θεῶν,} wofür auch \textgreek[variant=ancient]{πρὸς
τοῦ Διός} u. Ähnliches eintritt. \textgreek[variant=ancient]{πρὸς
θεῶν} ist keines\textcompwordmark{}wegs, wie gewöhnlich angegeben
wird, \emph{\quotedblbase Versicherung} bei den Göttern``, sondern
\emph{Bitt}formel.
% 76
\item Es giebt nicht bloß, wie es nach den Grammatiken scheint, einen Irrealis
der Gegenwart und Irrealis der Vergangenheit (z. B. ich wäre (jetzt)
zufrieden, ich wäre (damals) zufrieden gewesen, wenn . . .), sondern
es muß auch einen Irrealis der \emph{Zukunft} geben. Ich sage z. B.:
\quotedblbase Wenn ich morgen in New-York wäre, würde ich mich an
dem Feste betheiligen,`` obgleich ich weiß, daß ich morgen unmöglich
dort sein kann. Diesen Irrealis der Zukunft drückt der Grieche im
Nebensatze durch \textgreek[variant=ancient]{εἰ} mit dem Optativ,
im regierenden Satze durch Optativ mit \textgreek[variant=ancient]{ἄν}
aus. 
\end{enumerate}
\emph{Anmerkung} In Beispielen, wie \textgreek[variant=ancient]{φαίη
δ᾽ ἂν ἡ θανοῦσα, εἰ φωνὴν λάβοι} steht also nicht der Optativ ungewöhnlich
für das Präteritum, sondern er bezeichnet regelrecht, wie in zahllosen
ähnlichen Fällen, den Irrealis der Zukunft: \quotedblbase wenn die
Verstorbene \emph{künftig einmal} wiederkäme, so würde sie es bestätigen.`` 
