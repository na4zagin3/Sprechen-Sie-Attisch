\section*{Vorbemerkungen\addcontentsline{toc}{section}{\emph{Vorbemerkungen} über die Bedeutung der attischen Umgangs\textcompwordmark{}sprache für das Erlernen des Griechischen}}

Griechisch gilt den Allermeisten für eine im Grunde unlernbare Sprache,
deren man nimmermehr so mächtig werden könne, wie einer neueren, die
man leidlich beherrscht. Vorliegendes Büchlein, das fröhlicher Ferienlaune
seinen Ursprung verdankt, möchte den Gegenbeweis führen, indem es
einem ersten Versuch macht, attische Umgangs\textcompwordmark{}sprache
in ihren gebräuchlichsten Wendungen zu lehren.

Wer die Umgangs\textcompwordmark{}sprache eines Volkes kennt, hat
den Schlüssel zum Verständniß seiner Schriftwerke gleich den Volks\textcompwordmark{}genossen
selbst.

Der attische Knabe brachte zur Lectüre griechischer Dichter, der attische
Bauer in sein Theater oder in die Volks\-ver\-samm\-lung nur die
Kenntniß der attischen Umgangs\textcompwordmark{}sprache in ihrer
einfachsten Form mit; \emph{sie} befähigte zum Verständniß sophokleïscher
Dramen und perikleïsche Reden. Die Sprache des Alltags\textcompwordmark{}lebens
lieferte diejenigen Analogien, welche zum Erfassen der höheren Erzeugnisse
in Rede und Schrift nothwendig waren.

Man hat oft behauptet, daß es erstaunlich wenig Worte und Wendungen
sind, mit denen der gemeine Mann in seiner Muttersprache aus\textcompwordmark{}kommt
und die ihn befähigen, auch das zu verstehen, was für ihn Neubildung
ist. Sollte es nicht möglich sein, dem Athener seinen verhältnißmäßig
kleinen Urvorrath abzulauschen, somit die Sprache in ihrem \emph{Kerne}
zu erfassen und diese Worte und Wendungen demjenigen, der Griechisch
wirklich lernen will, geläufig zu machen?

Aristophanes bietet für diesen Zweck in denjenigen Partien, wo er
den gemeinen Mann im volks\textcompwordmark{}thümlichen Verkehrs\textcompwordmark{}tone
reden läßt, sprachlichen Stoff genug, und auch in der übrigen Literatur
finden sich verstreut Stellen, welche für treue Nachahmungen der Sprache
des gemeinen Lebens gelten müssen. Die Aufgabe kann also nicht unlös\textcompwordmark{}bar
sein, wenn auch das vorliegende Schrift\textcompwordmark{}chen nur
erst einen kleinen Beitrag zu ihrer Lösung bringt.

Die Worte und Wendungen in den nachstehenden Gesprächen sind in der
Hauptsache der aristphanischen Sprache entnommen. Einiges mußte aus
der späteren Gräcität beigefügt werden. Die dem Neugriechischen entlehnten
Ergänzungen, welche zur Bezeichnung moderner Begriffe verwandt wurden,
sind durch {*} besonders kenntlich gemacht.

Auch wer nicht die Absicht hat, attisch conversiren zu lernen, wird
mit vielem Nutzen für sein Verständniß des Griechischen sich mit der
attischen Umgangs\textcompwordmark{}sprache beschäftigen. Denn während
man auf unseren Gymnasien im Lateinischen fast nur solche Schriften
liest, welche der höheren Kunst\textcompwordmark{}sprache angehören
--- man denke nur and Cicero und Tacitus --- und in welchen die Volks\textcompwordmark{}sprache
kaum hier und da er\textcompwordmark{}kennbar ist, werden wir im
Griechischen weit mehr auf die Sprache des gewöhnlichen Lebens hin\textcompwordmark{}gewiesen.
Im Griechischen lesen wir Gespräche bei den Dramatikern, Gespräche
bei Plato; die Stimme des gemeinsten Mannes, --- schon \emph{dies}
nöthigt sie, seiner Sprache nahe zu bleiben, und schon dies muß die
Kenntniß der Aus\textcompwordmark{}drucks\textcompwordmark{}weise
des täglichen Lebens im Griechischen nützlich machen zum feinfühligeren
Verständniß der Texte.

Zweitens aber ist die \emph{Färbung} der Sprache und die Stil\textcompwordmark{}gattung
eines Literatur\textcompwordmark{}werkes nur demjenigen recht erkennbar,
der ermessen kann, wie weit dessen Sprache sich \emph{abhebt} von
der Alltags\textcompwordmark{}sprache. Wer das Deutsche nur aus Schiller
gelernt hätte, dem würde das Verständniß ab\textcompwordmark{}gehen
für die Eigenart und die Höhe der Schiller'schen Diction. Erst wer
von der Sprache der \emph{Alltäglichkeit} aus an sie herantritt,
bringt den Maßstab für sie mit. Es wird im Griechischen nicht anders
sein.

Drittens zwingt ganz besonders die Beschäftigung mit der griechischen
Um\textcompwordmark{}gangs\textcompwordmark{}sprache zur \emph{Vergleichung}
des deutschen und griechischen Aus\textcompwordmark{}druckes und
fördert dadurch die Sicherheit und Natürlichkeit der Über\textcompwordmark{}setzungen
aus dem Griechischen, die auf der Leichtigkeit und Bereit\textcompwordmark{}schaft
der Wort\textcompwordmark{}ver\textcompwordmark{}gleichungen der
beruht. Was man den \emph{Geist} der Sprache nennt, das zeigt sich
am Auf\textcompwordmark{}fallendsten da, wo die Vergleichung der
Sprachen unter einander \emph{leicht} und \emph{nahe}liegend ist:
das ist auf dem Gebiete des Alltäglichen. Den jocosen Ton, der sich
von selbst ergiebt, sobald man die alltägliche Aus\textcompwordmark{}drucks\textcompwordmark{}weise
des modernen Lebens mit der Sprechweise der Alten in Vergleich stellt,
wird man als bei diesem Studium unvermeidlich um der Sache willen
mit in den Kauf nehmen.

Endlich aber sei darauf hingewiesen, daß nichts dem Erlernen des
Griechischen an unseren Gymnasien so viele \emph{Gegner} geschaffen,
als eben die Thatsache, daß Griechisch im Grunde für eine unlernbare
Sprache gilt. Was der belgische Professor Emil de Laveleye über die
von ihm beobachteten Ergebnisse des Gymnasial\textcompwordmark{}unter\textcompwordmark{}richtes
sagt: \quotedblbase \textfrench{résultat net et incontestable: on
sait peu le latin et point du tout le grec,}`` das, behaupten Viele,
trifft annähernd auch bei den deutschen Gymnasien zu. Erstaunlich
Wenige, die \quotedblbase Griechisch gelernt`` haben, wissen mit
einiger Bestimmt\textcompwordmark{}heit anzugeben, wie der Attiker
die einfachsten Begriffe, z.\,B. \quotedblbase Ich werde zu dir
kommen``, aus\textcompwordmark{}zudrücken pflegt. Wenn im Lateinischen
Jemand nicht sofort auf \quotedblbase \textlatin{veniam}`` käme,
würde man meinen, daß ihm die allerersten Anfangs\textcompwordmark{}gründe
mangeln, und wenn er nicht verstünde, \quotedblbase \textlatin{veniam}``
und \quotedblbase \textlatin{ibo}`` aus\textcompwordmark{}ein\textcompwordmark{}an\textcompwordmark{}der\textcompwordmark{}zu\textcompwordmark{}halten,
so würde man über Unzulänglichkeit des Unterrichtes mit vollem Rechte
Klage führen und glauben, daß solche Unsicherheit auch dem sicheren
Erfassen des \emph{Sinnes} lateinischer \emph{Schriftwerke Eintrag
}thun müsse. Aber  im Griechischen? Man mache den Versuch, und man
wird überraschend Wenige finden, die das im Gebrauche des Attikers
alltägliche \quotedblbase \textgreek[variant=ancient]{ἥξω παρὰ σέ}``
in Be\textcompwordmark{}reit\textcompwordmark{}schaft haben. Man
studirt im Griechischen eifrig die Sprach\emph{gesetze}, aber gar
wenig die \emph{Sprache}, und doch lernt man es nicht um der grammatischen
Schulung willen, --- für diese sorgt aus\textcompwordmark{}reichend
das Latein, --- sondern der Sprache wegen. Man setze einem jungen
Manne, der die Schule  mit dem Zeugniß der Reife im Griechischen
verlassen hat, ein Glas griechischen Weines vor: er wird schwerlich
im Stande sein, auf Griechisch mit nur einigermaßen passendem Worte
dafür zu danken, oder zu sagen, daß ihm der Wein gut schmeckt. Allerdings
ist solche Sprachfertigkeit nicht das Ziel und die Aufgabe des griechischen
Unterrichts im Gymnasium aber daß sie bei den langen und angestrengten
Studien nicht nebenbei mit abfällt und so völlig fern zu bleiben
scheint, läßt das Gefühl des Griechischkönnens nicht aufkommen. Der
\quotedblbase Reife`` ist sich gar wohl bewußt, daß es ihm unsägliche
Mühe macht, ganz einfache Gedanken in wirklich griechischen Wendungen
wiederzugeben. Das macht unzufrieden und trägt viel dazu bei, dem
Griechischen Gegner zu schaffen. Auch aus diesem Grunde soll mein
Büchlein zeigen, daß es leicht angeht, sich mit den Kenntnissen,
die das Gymnasium bietet, des Griechischen so zu bemächtigen, daß
man sich darin verständlich machen könnte. 

Die Hauptsache aber bleibt: die allergewöhnlichsten Wörter und Wendungen
in der Ver\textcompwordmark{}kehrs\textcompwordmark{}sprache des
täglichen Lebens sind der Urvorrath, der Krystallisations\textcompwordmark{}kern,
an den und um den sich die weiteren sprachlichen Bildungen angesetzt
und angeschlossen haben. Schon darum verdienen sie unsere Achtung.
\emph{Hier} gilt es, die Sprache zu fassen, für den, der sie wirklich
lernen will.

Eras\textcompwordmark{}mus und die Leute seiner Zeit, deren Kenntniß
des Griechischen wir bewundern, lernten es durch Ver\textcompwordmark{}kehr
mit Griechisch sprechenden Lehrern aus den Gesprächen über Gegenstände
des gewöhnlichen Lebens. Aus der Grammatik und Lectüre allein hat
noch Niemand Griechisch wirklich gelernt. Aber die Sprache verdient
es, daß wer sie lernen will, sie wirklich und nicht bloß zum Scheine
zu lernen sucht; denn Griechisch ist, wie der treffliche Wilhelm
Roscher, der berühmte Leipziger Nationalökonom, in seinem Buche über
Thukydides einst gesagt hat,

\begin{quotedquotation}\noindent die Sprache aller Sprachen, worin
die köstlichsten Menschenworte geredet sind. Die feierliche Grandezza
des Spaniers, die feine Süßigkeit des Italieners, des Franzosen geläufige
Anmuth, des Engländers pathetische Kraft, des Deutschen unergründlicher
Reichthum, ja selbst die Würde der römischen Senatorensprache, hier
sind sie vereinigt, sind geläutert im Feuer des Geistes und zum edelsten
Erze zusammengeschmolzen.\unskip``\end{quotedquotation} 
