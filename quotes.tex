\section*{Zum Merken und Citiren.\addcontentsline{toc}{section}{Allerlei zum Merken und Citiren}}

\noindent \begin{center}
\emph{Die neun Musen:}
\par\end{center}
\begin{greek}[variant=ancient]%
\begin{quote}
Κλειώ τ᾽ Εὐτέρπη τε Θάλειά τε Μελπονένη τε\\
Τερψιχόρη τ᾽ Ἐρατώ τε Πολύμνιά τ᾽ Οὐρανίη τε,\\
Καλλιόπη θ᾽· ἡ δὲ προφερεστάτη ἐστὶν ἁπασέων.

\source{\textgerman[spelling=old,babelshorthands=true]{Lateinisches
Merkwort: }\textlatin{TUM PECCET. (Hesiod. Theog. 77.)}}
\end{quote}
\end{greek}%
\noindent \begin{center}
\emph{Die drei Grazien:}
\par\end{center}
\begin{greek}[variant=ancient]%
\begin{quote}
Ἀγλαΐη τε καὶ Εὐφροσύνη Θαλίη τ᾽ ἐρατείνη. 

\source{\textlatin{(Hesiod. Theog. 909.)}}
\end{quote}
\end{greek}%
\noindent \begin{center}
\emph{Die drei Parzen:}
\par\end{center}
\begin{greek}[variant=ancient]%
\begin{quote}
Κλωθώ τε Λάχεσίς τε καὶ Ἄτροπος, αἵ τε διδοῦσι\\
θνητοῖς ἀνθρώποισιν ἔχειν ἀγαθόν τε κακόν τε. 

\source{\textlatin{(Hesiod. Theog. 905.)}}
\end{quote}
\end{greek}%
\noindent \begin{center}
\emph{Die drei Gorgonen:}
\par\end{center}
\begin{greek}[variant=ancient]%
\begin{quote}
Σθεινώ τ᾽ Εὐρυάλη τε Μέδουσά τε λυγρὰ παθοῦσα.

\source{\textlatin{(Hesiod. Theog. 276.)}}
\end{quote}
\end{greek}%
\noindent \begin{center}
\emph{Scipio bei Numantia über Gracchus:}
\par\end{center}
\begin{greek}[variant=ancient]%
\begin{quote}
ὥς ἀπόλοιτο καὶ ἄλλος, ὅτις τοιαῦτά γε ῥέζοι. 

\source{\textlatin{(Hom. Od. 1, 47.)}}
\end{quote}
\end{greek}%
\noindent \begin{center}
\emph{Cicero's Wahlspruch:}
\par\end{center}
\begin{greek}[variant=ancient]%
\begin{quote}
αἲεν ἀριστεύειν καὶ ὑπείροχον ἔμμεναι ἄλλων. 

\source{\textlatin{(Hom. Il. 6, 208.)}}
\end{quote}
\end{greek}%
\noindent \begin{center}
\emph{Hector's Wahlspruch:}
\par\end{center}
\begin{greek}[variant=ancient]%
\begin{quote}
εἷς οἰωνὸς ἄριστος, ἀμύνεσθαι περὶ πάτρης. 

\source{\textlatin{(Hom. Il. 12, 243.)}}
\end{quote}
\end{greek}%
\noindent \begin{center}
\emph{Alexander's des Großen Wahlspruch:}
\par\end{center}
\begin{greek}[variant=ancient]%
\begin{quote}
ἄμφότερον, βασιλεύς τ᾽ ἀγαθός κρατερός τ᾽ αἰχμητής. 

\source{\textlatin{(Hom. Il. 3, 197.)}}
\end{quote}
\end{greek}%
\noindent \begin{center}
\emph{Scipio auf den Trümmern Karthago's.}
\par\end{center}
\begin{greek}[variant=ancient]%
\begin{quote}
ἔσσεται ἦμαρ, ὅτ᾽ ἄν ποτ᾽ ὀλώλῃ Ἴλιος ἱρή\\
καὶ Πρίαμος καὶ λαὸς ἐϋμμελίω Πριάμοιο. 

\source{\textlatin{(Hom. Il. 6, 448.)}}
\end{quote}
\end{greek}%
\noindent \begin{center}
\emph{Die sieben Weisen:}
\par\end{center}
\begin{greek}[variant=ancient]%
\begin{quote}
Ἑπτὰ σοφῶν, Κλεόβουλε, σὲ μὲν τεκνώσατο Λίνδος·\\
φατὶ δὲ Συσιφία χθὼν Περίανδρον ἔχειν·\\
Πιττακὸν ἁ Μυτιλάνα· Βίαντα δὲ δῖα Πριήνη·\\
Μίλητος δὲ Θαλῆν, ἄκρον ἔρεισμα Δίκας·\\
ἁ Σπάρτα Χίλωνα· Σόλωνα δὲ Κεκροπὶς αἶα.\\
πάντας ἀριζάλου σωφροσύνας φύλακας. 
\end{quote}
\end{greek}%
\noindent \begin{center}
Die Aus\textcompwordmark{}sprüche der sieben weisen (nach Diogenes
Laërtius):
\par\end{center}

Thales: \textgreek[variant=ancient]{γνῶθι σαυτόν!} (Erkenne dich selbst!)

Solon: \textgreek[variant=ancient]{μηδὲν ἄγαν!} (Nichts übertreiben!)

Chilon: \textgreek[variant=ancient]{ἐγγύα πάρα δ᾽ ἄτα!} (Bürgen thut
würgen In Geldsachen hört die Gemüthlichkeit auf.)

Pittacus: \textgreek[variant=ancient]{καιρὸν γνῶθι!} (Nimm den Augenblick
wahr!)

Bias: \textgreek[variant=ancient]{οἱ πλεῖστοι κακοί.} (Viele Köche
verderben den Brei.)

Kleobulus: \textgreek[variant=ancient]{μέτρον ἄριστον.} (Maßhalten
ist gut.)

Periander: \textgreek[variant=ancient]{μελέτη τὸ πᾶν.} (Übung macht
den Meister.)

Das (angeblich) delphische Orakel über Sokrates:
\begin{greek}[variant=ancient]%
\begin{quote}
Σοφὸς Σοφοκλῆς, σοφώτερος δ᾽ Εὐριπίδης,\\
Ἀνδρῶν δὲ πάντων Σωκράτης σοφώτατος. 

\source{\textlatin{(Schol. Aristoph. Nub. v. 144.)}}
\end{quote}
\end{greek}%
\noindent \begin{center}
\emph{Die Worte des Archimedes:}
\par\end{center}
\begin{enumerate}
\item \textgreek[variant=ancient]{Εὕρηκα!}
\item \textgreek[variant=ancient]{δός μοι ποῦ στῶ καὶ τὰν γᾶν κινασῶ! }
\item \textlatin{noli istud disturbare!}
\end{enumerate}
\noindent \begin{center}
\emph{Kaiser Augustus auf dem Sterbebette:}
\par\end{center}
\begin{greek}[variant=ancient]%
\begin{quote}
— — εἰ δὲ πᾶν ἔχει καλῶς, τῷ παιγνίῳ\\
Δότε κρότον καὶ πάντες ὑμεῖς μετὰ χαρᾶς κτυπήσατε! 

\source{\textlatin{(Sueton. Octav. 99.)}}
\end{quote}
\end{greek}%
\noindent \begin{center}
\emph{Die spartanische Mutter zu ihrem Sohne:}
\par\end{center}
\begin{greek}[variant=ancient]%
\begin{quote}
Τέκνον, ἢ τὰν ἢ ἐπὶ τᾶς!

\source{\textlatin{(Plutarch. }Λακαινῶν ἀποφθέγματα.\textlatin{)}}
\end{quote}
\end{greek}%
\noindent \begin{center}
\emph{Weg mit den sorgen!}
\par\end{center}
\begin{greek}[variant=ancient]%
\begin{quote}
τὸ σήμερον μέλει μοι,\\
τὸ δ᾽ αὔριον τίς οἶδεν; 

\source{\textgerman[spelling=old,babelshorthands=true]{(Anakreon)}}
\end{quote}
\end{greek}%
\noindent \begin{center}
\emph{Griechische Tages\textcompwordmark{}eintheilung:}\\
6 Stunden für die Arbeit, 4 Stunden für den Lebens\textcompwordmark{}genuß:
\par\end{center}
\begin{greek}[variant=ancient]%
\begin{quote}
ἓξ ὧραι μόχθοις ἱκανώταται· αἱ δὲ μετ᾽ αὐτὰς\\
γράμμασι δεικνύμεναι ζῆθι λέγουσι βροτοῖς.

1— 6: αʹ. βʹ. γʹ. δʹ. εʹ. ϛʹ.

7—10: ζʹ. ηʹ. θʹ. ιʹ. 

\source{\textgerman[spelling=old,babelshorthands=true]{(Alter Spruch.)}}
\end{quote}
\end{greek}%


\noindent \begin{center}
Druck von Hesse \& Becker in Leipzig.
\par\end{center}

\pagebreak{}

\begin{center}
{\Huge{}(Das originale Buch hat Ankündigungen hier.)}\\

\par\end{center}{\Huge \par}

