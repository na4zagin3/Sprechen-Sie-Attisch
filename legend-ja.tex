\begin{paracol}{2}
%\section*{Vorbemerkungen\addcontentsline{toc}{section}{\emph{Vorbemerkungen} über die Bedeutung der attischen Umgangs\textcompwordmark{}sprache für das Erlernen des Griechischen}}
\section*{\textgerman{Vorbemerkungen}}
\switchcolumn
\section*{\textjapanese{前言\addcontentsline{toc}{section}{前言: ギリシャ語学習におけるアッティカ方言口語の重要性について}}}
\switchcolumn*

\begin{german}
Griechisch gilt den Allermeisten für eine im Grunde unlernbare Sprache,
deren man nimmermehr so mächtig werden könne, wie einer neueren, die
man leidlich beherrscht. Vorliegendes Büchlein, das fröhlicher Ferienlaune
seinen Ursprung verdankt, möchte den Gegenbeweis führen, indem es
einem ersten Versuch macht, attische Umgangs\textcompwordmark{}sprache
in ihren gebräuchlichsten Wendungen zu lehren.
\end{german}
\switchcolumn

\begin{japanese}
ギリシャ語は、ほとんどの人にとり、基本的に学べない言語と考えられており、
まともに把握している現代言語ほどには決して習熟できないものと思われている。
本小書は、明るい休日の気分がその発端となったもので、
アッティカ方言口語を最も一般的な言い囘しで教えることで、その反証を試みるものだ。
\end{japanese}
\switchcolumn*

\begin{german}
Wer die Umgangs\textcompwordmark{}sprache eines Volkes kennt, hat
den Schlüssel zum Verständniß seiner Schriftwerke gleich den Volks\textcompwordmark{}genossen
selbst.
\end{german}
\switchcolumn

\begin{japanese}
ある民族の口語を知っている人は、その民族の書かれた作品をその民族同胞と同じように
理解する鍵を持っている。
\end{japanese}
\switchcolumn*

\begin{german}
Der attische Knabe brachte zur Lectüre griechischer Dichter, der attische
Bauer in sein Theater oder in die Volksversammlung nur die
Kenntniß der attischen Umgangssprache in ihrer
einfachsten Form mit; \emph{sie} befähigte zum Verständniß sophokleïscher
Dramen und perikleïsche Reden. Die Sprache des Alltagslebens
lieferte diejenigen Analogien, welche zum Erfassen der höheren Erzeugnisse
in Rede und Schrift nothwendig waren.
\end{german}
\switchcolumn

\begin{japanese}
アッティカの少年はギリシャ詩人の朗読に、また、アッティカの農民は劇場や公会に、アッ
ティカ方言口語の最も単純な形の知識だけを持っていった。
\emph{その}おかげで、ソ\jdb{ポ}クレースの劇やペリクレースの演説が理解できたのだ。
日常生活の言語は、演説や書物における高度な作品を理解するために必要な\ruby{類似}{アナロジー}を与えていたのだ。
\end{japanese}
\switchcolumn*

\begin{german}
Man hat oft behauptet, daß es erstaunlich wenig Worte und Wendungen
sind, mit denen der gemeine Mann in seiner Muttersprache aus\textcompwordmark{}kommt
und die ihn befähigen, auch das zu verstehen, was für ihn Neubildung
ist. Sollte es nicht möglich sein, dem Athener seinen verhältnißmäßig
kleinen Urvorrath abzulauschen, somit die Sprache in ihrem \emph{Kerne}
zu erfassen und diese Worte und Wendungen demjenigen, der Griechisch
wirklich lernen will, geläufig zu machen?
\end{german}
\switchcolumn

\begin{japanese}
新造語の意味すら理解できる母語として民衆が獲得し使いこなす言辞や言いまわしの数は驚くほど少ないことがしばしば指摘される。
それならばア\jdb{テ}ーナイ人から比較的少数の根源的語彙を聴き写し、
言語の\emph{核心}を捉え、
ギリシャ語を本当に学びたい人にこれらの言辞や言いまわしを馴染ませることも不可能ではなかろう。
\end{japanese}
\switchcolumn*

\begin{german}
Aristophanes bietet für diesen Zweck in denjenigen Partien, wo er
den gemeinen Mann im volks\textcompwordmark{}thümlichen Verkehrs\textcompwordmark{}tone
reden läßt, sprachlichen Stoff genug, und auch in der übrigen Literatur
finden sich verstreut Stellen, welche für treue Nachahmungen der Sprache
des gemeinen Lebens gelten müssen. Die Aufgabe kann also nicht unlösbar
sein, wenn auch das vorliegende Schriftchen nur
erst einen kleinen Beitrag zu ihrer Lösung bringt.
\end{german}
\switchcolumn

\begin{japanese}
アリスト\jdb{パ}ネースは、民衆を庶民的な調子で語らせる箇所で、この目的に十分な言
語資料を提供しており。
その他の文学作品にも、庶民生活の言葉の忠実な模倣と見なせる箇所が散在している。\todo{この節ここまで校了}%
従って、この課題は解けないわけではない。ただ、ここに掲げた小冊子がその解決にまだ小さな一歩しか示せていないだけだ。
\end{japanese}
\switchcolumn*

\begin{german}
Die Worte und Wendungen in den nachstehenden Gesprächen sind in der
Hauptsache der aristphanischen Sprache entnommen. Einiges mußte aus
der späteren Gräcität beigefügt werden. Die dem Neugriechischen entlehnten
Ergänzungen, welche zur Bezeichnung moderner Begriffe verwandt wurden,
sind durch {*} besonders kenntlich gemacht.
\end{german}
\switchcolumn

\begin{japanese}
以下の会話に出てくる語と表現は、基本的にアリストパネスの語法から取った。
いくらかは後代のギリシャ語から補う必要があった。
現代ギリシャ語に由来し、近代的概念を表すために用いられた補遺は、特に {*} で示してある。
\end{japanese}
\switchcolumn*

\begin{german}
Auch wer nicht die Absicht hat, attisch conversiren zu lernen, wird
mit vielem Nutzen für sein Verständniß des Griechischen sich mit der
attischen Umgangssprache beschäftigen. Denn während
man auf unseren Gymnasien im Lateinischen fast nur solche Schriften
liest, welche der höheren Kunstsprache angehören
--- man denke nur and Cicero und Tacitus --- und in welchen die Volkssprache
kaum hier und da erkennbar ist, werden wir im
Griechischen weit mehr auf die Sprache des gewöhnlichen Lebens hingewiesen.
Im Griechischen lesen wir Gespräche bei den Dramatikern, Gespräche
bei Plato; die Stimme des gemeinsten Mannes, --- schon \emph{dies}
nöthigt sie, seiner Sprache nahe zu bleiben, und schon dies muß die
Kenntniß der Ausdrucksweise
des täglichen Lebens im Griechischen nützlich machen zum feinfühligeren
Verständniß der Texte.
\end{german}
\switchcolumn

\begin{japanese}
アッティカ語で会話を学ぶつもりのない人でも、ギリシャ語解のためにアッティカ方言口語を学ぶことは大いに役立つ。
というのも、私たちのギムナジウムではラテン語の授業でほとんど高尚な文語ばかりを読む
--- シケロやタキトゥスを思い起こせばよい --- その中に民衆語がわずかに垣間見えるだけである。
それに比べてギリシャ語では、日常生活の言葉にずっと多く向き合わされる。
ギリシャ語では、劇作家の対話もプラトンの対話も読む。
最も庶民的な人々の声がすでにその言葉遣いの近くにとどまることを強い、
それが古典文献をより繊細に理解する助けとなるのだ。
\end{japanese}
\switchcolumn*

\begin{german}
Zweitens aber ist die \emph{Färbung} der Sprache und die Stil\textcompwordmark{}gattung
eines Literatur\textcompwordmark{}werkes nur demjenigen recht erkennbar,
der ermessen kann, wie weit dessen Sprache sich \emph{abhebt} von
der Alltagssprache. Wer das Deutsche nur aus Schiller
gelernt hätte, dem würde das Verständniß abgehen
für die Eigenart und die Höhe der Schiller'schen Diction. Erst wer
von der Sprache der \emph{Alltäglichkeit} aus an sie herantritt,
bringt den Maßstab für sie mit. Es wird im Griechischen nicht anders
sein.
\end{german}
\switchcolumn

\begin{japanese}
第二に、言語の\emph{色合い}や文体の種類は、日常語からどれほど\emph{離れている}かを測れる者にしか本当にはわからない。
もしドイツ語をシラーだけで学んだなら、シラーの語法の独自性や高さを理解する感覚が欠けるだろう。
\emph{日常}の言葉から近づいてこそ、その尺度を持てるのであり、ギリシャ語でも同じである。
\end{japanese}
\switchcolumn*

\begin{german}
Drittens zwingt ganz besonders die Beschäftigung mit der griechischen
Um\textcompwordmark{}gangs\textcompwordmark{}sprache zur \emph{Vergleichung}
des deutschen und griechischen Ausdruckes und
fördert dadurch die Sicherheit und Natürlichkeit der Übersetzungen
aus dem Griechischen, die auf der Leichtigkeit und Bereitschaft
der Wortvergleichungen der
beruht. Was man den \emph{Geist} der Sprache nennt, das zeigt sich
am Auffallendsten da, wo die Vergleichung der
Sprachen unter einander \emph{leicht} und \emph{nahe}liegend ist:
das ist auf dem Gebiete des Alltäglichen. Den jocosen Ton, der sich
von selbst ergiebt, sobald man die alltägliche Ausdrucksweise
des modernen Lebens mit der Sprechweise der Alten in Vergleich stellt,
wird man als bei diesem Studium unvermeidlich um der Sache willen
mit in den Kauf nehmen.
\end{german}
\switchcolumn

\begin{japanese}
第三に、ギリシャ方言口語の学習は、特にドイツ語とギリシャ語の\emph{表現の比較}を迫り、
語の置き換えを軽やかにすることでギリシャ語からの翻訳の確かさと自然さを高める。
人が言うところの言語の\emph{精神}とは、言語同士の比較が\emph{容易}で\emph{身近}なところ、
すなわち日常の領域で最も目立って現れる。
現代生活の平凡な表現を古代人の言い回しと並べると自然に滑稽な調子が生まれるが、
この勉強では肝心の目的のためにそれを受け入れるしかないだろう。
\end{japanese}
\switchcolumn*

\begin{german}
Endlich aber sei darauf hingewiesen, daß nichts dem Erlernen des
Griechischen an unseren Gymnasien so viele \emph{Gegner} geschaffen,
als eben die Thatsache, daß Griechisch im Grunde für eine unlernbare
Sprache gilt. Was der belgische Professor Emil de Laveleye über die
von ihm beobachteten Ergebnisse des Gymnasialunterrichtes
sagt: \quotedblbase \textfrench{résultat net et incontestable: on
sait peu le latin et point du tout le grec,}`` das, behaupten Viele,
trifft annähernd auch bei den deutschen Gymnasien zu. Erstaunlich
Wenige, die \quotedblbase Griechisch gelernt`` haben, wissen mit
einiger Bestimmtheit anzugeben, wie der Attiker
die einfachsten Begriffe, z.\,B. \quotedblbase Ich werde zu dir
kommen``, auszudrücken pflegt. Wenn im Lateinischen
Jemand nicht sofort auf \quotedblbase \textlatin{veniam}`` käme,
würde man meinen, daß ihm die allerersten Anfangsgründe
mangeln, und wenn er nicht verstünde, \quotedblbase \textlatin{veniam}``
und \quotedblbase \textlatin{ibo}`` auseinanderzuhalten,
so würde man über Unzulänglichkeit des Unterrichtes mit vollem Rechte
Klage führen und glauben, daß solche Unsicherheit auch dem sicheren
Erfassen des \emph{Sinnes} lateinischer \emph{Schriftwerke Eintrag
}thun müsse. Aber  im Griechischen? Man mache den Versuch, und man
wird überraschend Wenige finden, die das im Gebrauche des Attikers
alltägliche \quotedblbase \textgreek[variant=ancient]{ἥξω παρὰ σέ}``
in Bereitschaft haben. Man
studirt im Griechischen eifrig die Sprach\emph{gesetze}, aber gar
wenig die \emph{Sprache}, und doch lernt man es nicht um der grammatischen
Schulung willen, --- für diese sorgt ausreichend
das Latein, --- sondern der Sprache wegen. Man setze einem jungen
Manne, der die Schule  mit dem Zeugniß der Reife im Griechischen
verlassen hat, ein Glas griechischen Weines vor: er wird schwerlich
im Stande sein, auf Griechisch mit nur einigermaßen passendem Worte
dafür zu danken, oder zu sagen, daß ihm der Wein gut schmeckt. Allerdings
ist solche Sprachfertigkeit nicht das Ziel und die Aufgabe des griechischen
Unterrichts im Gymnasium aber daß sie bei den langen und angestrengten
Studien nicht nebenbei mit abfällt und so völlig fern zu bleiben
scheint, läßt das Gefühl des Griechischkönnens nicht aufkommen. Der
\quotedblbase Reife`` ist sich gar wohl bewußt, daß es ihm unsägliche
Mühe macht, ganz einfache Gedanken in wirklich griechischen Wendungen
wiederzugeben. Das macht unzufrieden und trägt viel dazu bei, dem
Griechischen Gegner zu schaffen. Auch aus diesem Grunde soll mein
Büchlein zeigen, daß es leicht angeht, sich mit den Kenntnissen,
die das Gymnasium bietet, des Griechischen so zu bemächtigen, daß
man sich darin verständlich machen könnte. 
\end{german}
\switchcolumn

\begin{japanese}
最後に指摘しておきたいのは、ギムナジウムでのギリシャ語学習にこれほど多くの\emph{反対者}を生んだのは、
ギリシャ語が本質的に習得不可能な言語だと見なされている事実にほかならないという点だ。
ベルギーのエミール・ド・ラヴェレイ教授が観察したギムナジウム教育の成果について
「\textfrench{résultat net et incontestable: on sait peu le latin et point du tout le grec,}」
と述べていることは、多くの人の主張ではドイツのギムナジウムにも大体当てはまる。
「ギリシャ語を学んだ」と言う人でも、アッティカ人が
「私は君のところへ行くだろう」のような最も単純な概念をどう表すか、
たとえば「\textlatin{veniam}」と「\textlatin{ibo}」を区別できないとすれば、
自信を持って答えられる人は驚くほど少ない。
ラテン語で「\textlatin{veniam}」がすぐに出てこなければ、
初歩が欠けていると考えられるし、\textlatin{veniam} と \textlatin{ibo} の違いがわからなければ、
授業の不備だと正当に非難され、そのような不確かさはラテン語の\emph{文献}の\emph{意味}を確実に把握する上でも害になるとみなされる。
しかしギリシャ語ではどうか。
試してみれば、アッティカ人の日常語である
「\textgreek[variant=ancient]{ἥξω παρὰ σέ}」をすぐに口にできる人は驚くほど少ないだろう。
ギリシャ語では熱心に\emph{言語法則}を学ぶが、\emph{言語そのもの}はあまり学ばない。
しかもギリシャ語を学ぶのは文法訓練のためではなく --- それならラテン語で十分で --- 言語そのもののためだ。
ギリシャ語の修了証を持って学校を出た若者に、ギリシャ産のワインを一杯差し出してみるといい。
ギリシャ語でそれに礼を言うことも、ワインが美味しいと伝えることも、まずできないだろう。
もちろん、そのような話術はギムナジウムのギリシャ語教育の目標ではない。
だが長く厳しい学習の中でそれが少しも身につかず、全く身から離れているように見えるのでは、
ギリシャ語を使いこなせるという感覚は生まれない。
「修了生」は、簡単な思考を本当にギリシャ語らしい言い回しで表すのに途方もない苦労をすることを自覚している。
それが不満を生み、ギリシャ語への敵意を育てる一因ともなる。
そうした理由からも、この小冊子は、ギムナジウムで得られる知識を手がかりにすれば、
ギリシャ語を自在に扱って意思疎通できるようになるのはたやすいことを示したいのだ。
\end{japanese}
\switchcolumn*

\begin{german}
Die Hauptsache aber bleibt: die allergewöhnlichsten Wörter und Wendungen
in der Ver\textcompwordmark{}kehrs\textcompwordmark{}sprache des
täglichen Lebens sind der Urvorrath, der Krystallisationskern,
an den und um den sich die weiteren sprachlichen Bildungen angesetzt
und angeschlossen haben. Schon darum verdienen sie unsere Achtung.
\emph{Hier} gilt es, die Sprache zu fassen, für den, der sie wirklich
lernen will.
\end{german}
\switchcolumn

\begin{japanese}
とはいえ肝心なのはこうだ。日常生活の共通語で使われるごくありふれた単語や言い回しこそが、
語の備蓄であり、結晶化の核であり、その周りにさらに複雑な言語表現が作られ、接続してきたのだ。
それだけでも十分に敬意に値する。\emph{ここ}にこそ、言語をつかもうとする者が本当に学ぶべきものがある。
\end{japanese}
\switchcolumn*

\begin{german}
Eras\textcompwordmark{}mus und die Leute seiner Zeit, deren Kenntniß
des Griechischen wir bewundern, lernten es durch Ver\textcompwordmark{}kehr
mit Griechisch sprechenden Lehrern aus den Gesprächen über Gegenstände
des gewöhnlichen Lebens. Aus der Grammatik und Lectüre allein hat
noch Niemand Griechisch wirklich gelernt. Aber die Sprache verdient
es, daß wer sie lernen will, sie wirklich und nicht bloß zum Scheine
zu lernen sucht; denn Griechisch ist, wie der treffliche Wilhelm
Roscher, der berühmte Leipziger Nationalökonom, in seinem Buche über
Thukydides einst gesagt hat,
\end{german}
\switchcolumn

\begin{japanese}
私たちがそのギリシャ語力に感嘆するエラスムスやその同時代の人々は、
ギリシャ語を話す教師たちとの交流を通じ、日常の事柄についての会話から学んだ。
文法書と講読だけでギリシャ語を本当に身につけた人は一人もいない。
しかしギリシャ語は、それを学びたい人が、形だけでなく本当に学ぼうとする価値のある言語だ。
優れた経済学者であるライプツィヒ大学のウィルヘルム・ロッシャーがトゥキュディデス論でかつて述べたように、
\end{japanese}
\switchcolumn*

\begin{german}
\begin{quotedquotation}\noindent die Sprache aller Sprachen, worin
die köstlichsten Menschenworte geredet sind. Die feierliche Grandezza
des Spaniers, die feine Süßigkeit des Italieners, des Franzosen geläufige
Anmuth, des Engländers pathetische Kraft, des Deutschen unergründlicher
Reichthum, ja selbst die Würde der römischen Senatorensprache, hier
sind sie vereinigt, sind geläutert im Feuer des Geistes und zum edelsten
Erze zusammengeschmolzen.\unskip``\end{quotedquotation}
\end{german}
\switchcolumn

\begin{japanese}
\begin{quotedquotation}\noindent それはありとあらゆる言語のうち、最も尊い人間の言葉が語られた言語である。
スペイン語の荘重な威厳、イタリア語の繊細な甘美さ、フランス語の軽やかな優雅さ、
英語の力強い熱情、ドイツ語の底知れぬ豊かさ、さらにはローマの元老院語の威厳さえも、
ここに集い、精神の火で精錬され、もっとも貴い金属へと溶け合っているのだ。\end{quotedquotation}
\end{japanese}
\switchcolumn*

\end{paracol}
