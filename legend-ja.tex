\section*{前言\addcontentsline{toc}{section}{前言: ギリシャ語学習におけるアッティカ方言口語の重要性について}}

ギリシャ語は、ほとんどの人にとり、基本的に学べない言語と考えられており、
まともに把握している現代言語ほどには決して習熟できないものと思われている。
本小書は、明るい休日の気分がその発端となったもので、
アッティカ方言口語を最も一般的な言い囘しで教えることで、その反証を試みるものだ。

ある民族の口語を知っている人は、その民族の書かれた作品をその民族同胞と同じように
理解する鍵を持っている。


Der attische Knabe brachte zur Lectüre griechischer Dichter, der attische
Bauer in sein Theater oder in die Volksversammlung nur die
Kenntniß der attischen Umgangssprache in ihrer
einfachsten Form mit; \emph{sie} befähigte zum Verständniß sophokleïscher
Dramen und perikleïsche Reden. Die Sprache des Alltagslebens
lieferte diejenigen Analogien, welche zum Erfassen der höheren Erzeugnisse
in Rede und Schrift nothwendig waren.

アッティカの少年はギリシャ詩人の朗読に、また、アッティカの農民は劇場や公会に、アッ
ティカ方言口語の最も単純な形の知識だけを持っていった。
それにより、ソポクレースの劇やペリクレースの演説が理解できたのだ。
日常生活の言語は、演説や書物における高度な作品を理解するために必要な\ruby{類似}{アナロジー}をもたらしたのだ。

Man hat oft behauptet, daß es erstaunlich wenig Worte und Wendungen
sind, mit denen der gemeine Mann in seiner Muttersprache auskommt
und die ihn befähigen, auch das zu verstehen, was für ihn Neubildung
ist. Sollte es nicht möglich sein, dem Athener seinen verhältnißmäßig
kleinen Urvorrath abzulauschen, somit die Sprache in ihrem \emph{Kerne}
zu erfassen und diese Worte und Wendungen demjenigen, der Griechisch
wirklich lernen will, geläufig zu machen?

Aristophanes bietet für diesen Zweck in denjenigen Partien, wo er
den gemeinen Mann im volksthümlichen Verkehrstone
reden läßt, sprachlichen Stoff genug, und auch in der übrigen Literatur
finden sich verstreut Stellen, welche für treue Nachahmungen der Sprache
des gemeinen Lebens gelten müssen. Die Aufgabe kann also nicht unlösbar
sein, wenn auch das vorliegende Schriftchen nur
erst einen kleinen Beitrag zu ihrer Lösung bringt.

Die Worte und Wendungen in den nachstehenden Gesprächen sind in der
Hauptsache der aristphanischen Sprache entnommen. Einiges mußte aus
der späteren Gräcität beigefügt werden. Die dem Neugriechischen entlehnten
Ergänzungen, welche zur Bezeichnung moderner Begriffe verwandt wurden,
sind durch {*} besonders kenntlich gemacht.

Auch wer nicht die Absicht hat, attisch conversiren zu lernen, wird
mit vielem Nutzen für sein Verständniß des Griechischen sich mit der
attischen Umgangssprache beschäftigen. Denn während
man auf unseren Gymnasien im Lateinischen fast nur solche Schriften
liest, welche der höheren Kunstsprache angehören
--- man denke nur and Cicero und Tacitus --- und in welchen die Volkssprache
kaum hier und da erkennbar ist, werden wir im
Griechischen weit mehr auf die Sprache des gewöhnlichen Lebens hingewiesen.
Im Griechischen lesen wir Gespräche bei den Dramatikern, Gespräche
bei Plato; die Stimme des gemeinsten Mannes, --- schon \emph{dies}
nöthigt sie, seiner Sprache nahe zu bleiben, und schon dies muß die
Kenntniß der Ausdrucksweise
des täglichen Lebens im Griechischen nützlich machen zum feinfühligeren
Verständniß der Texte.

Zweitens aber ist die \emph{Färbung} der Sprache und die Stilgattung
eines Literaturwerkes nur demjenigen recht erkennbar,
der ermessen kann, wie weit dessen Sprache sich \emph{abhebt} von
der Alltagssprache. Wer das Deutsche nur aus Schiller
gelernt hätte, dem würde das Verständniß abgehen
für die Eigenart und die Höhe der Schiller'schen Diction. Erst wer
von der Sprache der \emph{Alltäglichkeit} aus an sie herantritt,
bringt den Maßstab für sie mit. Es wird im Griechischen nicht anders
sein.

Drittens zwingt ganz besonders die Beschäftigung mit der griechischen
Umgangssprache zur \emph{Vergleichung}
des deutschen und griechischen Ausdruckes und
fördert dadurch die Sicherheit und Natürlichkeit der Übersetzungen
aus dem Griechischen, die auf der Leichtigkeit und Bereitschaft
der Wortvergleichungen der
beruht. Was man den \emph{Geist} der Sprache nennt, das zeigt sich
am Auffallendsten da, wo die Vergleichung der
Sprachen unter einander \emph{leicht} und \emph{nahe}liegend ist:
das ist auf dem Gebiete des Alltäglichen. Den jocosen Ton, der sich
von selbst ergiebt, sobald man die alltägliche Ausdrucksweise
des modernen Lebens mit der Sprechweise der Alten in Vergleich stellt,
wird man als bei diesem Studium unvermeidlich um der Sache willen
mit in den Kauf nehmen.

Endlich aber sei darauf hingewiesen, daß nichts dem Erlernen des
Griechischen an unseren Gymnasien so viele \emph{Gegner} geschaffen,
als eben die Thatsache, daß Griechisch im Grunde für eine unlernbare
Sprache gilt. Was der belgische Professor Emil de Laveleye über die
von ihm beobachteten Ergebnisse des Gymnasialunterrichtes
sagt: \quotedblbase \textfrench{résultat net et incontestable: on
sait peu le latin et point du tout le grec,}`` das, behaupten Viele,
trifft annähernd auch bei den deutschen Gymnasien zu. Erstaunlich
Wenige, die \quotedblbase Griechisch gelernt`` haben, wissen mit
einiger Bestimmtheit anzugeben, wie der Attiker
die einfachsten Begriffe, z.\,B. \quotedblbase Ich werde zu dir
kommen``, auszudrücken pflegt. Wenn im Lateinischen
Jemand nicht sofort auf \quotedblbase \textlatin{veniam}`` käme,
würde man meinen, daß ihm die allerersten Anfangsgründe
mangeln, und wenn er nicht verstünde, \quotedblbase \textlatin{veniam}``
und \quotedblbase \textlatin{ibo}`` auseinanderzuhalten,
so würde man über Unzulänglichkeit des Unterrichtes mit vollem Rechte
Klage führen und glauben, daß solche Unsicherheit auch dem sicheren
Erfassen des \emph{Sinnes} lateinischer \emph{Schriftwerke Eintrag
}thun müsse. Aber  im Griechischen? Man mache den Versuch, und man
wird überraschend Wenige finden, die das im Gebrauche des Attikers
alltägliche \quotedblbase \textgreek[variant=ancient]{ἥξω παρὰ σέ}``
in Bereitschaft haben. Man
studirt im Griechischen eifrig die Sprach\emph{gesetze}, aber gar
wenig die \emph{Sprache}, und doch lernt man es nicht um der grammatischen
Schulung willen, --- für diese sorgt ausreichend
das Latein, --- sondern der Sprache wegen. Man setze einem jungen
Manne, der die Schule  mit dem Zeugniß der Reife im Griechischen
verlassen hat, ein Glas griechischen Weines vor: er wird schwerlich
im Stande sein, auf Griechisch mit nur einigermaßen passendem Worte
dafür zu danken, oder zu sagen, daß ihm der Wein gut schmeckt. Allerdings
ist solche Sprachfertigkeit nicht das Ziel und die Aufgabe des griechischen
Unterrichts im Gymnasium aber daß sie bei den langen und angestrengten
Studien nicht nebenbei mit abfällt und so völlig fern zu bleiben
scheint, läßt das Gefühl des Griechischkönnens nicht aufkommen. Der
\quotedblbase Reife`` ist sich gar wohl bewußt, daß es ihm unsägliche
Mühe macht, ganz einfache Gedanken in wirklich griechischen Wendungen
wiederzugeben. Das macht unzufrieden und trägt viel dazu bei, dem
Griechischen Gegner zu schaffen. Auch aus diesem Grunde soll mein
Büchlein zeigen, daß es leicht angeht, sich mit den Kenntnissen,
die das Gymnasium bietet, des Griechischen so zu bemächtigen, daß
man sich darin verständlich machen könnte. 

Die Hauptsache aber bleibt: die allergewöhnlichsten Wörter und Wendungen
in der Verkehrssprache des
täglichen Lebens sind der Urvorrath, der Krystallisationskern,
an den und um den sich die weiteren sprachlichen Bildungen angesetzt
und angeschlossen haben. Schon darum verdienen sie unsere Achtung.
\emph{Hier} gilt es, die Sprache zu fassen, für den, der sie wirklich
lernen will.

Erasmus und die Leute seiner Zeit, deren Kenntniß
des Griechischen wir bewundern, lernten es durch Verkehr
mit Griechisch sprechenden Lehrern aus den Gesprächen über Gegenstände
des gewöhnlichen Lebens. Aus der Grammatik und Lectüre allein hat
noch Niemand Griechisch wirklich gelernt. Aber die Sprache verdient
es, daß wer sie lernen will, sie wirklich und nicht bloß zum Scheine
zu lernen sucht; denn Griechisch ist, wie der treffliche Wilhelm
Roscher, der berühmte Leipziger Nationalökonom, in seinem Buche über
Thukydides einst gesagt hat,

\begin{quotedquotation}\noindent die Sprache aller Sprachen, worin
die köstlichsten Menschenworte geredet sind. Die feierliche Grandezza
des Spaniers, die feine Süßigkeit des Italieners, des Franzosen geläufige
Anmuth, des Engländers pathetische Kraft, des Deutschen unergründlicher
Reichthum, ja selbst die Würde der römischen Senatorensprache, hier
sind sie vereinigt, sind geläutert im Feuer des Geistes und zum edelsten
Erze zusammengeschmolzen.\unskip``\end{quotedquotation} 
