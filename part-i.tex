\switchcolumn*[{


\part[Sprichwörtliches aus der Umgangs\textcompwordmark{}sprache Altgriechische
Bezeichnungen für moderne Begriffe]{Sprichwörtliches aus der Umgangs\textcompwordmark{}sprache.}

}]Mensch, ärgere dich nicht!

\switchcolumn

\begin{greek}[variant=ancient]%
μὴ σεαυτὸν ἔσθιε, ὦ ᾽γαθέ!

\end{greek}%
\switchcolumn*

Eines Mannes Rede ist keine Rede.

\switchcolumn

\begin{greek}[variant=ancient]%
πρὶν ἂν ἀμφοῖν μῦθον ἀκούσῃς, οὐκ ἂν δικάσαις.

\end{greek}%
\switchcolumn*

Das hieße Eulen nach Athen tragen.

\switchcolumn

\begin{greek}[variant=ancient]%
τίς γλαῦκ᾽ Ἀθήναζε ἄγαγεν;

\end{greek}%
\switchcolumn*

Vorsicht ist die Mutter der Weis\textcompwordmark{}heit. 

\switchcolumn

\begin{greek}[variant=ancient]%
ἡ (γὰρ) εὐλάβεια πάντα σώζει.

\end{greek}%
\switchcolumn*

Eine Schwalbe macht noch keinen Sommer.

\switchcolumn

\begin{greek}[variant=ancient]%
μία χελιδὼν ἔαρ οὐ ποιεῖ.

\end{greek}%
\switchcolumn*

Menge dich nicht in meine Sachen!

\switchcolumn

\begin{greek}[variant=ancient]%
μὴ τὸν ἐμὸν οἴκει οἶκον!!

\end{greek}%
\switchcolumn*

Der reine Menschenfeind (Timon)!

\switchcolumn

\begin{greek}[variant=ancient]%
Τίμων \emph{καθαρός!}

\end{greek}%
\switchcolumn*

Immer das alte Lied!

\switchcolumn

\begin{greek}[variant=ancient]%
ὁ Διὸς Κόρινθος!

\end{greek}%
\switchcolumn*

\begin{latin}%
Hic Rhodus, hic salta!

\end{latin}%
\switchcolumn

\begin{greek}[variant=ancient]%
ἰδοὺ ἡ Ῥόδος\footnote{\begin{latin}%
orig. \textgreek[variant=ancient]{ʼΡόδος}\end{latin}%
}, ἰδοὺ καὶ τὸ πήδημα!

\end{greek}%
\switchcolumn*

Ein trauriger Peter (Japper)!

\switchcolumn

\begin{greek}[variant=ancient]%
Μυσῶν ἔσχατος!

\end{greek}%
\switchcolumn*

Das Gute ist rar.

\switchcolumn

\begin{greek}[variant=ancient]%
ὀλίγον τὸ χρηστόν ἐστιν.

\end{greek}%
\switchcolumn*

Es ist kein Vorwärts\textcompwordmark{}kommen (für uns).

\switchcolumn

\begin{greek}[variant=ancient]%
οὔτε θέομεν οὔτ᾽ ἐλαύνομεν.!

\end{greek}%
\switchcolumn*

Geld regiert die Welt.

\switchcolumn

\begin{greek}[variant=ancient]%
ἅπαντα (γὰρ) τῷ πλουτεῖν ὑπήκοα.!

\end{greek}%
\switchcolumn*

\begin{latin}%
Donec eris felix, multos numerabis amicos.

\end{latin}%
\switchcolumn

\begin{greek}[variant=ancient]%
ζεῖ χύτρα, ζῇ φιλία.!

\end{greek}%
\switchcolumn*

Durch Schaden wird man klug!

\switchcolumn

\begin{greek}[variant=ancient]%
\quotedblbase παθὼν δέ τε νήπιος ἔγνω.``

\end{greek}%
\switchcolumn*

\begin{latin}%
Tempi passati!

\end{latin}%
\switchcolumn

\begin{greek}[variant=ancient]%
πάλαι ποτ᾽ ἦσαν ἄλκιμοι Μιλήσιοι.

\end{greek}%
\switchcolumn*

\begin{latin}%
Ubi bene, ibi patria!

\end{latin}%
\switchcolumn

\begin{greek}[variant=ancient]%
πατρὶς γάρ ἐστι πᾶσ᾽, ἵν ἂν πράττῃ τις εὖ.

\end{greek}%
\switchcolumn*

Er ist der beste Bruder auch nicht!

\switchcolumn

\begin{greek}[variant=ancient]%
ἐστὶ τοῦ πονηροῦ κόμματος.

\end{greek}%
\switchcolumn*

\begin{latin}%
Parturiunt montes etc.

\end{latin}%
\switchcolumn

\begin{greek}[variant=ancient]%
ὤδινεν ὄρος, εἶτα μῦν ἀπέτεκεν.

\end{greek}%
\switchcolumn*

Du giebst dir vergebliche Mühe.

\switchcolumn

\begin{greek}[variant=ancient]%
λίθον ἕψεις.

\end{greek}%
\switchcolumn*

Das Übel ärger machen.

\switchcolumn

\begin{greek}[variant=ancient]%
πλέον θάτερον ποιεῖν.

\end{greek}%
\switchcolumn*

Eile mit Weile.

\switchcolumn

\begin{greek}[variant=ancient]%
σπεῦδε βραδέως!\textgerman[spelling=old,babelshorthands=true]{ (Wahlspruch
des Kaisers Augustus.)}

\end{greek}%
\switchcolumn*

Laß dir genügen!

\switchcolumn

\begin{greek}[variant=ancient]%
πλέον ἥμισυ παντός!

\end{greek}%
