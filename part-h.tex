\switchcolumn*[


\part{Beim Skat\textcompwordmark{}spiel.}


\section{Ein Spiel mit Redens\textcompwordmark{}arten}

]Wollen wir nicht ein Spielchen machen?

\switchcolumn

\begin{greek}[variant=ancient]%
βούλεσθε παιδιὰν παίζωμεν;

\end{greek}%
\switchcolumn*

Meinetwegen.

\switchcolumn

\begin{greek}[variant=ancient]%
\emph{οὐδὲν κωλύει.}

\end{greek}%
\switchcolumn*

Was wollen wir spielen?

\switchcolumn

\begin{greek}[variant=ancient]%
παιδιὰν τίνα;

\end{greek}%
\switchcolumn*

Einen Skat wollen wir machen.

\switchcolumn

\begin{greek}[variant=ancient]%
(σκατιούμεθα).

\end{greek}%
\switchcolumn*

Wer giebt?

\switchcolumn

\begin{greek}[variant=ancient]%
τίς ὁ διαδώσων;

\end{greek}%
\switchcolumn*

Ich frage.

\switchcolumn

\begin{greek}[variant=ancient]%
ἐμὸν τὸ ἐρωτᾶν.

\end{greek}%
\switchcolumn*

Eichel, Grün, Roth, Schellen.

\switchcolumn

\begin{greek}[variant=ancient]%
τὰ βαλάνια, τὰ φυλλεῖα, τὰ ἐρυθρά, τά κρόταλα.

\end{greek}%
\switchcolumn*

Eichel sticht.

\switchcolumn

\begin{greek}[variant=ancient]%
κρατεῖ τὰ βαλάνια.

\end{greek}%
\switchcolumn*

Geben Sie Grün zu!

\switchcolumn

\begin{greek}[variant=ancient]%
ἀπόδος φυλλεῖα!

\end{greek}%
\switchcolumn*

Ich?

\switchcolumn

\begin{greek}[variant=ancient]%
ἐγώ;

\end{greek}%
\switchcolumn*

Freilich (Sie)!

\switchcolumn

\begin{greek}[variant=ancient]%
σὺ μέντοι!

\end{greek}%
\switchcolumn*

Was habe ich davon?

\switchcolumn

\begin{greek}[variant=ancient]%
τί κερδανῶ;

\end{greek}%
\switchcolumn*

Was ich für ein Pech habe!

\switchcolumn

\begin{greek}[variant=ancient]%
ὡς δυστυκής εἰμι!

\end{greek}%
\switchcolumn*

Nur nicht ängstlich!

\switchcolumn

\begin{greek}[variant=ancient]%
μὴ δέδιθι!

\end{greek}%
\switchcolumn*

Sehen Sie sich vor, daß Ihnen der rothe Wenzel nicht entgeht!

\switchcolumn

\begin{greek}[variant=ancient]%
εὐλαβοῦ, μὴ ἐκφύγῃ σε τῶν ἐρυθρῶν ὁ κράτιστος!

\end{greek}%
\switchcolumn*

Jetzt ist's an Ihnen, zu sehen, wie wir gewinnen!

\switchcolumn

\begin{greek}[variant=ancient]%
σὸν \emph{ἔργον} φροντίζειν, ὅπως κρατήσομεν.

\end{greek}%
\switchcolumn*

Jetzt gilt es!

\switchcolumn

\begin{greek}[variant=ancient]%
νῶν ὁ καιρός!

\end{greek}%
\switchcolumn*

Jetzt haben wir ihn! 

\switchcolumn

\begin{greek}[variant=ancient]%
νῶν ἔχεται μέσος!

\end{greek}%
\switchcolumn*

Hau' ihm, Lucas!

\switchcolumn

\begin{greek}[variant=ancient]%
παῖε, παῖε τὸν πανοῦργον!

\end{greek}%
\switchcolumn*

Das soll Ihnen schlecht bekommen, daß Sie das rothe Daus gestochen
haben!

\switchcolumn

\begin{greek}[variant=ancient]%
οὔ τοι μὰ Δία χαιρήσεις, ὁτιὴ τοῦτ᾽ ἔδρασας.!

\end{greek}%
\switchcolumn*

Verwünscht! Das ist zum Haaraus\textcompwordmark{}raufen!

\switchcolumn

\begin{greek}[variant=ancient]%
οἴμοι, διαρραγήσομαι!

\end{greek}%
\switchcolumn*

Ich weiß schon, wie Sie es machen.

\switchcolumn

\begin{greek}[variant=ancient]%
τοὺς τρόπους σου ἐπίσταμαι.

\end{greek}%
\switchcolumn*

Feine Rase!

\switchcolumn

\begin{greek}[variant=ancient]%
εὖ γε ξυνέβαλες!

\end{greek}%
\switchcolumn*

Du wunderst dich?

\switchcolumn

\begin{greek}[variant=ancient]%
ἐθαύμασας;!

\end{greek}%
\switchcolumn*

Darin bin ich Meister.

\switchcolumn

\begin{greek}[variant=ancient]%
ταύτεῃ κράτιστός εἰμι.

\end{greek}%
\switchcolumn*

Sie spielen falsch!

\switchcolumn

\begin{greek}[variant=ancient]%
ἀδικεῖς!

\end{greek}%
\switchcolumn*

Du hast die Mogelei nicht bemerkt.

\switchcolumn

\begin{greek}[variant=ancient]%
τὸ πραττόμενόν σε λέληθεν.

\end{greek}%
\switchcolumn*

Ist das wahr?

\switchcolumn

\begin{greek}[variant=ancient]%
τί λέγεις;

\end{greek}%
\switchcolumn*

Ent\textcompwordmark{}schuldigen Sie!

\switchcolumn

\begin{greek}[variant=ancient]%
σύγγνωθί μοι!

\end{greek}%
\switchcolumn*

Kellner, zünden Sie Licht an!

\switchcolumn

\begin{greek}[variant=ancient]%
ἅπτε, παῖ, λύχνον!

\end{greek}%
\switchcolumn*

Was fällt Ihnen denn ein, daß Sie die Zehn aus\textcompwordmark{}spielen? 

\switchcolumn

\begin{greek}[variant=ancient]%
τί δὴ μαθὼν τοῦτο ποιεῖς;

\end{greek}%
\switchcolumn*

Die Noth zwingt mich dazu.

\switchcolumn

\begin{greek}[variant=ancient]%
ἡ ἀνάγκη με πιέζει.

\end{greek}%
\switchcolumn*

Verwünscht! was soll ich thun?

\switchcolumn

\begin{greek}[variant=ancient]%
οἴ μοι, τί δράςω;

\end{greek}%
\switchcolumn*

Geben Sie mir einen guten Rath!

\switchcolumn

\begin{greek}[variant=ancient]%
χρηστόν τι συμβούλευσον.

\end{greek}%
\switchcolumn*

Er will's gewinnen.

\switchcolumn

\begin{greek}[variant=ancient]%
ἐθέλει οὗτος κρατῆσαι.

\end{greek}%
\switchcolumn*

Geben Sie sich keine vergebliche Mühe!

\switchcolumn

\begin{greek}[variant=ancient]%
λίθον ἕψεις!

\end{greek}%
\switchcolumn*

Hilf Himmel!

\switchcolumn

\begin{greek}[variant=ancient]%
Ἄπολλον ἀποτρόπαιε!

\end{greek}%
\switchcolumn*

O weh! Jetzt geht's uns (zweien) schlecht!

\switchcolumn

\begin{greek}[variant=ancient]%
ἒ ἔ, παρὰ νῷν στενάζειν!

\end{greek}%
\switchcolumn*

Gerade das will ich ja!

\switchcolumn

\begin{greek}[variant=ancient]%
τοῦτ᾽ αὐτὸ γὰρ καὶ βούλομαι!

\end{greek}%
\switchcolumn*

Zähle einmal!

\switchcolumn

\begin{greek}[variant=ancient]%
λόγισαι!

\end{greek}%
\switchcolumn*

Wir haben verspielt!

\switchcolumn

\begin{greek}[variant=ancient]%
\emph{ἀπολώλαμεν} ἡμεῖς.

\end{greek}%
\switchcolumn*

Bitte, bezahlen Sie!

\switchcolumn

\begin{greek}[variant=ancient]%
ἀπότισον \emph{δῆτα!}

\end{greek}%
\switchcolumn*

Mein Geld ist futsch!

\switchcolumn

\begin{greek}[variant=ancient]%
φροῦδα τὰ χρήματα!

\end{greek}%
\switchcolumn*

Es steht schlecht mit mir.

\switchcolumn

\begin{greek}[variant=ancient]%
φαῦλόν ἐστι τὸ ἐμὸν πρᾶγμα.

\end{greek}%
\switchcolumn*

Wir machen miserable Geschäfte.

\switchcolumn

\begin{greek}[variant=ancient]%
ἀθλίως πεπράγαμεν.

\end{greek}%
\switchcolumn*[


\section{Ein Grand}

](Ein Grand.)

\switchcolumn

\begin{greek}[variant=ancient]%
(τὸ παμμέγιστον.)

\end{greek}%
\switchcolumn*

A. Wer giebt denn?

\switchcolumn

\begin{greek}[variant=ancient]%
τίς ὁ διαδώσων;

\end{greek}%
\switchcolumn*

B. Du selbst.

\switchcolumn

\begin{greek}[variant=ancient]%
αὐτὸς σύ.

\end{greek}%
\switchcolumn*

C. Immer, wer fragt.

\switchcolumn

\begin{greek}[variant=ancient]%
ὁ ἀεὶ ἐρωτήσας.

\end{greek}%
\switchcolumn*

B. Nun gieb mir aber einmal anständige Karten; ich habe den ganzen
Abend noch kein Spiel gehabt!

\switchcolumn

\begin{greek}[variant=ancient]%
δός τι δῆτ᾽ ἐμοὶ· οὐδὲν γὰρ πώποτ᾽ ἔλαβον ἔγωγε τῇδε τῇ\footnote{\begin{latin}%
\textgreek[variant=ancient]{τὸ ῥῆμα οὐ δύναμαι διαγνῶναι.}\end{latin}%
} ἑσπέρᾳ!

\end{greek}%
\switchcolumn*

C. Ich frage. Grün Solo!

\switchcolumn

\begin{greek}[variant=ancient]%
ἐμὸν τὸ ἐρωτᾶν. τὰ φυλλεῖα αὐτὰ\footnote{\begin{latin}%
\textgreek[variant=ancient]{τὸ ῥῆμα οὐ δύναμαι διαγνῶναι.}\end{latin}%
} καθ᾽ αὑτά!

\end{greek}%
\switchcolumn*

B. Das halt' ich!

\switchcolumn

\begin{greek}[variant=ancient]%
ἔχω ἔγωγε!

\end{greek}%
\switchcolumn*

C. Null?

\switchcolumn

\begin{greek}[variant=ancient]%
τὸ μηδέν;

\end{greek}%
\switchcolumn*

B. Auch das.

\switchcolumn

\begin{greek}[variant=ancient]%
καὶ τοῦτό γε.

\end{greek}%
\switchcolumn*

C. Passe.

\switchcolumn

\begin{greek}[variant=ancient]%
παραχωρῶ ἔγωγε.

\end{greek}%
\switchcolumn*

A. Ich auch.

\switchcolumn

\begin{greek}[variant=ancient]%
κἀγώ.

\end{greek}%
\switchcolumn*

B. Grand.

\switchcolumn

\begin{greek}[variant=ancient]%
τὸ παμμέγιστον.

\end{greek}%
\switchcolumn*

B. Ich spiele selbst aus. Hier! Wenzel 'raus!

\switchcolumn

\begin{greek}[variant=ancient]%
ἐμὸν τὸ ἐξάρχειν. ἰδού. ἀπόδοτε δὴ τοὺς κρατίστους!

\end{greek}%
\switchcolumn*

C. Ja, den kann ich nicht!

\switchcolumn

\begin{greek}[variant=ancient]%
οὐ δυνατὸς ἐγὼ μὰ Δία ὑπὲρ τοῦτον.

\end{greek}%
\switchcolumn*

A. Nanu?!

\switchcolumn

\begin{greek}[variant=ancient]%
τί φής;

\end{greek}%
\switchcolumn*

B. Hurrah! Der Alte liegt im Skat! Hier!

\switchcolumn

\begin{greek}[variant=ancient]%
βαβαιάξ! ἀπόκειται ὁ παγκράτιστος! ἰδού

\end{greek}%
\switchcolumn*

C. Himmeldonnerwetter!

\switchcolumn

\begin{greek}[variant=ancient]%
ἐς κόρακας!

\end{greek}%
\switchcolumn*

A. Kreuzmillionen . . .!

\switchcolumn

\begin{greek}[variant=ancient]%
Ἄπολλον ἀποτρόπαιε!

\end{greek}%
\switchcolumn*

C. Ih, da soll doch der Deiwel 'reinfahren!

\switchcolumn

\begin{greek}[variant=ancient]%
οἴμοι κακοδαίμων!

\end{greek}%
\switchcolumn*

A. Heiliges Gewitter! Hast du denn gar nichts?

\switchcolumn

\begin{greek}[variant=ancient]%
ὦ Ζεῦ βασιλεῦ! οὐκ ἄρ᾽ ἔχεις\footnote{\begin{latin}%
\textgreek[variant=ancient]{τὸ ῥῆμα οὐ δύναμαι διαγνῶναι.}\end{latin}%
} οὐδέν;

\end{greek}%
\switchcolumn*

C. Dieser ist unser! 'rin, was Beine hat!

\switchcolumn

\begin{greek}[variant=ancient]%
ἀλλὰ τοῦτό γε γίγνεται ἥμῖν. νῦν ὁ καιρὸς ἐπιδοῦναι!

\end{greek}%
\switchcolumn*

B. Halt! Gesprochen wird nicht beim Spiel!

\switchcolumn

\begin{greek}[variant=ancient]%
μὴ δῆτα — οὐ γὰρ ἔστι λαλεῖν τῷ παίζοντι!

\end{greek}%
\switchcolumn*

C. So, das ist auch unser!\\
Gottlob! Aus dem Schneider wären wir!

\switchcolumn

\begin{greek}[variant=ancient]%
ἰδοὺ καὶ τοῦτο ἡμῖν!

τὸ μέσον καλῶς τετμήκαμεν!

\end{greek}%
\switchcolumn*

A. Oh, wir kriegen noch viel mehr!

\switchcolumn

\begin{greek}[variant=ancient]%
ἕξομεν ἔτι πολλῷ πλέον, ὦ τάν.

\end{greek}%
\switchcolumn*

B. Keinen Stich! Der Rest ist mein!

\switchcolumn

\begin{greek}[variant=ancient]%
οὐκ ἄλλ᾽ οὐδὲ ἕν. ἐμὰ γὰρ τὰ λοιπά!

\end{greek}%
\switchcolumn*

A. u. C. Oho! — Wahrhaftig!

\switchcolumn

\begin{greek}[variant=ancient]%
οὐδὲν λέγεις! — μὰ τὸν Δί᾽ οὐ τοίνυν!

\end{greek}%
\switchcolumn*

A. Ja wie konntest du aber auch \emph{die} Farbe spielen? Wir mußten
ja dicke gewinnen! 

\switchcolumn

\begin{greek}[variant=ancient]%
πῶς ἄρ᾽ οὖν ἐπὶ ταῦτα ἦλθες; ἐμέλλομεν γάρ τοι σφοδρῶς ὑπερέχειν!

\end{greek}%
\switchcolumn*

Ich sitze hier mit der ganzen Grün.

\switchcolumn

\begin{greek}[variant=ancient]%
ἐγὼ δὲ κάθημαι οὕτω πάντα τὰ φυλλεῖα ἔχων.

\end{greek}%
\switchcolumn*

C. So? Warum stichst du denn nicht? Ich habe ganz richtig aus\textcompwordmark{}gespielt.
— du bist schuld!

\switchcolumn

\begin{greek}[variant=ancient]%
ἄληθες; τί δὴ παθὴν οὐχ ὑπερέβαλες\footnote{\begin{latin}%
orig. \textgreek[variant=ancient]{ὑπερ-|έβαλες}\end{latin}%
} σύ; εὖ γὰρ ἐμοίησα ἔγωγε. — σὺ δὲ τούτου αἴτιος!

\end{greek}%
\switchcolumn*

B. Das war Grand mit Vieren! Sechzig. Wer giebt?

\switchcolumn

\begin{greek}[variant=ancient]%
παμμέγιστον τοῦτ᾽ ἦν μετὰ τεσσάρων! ἑξήκοντα. τίς ὁ διαδώσων;

\end{greek}%
